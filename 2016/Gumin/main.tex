\title{Использование символьных конечных преобразователей для лексического анализа динамически формируемого кода}

\titlerunning{Использование символьных конечных преобразователей для лексического анализа динамически формируемого кода}

\author{Гумин Егор Дмитриевич}

\authorrunning{Е.Д.Гумин}

\tocauthor{Е.Д.Гумин}
\institute{Санкт-Петербургский государственный университет\\
\email{gumin.egor@gmail.com}}

\maketitle             

\begin{abstract}
Существует подход к написанию программного кода, при котором код на одном языке программирования формируется программой на другом языке (такой код называется динамически формируемым). Подсветка синтаксиса и диагностика ошибок в динамически формируемом коде позволила бы упростить его написание и поддержку. Проект YaccConstructor, в рамках которого выполнена данная работа, позволяет производить статический анализ такого кода, что необходимо для обеспечения вышеуказанной функциональности. Но механизм лексического анализа в YaccConstructor обладает недостаточной производительностью. Вероятная причина этой проблемы -разрастание на больших алфавитах конечных преобразователей, используемых при лексическом анализе.В рамках данной работы исследован вопрос о возможности увеличения его производительности с помощью использования символьных конечных преобразователей из библиотеки Microsoft.Automata, которые позволяют представить структуры данных более компактно.
\end{abstract}

\section*{Введение}
Графы и графовые базы данных имеют широкое применение в таких областях, как биоинформатика, логистика, социальные сети и многие другие. Запросы к таким базам формулируются как задача поиска путей в графе, удовлетворяющих некоторым ограничениям. Во многих случаях такие ограничения формулируются в виде некоторой грамматики, наиболее выразительным её представлением является контекстно-свободная (КС) грамматика. В таком случае задача сводится к поиску путей в графе, которые бы соответствовали строкам в контекстно-свободном языке. Такую задачу назовем синтаксическим анализом графа. 

Одним из примеров применения синтаксического анализа графа является поиск подпоследовательности геномов в задаче биоинформатики. Из окружающей среды берется образец, по нему необходимо выявить подпоследовательности генов для классификации организмов в данном образце. Для этого по образцу строится метагеномная сборка, являющаяся комбинацией генов. Сборка представляет собой граф с последовательностями символов на ребрах. В таком графе необходимо найти подстроки, позволяющие провести классификацию. Такую задачу можно решить при помощи синтаксического анализа графов.

Существуют различные подходы к синтаксическому анализу графов (например,~\cite{GrigRagCFPQuerying},~\cite{Hellings},~\cite{Sevon}), одно из которых основано на алгоритма Generalised LL (GLL)~\cite{GrigRagCFPQuerying}. Алгоритм GLL позволяет без модификаций грамматики анализировать все КС-языки. Результатом работы данного алгоритма является компактное представление леса разбора Shared Packed Parse Forest (SPPF)~\cite{SPPF}. Данное представление разбора позволяет производить дополнительный анализ и получать таким образом дополнительную информацию, а так же производить семантические действия уже после разбора. Данное решение построено по принципу генераторов синтаксических анализаторов, когда по грамматике генерируется код анализаторов, что не всегда удобно, при написании запросов к графовым базам данных.

Существуют различные решения для поиска путей в графовых базах данных. Это встроенные инструменты и языки для запросов. К примеру, для базы данных Neo4j~\cite{Neo4j} существуют такие языки запросов как Cypher~\cite{Cypher} и openCypher~\cite{openCypher}. Однако они не поддерживают синтаксис запросов в стилеконтекстно-свободной грамматики и не возвращают пути в виде деревьев разбора. При работе с графовыми базами данных было бы удобно строить запрос к ним на языке, на котором написано целевое приложение. Это возможно реализовать техникой парсер-комбинаторов. Парсер-комбинаторы позволяют создавать синтаксические анализаторы динамически непосредственно в коде программы на некотором языке. Все существующие библиотеки парсер-комбинаторов анализируют только линейный вход — строки. Нашей задачей стала разработка библиотеки для синтаксического анализа графов. Существует библиотека Meerkat~\cite{Meerkat} на языке Scala~\cite{Scala}, реализующая синтаксический анализ строк методом парсер-комбинаторов, используя идеи, схожие с используемыми в алгоритме GLL~\cite{GLL}. Она обладает рядом преимуществ:

\begin{itemize} 
\item результатом работы библиотеки является лес разбора SPPF;
\item библиотеку возможно запустить на JVM;
\item разбор происходит в худшем случае за $O(n^3)$, где n – длина входной последовательности.
\end{itemize}

Было принято решение использовать данную библиотеку для решения задачи.

Таким образом в рамках данной работы была поставлена задача модифицировать существующую библиотеку для синтаксического анализа графов с помощью техники парсер-комбинаторов.

\section{Постановка задачи}
Целью данной работы является создание решения для синтаксического анализа регулярных множеств, применимого для работы со входными данными большого размера. Для достижения поставленной цели были поставлены следующие задачи:

\begin{itemize}  
\item Разработать алгоритм синтаксического анализа динамически  формируемого кода на основе алгоритма GLL. 
\item Доказать корректность предложенного алгоритма.
\item Применить к задаче поиска на входных данных большого размера.
\item Реализовать предложенный алгоритм в рамках проекта YaccConstructor.
\item Произвести эксперименты и сравнение.
\end{itemize}

\section{Обзор}
\subsection{Синтаксический анализ графов}
При работе с графами, например в графовых базах данных, возникает необходимость выполнения запросов поиска путей, удовлетворяющих заданным ограничениям. Ограничения задаются, как правило, регулярной грамматикой, однако контекстно-свободные грамматики представляют собой более выразительный язык запросов. Контекстно-свободная грамматика (КС-грамматика) $G$ --- четверка $(T, N, P, S)$, где $N$ --- множество нетерминалов, $T$ --- множество терминалов ($T \cap N = \varnothing$), $P = \{ A \rightarrow \alpha \mid A \in N, \alpha \in (N \cup T)^*\}$ --- множество правил грамматики и $S \in N$ --- стартовый символ. Грамматика $G = (\{+, -, a\}, \{ E, N \}, P, E)$, множество правил которой $P$ представлены на листинге~\ref{grmG1}, задает язык арифметических выражений со сложением и умножением над переменными $a$.

\begin{listing}
\caption{Правила грамматики $G$}
\label{grmG1}
\centering
$\begin{array}{ll}
E \rightarrow & N \ + \ N \mid N \ - \ N
\\
N \rightarrow & a
\end{array}$
 \end{listing}

Итак, задача выполнения запросов является задачей поиска в ориентированном графе всех путей, представляющих собой строки языка, заданные КС-грамматикой. Одно из возможных решений такой задачи --- это модификация алгоритмов классического синтаксического анализа строк. Работа~\cite{Sevon} модифицирует алгоритм Эрли. Данный подход позволяет записывать запросы к графовым структурам данных с указанием направления поиска: прямое направление, когда поиск производится в направлении ребер графа, или обратном --- по обратным ребрам. Данный алгоритм строит только некоторое приближение к результату: обработка циклов входного графа осуществляется только до некоторой глубины, специфицируемой пользователем, в результате чего некоторые пути могут быть утеряны. Результатом работы алгоритма является подграф, содержащий пути разбора. 

Для последующего анализа путей из результата удобно иметь информацию об их синтаксической структуре, например в форме абстрактных синтаксических деревьев. Так как множество путей может быть бесконечным, то и синтаксических деревьев может быть бесконечно много, поэтому возникает вопрос об их представлении. В алгоритмах обобщенного синтаксического анализа, в случае существования нескольких деревьев разбора для одной строки (в виду неоднозначностей грамматики), используется компактное представление леса разбора SPPF (Shared Packed Parse Forest). В структуре SPPF переиспользуются общие фрагменты разных выводов, за счет его размер полиномиален от размера входной строки. Существуют модификации обобщенных алгоритмов, применимые для выполнения запросов в контекстно-свободных ограничениях к графам: алгоритмы на основе RNGLR~\cite{RNGLR} и GLL~\cite{GrigRagCFPQuerying}. Первый позволяет построить лес разбора SPPF, второй --- бинаризованный лес разбора Binarized Shared Packed Parse Forest~\cite{SPPF} с лучшими пространственными характеристиками. 

Binarized Shared Packed Parse Forest имеет размер $O(n^3)$, где n --- длина входной строки. В отличие от обычного дерева разбора, внутренние узлы которого всегда соответствуют нетерминалам грамматик, в BSPPF используются дополнительные типы узлов: упакованный узел (packed node) и промежуточный узел (intermediate node). Упакованный узел создаётся для представления неоднозначностей вывода (его дети соответствуют разным продукциям). Промежуточный узел используется для бинаризации, когда правило длины больше 2 представляется как цепочка применений правил длины 2. Именно за счет бинаризации достигается кубический размер представления леса разбора. В SPPF могут быть циклы. На рис.~\ref{fig:sppfV} представлено, как выглядит SPPF, состоящее из деревьев на рис.~\ref{fig:sppfA} и рис.~\ref{fig:sppfB}. А на рис.~\ref{fig:sppfG} представлена его бинаризованная форма. Пример взят из статьи~\cite{IzmCombinator}.

 \begin{figure}[t]
 \centering
    \subfloat[Дерево разбора 1]{
        \label{fig:sppfA}
        \includegraphics[width=0.35\textwidth]{Smolina/pics/SppfA.png}
    }
    \subfloat[Дерево разбора 2]{
        \label{fig:sppfB}
        \includegraphics[width=0.35\textwidth]{Smolina/pics/SppfB.png}        
    }

    ~\\~
    \subfloat[SPPF]{
        \label{fig:sppfV}
        \includegraphics[width=0.35\textwidth]{Smolina/pics/SppfV.png}        
    }
    \subfloat[Бинаризинное SPPF]{
        \label{fig:sppfG}
        \includegraphics[width=0.45\textwidth]{Smolina/pics/SppfG.png}        
    }
 \caption{Граф SPPF для двух вариантов вывода}
\end{figure}

Решение на RNGLR и GLL построено на основе генерации синтаксических анализаторов. Такое решение не является удобным при работе с графами и графовыми базами данных, так как при добавлении малейших изменений необходимо генерировать руками новый синтаксический анализатор, а также требуется использование дополнительного предметно-ориентированного языка для задания запроса. 

В сфере промышленных графовых баз данных существуют свои языки запросов. К примеру для графовой базы данных Neo4j~\cite{Neo4j} существуют языки Cypher~\cite{Cypher} и openCypher~\cite{openCypher}, а для OrientDB~\cite{OrientDB} используется язык SQL~\cite{Sql}. Ни один из них не поддерживает формат запроса в виде контекстно-свободной грамматики. Более того, результат запроса всегда --- простые строки, что усложняет их дальнейший анализ.

Таким образом, нашей целью стала разработка решения, в котором грамматику можно специфицировать в коде целевого приложения. Один из возможных подходов к решению данной задачи --- использование техники парсер-комбинаторов~\cite{HOFunParsing}.

\subsection{Техника парсер-комбинаторов}
Комбинатор --- это функция высшего порядка, которая из набора функций строит новую функцию. Возможность принимать функции как аргументы, комбинировать их и возвращать как результат является важной особенностью функциональных языков программирования. Парсер-комбинатор --- это функция высшего порядка, которая на вход получает множество синтаксических анализаторов и возвращает новый синтаксический анализатор. 

Для синтаксического анализа необходимо научиться анализировать элементарные сущности (терминалы, нетерминалы), осуществлять последовательное применение анализаторов и поддержать возможность осуществлять выбор анализатора для разбора суффикса сроки. Эти требования задают минимальный набор комбинаторов. Техника парсер-комбинаторов позволяет из элементарных анализаторов конструировать более сложные. Интеграция с языком программирования приложения, в котором применяется синтаксический анализатор, добавляет гибкости и расширяемости в сравнении с генераторами синтаксических анализаторов. Приведём пример реализации простейшего парсер-комбинатора на Scala.

В листинге~\ref{parser1} представлен синтаксический анализатор, который принимает на вход строковую последовательность, затем разбирает строку, начинающуюся с определенного терминала, и возвращает результат разбора и необработанный суффикс строки.

\begin{listing}
\caption{Синтаксический анализатор терминала}
\label{parser1}
\centering
\includegraphics[width=0.7\textwidth]{Smolina/pics/parser1.png}
\end{listing}

Для того чтобы получить синтаксический анализатор подстроки, можно воспользоваться парсер-комбинатором,
который составлял последовательность из анализаторов символов --- парсер-комбинатор последовательности seq. На листинге~\ref{parserSeq} приведен пример реализации.

\begin{listing}
\caption{Парсер-комбинатор последовательности}
\label{parserSeq}
\centering
\includegraphics[width=0.7\textwidth]{Smolina/pics/parserSeq.png}
\end{listing}

Теперь, чтобы получить синтаксический анализатор, начинающийся с подстроки ``ABC'', мы можем воспользоваться элементарными синтаксическими анализаторами символов и парсер-комбинатором последовательности (см. листиниг~\ref{parserABC}).

\begin{listing}
\caption{Парсер-комбинатор строки “ABC”}
\label{parserABC}
\centering
\includegraphics[width=0.9\textwidth]{Smolina/pics/parserABC.png}
\end{listing}

Простые парсер-комбинаторы, основанные на рекурсивном спуске, представляют собой интуитивно ясную модель и поэтому удобны для
отладки. Однако они имеют экспоненциальную сложность относительно размеров грамматики~\cite{Popov}. Это связано с тем, что в наивной реализации рекурсивного спуска при откате не сохраняются результаты и разбор префикса строки одним и тем же синтаксическим анализатором может происходить многократно. Мемоизация~\cite{Memoization} позволяет решить эту проблему за счет переиспользования результатов применения синтаксических анализаторов к подстрокам. Таким образом, однажды выполненное вычисление никогда не повторяется --- результат просто берется из таблицы.

Другой проблемой, ассоциируемой с парсер-комбинаторами, является трудность обработки леворекурсивных определений анализаторов. Например, синтаксический анализ наивным рекурсивным спуском в соответствии с грамматикой, представленной в листинге~\ref{grmG2}, никогда не завершится.

 \begin{listing}
\caption{Леворекурсивная грамматика}
\label{grmG2}
\centering
$\begin{array}{rl}
E \rightarrow E \ + \ a \ | \ a
\end{array}$
 \end{listing}

Данная проблема имеет несколько решений, одно из которых основано на ограничении числа вызовов нетерминала некоторой константой, связанной с длиной входной последовательности ~\cite{ParserComb}. Количество применений каждого распознавателя к каждой позиции в строке ограничивается длиной неразобранного суффикса строки. Данный подход не применим, если длина последовательности неизвестна, например, при считывании символов с сетевого сокета. Во-вторых, такой подход обладает сложностью $O(n^4)$, вместо ожидаемой $O(n^3)$, где n --- длина последовательности. Подобные проблемы решаются техникой Continuation Parsing Style (CPS)~\cite{MemoizationInTopDown}. В отличие от первого метода, цепочка леворекурсивных вызовов завершается, когда происходит второй вызов синтаксического анализатора в данной позиции входа. Затем результаты для леворекурсивных синтаксических анализаторов эффективно вычисляются в цикле: пока создается новый результат, завершенные пути синтаксического анализа, записанные как продолжения, перезапускаются в новой входной позиции. Как результат, обработка рекурсивных правил более эффективна и не требует знания длины последовательности. Более подробно об этом речь пойдет в следующей главе.



\section{Эксперименты}

Были проведены экспериментальные исследования, целью которых являлась проверка того, что конъюнктивные грамматики позволяют задавать структуру тРНК так, что синтаксический анализатор находит меньше некорректных цепочек.

\begin{figure}[h]
\begin{center}
\begin{verbatim}

[<Start>]
folded: stem<(any*[1..3] 
              stem<any*[7..10]> 
              any*[1..3] 
              stem<any*[5..8]> 
              any*[3..5] 
              stem<any*[5..8]>
              )>

stem<s>: 
      A stem<s> U
    | U stem<s> A
    | C stem<s> G
    | G stem<s> C
    | G stem<s> U
    | U stem<s> G
    | s

any: A | U | G | C

\end{verbatim}
\caption{КС-грамматика вторичной структуры тРНК}
\label{TRNAgrammar}
\end{center}
\end{figure}


\begin{figure}
\begin{center}
ACACCCCCCCUCACCCCCUCCCACCCCCUU
\end{center}
\caption{Пример цепочки нуклеотидов, сгенеированной для экспериментов}
\label{rnachain}
\end{figure}


\begin{figure}
\begin{center}
\begin{verbatim}
[<Start>]
folded: stem<subseq> & (any*[7..9] subseq any*[7..9])

subseq: any*[1..3] 
        stem<any*[7..10]> & (any*[4..6] any*[7..10] any*[4..6])
        any*[1..3] 
        stem<any*[5..8]> & (any*[6] any*[5..8] any*[6])
        any*[3..5] 
        stem<any*[5..8]> & (any*[4..5] any*[5..8] any*[4..5])
        
stem<s>:
      A stem<s> U
    | U stem<s> A
    | C stem<s> G
    | G stem<s> C
    | G stem<s> U
    | U stem<s> G
    | s

any: A | U | G | C

\end{verbatim}
\caption{Конъюнктивная грамматика структуры тРНК}
\label{TRNAgrammarConj}
\end{center}
\end{figure}


\begin{figure}
\begin{center}
\begin{tikzpicture}
\begin{axis}[
    legend pos = north west,
  xlabel = {Количество лексем},
  ylabel = {Время, с}
]
\addplot coordinates {
  (100,2) (200,17) (300,42) (400,81) (500,128) (600,190) (700,264) (800,345) (900,446) (1000,562)
};
\addplot coordinates {
  (100,1) (200,2) (300,4) (400,8) (500,12) (600,18) (700,22) (800,28) (900,36) (1000,43)
};
\legend{ 
  грамматика $G_{2}$, 
  грамматика $G_{3}$
};
\end{axis}
\end{tikzpicture}
\end{center}
\caption{Среднее время работы алгоритма на конъюнктивной и контекстно-свободной грамматиках тРНК}
\label{time}
\end{figure}

\begin{table}[h]
\begin{center}
  \begin{tabular}{ c | c | c }
    \hline
     & КС-грамматика & Конъюнктивная грамматика \\ \hline
    Тест 1. & 15 & 0 \\\hline
    Тест 2. & 5 & 0 \\\hline
    Тест 3. & 11 & 0 \\
    \hline
  \end{tabular}
\end{center}
\caption{Количество некорректных цепочек, распознанных синтаксическим анализатором}
\label{mistakes}
\end{table}

На рис.~\ref{TRNAgrammar} и~\ref{TRNAgrammarConj} представлены грамматки, описывающие структуру тРНК. Грамматика $G_2$ является контекстно-свободной, а грамматика $G_3$~--- конъюнктивной. По данным грамматикам были сгенерированы соответствующие синтаксические анализаторы.

На вход построенным синтаксическим анализаторам подавались сгенерированные цепочки ДНК длиной от 100 до 1000 симовлов. Эти цепочки содержали в себе последовательности тРНК, а также другие последовательности, которые можно ложно признать за тРНК. Напимер, цепочка на рис.~\ref{rnachain}, хоть и не является тРНК, распознаётся грамматикой $G_2$, но не распознаётся грамматикой $G_3$.

Результаты экспериментов приведены в таблице~\ref{mistakes}. Из них ясно, что грамматика $G_2$ не распознаёт ложные цепочки, распознаваемые грамматикой $G_3$. Время работы синтаксических анализаторов показано на графике, изображённом на рис.~\ref{time}. По графику видно, что время работы синтаксического анализатора, построенного по грамматике $G_3$, значительно превышает время работы другого.

Таким образом, конъюнктивная грамматика позволяет отсеивать цепочки, ложно распозначаемые КС-грамматикой, но за время, значительно большее, чем время работы анализатора по КС-грамматике.

\section*{Заключение}
В ходе работы получены следующие результаты:
\begin{itemize}
    \item реализована поддержка конъюнктивных грамматик в языке спецификации грамматик YARD;
    \item реализована поддержка конъюнктивных грамматик в генераторе GLL-анализаторов;
    \item результаты экспериментально проверены на небольших метагеномных сборках.
\end{itemize}

\subsubsection*{Дальнейшее направление работы}

В первую очередь, необходимо снизить время работы алгоритма. Полученная реализация предполагает полный обход дерева разбора, что, безусловно, влияет на производительность алгоритма. Кроме того, можно исследовать возможность расширения класса распознаваемых языков до булевых, на основе полученных результатов.



\begin{thebibliography}{99}

\bibitem{st}
  Nikolaj Bj{\o}rner, Margus Veanes.
  Symbolic transducers: Technical Report MSR-TR-2011-3.
  Microsoft Research, 2011.

\bibitem{YCUrl}
    Сайт проекта YaccConstructor.
    \url{ https://github.com/YaccConstructor/ }

\bibitem{MSAUrl}
    Сайт проекта Mircosoft.Automata.
    \url{ http://research.microsoft.com/en-us/projects/automata/ }

\bibitem{Z3Url}
    Сайт проекта Z3.
    \url{ https://github.com/Z3Prover/z3/ }

\bibitem{BekUrl}
    Сайт проекта Bek.
    \url{ http://research.microsoft.com/en-us/projects/bek/ }

\bibitem{BekArticle}
  Pieter Hooimeijer, Benjamin Livshits, David Molnar et al.
  Fast and precise sanitizer analysis with BEK. 
  Proceedings of the 20th USENIX conference on Security, USENIX Association. 2011.

\bibitem{polubelova}
  Полубелова М.И. 
  Лексический анализ динамически формируемых строковых выражений.
  \url{http://se.math.spbu.ru/SE/diploma/2015/bmo/444-Polubelova-report.pdf}



\bibitem{articleYC}
  Я.А. Кириленко, С.В. Григорьев, Д.А. Авдюхин. 
  Разработка синтаксических анализаторов в проектах по автоматизированному реинжинирингу информационных систем. 
  2013. 

\bibitem{articleZ3}
  De Moura Leonardo, Bj{\o}rner Nikolaj. 
  Z3: An efficient SMT solver.
  Tools and Algorithms for the Construction and Analysis of Systems. –– Springer, 2008. –– P. 337–340.

\bibitem{FST}
  Hanneforth Thomas. 
  Finite-state Machines: Theory and Applications Unweighted Finite-state Automata.
  Institut fur Linguistik Universitat Potsdam.


\bibitem{GrigorievPhd}
  Григорьев С.В.
  Синтаксический анализ динамически формируемых программ.
  2016.

\bibitem{STcompose}
  Margus Veanes, Pieter Hooimeijer, Benjamin Livshits et al.
  Symbolic finite state transducers: Algorithms and applications.
  ACM SIGPLAN Notices. –– 2012. –– Vol. 47, no. 1. –– P. 137–150.  

\bibitem{Alvor}
  Aivar Annamaa, Andrey Breslav, Jevgeni Kabanov, Varmo Vene.
  An Interactive Tool for Analyzing Embedded SQL Queries. 
  Programming Languages and Systems. –– Springer: Berlin, 2010. –– P. 131–138.

\bibitem{JSA}
  Christensen Aske Simon, M{\o}ller Anders, I.Schwartzbach Michael.
  Precise Analysis of String Expressions. 
  Proc. 10th International Static Analysis Symposium (SAS). –– Springer-Verlag: Berlin, 2003. ––  June. –– P. 1–18.

\bibitem{PHPSA}
  Minamide Yasuhiko.
  Static approximation of dynamically generated web pages.
  In Proceedings of the 14th International Conference on World Wide Web, WWW ’05. –– ACM, 2005. –– P. 432–441.

\end{thebibliography}

