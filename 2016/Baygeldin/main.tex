\title{Оптимизация алгоритма лексического анализа динамически формируемого кода}

\titlerunning{Оптимизация алгоритма лексического анализа динамически формируемого кода}

\author{Байгельдин Александр Юрьевич}

\authorrunning{А.Ю.Байгельдин}

\tocauthor{А.Ю.Байгельдин}
\institute{Санкт-Петербургский государственный университет\\
\email{a.baygeldin@gmail.com}}

\maketitle             

\begin{abstract}
Динамически формируемый код --- это код, который может быть получен и использован внутри другого кода при помощи строковых операций, таких как конкатенация, циклы, замена подстроки. Пример такого кода --- запросы SQL, которые составляются динамически в языках общего назначения. Поток лексем при анализе динамически формируемого кода, в отличие от классического лексического анализа, нелинеен, что делает невозможным непосредственное применение существующих подходов. Одно из возможных решений заключается в применении операции композиции к двум конечным преобразователям: один их них построен по регулярной аппроксимации множества значений динамически генерируемого кода, второй является лексическим анализатором. Данная работа выполнена в рамках проекта YaccConstructor в котором реализовано такое решение, однако, его производительность оказалась недостаточной. В рамках данной работы был реализован алгоритм композиции, обладающий лучшей производительностью, и произведено сравнение с существующим алгоритмом.
\end{abstract}

\section*{Введение}
Графы и графовые базы данных имеют широкое применение в таких областях, как биоинформатика, логистика, социальные сети и многие другие. Запросы к таким базам формулируются как задача поиска путей в графе, удовлетворяющих некоторым ограничениям. Во многих случаях такие ограничения формулируются в виде некоторой грамматики, наиболее выразительным её представлением является контекстно-свободная (КС) грамматика. В таком случае задача сводится к поиску путей в графе, которые бы соответствовали строкам в контекстно-свободном языке. Такую задачу назовем синтаксическим анализом графа. 

Одним из примеров применения синтаксического анализа графа является поиск подпоследовательности геномов в задаче биоинформатики. Из окружающей среды берется образец, по нему необходимо выявить подпоследовательности генов для классификации организмов в данном образце. Для этого по образцу строится метагеномная сборка, являющаяся комбинацией генов. Сборка представляет собой граф с последовательностями символов на ребрах. В таком графе необходимо найти подстроки, позволяющие провести классификацию. Такую задачу можно решить при помощи синтаксического анализа графов.

Существуют различные подходы к синтаксическому анализу графов (например,~\cite{GrigRagCFPQuerying},~\cite{Hellings},~\cite{Sevon}), одно из которых основано на алгоритма Generalised LL (GLL)~\cite{GrigRagCFPQuerying}. Алгоритм GLL позволяет без модификаций грамматики анализировать все КС-языки. Результатом работы данного алгоритма является компактное представление леса разбора Shared Packed Parse Forest (SPPF)~\cite{SPPF}. Данное представление разбора позволяет производить дополнительный анализ и получать таким образом дополнительную информацию, а так же производить семантические действия уже после разбора. Данное решение построено по принципу генераторов синтаксических анализаторов, когда по грамматике генерируется код анализаторов, что не всегда удобно, при написании запросов к графовым базам данных.

Существуют различные решения для поиска путей в графовых базах данных. Это встроенные инструменты и языки для запросов. К примеру, для базы данных Neo4j~\cite{Neo4j} существуют такие языки запросов как Cypher~\cite{Cypher} и openCypher~\cite{openCypher}. Однако они не поддерживают синтаксис запросов в стилеконтекстно-свободной грамматики и не возвращают пути в виде деревьев разбора. При работе с графовыми базами данных было бы удобно строить запрос к ним на языке, на котором написано целевое приложение. Это возможно реализовать техникой парсер-комбинаторов. Парсер-комбинаторы позволяют создавать синтаксические анализаторы динамически непосредственно в коде программы на некотором языке. Все существующие библиотеки парсер-комбинаторов анализируют только линейный вход — строки. Нашей задачей стала разработка библиотеки для синтаксического анализа графов. Существует библиотека Meerkat~\cite{Meerkat} на языке Scala~\cite{Scala}, реализующая синтаксический анализ строк методом парсер-комбинаторов, используя идеи, схожие с используемыми в алгоритме GLL~\cite{GLL}. Она обладает рядом преимуществ:

\begin{itemize} 
\item результатом работы библиотеки является лес разбора SPPF;
\item библиотеку возможно запустить на JVM;
\item разбор происходит в худшем случае за $O(n^3)$, где n – длина входной последовательности.
\end{itemize}

Было принято решение использовать данную библиотеку для решения задачи.

Таким образом в рамках данной работы была поставлена задача модифицировать существующую библиотеку для синтаксического анализа графов с помощью техники парсер-комбинаторов.

\section{Постановка задачи}
Целью данной работы является создание решения для синтаксического анализа регулярных множеств, применимого для работы со входными данными большого размера. Для достижения поставленной цели были поставлены следующие задачи:

\begin{itemize}  
\item Разработать алгоритм синтаксического анализа динамически  формируемого кода на основе алгоритма GLL. 
\item Доказать корректность предложенного алгоритма.
\item Применить к задаче поиска на входных данных большого размера.
\item Реализовать предложенный алгоритм в рамках проекта YaccConstructor.
\item Произвести эксперименты и сравнение.
\end{itemize}

\section{Обзор}
\subsection{Синтаксический анализ графов}
При работе с графами, например в графовых базах данных, возникает необходимость выполнения запросов поиска путей, удовлетворяющих заданным ограничениям. Ограничения задаются, как правило, регулярной грамматикой, однако контекстно-свободные грамматики представляют собой более выразительный язык запросов. Контекстно-свободная грамматика (КС-грамматика) $G$ --- четверка $(T, N, P, S)$, где $N$ --- множество нетерминалов, $T$ --- множество терминалов ($T \cap N = \varnothing$), $P = \{ A \rightarrow \alpha \mid A \in N, \alpha \in (N \cup T)^*\}$ --- множество правил грамматики и $S \in N$ --- стартовый символ. Грамматика $G = (\{+, -, a\}, \{ E, N \}, P, E)$, множество правил которой $P$ представлены на листинге~\ref{grmG1}, задает язык арифметических выражений со сложением и умножением над переменными $a$.

\begin{listing}
\caption{Правила грамматики $G$}
\label{grmG1}
\centering
$\begin{array}{ll}
E \rightarrow & N \ + \ N \mid N \ - \ N
\\
N \rightarrow & a
\end{array}$
 \end{listing}

Итак, задача выполнения запросов является задачей поиска в ориентированном графе всех путей, представляющих собой строки языка, заданные КС-грамматикой. Одно из возможных решений такой задачи --- это модификация алгоритмов классического синтаксического анализа строк. Работа~\cite{Sevon} модифицирует алгоритм Эрли. Данный подход позволяет записывать запросы к графовым структурам данных с указанием направления поиска: прямое направление, когда поиск производится в направлении ребер графа, или обратном --- по обратным ребрам. Данный алгоритм строит только некоторое приближение к результату: обработка циклов входного графа осуществляется только до некоторой глубины, специфицируемой пользователем, в результате чего некоторые пути могут быть утеряны. Результатом работы алгоритма является подграф, содержащий пути разбора. 

Для последующего анализа путей из результата удобно иметь информацию об их синтаксической структуре, например в форме абстрактных синтаксических деревьев. Так как множество путей может быть бесконечным, то и синтаксических деревьев может быть бесконечно много, поэтому возникает вопрос об их представлении. В алгоритмах обобщенного синтаксического анализа, в случае существования нескольких деревьев разбора для одной строки (в виду неоднозначностей грамматики), используется компактное представление леса разбора SPPF (Shared Packed Parse Forest). В структуре SPPF переиспользуются общие фрагменты разных выводов, за счет его размер полиномиален от размера входной строки. Существуют модификации обобщенных алгоритмов, применимые для выполнения запросов в контекстно-свободных ограничениях к графам: алгоритмы на основе RNGLR~\cite{RNGLR} и GLL~\cite{GrigRagCFPQuerying}. Первый позволяет построить лес разбора SPPF, второй --- бинаризованный лес разбора Binarized Shared Packed Parse Forest~\cite{SPPF} с лучшими пространственными характеристиками. 

Binarized Shared Packed Parse Forest имеет размер $O(n^3)$, где n --- длина входной строки. В отличие от обычного дерева разбора, внутренние узлы которого всегда соответствуют нетерминалам грамматик, в BSPPF используются дополнительные типы узлов: упакованный узел (packed node) и промежуточный узел (intermediate node). Упакованный узел создаётся для представления неоднозначностей вывода (его дети соответствуют разным продукциям). Промежуточный узел используется для бинаризации, когда правило длины больше 2 представляется как цепочка применений правил длины 2. Именно за счет бинаризации достигается кубический размер представления леса разбора. В SPPF могут быть циклы. На рис.~\ref{fig:sppfV} представлено, как выглядит SPPF, состоящее из деревьев на рис.~\ref{fig:sppfA} и рис.~\ref{fig:sppfB}. А на рис.~\ref{fig:sppfG} представлена его бинаризованная форма. Пример взят из статьи~\cite{IzmCombinator}.

 \begin{figure}[t]
 \centering
    \subfloat[Дерево разбора 1]{
        \label{fig:sppfA}
        \includegraphics[width=0.35\textwidth]{Smolina/pics/SppfA.png}
    }
    \subfloat[Дерево разбора 2]{
        \label{fig:sppfB}
        \includegraphics[width=0.35\textwidth]{Smolina/pics/SppfB.png}        
    }

    ~\\~
    \subfloat[SPPF]{
        \label{fig:sppfV}
        \includegraphics[width=0.35\textwidth]{Smolina/pics/SppfV.png}        
    }
    \subfloat[Бинаризинное SPPF]{
        \label{fig:sppfG}
        \includegraphics[width=0.45\textwidth]{Smolina/pics/SppfG.png}        
    }
 \caption{Граф SPPF для двух вариантов вывода}
\end{figure}

Решение на RNGLR и GLL построено на основе генерации синтаксических анализаторов. Такое решение не является удобным при работе с графами и графовыми базами данных, так как при добавлении малейших изменений необходимо генерировать руками новый синтаксический анализатор, а также требуется использование дополнительного предметно-ориентированного языка для задания запроса. 

В сфере промышленных графовых баз данных существуют свои языки запросов. К примеру для графовой базы данных Neo4j~\cite{Neo4j} существуют языки Cypher~\cite{Cypher} и openCypher~\cite{openCypher}, а для OrientDB~\cite{OrientDB} используется язык SQL~\cite{Sql}. Ни один из них не поддерживает формат запроса в виде контекстно-свободной грамматики. Более того, результат запроса всегда --- простые строки, что усложняет их дальнейший анализ.

Таким образом, нашей целью стала разработка решения, в котором грамматику можно специфицировать в коде целевого приложения. Один из возможных подходов к решению данной задачи --- использование техники парсер-комбинаторов~\cite{HOFunParsing}.

\subsection{Техника парсер-комбинаторов}
Комбинатор --- это функция высшего порядка, которая из набора функций строит новую функцию. Возможность принимать функции как аргументы, комбинировать их и возвращать как результат является важной особенностью функциональных языков программирования. Парсер-комбинатор --- это функция высшего порядка, которая на вход получает множество синтаксических анализаторов и возвращает новый синтаксический анализатор. 

Для синтаксического анализа необходимо научиться анализировать элементарные сущности (терминалы, нетерминалы), осуществлять последовательное применение анализаторов и поддержать возможность осуществлять выбор анализатора для разбора суффикса сроки. Эти требования задают минимальный набор комбинаторов. Техника парсер-комбинаторов позволяет из элементарных анализаторов конструировать более сложные. Интеграция с языком программирования приложения, в котором применяется синтаксический анализатор, добавляет гибкости и расширяемости в сравнении с генераторами синтаксических анализаторов. Приведём пример реализации простейшего парсер-комбинатора на Scala.

В листинге~\ref{parser1} представлен синтаксический анализатор, который принимает на вход строковую последовательность, затем разбирает строку, начинающуюся с определенного терминала, и возвращает результат разбора и необработанный суффикс строки.

\begin{listing}
\caption{Синтаксический анализатор терминала}
\label{parser1}
\centering
\includegraphics[width=0.7\textwidth]{Smolina/pics/parser1.png}
\end{listing}

Для того чтобы получить синтаксический анализатор подстроки, можно воспользоваться парсер-комбинатором,
который составлял последовательность из анализаторов символов --- парсер-комбинатор последовательности seq. На листинге~\ref{parserSeq} приведен пример реализации.

\begin{listing}
\caption{Парсер-комбинатор последовательности}
\label{parserSeq}
\centering
\includegraphics[width=0.7\textwidth]{Smolina/pics/parserSeq.png}
\end{listing}

Теперь, чтобы получить синтаксический анализатор, начинающийся с подстроки ``ABC'', мы можем воспользоваться элементарными синтаксическими анализаторами символов и парсер-комбинатором последовательности (см. листиниг~\ref{parserABC}).

\begin{listing}
\caption{Парсер-комбинатор строки “ABC”}
\label{parserABC}
\centering
\includegraphics[width=0.9\textwidth]{Smolina/pics/parserABC.png}
\end{listing}

Простые парсер-комбинаторы, основанные на рекурсивном спуске, представляют собой интуитивно ясную модель и поэтому удобны для
отладки. Однако они имеют экспоненциальную сложность относительно размеров грамматики~\cite{Popov}. Это связано с тем, что в наивной реализации рекурсивного спуска при откате не сохраняются результаты и разбор префикса строки одним и тем же синтаксическим анализатором может происходить многократно. Мемоизация~\cite{Memoization} позволяет решить эту проблему за счет переиспользования результатов применения синтаксических анализаторов к подстрокам. Таким образом, однажды выполненное вычисление никогда не повторяется --- результат просто берется из таблицы.

Другой проблемой, ассоциируемой с парсер-комбинаторами, является трудность обработки леворекурсивных определений анализаторов. Например, синтаксический анализ наивным рекурсивным спуском в соответствии с грамматикой, представленной в листинге~\ref{grmG2}, никогда не завершится.

 \begin{listing}
\caption{Леворекурсивная грамматика}
\label{grmG2}
\centering
$\begin{array}{rl}
E \rightarrow E \ + \ a \ | \ a
\end{array}$
 \end{listing}

Данная проблема имеет несколько решений, одно из которых основано на ограничении числа вызовов нетерминала некоторой константой, связанной с длиной входной последовательности ~\cite{ParserComb}. Количество применений каждого распознавателя к каждой позиции в строке ограничивается длиной неразобранного суффикса строки. Данный подход не применим, если длина последовательности неизвестна, например, при считывании символов с сетевого сокета. Во-вторых, такой подход обладает сложностью $O(n^4)$, вместо ожидаемой $O(n^3)$, где n --- длина последовательности. Подобные проблемы решаются техникой Continuation Parsing Style (CPS)~\cite{MemoizationInTopDown}. В отличие от первого метода, цепочка леворекурсивных вызовов завершается, когда происходит второй вызов синтаксического анализатора в данной позиции входа. Затем результаты для леворекурсивных синтаксических анализаторов эффективно вычисляются в цикле: пока создается новый результат, завершенные пути синтаксического анализа, записанные как продолжения, перезапускаются в новой входной позиции. Как результат, обработка рекурсивных правил более эффективна и не требует знания длины последовательности. Более подробно об этом речь пойдет в следующей главе.



\section{Реализация}
Предложенный алгоритм был реализован в рамках исследовательского проекта YaccConstructor. В данной главе описывается архитектура предложенного решения: основные модули и их взаимодействие. Кроме того, рассматриваются особенности практической реализации.

\subsection{Архитектура предложенного решения}
На основе предложенного алгоритма разработан новый модуль инструмента YaccConstructor, который является генератором в терминах, принятых в этом проекте. Это показано на рис.~\ref{Arch}, где изображена архитектура инструмента YaccConstructor и цветом выделен реализованный модуль.

\begin{figure}[h]
 \centering
 \includegraphics[width=\textwidth]{Ragozina/pics/Arch.pdf}
 \caption{Архитектура инструмента YaccConstructor (рисунок взят из работы~\cite{GrigorievPhd})}
 \label{Arch}
\end{figure}

Внутреннее устройство этого модуля показано на рис.~\ref{Arch2}. Основными компонентами являются генератор, который по грамматике строит управляющие таблицы и дополнительные структуры данных, компонента с описанием SPPF и функциями работы с ним, два интерпретатора управляющих таблиц, различающиеся тем, что один из них строит лес разбора, а другой нет. Интерпретаторы разделены в силу того, что структуры для хранения элементов дерева тесно связаны с другими структурами, используемыми при анализе, например, стеком, а отказ от построения леса был вызван необходимостью получить алгоритм, расходующий меньше памяти. По этой причине реализовано два набора структур данных, каждая из которых оптимальна при решении соответствующей задачи.

На вход генератор принимает внутреннее представление в формате IL, которое строится по грамматике и может быть получено с помощью соответствующего фронтенда. Так как в язык описания грамматики позволяет использовать конструкции, которые не обрабатываются генератором (например, метаправила), то необходимо применить соответствующие преобразования, что достигается заданием специальных параметров при запуске инструмента. Результатом работы генератора является файл с исходным кодом, в котором описаны управляющие таблицы и вспомогательная информация, которая в дальнейшем используется интерпретатором. 

Интерпретатор написан вручную и содержит в себе основную логику алгоритма. Он подключается в виде отдельной сборки к целевому приложению и позволяет на основе сгенерированнх данных выполнять анализ входа.

Пользователь при создании приложения, использующего модуль, добавляет в свой проект сгенерированный файл, ссылку на интерпретатор и файл, содержащий лексический анализатор (полученный с помощью другого модуля YC, который не описывается в данной работе) и вызывает соответствующую функцию для синтаксического анализа. Результатом работы такой функции является либо SPPF, либо набор координат во входном графе, позволяющих определить положение в нём участка, порождающего строку, принимаемую соответствующей грамматикой.

\begin{figure}
 \centering
 \includegraphics[width=\textwidth]{Ragozina/pics/GLL_Proc.pdf}
 \caption{Принцип работы реализованного модуля (рисунок взят и модифицирован из работы~\cite{GrigorievPhd})}
 \label{Arch2}
\end{figure}

\subsection{Особенности используемых структур данных}
Алгоритм реализован в рамках проекта YaccConstructor на языке программирования F\#. Исходный код свободно доступен в репозитории \url{https://github.com/YaccConstructor/YaccConstructor}, автор вёл разработку под учётной записью {\it AnastasiyaRagozina}.

Заявленная производительность алгоритма~--- в худшем случае куб по памяти и времени~--- обоснована теоретически~\cite{Johnstone201564}. На практике же, для достижения высокой производительности алгоритма, написанного с использованием языков высокого уровня, необходимо приложить некоторые усилия. Рассуждениям на данную тему и описанию эффективных структур данных посвящена работа~\cite{Johnstone2011}. При реализации описанного алгоритма подобные проблемы так же возникли: высокий расход памяти и медленные структуры данных. Основной проблемой было хранение леса разбора и поиск уже существующих узлов. Хранение узлов в многомерных массивах, как было предложено в~\cite{Johnstone2011}, накладывало значительные ограничения на длину входа. Кроме того, хранение в каждом узле дерева нескольких чисел (имени нетерминала и координаты начала и конца подцепочки, соответствующей данному поддереву) делало его громоздким. В результате для того, чтобы уменьшить расход памяти при хранении SPPF, было использовано сжатие хранимых в узлах координат в одно число. Это позволило вместо хранения двух чисел хранить одно, которое можно было использовать в качестве ключа при поиске уже созданных поддеревьев. Аналогичное сжатие использовалось для хранения слотов. Для хранения терминальных узлов в алгоритме было предложено использовать динамически изменяемый массив,  размер которого сравним с размером входных данных, что приводит к выделению большого количества лишней памяти при использовании стандартного типа \verb|ResizeArray<_>| при больших размерах входа. Для решения этой проблемы использовалась модификация динамически изменяемого массива, в которой память выделяется блоками константного размера. Данная структура данных была реализована в рамках работы над RNGLR-алгоритмом. В рамках данной работы она была выделена в библиотеку структур данных \texttt{FSharpx.Collections}, поддерживаемую FSharp-сообществом~\cite{FsharpX}. Подобные задачи являются интересными с инженерной точки зрения и часто возникают на практике.

Важной задачей также является представление метагеномной сборки и её обработка, так как, в отличие от графа, являющегося аппроксимацией встроенных языков, граф, представляющий метагеномную сборку, как правило, существенно большего размера. Для того, чтобы уменьшить размер самого графа, на рёбрах хранится не по одному токену, а цепочки токенов. Это приводит к тому, что теперь в качестве координаты начала и конца подстроки используется не два числа, а четыре~--- номера рёбер и позиция на них. По аналогии, эти числа сжимались. Для того, чтобы эффективно использовать такие индексы, была создана структура данных, доступ к элементам которой как у массива, но по сжатому числу. 

При обработке графа метагеномной сборки были использованы следующие знания о его структуре и особенностях решаемой задачи. В графе есть рёбра, на которых лежат подстроки длины большей, чем длина искомой подстроки. Это означает, что такие рёбра можно удалить из графа и обработать отдельно, как линейные данные. При этом граф распадается на набор связанных компонент, что позволяет обрабатывать части входного графа полностью независимо. Это, в свою очередь, существенно упрощает параллельную обработку данных: возникает классическая параллельность по данным, когда к большому количеству независимых данных нужно применить одну и ту же функцию обработки. 

Однако, несмотря на то, что параллельность по данным может быть реализована очевидным образом, использование нескольких потоков в рамках одного многоядерного процессора не даёт ожидаемого прироста производительности, что наглядно продемонстрировано на рис.~\ref{StackExp}.

\begin{figure}
 \centering
 \includegraphics[width=\textwidth]{Ragozina/pics/Stack.pdf}
 \caption{Сравнение производительности предложенного решения при запуске на нескольких потоках}
 \label{StackExp}
\end{figure} 

Замеры, результаты которых представлены на рис.~\ref{StackExp} и рис.~\ref{StackExp2} проводились на машине со следующей конфигурацией:

\begin{itemize}
\item OS Name Microsoft Windows 10 Pro;
\item System Type x64-based PC;
\item Processor	Intel(R) Core(TM) i7-4790 CPU 3.60GHz, 3601 Mhz, 4 Core(s), 4 Logical Processor(s);
\item Installed Physical Memory (RAM) 32.0 GB.
\end{itemize}

Из рис.~\ref{StackExp} видно, что максимальная производительность наблюдается при использовании двух потоков. Однако прирост производительности по сравнению с использованием одного потока составляет всего 6.4\%, что значительно меньше теоретически возможного.

Такое поведение системы связано с тем, что при обработке одного графа происходит активное обращение к вспомогательным структурам данных большого объёма. При этом обращения, в силу особенностей алгоритма, плохо локализованы. В связи с чем, при попытке обработать несколько графов одновременно на одном процессоре, учащаются промахи кэшей. Частично решить эту проблему удалось заменив очередь дескрипторов на стек, это сделало обращения к данным более локализованными и позволило улучшить производительность решения. 

Результаты измерений после замены очереди на стек представлены на рис.~\ref{StackExp2}. Максимальная производительность достигается при использовании трёх потоков и прирост производительность составляет 14.2\%. На рис.~\ref{StackExp2} представлено сравнение производительности до и после модификации.



\begin{figure}
 \centering
 \includegraphics[width=\textwidth]{Ragozina/pics/StackVSQueue.pdf}
 \caption{Сравнение производительности предложенного решения замене стека на очередь для хранения дескрипторов }
 \label{StackExp2}
\end{figure}

Для того, чтобы избавиться от проблем с кэшами при многопоточной обработке, можно использовать многопроцессорные системы, такие как вычислительные кластеры. При этом, как показали проведённые ранее эксперименты, имеет смысл запускать не более двух потоков на одном процессоре. Так как время обработки одного подграфа занимает время порядка нескольких секунд, то затраты на передачу по сети не должны заметно уменьшать выигрыш, получаемый за счёт параллельной обработки при достаточном количестве графов для обработки на одном узле.  

Для реализации вычислений в кластере была выбрана технология MBrace, которое позволяет, с одной стороны, управлять кластером в облаке Microsoft.Azure с помощью скриптов на F\#. Предоставляется полный набор функций, позволяющий сконфигурировать кластер ``с нуля'', а затем управлять им (например, изменять количество машин). С другой стороны, MBrace позволяет прозрачно использовать кластер в коде на F\#. Это достигается благодаря предоставлению набора высокоуровневых функций и окружения \texttt{cloud}, благодаря которому код, предназначенный для выполнения в кластере можно задать следующим образом.

\begin{listing}
    \begin{pyglist}[language=ocaml,numbers=left,numbersep=5pt]
    
let parallelTask = 
    [ for i in 1 .. 10 -> 
          cloud { return sprintf "i'm work item %d" i } ]
    |> Cloud.Parallel
    |> cluster.CreateProcess

\end{pyglist}
\caption{Код для запуска предложенного решения в кластере}
\label{lst:mbraceExample}
\end{listing}

В скобках \verb|cloud { }| может находиться произвольный код на F\#. Все необходимые для выполнения этого кода в кластере дополнительные действия и коммуникации (передача данных, подготовка и передача бинарных файлов) осуществляется автоматически и не требует участия разработчика. Таким образом, в предположении, что функция обработки графа \texttt{processGraph}  реализована, всё, что необходимо для реализации обработки массива графов о кластере, это ``завернуть'' её вызов в окружение \texttt{cloud} следующим образом.

\begin{listing}
    \begin{pyglist}[language=ocaml,numbers=left,numbersep=5pt]
    
let parallelGraphProcessing graphs = 
    [ for g in graphs -> cloud { return processGraph g } ]
    |> Cloud.Parallel
    |> cluster.CreateProcess

\end{pyglist}
\caption{Код для запуска предложенного решения в кластере с параметризацией входных данных}
\label{lst:mbraceExample}
\end{listing}

\section*{Заключение}
В ходе работы получены следующие результаты:
\begin{itemize}
    \item реализована поддержка конъюнктивных грамматик в языке спецификации грамматик YARD;
    \item реализована поддержка конъюнктивных грамматик в генераторе GLL-анализаторов;
    \item результаты экспериментально проверены на небольших метагеномных сборках.
\end{itemize}

\subsubsection*{Дальнейшее направление работы}

В первую очередь, необходимо снизить время работы алгоритма. Полученная реализация предполагает полный обход дерева разбора, что, безусловно, влияет на производительность алгоритма. Кроме того, можно исследовать возможность расширения класса распознаваемых языков до булевых, на основе полученных результатов.



\begin{thebibliography}{99}

\bibitem{string_embedded}
  Semen Grigorev, Ekaterina Verbitskaia, Andrei Ivanov et al.
  String-embedded language support in integrated development environment.
  In Proceedings of the 10th Central and Eastern European Software Engineering Conference in Russia (CEE-SECR ’14). –– 2014.

\bibitem{polubelova}
  Полубелова Марина.
  Лексический анализ динамически формируемых строковых выражений.
  Дипломная работа кафедры системного программирования СПбГУ. –– 2015.

\bibitem{handbook_automata}
  Mohri Mehryar. 
  Handbook of Weighted Automata.
  Monographs in Theoretical Computer Science. Springer, 2009. –– P. 213–254. 

\bibitem{yacc_article}
  Я.А. Кириленко, С.В. Григорьев, Д.А. Авдюхин. 
  Разработка синтаксических анализаторов в проектах по автоматизированному реинжинирингу информационных систем. 
  2013. 

\bibitem{yacc_www}
  Сайт проекта YaccConstructor.
  \url{http://yaccconstructor.github.io}

\bibitem{yacc_git}
  Репозиторий проекта YaccConstructor
  \url{https://github.com/YaccConstructor/YaccConstructor}

\bibitem{fslex}
  Сайт проекта FsLex
  \url{http://fsprojects.github.io/FsLexYacc/}

\bibitem{source_text}
  Репозиторий проекта YC.Utils.SourceText
  \url{https://github.com/YaccConstructor/YC.Utils.SourceText}

\bibitem{quick_graph}
  Сайт проекта QuickGraph
  \url{http://yaccconstructor.github.io/QuickGraph/}

\end{thebibliography}

