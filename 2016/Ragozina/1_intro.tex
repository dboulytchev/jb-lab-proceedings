\section*{Введение}
При работе с формальными языками и грамматиками выводимость цепочки в грамматике можно рассматривать как следующее свойство: цепочка $\omega$ обладает свойством $S$, если $\omega$ выводима из $S$: ($S \Rightarrow^* \omega $). При решении практических задач, как правило, выполняют проверку свойства выводимости в грамматике для отдельно взятых цепочек, либо же для конечного множества цепочек, представленных в явном виде (например, в виде множества файлов с исходным текстом программ). На практике такие множества могут оказаться бесконечными, что делает такую проверку невозможной. Подобная ситуация может возникнуть, если цепочки генерируются автоматически. Множество порождаемых генератором цепочек в этом случае будет регулярным. Для описания регулярных множеств часто используются конечные автоматы. Таким образом возникает задача проверки свойства выводимости в КС-грамматике для всех элементов множества, заданного конечным автоматом. Такую задачу будем называть синтаксическим анализом регулярных множеств. 
	
Описанная задача имеет практическое применение и возникает в ряде областей. Например, широкое распространение при разработке информационных систем получил подход, использующий динамически формируемые программы. Код таких программ формируется в процессе выполнения внешней программы из строковых литералов и в дальнейшем выполняется соответствующим окружением. Такой подход может использоваться для генерации SQL-запросов, Web-страниц, запросов к XML-подобным структурам данных. Однако проблема заключается в том, что динамически формируемый код не подвергается статической проверке стандартными инструментами, так как компилятором он воспринимается как обычные строки. Статический анализ встроенного кода позволил бы выявлять ошибки до того, как программа будет запущена. Кроме того, общепринятой практикой при разработке информационных систем является использование интегрированных сред разработки, которые упрощают процесс разработки путём подсветки синтаксиса, автодополнения и других функций. Для встроенных языков подобные возможности также были бы полезны. Для решения таких задач необходимо иметь структурное представление кода (дерево разбора), которое строится в процессе синтаксического анализа. При этом задача анализа динамически формируемого кода осложняется тем, что все возможные значения программы нельзя задать простым перечислением. Это происходит из-за того, что для формирования кода могут использоваться циклы, потенциально бесконечные. Для представления такого кода можно построить регулярную аппроксимацию сверху, представленную в виде конечного автомата над алфавитом встроенного языка. Таким образом мы приходим к задаче синтаксического анализа регулярных множеств при обработке динамически формируемого кода. Важной особенностью задач в данной области является необходимость построения дерева разбора, что выдвигает дополнительные требования к алгоритму анализа.

Другим примером использования синтаксического анализа регулярных множеств является поиск подпоследовательностей генома (таких как РНК, например) в задачах биоинформатики. Для того, чтобы классифицировать образцы, взятые из окружающей среды, для них строится метагеномная сборка, являющаяся комбинацией генов всех организмов, находящихся в образце. Сборка представляется в виде графа с последовательностями символов на рёбрах. В таком графе необходимо найти подстроки, позволяющие провести классификацию. Искомые подстроки могут быть описаны КС-грамматикой~\cite{Anderson}, то есть необходимо искать подстроки, обладающие свойством выводимости. В данном случае не обязательно строить дерево разбора, необходимо лишь ответить на вопрос, порождается ли цепочка данным автоматом. Данная задача может быть решена с помощью синтаксического анализа регулярных множеств.

Анализу динамически формируемых программ посвящён ряд работ~\cite{LrAbstract1, LrAbstract2, LRAbstractParsingSema}.  Изучению данной проблемы посвящена также работа~\cite{RelaxedARNGLR}, в которой описан алгоритм анализа встроенных языков с использованием алгоритма RNGLR, строящий структурное представление динамически формируемого кода. Однако в этой работе указано, что у предложенного решения существуют проблемы с производительностью. Синтаксический анализ также применяется в задачах биоинформатики, однако только для линейных входных данных. При этом отдельное внимание необходимо уделять производительности решения. Это обуславливается тем, что входные данные при анализе метагеномной сборки, как правило, имеют очень большой размер: порядка $10^5$ рёбер, $10^5$ вершин, $10^8$~--- суммарного количества символов в метках рёбер. 
