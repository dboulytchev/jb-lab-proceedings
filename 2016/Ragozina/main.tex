\title{Ослабленный синтаксический анализ динамически формируемых выражений на основе алгоритма GLL}

\titlerunning{Ослабленный синтаксический анализ}

\author{Рагозина Анастасия Константиновна}

\authorrunning{А.К.Рагозина}

\tocauthor{А.К.Рагозина}
\institute{Санкт-Петербургский государственный университет\\
\email{ragozina.anastasiya@gmail.com }}

\maketitle             

\begin{abstract}
Данная работа посвящена описанию подхода к синтаксическому анализу регулярных множеств. 
Подобная задача возникает при анализе встроенных языков или же поиске соответствий в 
метагеномных сборках в задачах биоинформатики. Регулярное множество задаётся с помощью 
конечного автомата, порождающего цепочки, и нужно проверить, принадлежат ли эти цепочки 
языку, задаваемому грамматикой. В качестве основы использовался алгоритм обобщённого 
синтаксического анализа GLL из-за его высокой скорости работы.
\end{abstract}

\section*{Введение}
Графы и графовые базы данных имеют широкое применение в таких областях, как биоинформатика, логистика, социальные сети и многие другие. Запросы к таким базам формулируются как задача поиска путей в графе, удовлетворяющих некоторым ограничениям. Во многих случаях такие ограничения формулируются в виде некоторой грамматики, наиболее выразительным её представлением является контекстно-свободная (КС) грамматика. В таком случае задача сводится к поиску путей в графе, которые бы соответствовали строкам в контекстно-свободном языке. Такую задачу назовем синтаксическим анализом графа. 

Одним из примеров применения синтаксического анализа графа является поиск подпоследовательности геномов в задаче биоинформатики. Из окружающей среды берется образец, по нему необходимо выявить подпоследовательности генов для классификации организмов в данном образце. Для этого по образцу строится метагеномная сборка, являющаяся комбинацией генов. Сборка представляет собой граф с последовательностями символов на ребрах. В таком графе необходимо найти подстроки, позволяющие провести классификацию. Такую задачу можно решить при помощи синтаксического анализа графов.

Существуют различные подходы к синтаксическому анализу графов (например,~\cite{GrigRagCFPQuerying},~\cite{Hellings},~\cite{Sevon}), одно из которых основано на алгоритма Generalised LL (GLL)~\cite{GrigRagCFPQuerying}. Алгоритм GLL позволяет без модификаций грамматики анализировать все КС-языки. Результатом работы данного алгоритма является компактное представление леса разбора Shared Packed Parse Forest (SPPF)~\cite{SPPF}. Данное представление разбора позволяет производить дополнительный анализ и получать таким образом дополнительную информацию, а так же производить семантические действия уже после разбора. Данное решение построено по принципу генераторов синтаксических анализаторов, когда по грамматике генерируется код анализаторов, что не всегда удобно, при написании запросов к графовым базам данных.

Существуют различные решения для поиска путей в графовых базах данных. Это встроенные инструменты и языки для запросов. К примеру, для базы данных Neo4j~\cite{Neo4j} существуют такие языки запросов как Cypher~\cite{Cypher} и openCypher~\cite{openCypher}. Однако они не поддерживают синтаксис запросов в стилеконтекстно-свободной грамматики и не возвращают пути в виде деревьев разбора. При работе с графовыми базами данных было бы удобно строить запрос к ним на языке, на котором написано целевое приложение. Это возможно реализовать техникой парсер-комбинаторов. Парсер-комбинаторы позволяют создавать синтаксические анализаторы динамически непосредственно в коде программы на некотором языке. Все существующие библиотеки парсер-комбинаторов анализируют только линейный вход — строки. Нашей задачей стала разработка библиотеки для синтаксического анализа графов. Существует библиотека Meerkat~\cite{Meerkat} на языке Scala~\cite{Scala}, реализующая синтаксический анализ строк методом парсер-комбинаторов, используя идеи, схожие с используемыми в алгоритме GLL~\cite{GLL}. Она обладает рядом преимуществ:

\begin{itemize} 
\item результатом работы библиотеки является лес разбора SPPF;
\item библиотеку возможно запустить на JVM;
\item разбор происходит в худшем случае за $O(n^3)$, где n – длина входной последовательности.
\end{itemize}

Было принято решение использовать данную библиотеку для решения задачи.

Таким образом в рамках данной работы была поставлена задача модифицировать существующую библиотеку для синтаксического анализа графов с помощью техники парсер-комбинаторов.

\section{Постановка задачи}
Целью данной работы является создание решения для синтаксического анализа регулярных множеств, применимого для работы со входными данными большого размера. Для достижения поставленной цели были поставлены следующие задачи:

\begin{itemize}  
\item Разработать алгоритм синтаксического анализа динамически  формируемого кода на основе алгоритма GLL. 
\item Доказать корректность предложенного алгоритма.
\item Применить к задаче поиска на входных данных большого размера.
\item Реализовать предложенный алгоритм в рамках проекта YaccConstructor.
\item Произвести эксперименты и сравнение.
\end{itemize}

\section{Обзор}
\subsection{Синтаксический анализ графов}
При работе с графами, например в графовых базах данных, возникает необходимость выполнения запросов поиска путей, удовлетворяющих заданным ограничениям. Ограничения задаются, как правило, регулярной грамматикой, однако контекстно-свободные грамматики представляют собой более выразительный язык запросов. Контекстно-свободная грамматика (КС-грамматика) $G$ --- четверка $(T, N, P, S)$, где $N$ --- множество нетерминалов, $T$ --- множество терминалов ($T \cap N = \varnothing$), $P = \{ A \rightarrow \alpha \mid A \in N, \alpha \in (N \cup T)^*\}$ --- множество правил грамматики и $S \in N$ --- стартовый символ. Грамматика $G = (\{+, -, a\}, \{ E, N \}, P, E)$, множество правил которой $P$ представлены на листинге~\ref{grmG1}, задает язык арифметических выражений со сложением и умножением над переменными $a$.

\begin{listing}
\caption{Правила грамматики $G$}
\label{grmG1}
\centering
$\begin{array}{ll}
E \rightarrow & N \ + \ N \mid N \ - \ N
\\
N \rightarrow & a
\end{array}$
 \end{listing}

Итак, задача выполнения запросов является задачей поиска в ориентированном графе всех путей, представляющих собой строки языка, заданные КС-грамматикой. Одно из возможных решений такой задачи --- это модификация алгоритмов классического синтаксического анализа строк. Работа~\cite{Sevon} модифицирует алгоритм Эрли. Данный подход позволяет записывать запросы к графовым структурам данных с указанием направления поиска: прямое направление, когда поиск производится в направлении ребер графа, или обратном --- по обратным ребрам. Данный алгоритм строит только некоторое приближение к результату: обработка циклов входного графа осуществляется только до некоторой глубины, специфицируемой пользователем, в результате чего некоторые пути могут быть утеряны. Результатом работы алгоритма является подграф, содержащий пути разбора. 

Для последующего анализа путей из результата удобно иметь информацию об их синтаксической структуре, например в форме абстрактных синтаксических деревьев. Так как множество путей может быть бесконечным, то и синтаксических деревьев может быть бесконечно много, поэтому возникает вопрос об их представлении. В алгоритмах обобщенного синтаксического анализа, в случае существования нескольких деревьев разбора для одной строки (в виду неоднозначностей грамматики), используется компактное представление леса разбора SPPF (Shared Packed Parse Forest). В структуре SPPF переиспользуются общие фрагменты разных выводов, за счет его размер полиномиален от размера входной строки. Существуют модификации обобщенных алгоритмов, применимые для выполнения запросов в контекстно-свободных ограничениях к графам: алгоритмы на основе RNGLR~\cite{RNGLR} и GLL~\cite{GrigRagCFPQuerying}. Первый позволяет построить лес разбора SPPF, второй --- бинаризованный лес разбора Binarized Shared Packed Parse Forest~\cite{SPPF} с лучшими пространственными характеристиками. 

Binarized Shared Packed Parse Forest имеет размер $O(n^3)$, где n --- длина входной строки. В отличие от обычного дерева разбора, внутренние узлы которого всегда соответствуют нетерминалам грамматик, в BSPPF используются дополнительные типы узлов: упакованный узел (packed node) и промежуточный узел (intermediate node). Упакованный узел создаётся для представления неоднозначностей вывода (его дети соответствуют разным продукциям). Промежуточный узел используется для бинаризации, когда правило длины больше 2 представляется как цепочка применений правил длины 2. Именно за счет бинаризации достигается кубический размер представления леса разбора. В SPPF могут быть циклы. На рис.~\ref{fig:sppfV} представлено, как выглядит SPPF, состоящее из деревьев на рис.~\ref{fig:sppfA} и рис.~\ref{fig:sppfB}. А на рис.~\ref{fig:sppfG} представлена его бинаризованная форма. Пример взят из статьи~\cite{IzmCombinator}.

 \begin{figure}[t]
 \centering
    \subfloat[Дерево разбора 1]{
        \label{fig:sppfA}
        \includegraphics[width=0.35\textwidth]{Smolina/pics/SppfA.png}
    }
    \subfloat[Дерево разбора 2]{
        \label{fig:sppfB}
        \includegraphics[width=0.35\textwidth]{Smolina/pics/SppfB.png}        
    }

    ~\\~
    \subfloat[SPPF]{
        \label{fig:sppfV}
        \includegraphics[width=0.35\textwidth]{Smolina/pics/SppfV.png}        
    }
    \subfloat[Бинаризинное SPPF]{
        \label{fig:sppfG}
        \includegraphics[width=0.45\textwidth]{Smolina/pics/SppfG.png}        
    }
 \caption{Граф SPPF для двух вариантов вывода}
\end{figure}

Решение на RNGLR и GLL построено на основе генерации синтаксических анализаторов. Такое решение не является удобным при работе с графами и графовыми базами данных, так как при добавлении малейших изменений необходимо генерировать руками новый синтаксический анализатор, а также требуется использование дополнительного предметно-ориентированного языка для задания запроса. 

В сфере промышленных графовых баз данных существуют свои языки запросов. К примеру для графовой базы данных Neo4j~\cite{Neo4j} существуют языки Cypher~\cite{Cypher} и openCypher~\cite{openCypher}, а для OrientDB~\cite{OrientDB} используется язык SQL~\cite{Sql}. Ни один из них не поддерживает формат запроса в виде контекстно-свободной грамматики. Более того, результат запроса всегда --- простые строки, что усложняет их дальнейший анализ.

Таким образом, нашей целью стала разработка решения, в котором грамматику можно специфицировать в коде целевого приложения. Один из возможных подходов к решению данной задачи --- использование техники парсер-комбинаторов~\cite{HOFunParsing}.

\subsection{Техника парсер-комбинаторов}
Комбинатор --- это функция высшего порядка, которая из набора функций строит новую функцию. Возможность принимать функции как аргументы, комбинировать их и возвращать как результат является важной особенностью функциональных языков программирования. Парсер-комбинатор --- это функция высшего порядка, которая на вход получает множество синтаксических анализаторов и возвращает новый синтаксический анализатор. 

Для синтаксического анализа необходимо научиться анализировать элементарные сущности (терминалы, нетерминалы), осуществлять последовательное применение анализаторов и поддержать возможность осуществлять выбор анализатора для разбора суффикса сроки. Эти требования задают минимальный набор комбинаторов. Техника парсер-комбинаторов позволяет из элементарных анализаторов конструировать более сложные. Интеграция с языком программирования приложения, в котором применяется синтаксический анализатор, добавляет гибкости и расширяемости в сравнении с генераторами синтаксических анализаторов. Приведём пример реализации простейшего парсер-комбинатора на Scala.

В листинге~\ref{parser1} представлен синтаксический анализатор, который принимает на вход строковую последовательность, затем разбирает строку, начинающуюся с определенного терминала, и возвращает результат разбора и необработанный суффикс строки.

\begin{listing}
\caption{Синтаксический анализатор терминала}
\label{parser1}
\centering
\includegraphics[width=0.7\textwidth]{Smolina/pics/parser1.png}
\end{listing}

Для того чтобы получить синтаксический анализатор подстроки, можно воспользоваться парсер-комбинатором,
который составлял последовательность из анализаторов символов --- парсер-комбинатор последовательности seq. На листинге~\ref{parserSeq} приведен пример реализации.

\begin{listing}
\caption{Парсер-комбинатор последовательности}
\label{parserSeq}
\centering
\includegraphics[width=0.7\textwidth]{Smolina/pics/parserSeq.png}
\end{listing}

Теперь, чтобы получить синтаксический анализатор, начинающийся с подстроки ``ABC'', мы можем воспользоваться элементарными синтаксическими анализаторами символов и парсер-комбинатором последовательности (см. листиниг~\ref{parserABC}).

\begin{listing}
\caption{Парсер-комбинатор строки “ABC”}
\label{parserABC}
\centering
\includegraphics[width=0.9\textwidth]{Smolina/pics/parserABC.png}
\end{listing}

Простые парсер-комбинаторы, основанные на рекурсивном спуске, представляют собой интуитивно ясную модель и поэтому удобны для
отладки. Однако они имеют экспоненциальную сложность относительно размеров грамматики~\cite{Popov}. Это связано с тем, что в наивной реализации рекурсивного спуска при откате не сохраняются результаты и разбор префикса строки одним и тем же синтаксическим анализатором может происходить многократно. Мемоизация~\cite{Memoization} позволяет решить эту проблему за счет переиспользования результатов применения синтаксических анализаторов к подстрокам. Таким образом, однажды выполненное вычисление никогда не повторяется --- результат просто берется из таблицы.

Другой проблемой, ассоциируемой с парсер-комбинаторами, является трудность обработки леворекурсивных определений анализаторов. Например, синтаксический анализ наивным рекурсивным спуском в соответствии с грамматикой, представленной в листинге~\ref{grmG2}, никогда не завершится.

 \begin{listing}
\caption{Леворекурсивная грамматика}
\label{grmG2}
\centering
$\begin{array}{rl}
E \rightarrow E \ + \ a \ | \ a
\end{array}$
 \end{listing}

Данная проблема имеет несколько решений, одно из которых основано на ограничении числа вызовов нетерминала некоторой константой, связанной с длиной входной последовательности ~\cite{ParserComb}. Количество применений каждого распознавателя к каждой позиции в строке ограничивается длиной неразобранного суффикса строки. Данный подход не применим, если длина последовательности неизвестна, например, при считывании символов с сетевого сокета. Во-вторых, такой подход обладает сложностью $O(n^4)$, вместо ожидаемой $O(n^3)$, где n --- длина последовательности. Подобные проблемы решаются техникой Continuation Parsing Style (CPS)~\cite{MemoizationInTopDown}. В отличие от первого метода, цепочка леворекурсивных вызовов завершается, когда происходит второй вызов синтаксического анализатора в данной позиции входа. Затем результаты для леворекурсивных синтаксических анализаторов эффективно вычисляются в цикле: пока создается новый результат, завершенные пути синтаксического анализа, записанные как продолжения, перезапускаются в новой входной позиции. Как результат, обработка рекурсивных правил более эффективна и не требует знания длины последовательности. Более подробно об этом речь пойдет в следующей главе.



\section{Алгоритм анализа регулярных множеств}
Целью данной работы является создание алгоритма синтаксического анализа регулярных множеств, который может быть применим для анализа встроенных языков. Как упоминалось ранее, встроенный код порождается в момент выполнения основной программы. Для генерации могут использоваться строковые операторы, условные операторы и циклы из-за чего такой код не может быть представлен линейно. Результатом работы лексического анализатора на таком коде является не линейный поток токенов, а конечный автомат над алфавитом токенов. В рамках данной работы на такие автоматы накладывается следующее ограничение: они должны быть детерминированными. Если это условие не будет выполняться, то нельзя будет однозначно выбрать, по какому пути производить разбор. Например, для встроенного кода на листинге~\ref{lst:brExpr} будет построен конечный автомат на рис.~\ref{input}.

%\fvset{frame=lines,framesep=5pt}
\begin{listing}
\begin{pyglist}[language=csharp,numbers=left,numbersep=5pt]
 string expr = "" ;
 for(int i = 0; i < len; i++) 
 {
     expr = "()" + expr;
 }
\end{pyglist}
\caption{Код на C\#, динамически формирующий скобочную последовательность}
\label{lst:brExpr}
\end{listing}

\begin{figure}[h]
 \centering
 \includegraphics[width=0.3\textwidth]{Ragozina/pics/input.pdf}
 \caption{Конечный автомат, представляющий аппроксимацию встроенного кода для листинга~\ref{lst:brExpr} }
 \label{input}
\end{figure}

В алгоритме вместо хранения позиции во входном потоке теперь хранится номер вершины во входном графе. Поскольку вход является нелинейным, то вместо того, чтобы просматривать один текущий входной символ на каждом шаге, рассматриваются все исходящие рёбра для текущей вершины и выбирается одно (как упоминалось ранее, автомат детерминирован, поэтому это возможно), соответствующее текущему терминальному символу в грамматике. Если такого ребра нет, то алгоритм просто продолжает свою работу~--- из очереди достаётся новый дескриптор и процесс возобновляется. 

Для автоматического создания синтаксических анализаторов существует несколько подходов. В рамках первого подхода весь код парсера генерируется по грамматике. Чаще всего такой подход используется при генерации анализаторов, построенных методом рекурсивного спуска. При генерации нисходящих анализаторов для каждого нетерминала генерируются функции, которые последовательно вызываются в процессе разбора. Несмотря на то, что нисходящие анализаторы просты для разработки, и поэтому чаще всего создаются вручную, существуют инструменты для автоматической генерации таких анализаторов. Например, инструмент ANTLR~\cite{antlr}~--- генератор парсеров, позволяющий автоматически создавать анализаторы на одном из целевых языков программирования по описанию LL(*)-грамматики на языке, близком к EBNF. Структура генераторов такого типа изображена на рис.~\ref{genTypes}{\it (а)}.

\begin{figure}
 \centering
 \includegraphics[width=\textwidth]{Ragozina/pics/GeneratorTypes.pdf}
 \caption{Подходы к генерации синтаксических анализаторов}
 \label{genTypes}
\end{figure}

Существует ещё один подход для генерации синтаксических анализаторов, который используется для получения табличных анализаторов. Отдельно создаётся интерпретатор, который содержит в себе основную логику алгоритма. Интерпретатор пишется вручную и переиспользуется. По грамматике каждый раз генерируется дополнительная информация, которая необходима интерпретатору в процессе работы. Структура такого генератора представлена на рис.~\ref{genTypes}{\it (б)}. Чаще всего в качестве дополнительной информации генерируются таблицы синтаксического анализа, управляющие процессом разбора.

В оригинальных работах, описывающих GLL-алгоритм, используется первый подход. В рамках данной работы был выбран второй подход из-за его гибкости и универсальности. Вместо генерации функций по слотам грамматики и их последовательного вызова, в главном цикле алгоритма просто рассматриваются все возможные состояния, в которых может находиться парсер. В зависимости от того, какой символ во входном потоке и какая позиция в грамматике, в процессе разбора рассматриваются следующие ситуации:

\begin{itemize}
\item Если текущий символ в грамматике является терминалом $x$ и существует исходящее из текущей вершины ребро, помеченное этим нетерминалом, то указатель в грамматике нужно сдвинуть на одну позицию вправо, $x \rightarrow \alpha X \cdot \beta$, и текущей вершиной назначить конечную вершину ребра. Никаких дополнительных действий со стеком при этом не производится. Иначе, если нет ребра, помеченного терминалом $x$, то текущая ветка разбора считается ошибочной, отбрасывается и  разбор продолжается с использованием следующего дескриптора.
\item Если текущий символ в грамматике является нетерминалом $a$, то необходимо в стек записать слот, по которому продолжить разбор после того, как правило для $a$ будет разобрано. Указатель в грамматике перемещается на $a \rightarrow \cdot \gamma $, а номер вершины во входном потоке остаётся без изменений.
\item Если указатель в грамматике имеет вид $x \rightarrow \alpha\cdot$ и стек не пуст, то слот вида $y \rightarrow \delta x \cdot \mu$, который хранится в этот момент в текущей вершине стека, извлекается и становится текущим.
\item Если текущий слот имеет вид $s \rightarrow \tau\cdot$, и весь входной поток рассмотрен, то разбор завершается успешно, иначе разбор заканчивается ошибкой. В случае успешного завершения разбора возвращается дерево, иначе сообщение об ошибке.
\end{itemize}

Наличие циклов во входном графе никак не влияет на процесс разбора. Дескрипторы позволяют без каких-либо изменений процесса разбора обработать их. Это делает результирующий алгоритм более простым в отличие от алгоритма, основанного на RNGLR, в который потребовалось внести существенные изменения для поддержки циклов~\cite{RelaxedARNGLR}. За счёт того, что каждый раз при добавлении дескриптора выполняется проверка всей четвёрки целиком (позиция во входе, слот, вершина стека и часть леса разбора), то лишние дескрипторы с одинаковыми деревьями не создаются. Переиспользование уже созданных узлов также позволяет избежать создания лишних деревьев: если дерево с определёнными координатами и соответствующим правилом вывода уже было создано, то повторно такое дерево создаваться не будет.

В алгоритме также поддерживается четвёрка: слот (вместо имени функции), номер вершины в графе, вершина стека и узел дерева. Поскольку вызов функций заменён на обработку ситуаций, возникающих в процессе анализа, в теле основной функции, то появилась необходимость определять, какое правило вывода использовать для разбора. Для определения правила используются LL-таблицы, где в каждой ячейке может быть несколько правил для разбора, что соответствует ситуации наличия в грамматике неоднозначностей. Анализатор состоит из функции, содержащей основной цикл алгоритма, функции, управляющей процессом разбора и функций для построения дерева и стека.

\begin{listing}[H]
\hrule
\begin{algorithmic}
\caption{Функция, содержащая в себе основную логику алгоритма}
\label{parsing}
\Function{parsing()}{}
	\State{$condition \gets true$}
	\If{$isEpsilonRule(cL.rule)$} 
		\State{$cR \gets$ $new TerminalNode("Epsilon", packExtension (cI, cI))$}
		\State{$cN \gets$ $getNodeP(cL, cN, cR)$}
		\State{$pop(cU, cI, cN)$}
    \Else
		\If{$isEndOfRule(cL.rule, cL.position)$}
			\State{$curSmb \gets$ $grammarRules[cL.rule][cL.position]$}
			\If{$isTerminal(curSmb)$}
				\State{$curSmb \gets$ $grammarRules[cL.rule][cL.position]$}
				\If{$cI.OutEdges$ contains edge labeled with $curSmb$}
					\State{$curEdge \gets$ edge labeled with $curSmb$}
					\State{$cR \gets$ $getNodeT(curEdge)$}
					\State{$cI \gets$ $curEdge.TargetVertex$}
					\State{$cL \gets$ $label(cL.rule, cL.position + 1)$}
					\State{$cN \gets$ $getNodeP(cL, cN, cR)$}
				    \State{$condition \gets$ $false$}
				\EndIf
			\Else
				\State{$cU \gets$ $create(cI, label(cL.rule, cL.position + 1), cU, cN)$}
				\ForAll{$edge$ in outgoing edges of $cI$}
					\ForAll{$rule in table[curSymbol, edge.Token]$}
						\State{$addContext(cI, packLabel(rule, 0), cU, \$)$}
					\EndFor
				\EndFor	
			\EndIf
		\Else
		    \State{$pop(cU, cI, cN)$}
		\EndIf
    \EndIf
\EndFunction
\end{algorithmic}

\hrule
\end{listing}

\begin{listing}
\hrule
\begin{algorithmic}
\caption{Функции, управляющие процессом разбора }
\label{control}
\Function{control()}{}
	\State{$condition \gets$ $true$}
	\While{not $stopr$}
		\If{$condition$} {$dispatcher()$}
		\Else {$processing()$} 
		\EndIf
	\EndWhile
\EndFunction

\Function{dispatcher()}{}
	\If{$\mathcal{Q}$ is not empty} 
		\State{$currentContext \gets$ $\mathcal{R}.Dequeue()$}
		\State{$cI \gets$ $currentContext.Index$}
		\State{$cU \gets$ $currentContext.GSSNode$}
		\State{$cL \gets$ $currentContext.Label$}
		\State{$cN \gets$ $currentContext.SPPFNode$}
		\State{$cR \gets$ $DummySPPFNode$}
		\State{$condition \gets$ $false$}
    \Else
		\State{$stop \gets$ $true$}
    \EndIf
\EndFunction
\end{algorithmic}
\hrule
\end{listing}

На листинге~\ref{control} приведены две функции управляющие разбором. Функция \texttt{control()} в зависимости от значений булевых переменных \texttt{stop} и \texttt{condition} вызывает функции \texttt{dispatcher()} или \texttt{parsing()}. Функция \texttt{dispatcher()} извлекает из очереди дескриптор, присваивает значения переменным. Функция \texttt{parsing()} на листинге~\ref{parsing} содержит в себе основную логику алгоритма.


\subsection{Пример работы алгоритма}
Рассмотрим следующий пример. В качестве входных данных будем использовать конечный автомат $M$, представленный на рис.~\ref{InputGraph}, который генерирует произвольные скобочные последовательности. Необходимо построить лес разбора для всех цепочек, порождаемых автоматом $M$, выводимых в грамматике $G_4$ (листинг~\ref{grmG4}), описывающей язык правильных скобочных последовательностей.
\begin{listing}
\caption{Грамматика $G_4$}
\label{grmG4}
\centering
$\begin{array}{rl}
s \rightarrow LBR \ s \ RBR \ s \ \varepsilon \  | \ \varepsilon 
\end{array}$
\end{listing}

\begin{figure}
 \centering
 \includegraphics[width=0.3\textwidth]{Ragozina/pics/ExampleInputGraph.pdf}
 \caption{Конечный автомат, подаваемый на вход анализатору }
 \label{InputGraph}
\end{figure}

В результате работы описанного алгоритма построено сжатое представление леса разбора, представленное на рис.~\ref{ExSppf}. Циклы в сжатом представлении леса разбора отображают наличие циклов во входном конечном автомате и позволяют извлекать потенциально бесконечное множество деревьев, каждое из которых соответствует цепочке, порождаемой автоматом.

\begin{figure}
 \centering
 \includegraphics[width=\textwidth]{Ragozina/pics/SppfExample.pdf}
 \caption{Сжатое представление леса разбора для грамматики $G_4$ и конечного автомата на рис.~\ref{InputGraph} }
 \label{ExSppf}
\end{figure}

\subsection{Доказательство корректности}
Для того чтобы показать, что предложенный алгоритм работает корректно, сначала нужно доказать, что процесс останавливается.

\textsc{Теорема 1.} 
\textit{Алгоритм завершает свою работу для произвольного детерминированного конечного автомата и контекстно-свободной грамматики.}

\textsc{Доказательство.}

Алгоритм завершает свою работу как только очередь дескрипторов становится пустой. Дескриптор с определённым набором значений полей в очередь добавляется лишь единожды. Таким образом, чтобы показать завершаемость алгоритма достаточно доказать, что количество дескрипторов конечно. 

Дескриптор состоит из четырёх элементов~--- слот, индекс во входном потоке, вершина стека, дерево. Таким образом, общее количество дескрипторов не превышает прямого произведения возможного количества каждого из этих элементов. Количество индексов не больше количества вершин входного графа. Количество слотов конечно, потому что грамматика конечна. Вершина стека определяется парой~--- слот и индекс, и, значит тоже конечно. Часть леса, хранимая в дескрипторе, определяется однозначно именем нетерминала или слотом и двумя координатами во входном графе. Обе составляющие конечны. $\square$

Таким образом было показано, что количество дескрипторов~--- конечное число.

\textsc{Определение 1.} 
\emph{Корректное дерево}~--- это упорядоченное дерево со следующими свойствами:
\begin{enumerate}
  \item Корень дерева соответствует стартовому нетерминалу грамматики $G$.
  \item Листья соответствуют терминалам грамматики $G$. Упорядоченная последовательность листьев соответствует некоторому пути во входном графе.
  \item Внутренние узлы соответствуют нетерминалам грамматики $G$. Потомки внутреннего узла (для нетерминала $N$) соответствуют символам правой части некоторой продукции для $N$ в грамматике $G$.
\end{enumerate}

Для того, чтобы доказать, что SPPF содержит только корректные деревья, сначала необходимо доказать следующую лемму.

\textsc{Лемма 1.} 
\textit{Для любой части леса $t$, построенного в процессе вывода, существует путь в графе $р$, такой, что крона $t$ покрывает этот путь. }

\textsc{Доказательство.}

Для доказательство используется индукция по построению SPPF. 

\textsc{База.}

Для терминальных узлов утверждение очевидно. Терминальный узел соответствует ровно одному ребру во входном графе и строится только после прохода по этому ребру. Построение эпсилон-узлов никак не зависит от входного графа, а производится только в соответствии с грамматикой. 

\textsc{Переход.}

Достаточно доказать для упакованных ячеек, всё остальное доказывается аналогично. Создание упакованных ячеек происходит в двух случаях~--- при чтении нового терминала из входного потока или изъятии вершины стека, что значит, что текущий нетерминал был разобран и необходимо вернуться к точке, с которой этот разбор начался. 

Рассмотрим первый случай. У нас есть часть леса, которая соответствует какому-то пути $p$ в графе от вершины $v_0$ до $v_1$. Текущая позиция во входном потоке соответствует правой координате для этой части SPPF. При считывании нового терминала создаётся упакованная ячейка, левым сыном которой становится уже построенная часть SPPF, а правым~--- терминал. Получаем новую часть леса, соответствующее пути $P_1 = v_0 \dots v_1 v_{1+1}$.

Рассмотрим второй случай~--- изъятие вершины со стека. Первая часть леса разбора Т1 хранится на ребре стека. Вторая часть Т2 построена по только что разобранному правилу. Каждая из этих частей соответствует какому-то подпути в графе и необходимо показать, что правая координата T1 совпадает с левой координатой T2. Это соответствует тому факту, что объединение этих частей леса даст часть леса, покрывающую путь в графе без ``дыр'', то есть если в графе была цепочка {\it ``abcd''} и T1 соответствует {\it ``ab''}, то T2 будет соответствовать {\it ``bc''}. Для того, чтобы показать, что это условие выполняется, достаточно рассмотреть, как происходит процесс разбора. Как только в процессе обхода грамматики (в слоте) встречается нетерминал, создаётся новая вершина стека, которая хранит в себе слот с позицией за этим нетерминалом, на ребре хранится уже построенная часть леса. Правая координата этой части SPPF является номером вершины, с которой будет происходить дальнейший разбор, это число и записывается в новый дескриптор. Таким образом, после того, как нетерминал будет разобран до конца, будет создан новый упакованный узел, в котором в качестве левого потомка будет часть леса с ребра, а в качестве правого~--- нетерминальный узел, левая координата которого совпадает с правой координатой левой части леса, так как именно с того места и начался разбор этого нетерминала. $\square$

Таким образом для упакованных узлов в дереве доказали необходимое. Доказательство для остальных видов узлов проводится аналогично.

\textsc{Теорема 2.} 
\textit{Любое дерево, извлечённое из SPPF, является корректным.}

\textsc{Доказательство.}

Рассмотрим произвольное извлечённое из SPPF дерево и докажем, что оно удовлетворяет определению. Первый и третий пункт определения корректного дерева следует из определения SPPF. 

Второй пункт следует из Леммы 1. Необходимо только показать, что такой путь начинается в начальной вершине и заканчивается в конечной. Действительно, так как работа алгоритма может быть начата только из начальной вершины, то левой координатой для неё будет стартовая вершина. Результатом работы алгоритма является SPPF. Узел помечается, как результирующий, если он помечен стартовым нетерминалом и его левая координата является стартовой вершиной, а правая~--- финальной. $\square$

\textsc{Теорема 3.} 
\textit{Пусть грамматика Г порождает язык $L$. Тогда для каждого пути в графе $p$, соответствующего строке $s$ из $L$, из SPPF может быть изъято корректное дерево.}

\textsc{Доказательство.}

Необходимо доказать, что SPPF содержит все корректные деревья вывода для всех корректных цепочек из входа. Как только процесс разбора начинается, в очередь дескрипторов добавляются дескрипторы для всех альтернатив стартового правила, соответствующие терминалу во входном потоке. Аналогичная ситуация происходит, как только в грамматике встречается нетерминал. Рассматриваются все альтернативы нетерминала и добавляются те, по которым может быть продолжен синтаксический анализ в соответствии со входным символом. Это гарантирует, что все альтернативы в выводе будут рассмотрены. При этом во входном графе все пути, соответствующие входным цепочкам, тоже рассматриваются, так как переход по ребру осуществляется всегда, если оно продолжает корректный префикс.$\square$

\subsection{Анализ данных большого объёма}
Одной из задач, сформулированных в данной работе, является использование предложенного алгоритма для анализа больших данных. Это востребовано, например, в задачах биоинформатики. Прежде чем формулировать задачу, следует ввести основные определения.

Исследование геномов является одной из распространённых задач биоинформатики. Информацию, содержащуюся в геноме, можно представить в виде последовательностей символов и в дальнейшем эти последовательности анализировать. Геномы извлекаются из ДНК и позволяют характеризовать тот или иной организм. Для этого из генома необходимо выделить определённые участки, позволяющие сделать выводы о его свойствах. Геном (последовательность ДНК)~--- строка в алфавите $\{A, C, G, T\}$, однозначно определяющая организм (или штамм), к которому она относится. Сборка~--- набор подстрок генома, длина которых на порядки меньше длины самого генома. Метагеномная сборка~--- смесь сборок нескольких геномов, то есть набор небольших подстрок нескольких геномов. Поскольку геном состоит из повторяющихся участков, то его можно представить в виде конечного автомата с последовательностями символов на рёбрах, который на практике часто представляется в виде графа  Де Брауна~\cite{Bruijn}.

Как упоминалось ранее, для решения задач, возникающих в биоинформатике, не нужно структурное представление вывода. Это значит, что дерево разбора, которое является результатом работы синтаксического анализатора, не нужно. Нужно лишь ответить на вопрос: порождает ли входной автомат данную подстроку или нет и вернуть координаты участка, на котором это происходит. При этом геном можно описать с помощью грамматики, т.е. про подцепочки, порождаемые входным конечным автоматом, известно, что они описываются некоторой грамматикой. Необходимо найти подавтоматы, принимающие цепочки, задаваемые некоторой грамматикой. Таким образом, предложенный алгоритм необходимо модифицировать таким образом, чтобы он решал данную задачу. 

Для решения поставленной задачи не нужно строить лес разбора, поэтому от функций для его построения можно просто отказаться. Самое простое представление результата~--- набор путей. Однако для больших графов это может потребовать больших дополнительных расходов памяти. Чтобы этого избежать, можно предложить следующий подход: строить множество начальных и конечных вершин и контролировать длину путей. Это можно делать в процессе анализа, не накапливая дополнительной информации. Тогда после завершения работы можно будет выделить подграф, который, возможно, будет содержать лишние пути и потому потребуется его последующая обработка с накоплением путей. При этом извлечённые подграфы будут существенно меньше исходного графа и их повторная обработка не сильно скажется на производительности.

Таким образом, в местах, где раньше в алгоритме строились узлы дерева, теперь просто запоминаются координаты. Вместо хранения поддерева на рёбрах стека теперь хранится просто число~--- начало и конец подцепочки, созданной на момент создания вершины стека. Кроме координат начала и конца и длины можно ещё сохранять путь целиком. Для этого на рёбрах нужно просто сохранять цепочки, а не одно число. 

\section{Реализация}
Предложенный алгоритм был реализован в рамках исследовательского проекта YaccConstructor. В данной главе описывается архитектура предложенного решения: основные модули и их взаимодействие. Кроме того, рассматриваются особенности практической реализации.

\subsection{Архитектура предложенного решения}
На основе предложенного алгоритма разработан новый модуль инструмента YaccConstructor, который является генератором в терминах, принятых в этом проекте. Это показано на рис.~\ref{Arch}, где изображена архитектура инструмента YaccConstructor и цветом выделен реализованный модуль.

\begin{figure}[h]
 \centering
 \includegraphics[width=\textwidth]{Ragozina/pics/Arch.pdf}
 \caption{Архитектура инструмента YaccConstructor (рисунок взят из работы~\cite{GrigorievPhd})}
 \label{Arch}
\end{figure}

Внутреннее устройство этого модуля показано на рис.~\ref{Arch2}. Основными компонентами являются генератор, который по грамматике строит управляющие таблицы и дополнительные структуры данных, компонента с описанием SPPF и функциями работы с ним, два интерпретатора управляющих таблиц, различающиеся тем, что один из них строит лес разбора, а другой нет. Интерпретаторы разделены в силу того, что структуры для хранения элементов дерева тесно связаны с другими структурами, используемыми при анализе, например, стеком, а отказ от построения леса был вызван необходимостью получить алгоритм, расходующий меньше памяти. По этой причине реализовано два набора структур данных, каждая из которых оптимальна при решении соответствующей задачи.

На вход генератор принимает внутреннее представление в формате IL, которое строится по грамматике и может быть получено с помощью соответствующего фронтенда. Так как в язык описания грамматики позволяет использовать конструкции, которые не обрабатываются генератором (например, метаправила), то необходимо применить соответствующие преобразования, что достигается заданием специальных параметров при запуске инструмента. Результатом работы генератора является файл с исходным кодом, в котором описаны управляющие таблицы и вспомогательная информация, которая в дальнейшем используется интерпретатором. 

Интерпретатор написан вручную и содержит в себе основную логику алгоритма. Он подключается в виде отдельной сборки к целевому приложению и позволяет на основе сгенерированнх данных выполнять анализ входа.

Пользователь при создании приложения, использующего модуль, добавляет в свой проект сгенерированный файл, ссылку на интерпретатор и файл, содержащий лексический анализатор (полученный с помощью другого модуля YC, который не описывается в данной работе) и вызывает соответствующую функцию для синтаксического анализа. Результатом работы такой функции является либо SPPF, либо набор координат во входном графе, позволяющих определить положение в нём участка, порождающего строку, принимаемую соответствующей грамматикой.

\begin{figure}
 \centering
 \includegraphics[width=\textwidth]{Ragozina/pics/GLL_Proc.pdf}
 \caption{Принцип работы реализованного модуля (рисунок взят и модифицирован из работы~\cite{GrigorievPhd})}
 \label{Arch2}
\end{figure}

\subsection{Особенности используемых структур данных}
Алгоритм реализован в рамках проекта YaccConstructor на языке программирования F\#. Исходный код свободно доступен в репозитории \url{https://github.com/YaccConstructor/YaccConstructor}, автор вёл разработку под учётной записью {\it AnastasiyaRagozina}.

Заявленная производительность алгоритма~--- в худшем случае куб по памяти и времени~--- обоснована теоретически~\cite{Johnstone201564}. На практике же, для достижения высокой производительности алгоритма, написанного с использованием языков высокого уровня, необходимо приложить некоторые усилия. Рассуждениям на данную тему и описанию эффективных структур данных посвящена работа~\cite{Johnstone2011}. При реализации описанного алгоритма подобные проблемы так же возникли: высокий расход памяти и медленные структуры данных. Основной проблемой было хранение леса разбора и поиск уже существующих узлов. Хранение узлов в многомерных массивах, как было предложено в~\cite{Johnstone2011}, накладывало значительные ограничения на длину входа. Кроме того, хранение в каждом узле дерева нескольких чисел (имени нетерминала и координаты начала и конца подцепочки, соответствующей данному поддереву) делало его громоздким. В результате для того, чтобы уменьшить расход памяти при хранении SPPF, было использовано сжатие хранимых в узлах координат в одно число. Это позволило вместо хранения двух чисел хранить одно, которое можно было использовать в качестве ключа при поиске уже созданных поддеревьев. Аналогичное сжатие использовалось для хранения слотов. Для хранения терминальных узлов в алгоритме было предложено использовать динамически изменяемый массив,  размер которого сравним с размером входных данных, что приводит к выделению большого количества лишней памяти при использовании стандартного типа \verb|ResizeArray<_>| при больших размерах входа. Для решения этой проблемы использовалась модификация динамически изменяемого массива, в которой память выделяется блоками константного размера. Данная структура данных была реализована в рамках работы над RNGLR-алгоритмом. В рамках данной работы она была выделена в библиотеку структур данных \texttt{FSharpx.Collections}, поддерживаемую FSharp-сообществом~\cite{FsharpX}. Подобные задачи являются интересными с инженерной точки зрения и часто возникают на практике.

Важной задачей также является представление метагеномной сборки и её обработка, так как, в отличие от графа, являющегося аппроксимацией встроенных языков, граф, представляющий метагеномную сборку, как правило, существенно большего размера. Для того, чтобы уменьшить размер самого графа, на рёбрах хранится не по одному токену, а цепочки токенов. Это приводит к тому, что теперь в качестве координаты начала и конца подстроки используется не два числа, а четыре~--- номера рёбер и позиция на них. По аналогии, эти числа сжимались. Для того, чтобы эффективно использовать такие индексы, была создана структура данных, доступ к элементам которой как у массива, но по сжатому числу. 

При обработке графа метагеномной сборки были использованы следующие знания о его структуре и особенностях решаемой задачи. В графе есть рёбра, на которых лежат подстроки длины большей, чем длина искомой подстроки. Это означает, что такие рёбра можно удалить из графа и обработать отдельно, как линейные данные. При этом граф распадается на набор связанных компонент, что позволяет обрабатывать части входного графа полностью независимо. Это, в свою очередь, существенно упрощает параллельную обработку данных: возникает классическая параллельность по данным, когда к большому количеству независимых данных нужно применить одну и ту же функцию обработки. 

Однако, несмотря на то, что параллельность по данным может быть реализована очевидным образом, использование нескольких потоков в рамках одного многоядерного процессора не даёт ожидаемого прироста производительности, что наглядно продемонстрировано на рис.~\ref{StackExp}.

\begin{figure}
 \centering
 \includegraphics[width=\textwidth]{Ragozina/pics/Stack.pdf}
 \caption{Сравнение производительности предложенного решения при запуске на нескольких потоках}
 \label{StackExp}
\end{figure} 

Замеры, результаты которых представлены на рис.~\ref{StackExp} и рис.~\ref{StackExp2} проводились на машине со следующей конфигурацией:

\begin{itemize}
\item OS Name Microsoft Windows 10 Pro;
\item System Type x64-based PC;
\item Processor	Intel(R) Core(TM) i7-4790 CPU 3.60GHz, 3601 Mhz, 4 Core(s), 4 Logical Processor(s);
\item Installed Physical Memory (RAM) 32.0 GB.
\end{itemize}

Из рис.~\ref{StackExp} видно, что максимальная производительность наблюдается при использовании двух потоков. Однако прирост производительности по сравнению с использованием одного потока составляет всего 6.4\%, что значительно меньше теоретически возможного.

Такое поведение системы связано с тем, что при обработке одного графа происходит активное обращение к вспомогательным структурам данных большого объёма. При этом обращения, в силу особенностей алгоритма, плохо локализованы. В связи с чем, при попытке обработать несколько графов одновременно на одном процессоре, учащаются промахи кэшей. Частично решить эту проблему удалось заменив очередь дескрипторов на стек, это сделало обращения к данным более локализованными и позволило улучшить производительность решения. 

Результаты измерений после замены очереди на стек представлены на рис.~\ref{StackExp2}. Максимальная производительность достигается при использовании трёх потоков и прирост производительность составляет 14.2\%. На рис.~\ref{StackExp2} представлено сравнение производительности до и после модификации.



\begin{figure}
 \centering
 \includegraphics[width=\textwidth]{Ragozina/pics/StackVSQueue.pdf}
 \caption{Сравнение производительности предложенного решения замене стека на очередь для хранения дескрипторов }
 \label{StackExp2}
\end{figure}

Для того, чтобы избавиться от проблем с кэшами при многопоточной обработке, можно использовать многопроцессорные системы, такие как вычислительные кластеры. При этом, как показали проведённые ранее эксперименты, имеет смысл запускать не более двух потоков на одном процессоре. Так как время обработки одного подграфа занимает время порядка нескольких секунд, то затраты на передачу по сети не должны заметно уменьшать выигрыш, получаемый за счёт параллельной обработки при достаточном количестве графов для обработки на одном узле.  

Для реализации вычислений в кластере была выбрана технология MBrace, которое позволяет, с одной стороны, управлять кластером в облаке Microsoft.Azure с помощью скриптов на F\#. Предоставляется полный набор функций, позволяющий сконфигурировать кластер ``с нуля'', а затем управлять им (например, изменять количество машин). С другой стороны, MBrace позволяет прозрачно использовать кластер в коде на F\#. Это достигается благодаря предоставлению набора высокоуровневых функций и окружения \texttt{cloud}, благодаря которому код, предназначенный для выполнения в кластере можно задать следующим образом.

\begin{listing}
    \begin{pyglist}[language=ocaml,numbers=left,numbersep=5pt]
    
let parallelTask = 
    [ for i in 1 .. 10 -> 
          cloud { return sprintf "i'm work item %d" i } ]
    |> Cloud.Parallel
    |> cluster.CreateProcess

\end{pyglist}
\caption{Код для запуска предложенного решения в кластере}
\label{lst:mbraceExample}
\end{listing}

В скобках \verb|cloud { }| может находиться произвольный код на F\#. Все необходимые для выполнения этого кода в кластере дополнительные действия и коммуникации (передача данных, подготовка и передача бинарных файлов) осуществляется автоматически и не требует участия разработчика. Таким образом, в предположении, что функция обработки графа \texttt{processGraph}  реализована, всё, что необходимо для реализации обработки массива графов о кластере, это ``завернуть'' её вызов в окружение \texttt{cloud} следующим образом.

\begin{listing}
    \begin{pyglist}[language=ocaml,numbers=left,numbersep=5pt]
    
let parallelGraphProcessing graphs = 
    [ for g in graphs -> cloud { return processGraph g } ]
    |> Cloud.Parallel
    |> cluster.CreateProcess

\end{pyglist}
\caption{Код для запуска предложенного решения в кластере с параметризацией входных данных}
\label{lst:mbraceExample}
\end{listing}

\section{Эксперименты}

Были проведены экспериментальные исследования, целью которых являлась проверка того, что конъюнктивные грамматики позволяют задавать структуру тРНК так, что синтаксический анализатор находит меньше некорректных цепочек.

\begin{figure}[h]
\begin{center}
\begin{verbatim}

[<Start>]
folded: stem<(any*[1..3] 
              stem<any*[7..10]> 
              any*[1..3] 
              stem<any*[5..8]> 
              any*[3..5] 
              stem<any*[5..8]>
              )>

stem<s>: 
      A stem<s> U
    | U stem<s> A
    | C stem<s> G
    | G stem<s> C
    | G stem<s> U
    | U stem<s> G
    | s

any: A | U | G | C

\end{verbatim}
\caption{КС-грамматика вторичной структуры тРНК}
\label{TRNAgrammar}
\end{center}
\end{figure}


\begin{figure}
\begin{center}
ACACCCCCCCUCACCCCCUCCCACCCCCUU
\end{center}
\caption{Пример цепочки нуклеотидов, сгенеированной для экспериментов}
\label{rnachain}
\end{figure}


\begin{figure}
\begin{center}
\begin{verbatim}
[<Start>]
folded: stem<subseq> & (any*[7..9] subseq any*[7..9])

subseq: any*[1..3] 
        stem<any*[7..10]> & (any*[4..6] any*[7..10] any*[4..6])
        any*[1..3] 
        stem<any*[5..8]> & (any*[6] any*[5..8] any*[6])
        any*[3..5] 
        stem<any*[5..8]> & (any*[4..5] any*[5..8] any*[4..5])
        
stem<s>:
      A stem<s> U
    | U stem<s> A
    | C stem<s> G
    | G stem<s> C
    | G stem<s> U
    | U stem<s> G
    | s

any: A | U | G | C

\end{verbatim}
\caption{Конъюнктивная грамматика структуры тРНК}
\label{TRNAgrammarConj}
\end{center}
\end{figure}


\begin{figure}
\begin{center}
\begin{tikzpicture}
\begin{axis}[
    legend pos = north west,
  xlabel = {Количество лексем},
  ylabel = {Время, с}
]
\addplot coordinates {
  (100,2) (200,17) (300,42) (400,81) (500,128) (600,190) (700,264) (800,345) (900,446) (1000,562)
};
\addplot coordinates {
  (100,1) (200,2) (300,4) (400,8) (500,12) (600,18) (700,22) (800,28) (900,36) (1000,43)
};
\legend{ 
  грамматика $G_{2}$, 
  грамматика $G_{3}$
};
\end{axis}
\end{tikzpicture}
\end{center}
\caption{Среднее время работы алгоритма на конъюнктивной и контекстно-свободной грамматиках тРНК}
\label{time}
\end{figure}

\begin{table}[h]
\begin{center}
  \begin{tabular}{ c | c | c }
    \hline
     & КС-грамматика & Конъюнктивная грамматика \\ \hline
    Тест 1. & 15 & 0 \\\hline
    Тест 2. & 5 & 0 \\\hline
    Тест 3. & 11 & 0 \\
    \hline
  \end{tabular}
\end{center}
\caption{Количество некорректных цепочек, распознанных синтаксическим анализатором}
\label{mistakes}
\end{table}

На рис.~\ref{TRNAgrammar} и~\ref{TRNAgrammarConj} представлены грамматки, описывающие структуру тРНК. Грамматика $G_2$ является контекстно-свободной, а грамматика $G_3$~--- конъюнктивной. По данным грамматикам были сгенерированы соответствующие синтаксические анализаторы.

На вход построенным синтаксическим анализаторам подавались сгенерированные цепочки ДНК длиной от 100 до 1000 симовлов. Эти цепочки содержали в себе последовательности тРНК, а также другие последовательности, которые можно ложно признать за тРНК. Напимер, цепочка на рис.~\ref{rnachain}, хоть и не является тРНК, распознаётся грамматикой $G_2$, но не распознаётся грамматикой $G_3$.

Результаты экспериментов приведены в таблице~\ref{mistakes}. Из них ясно, что грамматика $G_2$ не распознаёт ложные цепочки, распознаваемые грамматикой $G_3$. Время работы синтаксических анализаторов показано на графике, изображённом на рис.~\ref{time}. По графику видно, что время работы синтаксического анализатора, построенного по грамматике $G_3$, значительно превышает время работы другого.

Таким образом, конъюнктивная грамматика позволяет отсеивать цепочки, ложно распозначаемые КС-грамматикой, но за время, значительно большее, чем время работы анализатора по КС-грамматике.

\section*{Заключение}
В ходе работы получены следующие результаты:
\begin{itemize}
    \item реализована поддержка конъюнктивных грамматик в языке спецификации грамматик YARD;
    \item реализована поддержка конъюнктивных грамматик в генераторе GLL-анализаторов;
    \item результаты экспериментально проверены на небольших метагеномных сборках.
\end{itemize}

\subsubsection*{Дальнейшее направление работы}

В первую очередь, необходимо снизить время работы алгоритма. Полученная реализация предполагает полный обход дерева разбора, что, безусловно, влияет на производительность алгоритма. Кроме того, можно исследовать возможность расширения класса распознаваемых языков до булевых, на основе полученных результатов.



\begin{thebibliography}{99}

\bibitem{LrAbstract1}
  Doh K. G., Kim H., Schmidt D. A. 
  Abstract Parsing: Static Analysis of Dynamically Generated String Output Using LR-Parsing Technology. 
  Proceedings of the 16th International Symposium on Static Analysis. –– SAS ’09. –– Berlin, Heidelberg : Springer-Verlag, 2009. –– P. 256–272.

\bibitem{LrAbstract2}
  Doh K. G., Kim H., Schmidt D. A. 
  Abstract LR-parsing. 
  Formal Modeling. Ed. by Gul Agha, José Meseguer, Olivier Danvy. –– Berlin, Heidelberg : Springer-Verlag, 2011. –– P. 90–109.

\bibitem{LRAbstractParsingSema}
  Doh K. G., Kim H., Schmidt D. A. 
  Static Validation of Dynamically Generated HTML Documents Based on Abstract Parsing and Semantic Processing.
  Static Analysis. –– Springer Berlin Heidelberg, 2013. –– Vol. 7935 of Lecture Notes in Computer Science. –– P. 194–214.
                                                                                                                         
\bibitem{Anderson}
  Anderson J. Nov{\'a}k {\'A}. Sükösd Z. 
  Quantifying variances in comparative RNA secondary structure prediction.
  BMC Bioinformatics. –– 2013. –– P. 14–149. 

\bibitem{SELforIDE}
  S. Grigorev, E. Verbitskaia, A. Ivanov et al.
  String-embedded Language Support in Integrated Development Environment.
  Proceedings of the 10th Central and Eastern European Software Engineering Conference in Russia. –– CEE-SECR ’14. –– New York, NY, USA : ACM, 2014. –– P. 21:1–21:11.

\bibitem{RNGLR}
  E. Scott, A. Johnstone.
  Right Nulled GLR Parsers. 
  ACM Trans. Program. Lang. Syst. — 2006. — Vol. 28, no. 4. — P. 577–618.
                                                                                                                                                          
\bibitem{FSharp}
  Syme D., Granicz A., Cisternino A. 
  Expert F\#.
  (Expert’s Voice in .Net). –– ISBN: 1590598504, 9781590598504.

\bibitem{Afroozeh2015}
  A. Afroozeh, A. Izmaylova.
  Faster, Practical GLL Parsing.
  Compiler Construction. Springer. — 2015. — P. 89–108.                                                                                                   

\bibitem{Alvor1} 
  A. Annamaa, A. Breslav, J. Kabanov, V. Vene.
  An Interactive Tool for Analyzing Embedded SQL Queries.
  Proceedings of the 8th Asian Conference on Programming Languages and Systems. –– APLAS’10. –– Berlin, Heidelberg : Springer-Verlag, 2010. –– P. 131–138.

\bibitem{AbstractInterpretation}
  Cousot P., Cousot R. 
  Abstract Interpretation: A Unified Lattice Model for Static Analysis of Programs by Construction or Approximation of Fixpoints.
  Proceedings of the 4th ACM SIGACT-SIGPLAN Symposium on Principles of Programming Languages. –– POPL ’77. –– New York, NY, USA : ACM, 1977. –– P. 238–252. 

\bibitem{Tomita}
  Tomita M.
  An Efficient Context-free Parsing Algorithm for Natural Languages.
  Proceedings of the 9th International Joint Conference on Artificial Intelligence - Volume 2. –– IJCAI’85. –– San Francisco, CA, USA : Morgan Kaufmann Publishers Inc., 1985. –– P. 756–764. 

\bibitem{SPPF}
  Rekers J. G. 
  Parser Generation for Interactive Environments: Ph. D. thesis. 
  J. G. Rekers ; Universiteit van Amsterdam. –– 1992.

\bibitem{GrigorievPhd}
  С. Григорьев. 
  Синтаксический анализ динамически формируемых программ: Ph. D. thesis.
  Санкт-Петербургский государственный университет. –– 2016.

\bibitem{GLL} 
  Scott E., Johnstone A. 
  GLL Parsing. 
  Electron. Notes Theor. Comput. Sci. –– 2010. –– Vol. 253, no. 7. –– P. 177–189.

\bibitem{BRNGLR} 
  Scott E., Johnstone A., Economopoulos R. 
  BRNGLR: A Cubic Tomitastyle GLR Parsing Algorithm. 
  Acta Inf. –– 2007. –– Vol. 44, no. 6. –– P. 427–461.

\bibitem{RIGLR} 
  Scott E., Johnstone A. 
  Generalized Bottom Up Parsers With Reduced Stack Activity.
  Comput. J. –– 2005. –– Vol. 48, no. 5. –– P. 565–587.

\bibitem{Alvor2} 
  Annamaa A., Breslav A., Vene V. 
  Using Abstract Lexical Analysis and Parsing to Detect Errors in String-Embedded DSL Statements. 
  Proceedings of the 22nd Nordic Workshop on Programming Theory. –– 2010. –– P. 20–22.
               
\bibitem{Varis}
  Nguyen H., K{\"a}stner C., Nguyen T. Varis.
  IDE Support for Embedded Client Code in PHP Web Applications.
  Proceedings of the 37th International Conference on Software Engineering (ICSE). –– New York, NY : ACM Press, 2015. –– Formal Demonstration paper.

\bibitem{Johnstone201564}
  Johnstone A., Scott E.
  Principled software microengineering.
  Science of Computer Programming. –– 2015. –– Vol. 97, Part 1. –– P. 64 – 68. –– Special Issue on New Ideas and Emerging Results in Understanding Software. 

\bibitem{Johnstone2011}
  Johnstone A., Scott E.
  Modelling GLL Parser Implementations.
  Software Language Engineering: Third International Conference, SLE 2010, Eindhoven, The Netherlands, October 12-13, 2010, Revised Selected Papers / Ed. by Brian Malloy, Steffen Staab, Mark van den Brand. –– Berlin, Heidelberg : Springer Berlin Heidelberg, 2011. –– P. 42–61. –– ISBN: 978-3-642-19440-5.

\bibitem{reago}
  C. Yuan, J. Lei, J. Cole, Y. Sun.
  Reconstructing 16S rRNA genes in metagenomic data.
  Bioinformatics. –– 2015. –– Vol. 31, no. 12. –– P. i35–i43.
    
\bibitem{Infernal}
  Nawrocki E., Eddy S. 
  Infernal 1.1: 100-fold faster RNA homology searches.
  Bioinformatics. –– 2013. –– Vol. 29, no. 22. –– P. 2933–2935.

\bibitem{Emirge}
  Miller C. Baker B. Thomas B. Singer S. Banfield J.
  EMIRGE: reconstruction of full-length ribosomal genes from microbial community short read sequencing data.
  Genome Biology. –– 2011. –– Vol. 12, no. 5. –– P. 1–14.

\bibitem{Wang2015}
  Qiong Wang, Jordan A. Fish, Mariah Gilman et al.
  Xander: employing a novel method for efficient gene-targeted metagenomic assembly.
  Microbiome. –– 2015. –– Vol. 3, no. 1. –– P. 1–13.
                                                     
\bibitem{antlr}
  ANTLR [Электронный ресурс].
  \url{http://www.antlr.org/ }
  urldate = {11.05.2016}

\bibitem{AlvorUrl}
  Alvor [Электронный ресурс].
  \url{ http://code.google.com/p/alvor/ }
  urldate = {11.05.2016}


\bibitem{FsharpX}
  Репозиторий FSharpx.Collections [Электронный ресурс].
  \url{ https://github.com/fsprojects/FSharpx.Collections/ }
  urldate = {13.05.2016}

\bibitem{PHPSAUrl}
  PHP String Analyzer [Электронный ресурс].
  \url{ http://www.score.cs.tsukuba.ac.jp/~minamide/phpsa/ }
  urldate = {11.05.2016}

\bibitem{JSAUrl}
  Java String Analyzer [Электронный ресурс].
  \url{ http://www.brics.dk/JSA/ }
  urldate = {11.05.2016}

\bibitem{MONAUrl}
  "MONA [Электронный ресурс]"
  \url{ http://www.brics.dk/mona/ }
  urldate = {11.05.2016}

\bibitem{YCUrl}
  "YaccConstructor [Электронный ресурс]"
  \url{https://github.com/YaccConstructor/ }
  urldate = {11.05.2015}

\bibitem{hmmer}
  HMMER [Электронный ресурс]
  \url{http://hmmer.org/ }
  urldate = {14.05.2016}

\bibitem{ReSharper}
  ReSharper [Электронный ресурс].
  \url{https://www.jetbrains.com/resharper/ }
  urldate = {11.05.2016}

\bibitem{ReSharperSDK}
  Документация проекта ReSharper SDK [Электронный ресурс]
  \url{ https://www.jetbrains.com/resharper/devguide/README.html }
  urldate = {11.05.2016}

\bibitem{QuickGraph}
  QuickGraph [Электронный ресурс].
  \url{ https://quickgraph.codeplex.com/ }
  urldate = {11.05.2016},

\bibitem{Graphviz}
  Сайт проекта Graphviz [Электронный ресурс]
  \url{ http://www.graphviz.org/ }
  urldate = {11.05.2016}
    
\bibitem{IntelliLang}
  IntelliLang [Электронный ресурс]
  \url{ https://www.jetbrains.com/idea/help/intellilang.html }
  urldate = {11.05.2016}
                        
\bibitem{PHPStorm}
  PHPStorm IDE [Электронный ресурс]
  \url{ https://www.jetbrains.com/phpstorm/ }
  urldate = {11.05.2016}
                                             
\bibitem{JSA}
  Christensen A. S., M{\o}ller A., Schwartzbach M. I.
  Precise Analysis of String Expressions.
   Proc. 10th International Static Analysis Symposium (SAS). –– Vol. 2694 of LNCS. –– Springer-Verlag, 2003. –– P. 1–18.

\bibitem{PHPSA} 
  Minamide Y. 
  Static Approximation of Dynamically Generated Web Pages.
  Proceedings of the 14th International Conference on World Wide Web. –– WWW ’05. –– New York, NY, USA : ACM, 2005. –– P. 432–441.
                                         
\bibitem{polubelova}
  Полубелова М. 
  Лексический анализ динамически формируемых строковых выражений.
  Санкт-Петербургский государственный университет. –– 2015.
  
\bibitem{IDEA}
  IntelliJ IDEA IDE [Электронный ресурс].
  \url{https://www.jetbrains.com/idea/}
  urldate = {29.04.2016}

\bibitem{YCZOO}
  Репозиторий проекта YaccConstructor, содержащий набор различных грамматик [Электронный ресурс].
  \url{https://github.com/YaccConstructor/YC.GrammarZOO}
  urldate = {29.04.2016}

\bibitem{RelaxedARNGLR} 
  Verbitskaia E., Grigorev S., Avdyukhin D. 
  Relaxed Parsing of Regular Approximations of String-Embedded Languages.
  Preliminary Proceedings of the PSI 2015: 10th International Andrei Ershov Memorial Conference. –– PSI’15. –– 2015. –– P. 1–12.

\bibitem{durbin}
  R. Durbin, S. Eddy, A. Krogh, G. Mitchison.
  Biological sequence analysis: probabilistic models of proteins and nucleic acids.
  Cambridge university press, 1998.

\bibitem{EddyComputationalRNA}
  Nawrocki E., Eddy S.
  Computational identification of functional RNA homologs in metagenomic data.
  RNA biology. –– 2013. –– Vol. 10, no. 7. –– P. 1170–1179.

\bibitem{Bruijn}
  Compeau P., Pevzner P., Tesler G. 
  How to apply de Bruijn graphs to genome assembly.
   Nature biotechnology. –– 2011. –– Vol. 29, no. 11. –– P. 987–991.

\end{thebibliography}
