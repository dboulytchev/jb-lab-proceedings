\documentclass[runningheads]{jblab}

\usepackage{etex}
\usepackage[utf8]{inputenc}
\usepackage[english,russian]{babel}
\usepackage[T2A]{fontenc}
\usepackage{amsmath}

\usepackage{tikz}
\usetikzlibrary{arrows, matrix}

\usepackage{indentfirst}
\usepackage{amsfonts}
\usepackage{amssymb}
\usepackage{graphicx}
\usepackage{listings}
\usepackage{algpseudocode}
\usepackage{algorithm}

\usepackage[compatibility=false]{caption}
\captionsetup{labelfont=bf}

\DeclareCaptionType{Listing}[Листинг]
\captionsetup[Listing]{within=none,labelsep=period}

\DeclareCaptionType{Figure}[Рисунок]
\captionsetup[Figure]{within=none,labelsep=period}

\usepackage{algorithmicx}
\usepackage{wrapfig}
\usepackage{supertabular}
\usepackage{epigraph}
\usepackage{euscript}
\usepackage{textcomp}
\usepackage{cite}
\usepackage{ucs}
\usepackage{verbments}
\usepackage{xcolor}
\usepackage{subcaption}
\usepackage[inference]{semantic}
\usepackage[pdfencoding=auto]{hyperref}
\usepackage{hyperref}
\usepackage[inline]{enumitem}
\usepackage{cmap}
\usepackage{bm}
\usepackage{cancel}
\usepackage{booktabs}
\usepackage{empheq}
\usepackage{enumitem}
\usepackage{thm-restate}
\usepackage[capitalise, english, russian]{cleveref}

\DeclareMathAlphabet{\mathcal}{OT1}{pzc}{m}{it}

\usepackage{bussproofs}
\usepackage{mathtools}
\usepackage{multirow}

\usepackage{url}
\let\proof\relax
\let\endproof\relax

\usepackage{amsthm}
\usepackage{stmaryrd}
\usepackage{placeins}
\usepackage{changepage}

\definecolor{dkgreen}{rgb}{0,0.6,0}
\definecolor{dred}{rgb}{0.545,0,0}
\definecolor{dblue}{rgb}{0,0,0.545}
\definecolor{lgrey}{rgb}{0.9,0.9,0.9}
\definecolor{gray}{rgb}{0.4,0.4,0.4}
\definecolor{darkblue}{rgb}{0.0,0.0,0.6}

\usepackage{verbatim}
\usepackage{float}
\usepackage{minted}
\usepackage{array}
\usepackage{thmtools}

%Terekhov
\usepackage{adjustbox}

%Artemeva
\usepackage{listings}
\usepackage{amssymb}
\usepackage{amsthm}
\usepackage{boxedminipage}

% Kuklina
\usepackage{longtable}
\usetikzlibrary{positioning}

\newcommand{\cd}[1]{\texttt{#1}}
\let\emptyset\varnothing

\textwidth=10cm
\textheight=15cm

\oddsidemargin=0pt
\evensidemargin=0pt

\topmargin=0pt

\begin{document}
\newcommand{\Issue}[0]{8}
\newcommand{\Year}[0]{2020}

\sloppy

\begin{titlepage}

\centering

\includegraphics[width=4cm]{logo_JetBrains_1.png}
\vskip 1mm
\mbox{\Large{\textsc{Труды лаборатории}}}
\vskip 0.5cm
\mbox{\Large{\textsc{языковых инструментов}}}
\vskip 2.5cm
\large{Выпуск \Issue}
\vskip 6cm
\large{Санкт-Петербург, \Year}
\end{titlepage}

\thispagestyle{empty}
\phantom{xx}
\pagebreak

\chapter*{Предисловие}

Предисловие

\vskip 2cm
\begin{flushright}
\textit{подпись}
\end{flushright}

{
\tableofcontents
\thispagestyle{plain}
}

% Кастомные леммы и теоремы для удобства перевода на русский
% сделано по аналогии с вариантом 2019 года, только тут это в
% главном файле, а там было у Ю.Сусаниной в её тексте
\newtheorem{defn}{Определение}%[section]
\newtheorem{lm}{Лемма}
\newtheorem{thrm}{Теорема}
\setcounter{lm}{0}
\setcounter{thrm}{0}


%\usepackage{listings}  
%\usepackage{hyperref}
%\usepackage{verbments}
%\usepackage{minted}
%\usepackage{caption}
%\usepackage{fancybox}
%\newfontfamily{\cyrillicfonttt}{CMU Serif}

\title{Декларативное форматирование в режиме онлайн}

\titlerunning{Декларативное форматирование в режиме онлайн}

\author{Озерных Игорь Станиславович}

\authorrunning{И.С.Озерных}

\tocauthor{И.С.Озерных}
\institute{Санкт-Петербургский государственный университет\\
\email{igor.ozernykh@gmail.com}}

\maketitle             

\begin{abstract}
Аккуратное, соответствующие выбранному стилю оформление (форматирование)
программных текстов критично для их восприятия.
Существуют средства автоматического форматирования кода. Одним
из них является метод синтаксических шаблонов. Его преимуществом является декларативность, которая
достигается за счет задания стиля форматирования с помощью образца кода. Так, естественным ограничением
описанного метода является необходимость наличия достаточно полного образца на целевом языке.
Данная работа посвящена расширению границ применимости метода синтаксических шаблонов
за счет возможности задания шаблонов в режиме онлайн.
\end{abstract}

\section*{Введение}

Этап поддержки и сопровождения в жизненном цикле программного обеспечения может занимать до 90\% времени существования программного продукта. На этом этапе особенно важно понимание программного текста поддерживаемой системы, что зачастую бывает сложной задачей. Необходимым условием для лучшего понимания программного текста является его аккуратное оформление, отражающее структуру 
программы. Существуют различные стандарты оформления кода (стили кодирования)~--- наборы правил и соглашений, используемые при написании исходного кода на некотором языке программирования.
Рассмотрим их на примере стандарта кодирования, принятого в компании Google\footnote{\texttt{https://google-styleguide.googlecode.com/svn/trunk/cppguide.html}}, и стандарта, используемого в проекте GNU\footnote{\texttt{http://www.gnu.org/prep/standards/standards.html}} (см. рис.~\ref{codingstandards}) для языка C++.


\fvset{frame=lines,framesep=7pt}
\begin{figure}[ht]
\noindent\begin{minipage}{.5\textwidth}
    \lstinputlisting[language=Java]{Ozernykh/codes/exGoogleCC.txt}
\caption*{а) Стиль кодирования Google}    
\end{minipage}\hfill
\begin{minipage}{.5\textwidth}
    \lstinputlisting[language=Java]{Ozernykh/codes/exGNUCC.txt}
\caption*{б) Стиль кодирования GNU}    
\end{minipage}
\caption{Различные стили кодирования для языка С++}    
\label{codingstandards}
\end{figure}

Когда программист работает над проектом, он придерживается некоторого стиля кодирования. 
Если файлы проекта с одним стилем кодирования присоединяется к уже существующему проекту с другим стилем кодирования, то их форматирование нужно привести к тому же виду. 
Для решения этой задачи можно воспользоваться уже существующими средствами, например, форматтерами (программами, которые форматируют исходный текст) в IDE таких, как Eclipse, IntelliJ IDEA, Visual Studio и др. 
Однако в этом случае необходимо вручную задавать настройки форматирования.
Кроме того, количество принципиально разных стилей форматирования, которые можно задать с помощью данных настроек, невелико.

%Однако в этом случае необходимо иметь настройки форматирования, заданные программистом, но в целом, количество принципиально разных стилей форматировани, которые можно задать с помощью данных настроек, невелико. %причем эти настройки могут быть различными в разных IDE.
% * <Anton Podkopaev> 12:15:51 22 May 2015 UTC+0300:
% Нужно переработать это предложение.

Еще одним примером может послужить принтер-плагин для IntelliJ IDEA~\cite{podkopaev:diploma1}, который позволяет форматировать исходный код проекта по образцу. 
В качестве образца используется код из некоторого репозитория. 
Из него выделяются шаблоны форматирования для структур языка, которые применяются к структурам целевого кода. 
Под \textit{шаблоном} понимаются данные, сопоставление которых с элементом синтаксического дерева дает текстовое представление этого элемента (и его потомков). 
Например, на рис.~\ref{fig:ifTree} представлено дерево разбора для оператора ветвления. 


\begin{figure}[h]
	\centering
	\includegraphics[width=\textwidth]{Ozernykh/images/ifTree.jpg}
	\caption{Представление оператора ветвления в виде дерева разбора}
	\label{fig:ifTree}
\end{figure}

На рис.~\ref{fig:tmpltcodeintro}а изображен шаблон форматирования, который может быть применен к нему (несколько подвыражений, соответствующих выполнению условия, и одно подвыражение, соответствующее невыполненению условия).
На рис.~\ref{fig:tmpltcodeintro}б представлен результат применения этого шаблона к дереву разбора для оператора ветвления.

% \begin{figure}[h]
% 	\centering
% 	\lstinputlisting[language=c++]{codes/ifTmpltIntro.txt}
% 	\caption{Шаблон для оператора ветвления}
% 	\label{fig:ifTmpltIntro}
% \end{figure}

% На рис.~\ref{fig:ifCodeIntro} представлен результат применения этого шаблона к дереву разбора для оператора ветвления.

% \begin{figure}[h]
% 	\centering
% 	\lstinputlisting[language=c++]{codes/ifCodeIntro.txt}
% 	\caption{Представление дерева разбора при применении шаблона}
% 	\label{fig:ifCodeIntro}
% \end{figure}


\fvset{frame=lines,framesep=7pt}
\begin{figure}[ht]
\noindent\begin{minipage}{.5\textwidth}
    \lstinputlisting[language=Java]{Ozernykh/codes/ifTmpltIntro.txt}
\caption*{а) Шаблон для оператора\\
ветвления}    
\end{minipage}\hfill
\begin{minipage}{.5\textwidth}
    \lstinputlisting[language=Java]{Ozernykh/codes/ifCodeIntro.txt}
\caption*{б) Текст, полученный при применении шаблона к дереву разбора}    
\end{minipage}
\caption{Оператор ветвления и шаблон для него}    
\label{fig:tmpltcodeintro}
\end{figure}

Однако не всегда этот репозиторий существует. Кроме того, необходимо наличие заданного форматирования для всех структур языка.
% * <Anton Podkopaev> 12:19:02 22 May 2015 UTC+0300:
% 'Однако не всегда удобно иметь этот репозиторий под рукой.' --- поменять. Не про удобство надо говорить.

Шаблоны можно извлекать из уже существующего кода и применять их к другим элементам того же типа, но отформатированных иным способом. 
Например, пользователь меняет форматирование некоторой структуры языка (в частности, оператор ветвления), программа извлекает из полученного участка кода новый шаблон и применяет его к структурам того же типа, содержащим старое форматирование. 
Назовем такой способ задания шаблонов~--- форматирование в режиме
онлайн.
%картинки

Целью данной работы является расширение функциональности описанного 
принтер-плагина путем добавления возможности задания шаблонов в режиме онлайн. 

\section{Обзор предметной области}

Одним из подходов к заданию принтеров являются принтер-комбинаторы.
В этой главе приведен обзор классических принтер-комбинаторов, а также
принтер-комбинаторов с дополнительным оператором \emph{выбора}, которые
в рамках данной работы использованы для реализации принтеров, задаваемых
с помощью шаблонов. Для решения проблем эффективности
комбинаторов с выбором в данной работе был использован подход BURS.
Апробация подхода задания принтеров с помощью шаблонов
была произведена для языка Java, для которого есть средства форматирования,
встроенные в IDE.

\subsection{Принтер-комбинаторы}

Базовыми принтер-комбинаторными библиотеками являются библиотеки
Джона Хьюза\cite{hughes} и Филиппа Вадлера\cite{wadler}, которые
представляют собой функциональную переработку алгоритма
Оппена\cite{oppen}.
Ограничимся рассмотрением библиотеки Вадлера, так как
библиотека Хьюза обладает теми же свойствами, которые принципиальны в контексте
данной работы.

В библиотеке ключевым типом является документ.
Он представляет сущность, которая потом может
быть переведена в строковое представление алгоритмом принтера.
Основные конструкторы для составления документа таковы:
\begin{itemize}
  \item атомарная строка, которая печатается как есть;
  \item \emph{разделитель};
  \item последовательная композиция двух документов;
  \item набор связанных документов.
\end{itemize}

Определяющей особенностью данного подхода является то, что все разделители
в рамках одного набора могут быть совместно заменены алгоритмом принтера
на пробельный символ или на перевод строки. Выбор для каждого набора разделителей
основывается на том, что вывод должен поместиться в заданную ширину,
используя минимальное число строк.

Основная проблема такого подхода заключается в его слабой выразительной силе.
Документы, построенные по синтаксическому дереву печатаемой программы,
обрабатываются слишком единообразно, что иногда приводит к нежелательному результату.
Пусть, к примеру, нужно напечатать программу на языке
Python\footnote{\cd{http://python.org}}.
Между последовательными операторами в случае их печати на одной строчке необходимо
добавить дополнительный разделитель (``;''), иначе программа станет некорректной
(см. рис.~\ref{fig:seqEx}). Однако, описанные принтер-комбинаторы не предоставляют
возможности задать такое поведение.
Кроме того, с помощью таких принтер-комбинаторов невозможно выразить разные проектные СК,
так как они всегда печатают текст в одном стиле, который жестко ``зашит'' в их код.

\begin{figure}[h!]
	\centering
	\null\hfill
	\subfloat[Корректный код]{
		\centering
    \makebox[.4\textwidth] {
		  \lstinputlisting[language=Python]{codes/pythonCode.py}
    }
	}
	\null\hfill
	\subfloat[Некорректный код]{
		\centering
    \makebox[.4\textwidth] {
	  	\lstinputlisting[language=Python]{codes/pythonCodeBad.py}
    }
	}
	\hfill\null
	\caption{Пример работы принтер-комбинаторов для языка Python}
  \label{fig:seqEx}	
\end{figure}

Более подробный обзор библиотек Хьюза и Вадлера представлен в \cite{myCoursePaper}.

Большинство принтер-комбинаторных библиотек\cite{
swierstraChitil, swierstra04, peytonJones, kiselyov, chitil}
являются развитиями работ Хьюза и Вадлера. Так, в отличие от базовых, среди
них есть реализации с линейной сложностью обработки документа от его размера и
\emph{online} алгоритмы, которые не требуют просмотра всего документа
для начала печати его текстового представления. Но все они обладают тем же
интерфейсом, а значит в них также неразрешимы задачи, в которых требуется
задавать варианты текстовых представлений, которые отличаются не только
пробелами и переводами строк.

\subsection{Принтер-комбинаторы с выбором}

Существенно от описанных выше отличаются библиотеки, предоставляющие в своем
интерфейсе комбинатор \emph{выбора}, который позволяет задавать для одного поддерева
принципиально разные варианты раскладок.
Так, в работах~\cite{jongeEveryOccasion, jongeReengine} используется
оператор \lstinline[language = Haskell]{ALT},
но алгоритм принтера устроен так, что среди двух альтернатив выбирается первая, если она
помещается в заданную ширину, и вторая --- иначе,
что не дает оптимальный результат на выходе.

Оптимальные принтер-комбинаторы с выбором были впервые представлены в работе~\cite{swierstra}.
Их реализация является частью Utrecht Tools
Library\footnote{\cd{http://www.cs.uu.nl/wiki/HUT/WebHome}}
(практическая реализация несколько изменена по
отношению к той, что описана в статье, но отличие несущественно).
В данном подходе текст строится из блоков прямоугольной формы с возможно неполной последней
строчкой (см. рис. \ref{fig:basicFormat}). В реализации на Haskell блоки представляются
структурой \lstinline[language = Haskell]{Format}:
\begin{lstlisting}
    data Format = Elem { height        :: Int
                       , lastLineWidth :: Int
                       , width         :: Int
                       , txtstr        :: Int -> String -> String
                       }
\end{lstlisting}

Первые три поля структуры определяют геометрические размеры блока, а последнее --- функция,
которая используется для преобразования блока в текст. Здесь используется функция, а не
просто строчка, чтобы можно было преобразовывать вложенные блоки за линейное время. Первый
аргумент \lstinline[language = Haskell]{txtstr} задает сдвиг блока.

\begin{figure}
  \centering
  \subfloat[Блок текста]{
    \raisebox{5mm}{
      \centering
      \vspace{0pt}
      \tikz[scale = 2.0]{
        \draw (0,0) -- (1,0) -- (1,0.2) -- (2,0.2) -- (2,1) -- (0, 1) -- cycle;
      }
    }
    \label{fig:basicFormat}
  }
  ~
  \subfloat[Горизонтальная композиция]{
    \centering
    \tikz[scale = 2.0]{
       \draw (0,0) -- (1,0) -- (1,0.2) -- (2,0.2) -- (2,1) -- (0, 1) -- cycle;
       \draw (1.1,-0.9) -- (2.1,-0.9) -- (2.1,-0.7) -- (3.1,-0.7) -- (3.1,0.1) -- (1.1,0.1) -- cycle;
     }
     \label{fig:beside}
  }
  %\hfill
  ~
  \subfloat[Вертикальная композиция]{
    \makebox[.28\textwidth] {
    \centering
    \tikz[scale = 2.0]{
       \draw (0,0) -- (1,0) -- (1,0.2) -- (2,0.2) -- (2,1) -- (0, 1) -- cycle;
       \draw (0,-1.1) -- (1,-1.1) -- (1,-0.9) -- (2,-0.9) -- (2,-0.1) -- (0,-0.1) -- cycle;
    }
    \label{fig:above}
    }
  }
  \caption{Примитивы блока текста}
  \label{fig:basicConcat}
\end{figure}

Для работы с блоками текста используется следующие четыре примитива:
\begin{lstlisting}
    s2fmt     :: String -> Format
    indentFmt :: Int -> Format -> Format
    aboveFmt  :: Format -> Format -> Format
    besideFmt :: Format -> Format -> Format
\end{lstlisting}

Функция \lstinline[language = Haskell]{s2fmt} создает блок текста,
состоящий из одной строчки; \lstinline[language = Haskell]{indentFmt} по блоку создает новый,
сдвинутый на заданное число позиций. Действие примитивов композиции
\lstinline[language = Haskell]{besideFmt} и \lstinline[language = Haskell]{aboveFmt}
показано на рис.~\ref{fig:beside} и \ref{fig:above} соответственно.

Также, как и в библиотеках без комбинатора выбора, в работе~\cite{swierstra} используется понятие
\emph{документа}. Здесь документ можно рассматривать как множество возможных раскладок
(набор \lstinline[language = Haskell]{Format}-элементов). Документы описываются типом
\lstinline[language = Haskell]{Doc}, экземпляры которого на этапе построения представляют собой
деревья применений приведенных ниже комбинаторов, а на этапе обработки алгоритмом
принтера им в соответствие ставится итоговое множество вариантов текстовых представлений.

Документ конструируется с помощью следующих комбинаторов, которые симметричны
примитивам построения блоков текста:
\begin{lstlisting}
    text   :: String -> Doc
    indent :: Int -> Doc -> Doc
    beside :: Doc -> Doc -> Doc
    above  :: Doc -> Doc -> Doc
\end{lstlisting}

В дополнение появляется пятый комбинатор для документов:
\begin{lstlisting}
    choice :: Doc -> Doc -> Doc
\end{lstlisting}

Этот комбинатор и является тем самым комбинатором выбора. Он представляет
\emph{объединение} множеств раскладок документов, которые были переданы как
аргументы комбинатора. Заметим, что только этот комбинатор может произвести
документ с несколькими раскладками из одновариантных аргументов.

Оригинальная реализация существенно опирается на ленивые вычисления. В \cite{swierstra}
множество вариантов, соответствующее экземпляру \lstinline[language = Haskell]{Doc},
представляется ленивым списком всех возможных раскладок, удовлетворяющих ограничению
на максимальную ширину. Этот список отсортирован в порядке ``ухудшения''
раскладок, то есть в голове списка лежит ``лучшая'', в терминах оптимальности,
раскладка из возможных при заданной ширине. В случае
\lstinline{beside}- и
\lstinline{above}-композиций
документов полная (без учета ленивости) сложность вычисления списка нового документа
составляет $O(n \times m)$, где $n$ и $m$ --- размеры списков, соответствующих 
соединяемым документам. Размер нового списка также порядка $n \times m$.
Он не обязательно точно равен $n \times m$, так как некоторые из полученных
представлений могут не подпадать под ограничение на ширину, соответственно их можно
отбросить на этапе построения нового списка.

Выбор лучшего представления для самого верхнеуровнего документа происходит
просто --- нужно из соответствующего списка взять первый элемент.
Это создает впечатление, что общее число операций,
благодаря ленивым вычислениям, существенно уменьшается.
Но это не так из-за реализации обработки документа,
построенного с помощью комбинатора \lstinline[language = Haskell]{beside}, которая
вынуждает полное вычисление дочерних списков. Более того, из-за свойств отношения
порядка, построенного по критерию оптимальности, в рамках данной модели списков
принципиально нельзя построить ленивую обработку комбинатора
\lstinline[language = Haskell]{beside}.
Так, выбор оптимальной раскладки документа в
\cite{swierstra} имеет в худшем случае экспоненциальную сложность от числа
комбинаторов, использованных при его построении.
В \cite{swiComb} приведены оптимизации для данной модели, но они
не улучшают асимптотику решения для деревьев в общем случае.

\subsection{BURS}

Bottom-Up Rewrite System (BURS)\cite{burs} --- это метод динамического
программирования на деревьях, изначально появившийся в контексте задачи выбора
инструкций для генерации машинного кода. Основой BURS является
регулярная грамматика с древовидными правилами\cite{tata}, для которых задана
стоимость применения подстановки,
то есть грамматика со следующим набором правил: 

$$
\begin{array}{rcll}
  N &:& \alpha& [c]\\
  N &:& \alpha\; (K_1,\dots,K_n)& [c]
\end{array}
$$

Здесь $N, K_i$ это нетерминалы, $\alpha$ --- терминал,
$c$ --- функция стоимости, заданная для каждого правила.
Как и для обычной линейной грамматики, вводится стартовый
нетерминал S. Считается, что терминальное дерево выводится в
данной грамматике, если его можно получить с помощью правил
подстановки из одноузлового дерева $S$.
Каждая подстановка заменяет нетерминал $N$, находящийся в листе дерева, на дерево 
$\alpha\;(K_1,\dots,K_n)$, если в грамматике есть правило
$N:\alpha\;(K_1,\dots,K_n)$. 
Для каждой подстановки вычисляется стоимость ее применения с помощью
функции стоимости $c$.
Аргументами функции могут служить терминальная метка $\alpha$ и стоимости
вывода поддеревьев.
Задачей, которую решает BURS, является поиск вывода наименьшей стоимости
для заданного дерева по заданной грамматике.
Такой вывод может быть найден двухпроходным алгоритмом.

Первый проход (\emph{пометка}) обрабатывает дерево снизу вверх и вычисляет
для каждого узла набор троек $(K,\;R,\;c)$, где $K$ --- нетерминал, из которого может быть
выведено поддерево с корнем в обрабатываемом узле,
$R$ --- первое правило, которое используется для вывода минимальной стоимости из $K$,
$c$ --- стоимость такого вывода.
Процесс пометки происходит следующим образом:

\begin{itemize}
\item для листовой вершины, помеченной терминалом $\alpha$, в множество троек
этого вершины добавляется $(K,\;R,\;c\:(\alpha))$ для каждого правила $R=K:\;\alpha\;[c]$;
\item для промежуточной вершины, помеченной терминалом $\alpha$,
с непосредственными поддеревьями $v_1,\dots,v_n$
в множество добавляется тройка $(K,\;R,\;c\:(\alpha,c_1,\dots,c_n))$ для каждого правила
$R=K:\;\alpha\;(K_1,\dots,K_n)\;[c]$, где $(K_i,\;R_i,\;c_i)$ входит в множество троек для
$v_i$; если есть несколько правил вывода из нетерминала $K$, то выбирается правило,
минимизирующее стоимость вывода.
\end{itemize}

Второй проход (\emph{свертка}) просматривает дерево сверху вниз, используя сделанные пометки.
Первое правило из минимального вывода определяется тройкой $(S,\;R,\;c)$ для корневого узла
(если такой тройки нет, то вывод из $S$ невозможен).
Это правило однозначно определяет нетерминалы $K_i$ для каждого непосредственного поддерева
и процесс повторяется.

Для пометки дерева потенциально необходимо каждое правило грамматики применить к каждому узлу.
При фиксированной грамматике алгоритмическая сложность первого прохода --- $O\:(|R|)$,
где $|R|$ --- количество правил
(размер множества троек для каждого узла ограничен числом нетерминалов, которое не больше,
чем количество правил). Свертка также имеет линейную сложность.


\newpage
\subsection{Средства форматирования кода в IDE}

Интегрированные среды разработки программного обеспечения
(IDE) предоставляют средства
для форматирования программного кода. Рассмотрим их на примере двух Java IDE ---
IntelliJ IDEA\footnote{\cd{http://jetbrains.com/idea/}} и
Eclipse\footnote{\cd{http://eclipse.org}}.
Для задания требуемого СК используется широкий набор настроек \emph{форматтера},
подпрограммы IDE, отвечающей за форматирование исходных текстов.
Примерами таких настроек являются:
\begin{itemize}
  \item помещать ли фигурную скобку на той же строке, что и
    предыдущее выражение;
  \item форматирование списков --- всегда на одной строчке, переводить строчку после
    каждого элемента или печатать на одной строке,
    пока строчка меньше заданной рекомендуемой ширины.
%   \item и т.д.
\end{itemize}
На рис. ~\ref{fig:ideaFormatter} и~\ref{fig:eclipseFormatter} приведены диалоги задания
параметров форматирования в IntelliJ IDEA и Eclipse соответственно.

\begin{figure}[p]
	\centering
	\includegraphics[width=\textwidth]{ideaFormatter}
	\caption{Окно настройки форматтера IntelliJ IDEA}
	\label{fig:ideaFormatter}
\end{figure}

\begin{figure}[p]
	\centering
	\includegraphics[width=\textwidth]{eclipseFormatter}
	\caption{Окно настройки форматтера Eclipse}
	\label{fig:eclipseFormatter}
\end{figure}

У встроенных форматтеров есть следующие особенности.
Во-первых, это невозможность выразить
нестандартный СК, так как настройки форматтеров, несмотря на их большое
количество,
дают ограниченную вариативность для текстовых представлений форматируемых
программ. В целом, количество принципиально разных
стилей форматирования, которые можно задать с помощью данных настроек,
невелико.
Во-вторых, в случае, если надо
придерживаться СК уже существующего проекта, то по коду этого проекта необходимо
вручную задать все настройки форматтера.
Причем наборы настроек в разных IDE не совпадают, что увеличивает сложность
поддержки единого СК, если разработчики используют отличные от друг друга IDE.
Для решения данной проблемы существуют плагины, позволяющие использовать
внешние фоматтеры\footnote{\cd{http://plugins.jetbrains.com/plugin/6546}}.
Кроме того, есть средства, позволяющие экспортировать настройки форматтера в
XML-файл для их использования в другой
IDE\footnote{\cd{http://blog.jetbrains.com/idea/2014/01/intellij-idea-13-importing-code-formatter-settings-from-eclipse/}}.

Важным достоинством описанных форматтеров является малое время работы.
Это достигается в том числе и за счет того, что часто форматируется не весь
файл, а только его часть (см. \cite{eclipse}),
и из-за детерминированности представлений узлов синтаксического дерева,
которая следует из специфики настроек форматтера.
Однако форматирование целого большого проекта занимает существенное время.
Например, обработка кода IntelliJ IDEA Community Edition занимает более часа
у встроенного в IDEA форматтера.


\section{Реализация}
В процессе работы над проектом было выделено две основных задачи: реализация
взаимодействия с пользователем и расширение функциональности для переформатирования программного текста с фильтрацией шаблонов.
%диаграмму что ли нарисавать?
Предполагаемый процесс взаимодействия с пользователем: пользователь выделяет
участок программного кода, в котором будут производиться изменение, и вызывает 
действие (Action), которое извлекает шаблон из выделенного участка
кода. Затем пользователь меняет форматирование этого же участка, выделяет его
снова и вызывает действие, которое извлекает новый шаблон форматирования и 
сохраняет его.

После этого происходит перепечатывание элементов (PsiElement), имеющих старое
форматирование, в файле, в котором пользователь менял программный текст.

\subsection{Переформатирование}
\subsubsection{Извлечение элемента для форматирования}
% тут будет диаграмма:
% 1) Получаем текст
% 2) Корректируем отступы + пример
% 3) Извлечение элемента (для того чтобы знать тип) -- вся рутина
% 4) Склеить текст и элемент
В первую очередь необходимо получить элемент, форматирование которого пользователь хочет изменить.
% Где-то надо ссылку на диаграмму, наверное, тут
Весь этот процесс изображен на рис.~\ref{fig:getElemDiag}.

\begin{figure}[h]
	\centering
	\includegraphics[width=\textwidth]{images/getElemDiag.jpg}
	\caption{Диаграмма этапов извлечения элемента из текста}
	\label{fig:getElemDiag}
\end{figure}

На первом этапе с помощью функций, предоставляемых IntelliJ IDEA API, можно получить текст элемента. 
Однако этот текст не всегда корректный, так как может содержать в себе лишние отступы.
Например, на рис. \ref{fig:offsetExample} изображены методы doSomething() и doAnotherThing(). 
Они имеют отступ относительно оператора ветвления, равный четырем пробелам.
Кроме того, они имеют отступ относительно метода foo(), равный восьми пробелам. 
Чтобы шаблон был корректным, на втором этапе убираются лишние пробелы (в данном случае четыре).
Далее получаем из текста сам элемент. 
Это необходимо, чтобы узнать его тип. %ссылку на дерево PsiElement'ов.
Затем (на четвертом этапе) из корректного текста и полученного элемента создается новый элемент того же типа, но с корректными отступами.

Все это необходимо проделать, так как IntelliJ IDEA API не предоставляет функционала по изменению текста элемента явным образом. 

\begin{figure}[h]
	\centering
	\lstinputlisting[language=C++]{codes/offsetExample.txt}
	\caption{Общий отступ элементов}
	\label{fig:offsetExample}
\end{figure}

Аналогичные действия необходимо повторить для получения элемента из измененного
пользователем текста. 
Из этого элемента извлекается шаблон и сохраняется принтером для дальнейшего использования.

%\subsubsection{Извлечение шаблона}
%Шаблон, как и прежде, извлекается методами класса \lstinline[language=C++]{PsiElementComponent}.

%\subsubsection{Добавление шаблона}
%В связи с тем, что возникла необходимость добавления одного шаблона, класс %Printer обзавелся публичным методом, реализующим этот процесс. Остальные
%действия аналогичны тем при заполнении списка шаблонов из файла.

\subsubsection{Сравнение с шаблоном}
Как уже было отмечено в обзоре, при перепечатывании принтер обходит все элементы (PsiElement) дерева разбора и применяет к ним новые шаблоны. 
В рамках реализуемого подхода происходит перепечатывание лишь тех элементов, которые имели старое форматирование.
Это необходимо, если хочется поддерживать два и более различных варианта форматирования элементов некоторого типа. 
На рис. \ref{templates} изображены три блока оператора ветвления. 
Без фильтрации шаблонов все три блока будут изменены и получат одинаковое форматирование. %рис. сделать 
Однако первые два блока и так легко читаемы и понятны. 
Третий блок, будучи однострочным, уже не будет восприниматься так же хорошо.
Кроме того, существуют ограничения на ширину вывода, то есть в одной строке не можеть быть символов больше некоторой наперед заданной величины.

%\begin{figure}[t]
%    \centering
%    \lstinputlisting[language=C++]{codes/sameTmpltElems.txt}
%    \caption{Шаблоны}
%    \label{fig:sameTmpltElems}
%\end{figure}
%добавить вторую картинку как стало бы
\fvset{frame=lines,framesep=7pt}
\begin{figure}[ht]
\noindent\begin{minipage}{.5\textwidth}
    \lstinputlisting[language=C++]{codes/sameTmpltElems.txt}
\caption*{а) До переформатирования}    
\end{minipage}\hfill
\begin{minipage}{.5\textwidth}
    \lstinputlisting[language=C++]{codes/sameTmpltElems2.txt}
\caption*{б) После переформатирования}    
\end{minipage}
\caption{Переформатирование элементов одного типа без фильтрации шаблонов} 
\label{templates}
\end{figure}


Чтобы сделать настройку форматирования более гибкой и избавиться от подобного рода проблем, необходимо производить сравнение шаблонов текущего элемента дерева разбора и элемента со старым форматированием. 
Переформатировать элемент -- только в случае совпадения шаблонов.
Шаблоны совпадают, если совпадет их текстовое представление: шаблон записывается в виде строки, которая содержит необходимые элементы: ключевые слова, пробелы, переносы строк, скобки, а так же метки для поддеревьев, в которых указывается дополнительная информация, например, количество выражений в поддереве (одно или несколько).
В случае оператора ветвления (рис.~\ref{iftmpltcode}а) его поддеревьями являются: условие (condition), ветка, соответствующая выполнению условия (then block) и содержащая несколько подвыражений, ветка, соответствующая невыполнению условия (else block) и содержащая одно подвыражение. %ifTmplt ifTmpltCode
Соответствующий ему шаблон изображен на рисунке~\ref{iftmpltcode}б.
\fvset{frame=lines,framesep=7pt}
\begin{figure}[ht]
\noindent\begin{minipage}{.5\textwidth}
    \lstinputlisting[language=C++]{codes/ifTmpltCode.txt}
\caption*{а) Оператор ветвления}    
\end{minipage}\hfill
\begin{minipage}{.5\textwidth}
    \lstinputlisting[language=C++]{codes/ifTmplt.txt}
\caption*{б) Шаблон для оператора ветвления}    
\end{minipage}
\caption{Оператор ветвления и соответствующий ему шаблон}    
\label{iftmpltcode}
\end{figure}

%\begin{figure}[t]
%    \centering
%    \lstinputlisting[language=C++]{codes/sameTmpltElems.txt}
%    \caption{Шаблоны}
%    \label{fig:sameTmpltElems}
%\end{figure}
%добавить вторую картинку как стало бы

\subsection{Оценка производительности}
Описанный выше способ  давал очень низкую производительность. 
В таблице~\ref{table:slow} приведено время, затрачиваемое принтером на переформатирование файлов различной длины. 
Измерения производились на примере двух блоков: оператора ветвления и метода.

\begin{table}[h]
\begin{tabular}{cc|c|c|c|c|}
\cline{2-5}
\multicolumn{1}{ c| }{ } & \multicolumn{2}{p{6cm} |}{Формат. операторов ветвления} & \multicolumn{2}{p{5cm} |}{Форматирование методов} \\
\hline
\multicolumn{1}{ |c| }{Размер файла (строк)} & Кол-во & Время (сек.) & Кол-во & Время (сек.) \\ 
\hline
\multicolumn{1}{ |c| }{>7000}           & 520   & 35    & 660  & 58 \\
\multicolumn{1}{ |c| }{1000..2000}       & 145   & 2.7   & 90   & 4.4 \\
\multicolumn{1}{ |c| }{500..1000}       & 54    & 0.73  & 72   & 1.34 \\
\multicolumn{1}{ |c| }{0..500}          & 35    & 0.2   & 36   & 0.67 \\
\hline
\end{tabular}
\caption{Время переформатирования файлов I}
\label{table:slow}
\end{table}


Причина столь низкой производительности кроется в том, что для переформатирования элементов используется тот же способ, что и при форматировании по образцу:
принтер обходит все дерево разбора и сравнивает шаблоны.
%Однако нет необходимости проверять на совпадение шаблоны, полученные из каждого элемента дерева разбора, и шаблон, полученного из элемента со старым форматированием.
Однако нет необходимости сравнивать шаблон, полученный из элемента со старым форматированием, с шаблоном, полученным из каждого элемента дерева разбора.
Количество вариантов для переформатирования можно сузить лишь для элементов того же типа.

Для уменьшения затрат времени при обходе дерева элементов создается список кандидатов на форматирование (элементы того же типа, что и элемент со старым форматированием). 
После этого из элементов созданного списка извлекаются шаблоны.
Если полученный шаблон совпадает с шаблоном элемента со старым форматированием, то происходит перепечатывание элемента и замена его в дереве разбора.
Время, затрачиваемое в этом случае на переформатирование, приведено в таблице~\ref{table:fast}.

%\begin{table}[h]
%\begin{tabular}{ l | p{4cm} | p{4cm} }
%  Размер файла (строк) & Формат. операторов ветвления (сек.) & Формат. методов %(сек.) \\ 
%  \hline
%  >7000         & 6.6   & 5.8 \\
%  1000..2000    & 0.72  & 1   \\
%  500..1000     & 0.125 & 0.67 \\
%  0..500        & 0.145 & 0.325 \\
%\end{tabular}
%\caption{Время переформатирования файлов II}
%\label{table:fast}
%\end{table}


\begin{table}[h]
\begin{tabular}{cc|c|c|c|c|}
\cline{2-5}
\multicolumn{1}{ c| }{ } & \multicolumn{2}{p{6cm} |}{Формат. операторов ветвления} & \multicolumn{2}{p{5cm} |}{Форматирование  методов} \\
\hline
\multicolumn{1}{ |c| }{Размер файла (строк)} & Кол-во & Время (сек.) & Кол-во & Время (сек.) \\ 
\hline
\multicolumn{1}{ |c| }{>7000}           & 520   & 6.6   & 660  & 5.8 \\
\multicolumn{1}{ |c| }{1000..2000}      & 145   & 0.72  & 90   & 1 \\
\multicolumn{1}{ |c| }{500..1000}       & 54    & 0.225 & 72   & 0.67 \\
\multicolumn{1}{ |c| }{0..500}          & 35    & 0.145 & 36   & 0.325 \\
\hline
\end{tabular}
\caption{Время переформатирования файлов II}
\label{table:fast}
\end{table}
\section{Заключение}

В данной работе описан подход к композициональному символьному исполнению без раскрутки. Была предложена концепция композициональной памяти с символьной адресацией. Был доказан некоторый набор свойств КСП, дающий основание для подхода в стиле систем переписывания, где символьные кучи могут сами выступать как символы. Это даёт возможность автоматически порождать уравнения на состояния, решения которых в точности отражают поведения функций, работающих с динамической памятью. Было показано как свести задачу решения уравнений на состояния к задаче проверки безопасности чистых функций второго порядка.

Данная работа нацелена на теоретические основания композиционального анализа динамической памяти. Мы оставляем апробацию этого подхода на будущее. Другим направлением будущих исследований может быть расширение нашего формализма на композициональный анализ параллельных программ.

\begin{thebibliography}{99}
  \bibitem{podkopaev:diploma1}
  А.В.Подкопаев. Полиномиальной сложности оптимальные принтер-комбинаторы с выбором.
  Дипломная работа, кафедра системного программирования, математико-механический факультет СПбГУ, 2014.

  \bibitem{stratego:xt}
  Stratego/XT. \url{http://strategoxt.org/Stratego/StrategoDocumentation}, 2015.

  \bibitem{intellij:plugdev}
  IntelliJ IDEA Plugin Development Guide. \url{https://confluence.jetbrains.com/display/IDEADEV/PluginDevelopment},
  2015.
\end{thebibliography}

% %\usepackage{listings}  
%\usepackage{hyperref}
%\usepackage{verbments}
%\usepackage{minted}
%\usepackage{caption}
%\usepackage{fancybox}
%\newfontfamily{\cyrillicfonttt}{CMU Serif}

\title{Декларативное форматирование в режиме онлайн}

\titlerunning{Декларативное форматирование в режиме онлайн}

\author{Озерных Игорь Станиславович}

\authorrunning{И.С.Озерных}

\tocauthor{И.С.Озерных}
\institute{Санкт-Петербургский государственный университет\\
\email{igor.ozernykh@gmail.com}}

\maketitle             

\begin{abstract}
Аккуратное, соответствующие выбранному стилю оформление (форматирование)
программных текстов критично для их восприятия.
Существуют средства автоматического форматирования кода. Одним
из них является метод синтаксических шаблонов. Его преимуществом является декларативность, которая
достигается за счет задания стиля форматирования с помощью образца кода. Так, естественным ограничением
описанного метода является необходимость наличия достаточно полного образца на целевом языке.
Данная работа посвящена расширению границ применимости метода синтаксических шаблонов
за счет возможности задания шаблонов в режиме онлайн.
\end{abstract}

\section*{Введение}

Этап поддержки и сопровождения в жизненном цикле программного обеспечения может занимать до 90\% времени существования программного продукта. На этом этапе особенно важно понимание программного текста поддерживаемой системы, что зачастую бывает сложной задачей. Необходимым условием для лучшего понимания программного текста является его аккуратное оформление, отражающее структуру 
программы. Существуют различные стандарты оформления кода (стили кодирования)~--- наборы правил и соглашений, используемые при написании исходного кода на некотором языке программирования.
Рассмотрим их на примере стандарта кодирования, принятого в компании Google\footnote{\texttt{https://google-styleguide.googlecode.com/svn/trunk/cppguide.html}}, и стандарта, используемого в проекте GNU\footnote{\texttt{http://www.gnu.org/prep/standards/standards.html}} (см. рис.~\ref{codingstandards}) для языка C++.


\fvset{frame=lines,framesep=7pt}
\begin{figure}[ht]
\noindent\begin{minipage}{.5\textwidth}
    \lstinputlisting[language=Java]{Ozernykh/codes/exGoogleCC.txt}
\caption*{а) Стиль кодирования Google}    
\end{minipage}\hfill
\begin{minipage}{.5\textwidth}
    \lstinputlisting[language=Java]{Ozernykh/codes/exGNUCC.txt}
\caption*{б) Стиль кодирования GNU}    
\end{minipage}
\caption{Различные стили кодирования для языка С++}    
\label{codingstandards}
\end{figure}

Когда программист работает над проектом, он придерживается некоторого стиля кодирования. 
Если файлы проекта с одним стилем кодирования присоединяется к уже существующему проекту с другим стилем кодирования, то их форматирование нужно привести к тому же виду. 
Для решения этой задачи можно воспользоваться уже существующими средствами, например, форматтерами (программами, которые форматируют исходный текст) в IDE таких, как Eclipse, IntelliJ IDEA, Visual Studio и др. 
Однако в этом случае необходимо вручную задавать настройки форматирования.
Кроме того, количество принципиально разных стилей форматирования, которые можно задать с помощью данных настроек, невелико.

%Однако в этом случае необходимо иметь настройки форматирования, заданные программистом, но в целом, количество принципиально разных стилей форматировани, которые можно задать с помощью данных настроек, невелико. %причем эти настройки могут быть различными в разных IDE.
% * <Anton Podkopaev> 12:15:51 22 May 2015 UTC+0300:
% Нужно переработать это предложение.

Еще одним примером может послужить принтер-плагин для IntelliJ IDEA~\cite{podkopaev:diploma1}, который позволяет форматировать исходный код проекта по образцу. 
В качестве образца используется код из некоторого репозитория. 
Из него выделяются шаблоны форматирования для структур языка, которые применяются к структурам целевого кода. 
Под \textit{шаблоном} понимаются данные, сопоставление которых с элементом синтаксического дерева дает текстовое представление этого элемента (и его потомков). 
Например, на рис.~\ref{fig:ifTree} представлено дерево разбора для оператора ветвления. 


\begin{figure}[h]
	\centering
	\includegraphics[width=\textwidth]{Ozernykh/images/ifTree.jpg}
	\caption{Представление оператора ветвления в виде дерева разбора}
	\label{fig:ifTree}
\end{figure}

На рис.~\ref{fig:tmpltcodeintro}а изображен шаблон форматирования, который может быть применен к нему (несколько подвыражений, соответствующих выполнению условия, и одно подвыражение, соответствующее невыполненению условия).
На рис.~\ref{fig:tmpltcodeintro}б представлен результат применения этого шаблона к дереву разбора для оператора ветвления.

% \begin{figure}[h]
% 	\centering
% 	\lstinputlisting[language=c++]{codes/ifTmpltIntro.txt}
% 	\caption{Шаблон для оператора ветвления}
% 	\label{fig:ifTmpltIntro}
% \end{figure}

% На рис.~\ref{fig:ifCodeIntro} представлен результат применения этого шаблона к дереву разбора для оператора ветвления.

% \begin{figure}[h]
% 	\centering
% 	\lstinputlisting[language=c++]{codes/ifCodeIntro.txt}
% 	\caption{Представление дерева разбора при применении шаблона}
% 	\label{fig:ifCodeIntro}
% \end{figure}


\fvset{frame=lines,framesep=7pt}
\begin{figure}[ht]
\noindent\begin{minipage}{.5\textwidth}
    \lstinputlisting[language=Java]{Ozernykh/codes/ifTmpltIntro.txt}
\caption*{а) Шаблон для оператора\\
ветвления}    
\end{minipage}\hfill
\begin{minipage}{.5\textwidth}
    \lstinputlisting[language=Java]{Ozernykh/codes/ifCodeIntro.txt}
\caption*{б) Текст, полученный при применении шаблона к дереву разбора}    
\end{minipage}
\caption{Оператор ветвления и шаблон для него}    
\label{fig:tmpltcodeintro}
\end{figure}

Однако не всегда этот репозиторий существует. Кроме того, необходимо наличие заданного форматирования для всех структур языка.
% * <Anton Podkopaev> 12:19:02 22 May 2015 UTC+0300:
% 'Однако не всегда удобно иметь этот репозиторий под рукой.' --- поменять. Не про удобство надо говорить.

Шаблоны можно извлекать из уже существующего кода и применять их к другим элементам того же типа, но отформатированных иным способом. 
Например, пользователь меняет форматирование некоторой структуры языка (в частности, оператор ветвления), программа извлекает из полученного участка кода новый шаблон и применяет его к структурам того же типа, содержащим старое форматирование. 
Назовем такой способ задания шаблонов~--- форматирование в режиме
онлайн.
%картинки

Целью данной работы является расширение функциональности описанного 
принтер-плагина путем добавления возможности задания шаблонов в режиме онлайн. 

\section{Обзор предметной области}

Одним из подходов к заданию принтеров являются принтер-комбинаторы.
В этой главе приведен обзор классических принтер-комбинаторов, а также
принтер-комбинаторов с дополнительным оператором \emph{выбора}, которые
в рамках данной работы использованы для реализации принтеров, задаваемых
с помощью шаблонов. Для решения проблем эффективности
комбинаторов с выбором в данной работе был использован подход BURS.
Апробация подхода задания принтеров с помощью шаблонов
была произведена для языка Java, для которого есть средства форматирования,
встроенные в IDE.

\subsection{Принтер-комбинаторы}

Базовыми принтер-комбинаторными библиотеками являются библиотеки
Джона Хьюза\cite{hughes} и Филиппа Вадлера\cite{wadler}, которые
представляют собой функциональную переработку алгоритма
Оппена\cite{oppen}.
Ограничимся рассмотрением библиотеки Вадлера, так как
библиотека Хьюза обладает теми же свойствами, которые принципиальны в контексте
данной работы.

В библиотеке ключевым типом является документ.
Он представляет сущность, которая потом может
быть переведена в строковое представление алгоритмом принтера.
Основные конструкторы для составления документа таковы:
\begin{itemize}
  \item атомарная строка, которая печатается как есть;
  \item \emph{разделитель};
  \item последовательная композиция двух документов;
  \item набор связанных документов.
\end{itemize}

Определяющей особенностью данного подхода является то, что все разделители
в рамках одного набора могут быть совместно заменены алгоритмом принтера
на пробельный символ или на перевод строки. Выбор для каждого набора разделителей
основывается на том, что вывод должен поместиться в заданную ширину,
используя минимальное число строк.

Основная проблема такого подхода заключается в его слабой выразительной силе.
Документы, построенные по синтаксическому дереву печатаемой программы,
обрабатываются слишком единообразно, что иногда приводит к нежелательному результату.
Пусть, к примеру, нужно напечатать программу на языке
Python\footnote{\cd{http://python.org}}.
Между последовательными операторами в случае их печати на одной строчке необходимо
добавить дополнительный разделитель (``;''), иначе программа станет некорректной
(см. рис.~\ref{fig:seqEx}). Однако, описанные принтер-комбинаторы не предоставляют
возможности задать такое поведение.
Кроме того, с помощью таких принтер-комбинаторов невозможно выразить разные проектные СК,
так как они всегда печатают текст в одном стиле, который жестко ``зашит'' в их код.

\begin{figure}[h!]
	\centering
	\null\hfill
	\subfloat[Корректный код]{
		\centering
    \makebox[.4\textwidth] {
		  \lstinputlisting[language=Python]{codes/pythonCode.py}
    }
	}
	\null\hfill
	\subfloat[Некорректный код]{
		\centering
    \makebox[.4\textwidth] {
	  	\lstinputlisting[language=Python]{codes/pythonCodeBad.py}
    }
	}
	\hfill\null
	\caption{Пример работы принтер-комбинаторов для языка Python}
  \label{fig:seqEx}	
\end{figure}

Более подробный обзор библиотек Хьюза и Вадлера представлен в \cite{myCoursePaper}.

Большинство принтер-комбинаторных библиотек\cite{
swierstraChitil, swierstra04, peytonJones, kiselyov, chitil}
являются развитиями работ Хьюза и Вадлера. Так, в отличие от базовых, среди
них есть реализации с линейной сложностью обработки документа от его размера и
\emph{online} алгоритмы, которые не требуют просмотра всего документа
для начала печати его текстового представления. Но все они обладают тем же
интерфейсом, а значит в них также неразрешимы задачи, в которых требуется
задавать варианты текстовых представлений, которые отличаются не только
пробелами и переводами строк.

\subsection{Принтер-комбинаторы с выбором}

Существенно от описанных выше отличаются библиотеки, предоставляющие в своем
интерфейсе комбинатор \emph{выбора}, который позволяет задавать для одного поддерева
принципиально разные варианты раскладок.
Так, в работах~\cite{jongeEveryOccasion, jongeReengine} используется
оператор \lstinline[language = Haskell]{ALT},
но алгоритм принтера устроен так, что среди двух альтернатив выбирается первая, если она
помещается в заданную ширину, и вторая --- иначе,
что не дает оптимальный результат на выходе.

Оптимальные принтер-комбинаторы с выбором были впервые представлены в работе~\cite{swierstra}.
Их реализация является частью Utrecht Tools
Library\footnote{\cd{http://www.cs.uu.nl/wiki/HUT/WebHome}}
(практическая реализация несколько изменена по
отношению к той, что описана в статье, но отличие несущественно).
В данном подходе текст строится из блоков прямоугольной формы с возможно неполной последней
строчкой (см. рис. \ref{fig:basicFormat}). В реализации на Haskell блоки представляются
структурой \lstinline[language = Haskell]{Format}:
\begin{lstlisting}
    data Format = Elem { height        :: Int
                       , lastLineWidth :: Int
                       , width         :: Int
                       , txtstr        :: Int -> String -> String
                       }
\end{lstlisting}

Первые три поля структуры определяют геометрические размеры блока, а последнее --- функция,
которая используется для преобразования блока в текст. Здесь используется функция, а не
просто строчка, чтобы можно было преобразовывать вложенные блоки за линейное время. Первый
аргумент \lstinline[language = Haskell]{txtstr} задает сдвиг блока.

\begin{figure}
  \centering
  \subfloat[Блок текста]{
    \raisebox{5mm}{
      \centering
      \vspace{0pt}
      \tikz[scale = 2.0]{
        \draw (0,0) -- (1,0) -- (1,0.2) -- (2,0.2) -- (2,1) -- (0, 1) -- cycle;
      }
    }
    \label{fig:basicFormat}
  }
  ~
  \subfloat[Горизонтальная композиция]{
    \centering
    \tikz[scale = 2.0]{
       \draw (0,0) -- (1,0) -- (1,0.2) -- (2,0.2) -- (2,1) -- (0, 1) -- cycle;
       \draw (1.1,-0.9) -- (2.1,-0.9) -- (2.1,-0.7) -- (3.1,-0.7) -- (3.1,0.1) -- (1.1,0.1) -- cycle;
     }
     \label{fig:beside}
  }
  %\hfill
  ~
  \subfloat[Вертикальная композиция]{
    \makebox[.28\textwidth] {
    \centering
    \tikz[scale = 2.0]{
       \draw (0,0) -- (1,0) -- (1,0.2) -- (2,0.2) -- (2,1) -- (0, 1) -- cycle;
       \draw (0,-1.1) -- (1,-1.1) -- (1,-0.9) -- (2,-0.9) -- (2,-0.1) -- (0,-0.1) -- cycle;
    }
    \label{fig:above}
    }
  }
  \caption{Примитивы блока текста}
  \label{fig:basicConcat}
\end{figure}

Для работы с блоками текста используется следующие четыре примитива:
\begin{lstlisting}
    s2fmt     :: String -> Format
    indentFmt :: Int -> Format -> Format
    aboveFmt  :: Format -> Format -> Format
    besideFmt :: Format -> Format -> Format
\end{lstlisting}

Функция \lstinline[language = Haskell]{s2fmt} создает блок текста,
состоящий из одной строчки; \lstinline[language = Haskell]{indentFmt} по блоку создает новый,
сдвинутый на заданное число позиций. Действие примитивов композиции
\lstinline[language = Haskell]{besideFmt} и \lstinline[language = Haskell]{aboveFmt}
показано на рис.~\ref{fig:beside} и \ref{fig:above} соответственно.

Также, как и в библиотеках без комбинатора выбора, в работе~\cite{swierstra} используется понятие
\emph{документа}. Здесь документ можно рассматривать как множество возможных раскладок
(набор \lstinline[language = Haskell]{Format}-элементов). Документы описываются типом
\lstinline[language = Haskell]{Doc}, экземпляры которого на этапе построения представляют собой
деревья применений приведенных ниже комбинаторов, а на этапе обработки алгоритмом
принтера им в соответствие ставится итоговое множество вариантов текстовых представлений.

Документ конструируется с помощью следующих комбинаторов, которые симметричны
примитивам построения блоков текста:
\begin{lstlisting}
    text   :: String -> Doc
    indent :: Int -> Doc -> Doc
    beside :: Doc -> Doc -> Doc
    above  :: Doc -> Doc -> Doc
\end{lstlisting}

В дополнение появляется пятый комбинатор для документов:
\begin{lstlisting}
    choice :: Doc -> Doc -> Doc
\end{lstlisting}

Этот комбинатор и является тем самым комбинатором выбора. Он представляет
\emph{объединение} множеств раскладок документов, которые были переданы как
аргументы комбинатора. Заметим, что только этот комбинатор может произвести
документ с несколькими раскладками из одновариантных аргументов.

Оригинальная реализация существенно опирается на ленивые вычисления. В \cite{swierstra}
множество вариантов, соответствующее экземпляру \lstinline[language = Haskell]{Doc},
представляется ленивым списком всех возможных раскладок, удовлетворяющих ограничению
на максимальную ширину. Этот список отсортирован в порядке ``ухудшения''
раскладок, то есть в голове списка лежит ``лучшая'', в терминах оптимальности,
раскладка из возможных при заданной ширине. В случае
\lstinline{beside}- и
\lstinline{above}-композиций
документов полная (без учета ленивости) сложность вычисления списка нового документа
составляет $O(n \times m)$, где $n$ и $m$ --- размеры списков, соответствующих 
соединяемым документам. Размер нового списка также порядка $n \times m$.
Он не обязательно точно равен $n \times m$, так как некоторые из полученных
представлений могут не подпадать под ограничение на ширину, соответственно их можно
отбросить на этапе построения нового списка.

Выбор лучшего представления для самого верхнеуровнего документа происходит
просто --- нужно из соответствующего списка взять первый элемент.
Это создает впечатление, что общее число операций,
благодаря ленивым вычислениям, существенно уменьшается.
Но это не так из-за реализации обработки документа,
построенного с помощью комбинатора \lstinline[language = Haskell]{beside}, которая
вынуждает полное вычисление дочерних списков. Более того, из-за свойств отношения
порядка, построенного по критерию оптимальности, в рамках данной модели списков
принципиально нельзя построить ленивую обработку комбинатора
\lstinline[language = Haskell]{beside}.
Так, выбор оптимальной раскладки документа в
\cite{swierstra} имеет в худшем случае экспоненциальную сложность от числа
комбинаторов, использованных при его построении.
В \cite{swiComb} приведены оптимизации для данной модели, но они
не улучшают асимптотику решения для деревьев в общем случае.

\subsection{BURS}

Bottom-Up Rewrite System (BURS)\cite{burs} --- это метод динамического
программирования на деревьях, изначально появившийся в контексте задачи выбора
инструкций для генерации машинного кода. Основой BURS является
регулярная грамматика с древовидными правилами\cite{tata}, для которых задана
стоимость применения подстановки,
то есть грамматика со следующим набором правил: 

$$
\begin{array}{rcll}
  N &:& \alpha& [c]\\
  N &:& \alpha\; (K_1,\dots,K_n)& [c]
\end{array}
$$

Здесь $N, K_i$ это нетерминалы, $\alpha$ --- терминал,
$c$ --- функция стоимости, заданная для каждого правила.
Как и для обычной линейной грамматики, вводится стартовый
нетерминал S. Считается, что терминальное дерево выводится в
данной грамматике, если его можно получить с помощью правил
подстановки из одноузлового дерева $S$.
Каждая подстановка заменяет нетерминал $N$, находящийся в листе дерева, на дерево 
$\alpha\;(K_1,\dots,K_n)$, если в грамматике есть правило
$N:\alpha\;(K_1,\dots,K_n)$. 
Для каждой подстановки вычисляется стоимость ее применения с помощью
функции стоимости $c$.
Аргументами функции могут служить терминальная метка $\alpha$ и стоимости
вывода поддеревьев.
Задачей, которую решает BURS, является поиск вывода наименьшей стоимости
для заданного дерева по заданной грамматике.
Такой вывод может быть найден двухпроходным алгоритмом.

Первый проход (\emph{пометка}) обрабатывает дерево снизу вверх и вычисляет
для каждого узла набор троек $(K,\;R,\;c)$, где $K$ --- нетерминал, из которого может быть
выведено поддерево с корнем в обрабатываемом узле,
$R$ --- первое правило, которое используется для вывода минимальной стоимости из $K$,
$c$ --- стоимость такого вывода.
Процесс пометки происходит следующим образом:

\begin{itemize}
\item для листовой вершины, помеченной терминалом $\alpha$, в множество троек
этого вершины добавляется $(K,\;R,\;c\:(\alpha))$ для каждого правила $R=K:\;\alpha\;[c]$;
\item для промежуточной вершины, помеченной терминалом $\alpha$,
с непосредственными поддеревьями $v_1,\dots,v_n$
в множество добавляется тройка $(K,\;R,\;c\:(\alpha,c_1,\dots,c_n))$ для каждого правила
$R=K:\;\alpha\;(K_1,\dots,K_n)\;[c]$, где $(K_i,\;R_i,\;c_i)$ входит в множество троек для
$v_i$; если есть несколько правил вывода из нетерминала $K$, то выбирается правило,
минимизирующее стоимость вывода.
\end{itemize}

Второй проход (\emph{свертка}) просматривает дерево сверху вниз, используя сделанные пометки.
Первое правило из минимального вывода определяется тройкой $(S,\;R,\;c)$ для корневого узла
(если такой тройки нет, то вывод из $S$ невозможен).
Это правило однозначно определяет нетерминалы $K_i$ для каждого непосредственного поддерева
и процесс повторяется.

Для пометки дерева потенциально необходимо каждое правило грамматики применить к каждому узлу.
При фиксированной грамматике алгоритмическая сложность первого прохода --- $O\:(|R|)$,
где $|R|$ --- количество правил
(размер множества троек для каждого узла ограничен числом нетерминалов, которое не больше,
чем количество правил). Свертка также имеет линейную сложность.


\newpage
\subsection{Средства форматирования кода в IDE}

Интегрированные среды разработки программного обеспечения
(IDE) предоставляют средства
для форматирования программного кода. Рассмотрим их на примере двух Java IDE ---
IntelliJ IDEA\footnote{\cd{http://jetbrains.com/idea/}} и
Eclipse\footnote{\cd{http://eclipse.org}}.
Для задания требуемого СК используется широкий набор настроек \emph{форматтера},
подпрограммы IDE, отвечающей за форматирование исходных текстов.
Примерами таких настроек являются:
\begin{itemize}
  \item помещать ли фигурную скобку на той же строке, что и
    предыдущее выражение;
  \item форматирование списков --- всегда на одной строчке, переводить строчку после
    каждого элемента или печатать на одной строке,
    пока строчка меньше заданной рекомендуемой ширины.
%   \item и т.д.
\end{itemize}
На рис. ~\ref{fig:ideaFormatter} и~\ref{fig:eclipseFormatter} приведены диалоги задания
параметров форматирования в IntelliJ IDEA и Eclipse соответственно.

\begin{figure}[p]
	\centering
	\includegraphics[width=\textwidth]{ideaFormatter}
	\caption{Окно настройки форматтера IntelliJ IDEA}
	\label{fig:ideaFormatter}
\end{figure}

\begin{figure}[p]
	\centering
	\includegraphics[width=\textwidth]{eclipseFormatter}
	\caption{Окно настройки форматтера Eclipse}
	\label{fig:eclipseFormatter}
\end{figure}

У встроенных форматтеров есть следующие особенности.
Во-первых, это невозможность выразить
нестандартный СК, так как настройки форматтеров, несмотря на их большое
количество,
дают ограниченную вариативность для текстовых представлений форматируемых
программ. В целом, количество принципиально разных
стилей форматирования, которые можно задать с помощью данных настроек,
невелико.
Во-вторых, в случае, если надо
придерживаться СК уже существующего проекта, то по коду этого проекта необходимо
вручную задать все настройки форматтера.
Причем наборы настроек в разных IDE не совпадают, что увеличивает сложность
поддержки единого СК, если разработчики используют отличные от друг друга IDE.
Для решения данной проблемы существуют плагины, позволяющие использовать
внешние фоматтеры\footnote{\cd{http://plugins.jetbrains.com/plugin/6546}}.
Кроме того, есть средства, позволяющие экспортировать настройки форматтера в
XML-файл для их использования в другой
IDE\footnote{\cd{http://blog.jetbrains.com/idea/2014/01/intellij-idea-13-importing-code-formatter-settings-from-eclipse/}}.

Важным достоинством описанных форматтеров является малое время работы.
Это достигается в том числе и за счет того, что часто форматируется не весь
файл, а только его часть (см. \cite{eclipse}),
и из-за детерминированности представлений узлов синтаксического дерева,
которая следует из специфики настроек форматтера.
Однако форматирование целого большого проекта занимает существенное время.
Например, обработка кода IntelliJ IDEA Community Edition занимает более часа
у встроенного в IDEA форматтера.


\section{Реализация}
В процессе работы над проектом было выделено две основных задачи: реализация
взаимодействия с пользователем и расширение функциональности для переформатирования программного текста с фильтрацией шаблонов.
%диаграмму что ли нарисавать?
Предполагаемый процесс взаимодействия с пользователем: пользователь выделяет
участок программного кода, в котором будут производиться изменение, и вызывает 
действие (Action), которое извлекает шаблон из выделенного участка
кода. Затем пользователь меняет форматирование этого же участка, выделяет его
снова и вызывает действие, которое извлекает новый шаблон форматирования и 
сохраняет его.

После этого происходит перепечатывание элементов (PsiElement), имеющих старое
форматирование, в файле, в котором пользователь менял программный текст.

\subsection{Переформатирование}
\subsubsection{Извлечение элемента для форматирования}
% тут будет диаграмма:
% 1) Получаем текст
% 2) Корректируем отступы + пример
% 3) Извлечение элемента (для того чтобы знать тип) -- вся рутина
% 4) Склеить текст и элемент
В первую очередь необходимо получить элемент, форматирование которого пользователь хочет изменить.
% Где-то надо ссылку на диаграмму, наверное, тут
Весь этот процесс изображен на рис.~\ref{fig:getElemDiag}.

\begin{figure}[h]
	\centering
	\includegraphics[width=\textwidth]{images/getElemDiag.jpg}
	\caption{Диаграмма этапов извлечения элемента из текста}
	\label{fig:getElemDiag}
\end{figure}

На первом этапе с помощью функций, предоставляемых IntelliJ IDEA API, можно получить текст элемента. 
Однако этот текст не всегда корректный, так как может содержать в себе лишние отступы.
Например, на рис. \ref{fig:offsetExample} изображены методы doSomething() и doAnotherThing(). 
Они имеют отступ относительно оператора ветвления, равный четырем пробелам.
Кроме того, они имеют отступ относительно метода foo(), равный восьми пробелам. 
Чтобы шаблон был корректным, на втором этапе убираются лишние пробелы (в данном случае четыре).
Далее получаем из текста сам элемент. 
Это необходимо, чтобы узнать его тип. %ссылку на дерево PsiElement'ов.
Затем (на четвертом этапе) из корректного текста и полученного элемента создается новый элемент того же типа, но с корректными отступами.

Все это необходимо проделать, так как IntelliJ IDEA API не предоставляет функционала по изменению текста элемента явным образом. 

\begin{figure}[h]
	\centering
	\lstinputlisting[language=C++]{codes/offsetExample.txt}
	\caption{Общий отступ элементов}
	\label{fig:offsetExample}
\end{figure}

Аналогичные действия необходимо повторить для получения элемента из измененного
пользователем текста. 
Из этого элемента извлекается шаблон и сохраняется принтером для дальнейшего использования.

%\subsubsection{Извлечение шаблона}
%Шаблон, как и прежде, извлекается методами класса \lstinline[language=C++]{PsiElementComponent}.

%\subsubsection{Добавление шаблона}
%В связи с тем, что возникла необходимость добавления одного шаблона, класс %Printer обзавелся публичным методом, реализующим этот процесс. Остальные
%действия аналогичны тем при заполнении списка шаблонов из файла.

\subsubsection{Сравнение с шаблоном}
Как уже было отмечено в обзоре, при перепечатывании принтер обходит все элементы (PsiElement) дерева разбора и применяет к ним новые шаблоны. 
В рамках реализуемого подхода происходит перепечатывание лишь тех элементов, которые имели старое форматирование.
Это необходимо, если хочется поддерживать два и более различных варианта форматирования элементов некоторого типа. 
На рис. \ref{templates} изображены три блока оператора ветвления. 
Без фильтрации шаблонов все три блока будут изменены и получат одинаковое форматирование. %рис. сделать 
Однако первые два блока и так легко читаемы и понятны. 
Третий блок, будучи однострочным, уже не будет восприниматься так же хорошо.
Кроме того, существуют ограничения на ширину вывода, то есть в одной строке не можеть быть символов больше некоторой наперед заданной величины.

%\begin{figure}[t]
%    \centering
%    \lstinputlisting[language=C++]{codes/sameTmpltElems.txt}
%    \caption{Шаблоны}
%    \label{fig:sameTmpltElems}
%\end{figure}
%добавить вторую картинку как стало бы
\fvset{frame=lines,framesep=7pt}
\begin{figure}[ht]
\noindent\begin{minipage}{.5\textwidth}
    \lstinputlisting[language=C++]{codes/sameTmpltElems.txt}
\caption*{а) До переформатирования}    
\end{minipage}\hfill
\begin{minipage}{.5\textwidth}
    \lstinputlisting[language=C++]{codes/sameTmpltElems2.txt}
\caption*{б) После переформатирования}    
\end{minipage}
\caption{Переформатирование элементов одного типа без фильтрации шаблонов} 
\label{templates}
\end{figure}


Чтобы сделать настройку форматирования более гибкой и избавиться от подобного рода проблем, необходимо производить сравнение шаблонов текущего элемента дерева разбора и элемента со старым форматированием. 
Переформатировать элемент -- только в случае совпадения шаблонов.
Шаблоны совпадают, если совпадет их текстовое представление: шаблон записывается в виде строки, которая содержит необходимые элементы: ключевые слова, пробелы, переносы строк, скобки, а так же метки для поддеревьев, в которых указывается дополнительная информация, например, количество выражений в поддереве (одно или несколько).
В случае оператора ветвления (рис.~\ref{iftmpltcode}а) его поддеревьями являются: условие (condition), ветка, соответствующая выполнению условия (then block) и содержащая несколько подвыражений, ветка, соответствующая невыполнению условия (else block) и содержащая одно подвыражение. %ifTmplt ifTmpltCode
Соответствующий ему шаблон изображен на рисунке~\ref{iftmpltcode}б.
\fvset{frame=lines,framesep=7pt}
\begin{figure}[ht]
\noindent\begin{minipage}{.5\textwidth}
    \lstinputlisting[language=C++]{codes/ifTmpltCode.txt}
\caption*{а) Оператор ветвления}    
\end{minipage}\hfill
\begin{minipage}{.5\textwidth}
    \lstinputlisting[language=C++]{codes/ifTmplt.txt}
\caption*{б) Шаблон для оператора ветвления}    
\end{minipage}
\caption{Оператор ветвления и соответствующий ему шаблон}    
\label{iftmpltcode}
\end{figure}

%\begin{figure}[t]
%    \centering
%    \lstinputlisting[language=C++]{codes/sameTmpltElems.txt}
%    \caption{Шаблоны}
%    \label{fig:sameTmpltElems}
%\end{figure}
%добавить вторую картинку как стало бы

\subsection{Оценка производительности}
Описанный выше способ  давал очень низкую производительность. 
В таблице~\ref{table:slow} приведено время, затрачиваемое принтером на переформатирование файлов различной длины. 
Измерения производились на примере двух блоков: оператора ветвления и метода.

\begin{table}[h]
\begin{tabular}{cc|c|c|c|c|}
\cline{2-5}
\multicolumn{1}{ c| }{ } & \multicolumn{2}{p{6cm} |}{Формат. операторов ветвления} & \multicolumn{2}{p{5cm} |}{Форматирование методов} \\
\hline
\multicolumn{1}{ |c| }{Размер файла (строк)} & Кол-во & Время (сек.) & Кол-во & Время (сек.) \\ 
\hline
\multicolumn{1}{ |c| }{>7000}           & 520   & 35    & 660  & 58 \\
\multicolumn{1}{ |c| }{1000..2000}       & 145   & 2.7   & 90   & 4.4 \\
\multicolumn{1}{ |c| }{500..1000}       & 54    & 0.73  & 72   & 1.34 \\
\multicolumn{1}{ |c| }{0..500}          & 35    & 0.2   & 36   & 0.67 \\
\hline
\end{tabular}
\caption{Время переформатирования файлов I}
\label{table:slow}
\end{table}


Причина столь низкой производительности кроется в том, что для переформатирования элементов используется тот же способ, что и при форматировании по образцу:
принтер обходит все дерево разбора и сравнивает шаблоны.
%Однако нет необходимости проверять на совпадение шаблоны, полученные из каждого элемента дерева разбора, и шаблон, полученного из элемента со старым форматированием.
Однако нет необходимости сравнивать шаблон, полученный из элемента со старым форматированием, с шаблоном, полученным из каждого элемента дерева разбора.
Количество вариантов для переформатирования можно сузить лишь для элементов того же типа.

Для уменьшения затрат времени при обходе дерева элементов создается список кандидатов на форматирование (элементы того же типа, что и элемент со старым форматированием). 
После этого из элементов созданного списка извлекаются шаблоны.
Если полученный шаблон совпадает с шаблоном элемента со старым форматированием, то происходит перепечатывание элемента и замена его в дереве разбора.
Время, затрачиваемое в этом случае на переформатирование, приведено в таблице~\ref{table:fast}.

%\begin{table}[h]
%\begin{tabular}{ l | p{4cm} | p{4cm} }
%  Размер файла (строк) & Формат. операторов ветвления (сек.) & Формат. методов %(сек.) \\ 
%  \hline
%  >7000         & 6.6   & 5.8 \\
%  1000..2000    & 0.72  & 1   \\
%  500..1000     & 0.125 & 0.67 \\
%  0..500        & 0.145 & 0.325 \\
%\end{tabular}
%\caption{Время переформатирования файлов II}
%\label{table:fast}
%\end{table}


\begin{table}[h]
\begin{tabular}{cc|c|c|c|c|}
\cline{2-5}
\multicolumn{1}{ c| }{ } & \multicolumn{2}{p{6cm} |}{Формат. операторов ветвления} & \multicolumn{2}{p{5cm} |}{Форматирование  методов} \\
\hline
\multicolumn{1}{ |c| }{Размер файла (строк)} & Кол-во & Время (сек.) & Кол-во & Время (сек.) \\ 
\hline
\multicolumn{1}{ |c| }{>7000}           & 520   & 6.6   & 660  & 5.8 \\
\multicolumn{1}{ |c| }{1000..2000}      & 145   & 0.72  & 90   & 1 \\
\multicolumn{1}{ |c| }{500..1000}       & 54    & 0.225 & 72   & 0.67 \\
\multicolumn{1}{ |c| }{0..500}          & 35    & 0.145 & 36   & 0.325 \\
\hline
\end{tabular}
\caption{Время переформатирования файлов II}
\label{table:fast}
\end{table}
\section{Заключение}

В данной работе описан подход к композициональному символьному исполнению без раскрутки. Была предложена концепция композициональной памяти с символьной адресацией. Был доказан некоторый набор свойств КСП, дающий основание для подхода в стиле систем переписывания, где символьные кучи могут сами выступать как символы. Это даёт возможность автоматически порождать уравнения на состояния, решения которых в точности отражают поведения функций, работающих с динамической памятью. Было показано как свести задачу решения уравнений на состояния к задаче проверки безопасности чистых функций второго порядка.

Данная работа нацелена на теоретические основания композиционального анализа динамической памяти. Мы оставляем апробацию этого подхода на будущее. Другим направлением будущих исследований может быть расширение нашего формализма на композициональный анализ параллельных программ.

\begin{thebibliography}{99}
  \bibitem{podkopaev:diploma1}
  А.В.Подкопаев. Полиномиальной сложности оптимальные принтер-комбинаторы с выбором.
  Дипломная работа, кафедра системного программирования, математико-механический факультет СПбГУ, 2014.

  \bibitem{stratego:xt}
  Stratego/XT. \url{http://strategoxt.org/Stratego/StrategoDocumentation}, 2015.

  \bibitem{intellij:plugdev}
  IntelliJ IDEA Plugin Development Guide. \url{https://confluence.jetbrains.com/display/IDEADEV/PluginDevelopment},
  2015.
\end{thebibliography}

%\usepackage{listings}  
%\usepackage{hyperref}
%\usepackage{verbments}
%\usepackage{minted}
%\usepackage{caption}
%\usepackage{fancybox}
%\newfontfamily{\cyrillicfonttt}{CMU Serif}

\title{Декларативное форматирование в режиме онлайн}

\titlerunning{Декларативное форматирование в режиме онлайн}

\author{Озерных Игорь Станиславович}

\authorrunning{И.С.Озерных}

\tocauthor{И.С.Озерных}
\institute{Санкт-Петербургский государственный университет\\
\email{igor.ozernykh@gmail.com}}

\maketitle             

\begin{abstract}
Аккуратное, соответствующие выбранному стилю оформление (форматирование)
программных текстов критично для их восприятия.
Существуют средства автоматического форматирования кода. Одним
из них является метод синтаксических шаблонов. Его преимуществом является декларативность, которая
достигается за счет задания стиля форматирования с помощью образца кода. Так, естественным ограничением
описанного метода является необходимость наличия достаточно полного образца на целевом языке.
Данная работа посвящена расширению границ применимости метода синтаксических шаблонов
за счет возможности задания шаблонов в режиме онлайн.
\end{abstract}

\section*{Введение}

Этап поддержки и сопровождения в жизненном цикле программного обеспечения может занимать до 90\% времени существования программного продукта. На этом этапе особенно важно понимание программного текста поддерживаемой системы, что зачастую бывает сложной задачей. Необходимым условием для лучшего понимания программного текста является его аккуратное оформление, отражающее структуру 
программы. Существуют различные стандарты оформления кода (стили кодирования)~--- наборы правил и соглашений, используемые при написании исходного кода на некотором языке программирования.
Рассмотрим их на примере стандарта кодирования, принятого в компании Google\footnote{\texttt{https://google-styleguide.googlecode.com/svn/trunk/cppguide.html}}, и стандарта, используемого в проекте GNU\footnote{\texttt{http://www.gnu.org/prep/standards/standards.html}} (см. рис.~\ref{codingstandards}) для языка C++.


\fvset{frame=lines,framesep=7pt}
\begin{figure}[ht]
\noindent\begin{minipage}{.5\textwidth}
    \lstinputlisting[language=Java]{Ozernykh/codes/exGoogleCC.txt}
\caption*{а) Стиль кодирования Google}    
\end{minipage}\hfill
\begin{minipage}{.5\textwidth}
    \lstinputlisting[language=Java]{Ozernykh/codes/exGNUCC.txt}
\caption*{б) Стиль кодирования GNU}    
\end{minipage}
\caption{Различные стили кодирования для языка С++}    
\label{codingstandards}
\end{figure}

Когда программист работает над проектом, он придерживается некоторого стиля кодирования. 
Если файлы проекта с одним стилем кодирования присоединяется к уже существующему проекту с другим стилем кодирования, то их форматирование нужно привести к тому же виду. 
Для решения этой задачи можно воспользоваться уже существующими средствами, например, форматтерами (программами, которые форматируют исходный текст) в IDE таких, как Eclipse, IntelliJ IDEA, Visual Studio и др. 
Однако в этом случае необходимо вручную задавать настройки форматирования.
Кроме того, количество принципиально разных стилей форматирования, которые можно задать с помощью данных настроек, невелико.

%Однако в этом случае необходимо иметь настройки форматирования, заданные программистом, но в целом, количество принципиально разных стилей форматировани, которые можно задать с помощью данных настроек, невелико. %причем эти настройки могут быть различными в разных IDE.
% * <Anton Podkopaev> 12:15:51 22 May 2015 UTC+0300:
% Нужно переработать это предложение.

Еще одним примером может послужить принтер-плагин для IntelliJ IDEA~\cite{podkopaev:diploma1}, который позволяет форматировать исходный код проекта по образцу. 
В качестве образца используется код из некоторого репозитория. 
Из него выделяются шаблоны форматирования для структур языка, которые применяются к структурам целевого кода. 
Под \textit{шаблоном} понимаются данные, сопоставление которых с элементом синтаксического дерева дает текстовое представление этого элемента (и его потомков). 
Например, на рис.~\ref{fig:ifTree} представлено дерево разбора для оператора ветвления. 


\begin{figure}[h]
	\centering
	\includegraphics[width=\textwidth]{Ozernykh/images/ifTree.jpg}
	\caption{Представление оператора ветвления в виде дерева разбора}
	\label{fig:ifTree}
\end{figure}

На рис.~\ref{fig:tmpltcodeintro}а изображен шаблон форматирования, который может быть применен к нему (несколько подвыражений, соответствующих выполнению условия, и одно подвыражение, соответствующее невыполненению условия).
На рис.~\ref{fig:tmpltcodeintro}б представлен результат применения этого шаблона к дереву разбора для оператора ветвления.

% \begin{figure}[h]
% 	\centering
% 	\lstinputlisting[language=c++]{codes/ifTmpltIntro.txt}
% 	\caption{Шаблон для оператора ветвления}
% 	\label{fig:ifTmpltIntro}
% \end{figure}

% На рис.~\ref{fig:ifCodeIntro} представлен результат применения этого шаблона к дереву разбора для оператора ветвления.

% \begin{figure}[h]
% 	\centering
% 	\lstinputlisting[language=c++]{codes/ifCodeIntro.txt}
% 	\caption{Представление дерева разбора при применении шаблона}
% 	\label{fig:ifCodeIntro}
% \end{figure}


\fvset{frame=lines,framesep=7pt}
\begin{figure}[ht]
\noindent\begin{minipage}{.5\textwidth}
    \lstinputlisting[language=Java]{Ozernykh/codes/ifTmpltIntro.txt}
\caption*{а) Шаблон для оператора\\
ветвления}    
\end{minipage}\hfill
\begin{minipage}{.5\textwidth}
    \lstinputlisting[language=Java]{Ozernykh/codes/ifCodeIntro.txt}
\caption*{б) Текст, полученный при применении шаблона к дереву разбора}    
\end{minipage}
\caption{Оператор ветвления и шаблон для него}    
\label{fig:tmpltcodeintro}
\end{figure}

Однако не всегда этот репозиторий существует. Кроме того, необходимо наличие заданного форматирования для всех структур языка.
% * <Anton Podkopaev> 12:19:02 22 May 2015 UTC+0300:
% 'Однако не всегда удобно иметь этот репозиторий под рукой.' --- поменять. Не про удобство надо говорить.

Шаблоны можно извлекать из уже существующего кода и применять их к другим элементам того же типа, но отформатированных иным способом. 
Например, пользователь меняет форматирование некоторой структуры языка (в частности, оператор ветвления), программа извлекает из полученного участка кода новый шаблон и применяет его к структурам того же типа, содержащим старое форматирование. 
Назовем такой способ задания шаблонов~--- форматирование в режиме
онлайн.
%картинки

Целью данной работы является расширение функциональности описанного 
принтер-плагина путем добавления возможности задания шаблонов в режиме онлайн. 

\section{Обзор предметной области}

Одним из подходов к заданию принтеров являются принтер-комбинаторы.
В этой главе приведен обзор классических принтер-комбинаторов, а также
принтер-комбинаторов с дополнительным оператором \emph{выбора}, которые
в рамках данной работы использованы для реализации принтеров, задаваемых
с помощью шаблонов. Для решения проблем эффективности
комбинаторов с выбором в данной работе был использован подход BURS.
Апробация подхода задания принтеров с помощью шаблонов
была произведена для языка Java, для которого есть средства форматирования,
встроенные в IDE.

\subsection{Принтер-комбинаторы}

Базовыми принтер-комбинаторными библиотеками являются библиотеки
Джона Хьюза\cite{hughes} и Филиппа Вадлера\cite{wadler}, которые
представляют собой функциональную переработку алгоритма
Оппена\cite{oppen}.
Ограничимся рассмотрением библиотеки Вадлера, так как
библиотека Хьюза обладает теми же свойствами, которые принципиальны в контексте
данной работы.

В библиотеке ключевым типом является документ.
Он представляет сущность, которая потом может
быть переведена в строковое представление алгоритмом принтера.
Основные конструкторы для составления документа таковы:
\begin{itemize}
  \item атомарная строка, которая печатается как есть;
  \item \emph{разделитель};
  \item последовательная композиция двух документов;
  \item набор связанных документов.
\end{itemize}

Определяющей особенностью данного подхода является то, что все разделители
в рамках одного набора могут быть совместно заменены алгоритмом принтера
на пробельный символ или на перевод строки. Выбор для каждого набора разделителей
основывается на том, что вывод должен поместиться в заданную ширину,
используя минимальное число строк.

Основная проблема такого подхода заключается в его слабой выразительной силе.
Документы, построенные по синтаксическому дереву печатаемой программы,
обрабатываются слишком единообразно, что иногда приводит к нежелательному результату.
Пусть, к примеру, нужно напечатать программу на языке
Python\footnote{\cd{http://python.org}}.
Между последовательными операторами в случае их печати на одной строчке необходимо
добавить дополнительный разделитель (``;''), иначе программа станет некорректной
(см. рис.~\ref{fig:seqEx}). Однако, описанные принтер-комбинаторы не предоставляют
возможности задать такое поведение.
Кроме того, с помощью таких принтер-комбинаторов невозможно выразить разные проектные СК,
так как они всегда печатают текст в одном стиле, который жестко ``зашит'' в их код.

\begin{figure}[h!]
	\centering
	\null\hfill
	\subfloat[Корректный код]{
		\centering
    \makebox[.4\textwidth] {
		  \lstinputlisting[language=Python]{codes/pythonCode.py}
    }
	}
	\null\hfill
	\subfloat[Некорректный код]{
		\centering
    \makebox[.4\textwidth] {
	  	\lstinputlisting[language=Python]{codes/pythonCodeBad.py}
    }
	}
	\hfill\null
	\caption{Пример работы принтер-комбинаторов для языка Python}
  \label{fig:seqEx}	
\end{figure}

Более подробный обзор библиотек Хьюза и Вадлера представлен в \cite{myCoursePaper}.

Большинство принтер-комбинаторных библиотек\cite{
swierstraChitil, swierstra04, peytonJones, kiselyov, chitil}
являются развитиями работ Хьюза и Вадлера. Так, в отличие от базовых, среди
них есть реализации с линейной сложностью обработки документа от его размера и
\emph{online} алгоритмы, которые не требуют просмотра всего документа
для начала печати его текстового представления. Но все они обладают тем же
интерфейсом, а значит в них также неразрешимы задачи, в которых требуется
задавать варианты текстовых представлений, которые отличаются не только
пробелами и переводами строк.

\subsection{Принтер-комбинаторы с выбором}

Существенно от описанных выше отличаются библиотеки, предоставляющие в своем
интерфейсе комбинатор \emph{выбора}, который позволяет задавать для одного поддерева
принципиально разные варианты раскладок.
Так, в работах~\cite{jongeEveryOccasion, jongeReengine} используется
оператор \lstinline[language = Haskell]{ALT},
но алгоритм принтера устроен так, что среди двух альтернатив выбирается первая, если она
помещается в заданную ширину, и вторая --- иначе,
что не дает оптимальный результат на выходе.

Оптимальные принтер-комбинаторы с выбором были впервые представлены в работе~\cite{swierstra}.
Их реализация является частью Utrecht Tools
Library\footnote{\cd{http://www.cs.uu.nl/wiki/HUT/WebHome}}
(практическая реализация несколько изменена по
отношению к той, что описана в статье, но отличие несущественно).
В данном подходе текст строится из блоков прямоугольной формы с возможно неполной последней
строчкой (см. рис. \ref{fig:basicFormat}). В реализации на Haskell блоки представляются
структурой \lstinline[language = Haskell]{Format}:
\begin{lstlisting}
    data Format = Elem { height        :: Int
                       , lastLineWidth :: Int
                       , width         :: Int
                       , txtstr        :: Int -> String -> String
                       }
\end{lstlisting}

Первые три поля структуры определяют геометрические размеры блока, а последнее --- функция,
которая используется для преобразования блока в текст. Здесь используется функция, а не
просто строчка, чтобы можно было преобразовывать вложенные блоки за линейное время. Первый
аргумент \lstinline[language = Haskell]{txtstr} задает сдвиг блока.

\begin{figure}
  \centering
  \subfloat[Блок текста]{
    \raisebox{5mm}{
      \centering
      \vspace{0pt}
      \tikz[scale = 2.0]{
        \draw (0,0) -- (1,0) -- (1,0.2) -- (2,0.2) -- (2,1) -- (0, 1) -- cycle;
      }
    }
    \label{fig:basicFormat}
  }
  ~
  \subfloat[Горизонтальная композиция]{
    \centering
    \tikz[scale = 2.0]{
       \draw (0,0) -- (1,0) -- (1,0.2) -- (2,0.2) -- (2,1) -- (0, 1) -- cycle;
       \draw (1.1,-0.9) -- (2.1,-0.9) -- (2.1,-0.7) -- (3.1,-0.7) -- (3.1,0.1) -- (1.1,0.1) -- cycle;
     }
     \label{fig:beside}
  }
  %\hfill
  ~
  \subfloat[Вертикальная композиция]{
    \makebox[.28\textwidth] {
    \centering
    \tikz[scale = 2.0]{
       \draw (0,0) -- (1,0) -- (1,0.2) -- (2,0.2) -- (2,1) -- (0, 1) -- cycle;
       \draw (0,-1.1) -- (1,-1.1) -- (1,-0.9) -- (2,-0.9) -- (2,-0.1) -- (0,-0.1) -- cycle;
    }
    \label{fig:above}
    }
  }
  \caption{Примитивы блока текста}
  \label{fig:basicConcat}
\end{figure}

Для работы с блоками текста используется следующие четыре примитива:
\begin{lstlisting}
    s2fmt     :: String -> Format
    indentFmt :: Int -> Format -> Format
    aboveFmt  :: Format -> Format -> Format
    besideFmt :: Format -> Format -> Format
\end{lstlisting}

Функция \lstinline[language = Haskell]{s2fmt} создает блок текста,
состоящий из одной строчки; \lstinline[language = Haskell]{indentFmt} по блоку создает новый,
сдвинутый на заданное число позиций. Действие примитивов композиции
\lstinline[language = Haskell]{besideFmt} и \lstinline[language = Haskell]{aboveFmt}
показано на рис.~\ref{fig:beside} и \ref{fig:above} соответственно.

Также, как и в библиотеках без комбинатора выбора, в работе~\cite{swierstra} используется понятие
\emph{документа}. Здесь документ можно рассматривать как множество возможных раскладок
(набор \lstinline[language = Haskell]{Format}-элементов). Документы описываются типом
\lstinline[language = Haskell]{Doc}, экземпляры которого на этапе построения представляют собой
деревья применений приведенных ниже комбинаторов, а на этапе обработки алгоритмом
принтера им в соответствие ставится итоговое множество вариантов текстовых представлений.

Документ конструируется с помощью следующих комбинаторов, которые симметричны
примитивам построения блоков текста:
\begin{lstlisting}
    text   :: String -> Doc
    indent :: Int -> Doc -> Doc
    beside :: Doc -> Doc -> Doc
    above  :: Doc -> Doc -> Doc
\end{lstlisting}

В дополнение появляется пятый комбинатор для документов:
\begin{lstlisting}
    choice :: Doc -> Doc -> Doc
\end{lstlisting}

Этот комбинатор и является тем самым комбинатором выбора. Он представляет
\emph{объединение} множеств раскладок документов, которые были переданы как
аргументы комбинатора. Заметим, что только этот комбинатор может произвести
документ с несколькими раскладками из одновариантных аргументов.

Оригинальная реализация существенно опирается на ленивые вычисления. В \cite{swierstra}
множество вариантов, соответствующее экземпляру \lstinline[language = Haskell]{Doc},
представляется ленивым списком всех возможных раскладок, удовлетворяющих ограничению
на максимальную ширину. Этот список отсортирован в порядке ``ухудшения''
раскладок, то есть в голове списка лежит ``лучшая'', в терминах оптимальности,
раскладка из возможных при заданной ширине. В случае
\lstinline{beside}- и
\lstinline{above}-композиций
документов полная (без учета ленивости) сложность вычисления списка нового документа
составляет $O(n \times m)$, где $n$ и $m$ --- размеры списков, соответствующих 
соединяемым документам. Размер нового списка также порядка $n \times m$.
Он не обязательно точно равен $n \times m$, так как некоторые из полученных
представлений могут не подпадать под ограничение на ширину, соответственно их можно
отбросить на этапе построения нового списка.

Выбор лучшего представления для самого верхнеуровнего документа происходит
просто --- нужно из соответствующего списка взять первый элемент.
Это создает впечатление, что общее число операций,
благодаря ленивым вычислениям, существенно уменьшается.
Но это не так из-за реализации обработки документа,
построенного с помощью комбинатора \lstinline[language = Haskell]{beside}, которая
вынуждает полное вычисление дочерних списков. Более того, из-за свойств отношения
порядка, построенного по критерию оптимальности, в рамках данной модели списков
принципиально нельзя построить ленивую обработку комбинатора
\lstinline[language = Haskell]{beside}.
Так, выбор оптимальной раскладки документа в
\cite{swierstra} имеет в худшем случае экспоненциальную сложность от числа
комбинаторов, использованных при его построении.
В \cite{swiComb} приведены оптимизации для данной модели, но они
не улучшают асимптотику решения для деревьев в общем случае.

\subsection{BURS}

Bottom-Up Rewrite System (BURS)\cite{burs} --- это метод динамического
программирования на деревьях, изначально появившийся в контексте задачи выбора
инструкций для генерации машинного кода. Основой BURS является
регулярная грамматика с древовидными правилами\cite{tata}, для которых задана
стоимость применения подстановки,
то есть грамматика со следующим набором правил: 

$$
\begin{array}{rcll}
  N &:& \alpha& [c]\\
  N &:& \alpha\; (K_1,\dots,K_n)& [c]
\end{array}
$$

Здесь $N, K_i$ это нетерминалы, $\alpha$ --- терминал,
$c$ --- функция стоимости, заданная для каждого правила.
Как и для обычной линейной грамматики, вводится стартовый
нетерминал S. Считается, что терминальное дерево выводится в
данной грамматике, если его можно получить с помощью правил
подстановки из одноузлового дерева $S$.
Каждая подстановка заменяет нетерминал $N$, находящийся в листе дерева, на дерево 
$\alpha\;(K_1,\dots,K_n)$, если в грамматике есть правило
$N:\alpha\;(K_1,\dots,K_n)$. 
Для каждой подстановки вычисляется стоимость ее применения с помощью
функции стоимости $c$.
Аргументами функции могут служить терминальная метка $\alpha$ и стоимости
вывода поддеревьев.
Задачей, которую решает BURS, является поиск вывода наименьшей стоимости
для заданного дерева по заданной грамматике.
Такой вывод может быть найден двухпроходным алгоритмом.

Первый проход (\emph{пометка}) обрабатывает дерево снизу вверх и вычисляет
для каждого узла набор троек $(K,\;R,\;c)$, где $K$ --- нетерминал, из которого может быть
выведено поддерево с корнем в обрабатываемом узле,
$R$ --- первое правило, которое используется для вывода минимальной стоимости из $K$,
$c$ --- стоимость такого вывода.
Процесс пометки происходит следующим образом:

\begin{itemize}
\item для листовой вершины, помеченной терминалом $\alpha$, в множество троек
этого вершины добавляется $(K,\;R,\;c\:(\alpha))$ для каждого правила $R=K:\;\alpha\;[c]$;
\item для промежуточной вершины, помеченной терминалом $\alpha$,
с непосредственными поддеревьями $v_1,\dots,v_n$
в множество добавляется тройка $(K,\;R,\;c\:(\alpha,c_1,\dots,c_n))$ для каждого правила
$R=K:\;\alpha\;(K_1,\dots,K_n)\;[c]$, где $(K_i,\;R_i,\;c_i)$ входит в множество троек для
$v_i$; если есть несколько правил вывода из нетерминала $K$, то выбирается правило,
минимизирующее стоимость вывода.
\end{itemize}

Второй проход (\emph{свертка}) просматривает дерево сверху вниз, используя сделанные пометки.
Первое правило из минимального вывода определяется тройкой $(S,\;R,\;c)$ для корневого узла
(если такой тройки нет, то вывод из $S$ невозможен).
Это правило однозначно определяет нетерминалы $K_i$ для каждого непосредственного поддерева
и процесс повторяется.

Для пометки дерева потенциально необходимо каждое правило грамматики применить к каждому узлу.
При фиксированной грамматике алгоритмическая сложность первого прохода --- $O\:(|R|)$,
где $|R|$ --- количество правил
(размер множества троек для каждого узла ограничен числом нетерминалов, которое не больше,
чем количество правил). Свертка также имеет линейную сложность.


\newpage
\subsection{Средства форматирования кода в IDE}

Интегрированные среды разработки программного обеспечения
(IDE) предоставляют средства
для форматирования программного кода. Рассмотрим их на примере двух Java IDE ---
IntelliJ IDEA\footnote{\cd{http://jetbrains.com/idea/}} и
Eclipse\footnote{\cd{http://eclipse.org}}.
Для задания требуемого СК используется широкий набор настроек \emph{форматтера},
подпрограммы IDE, отвечающей за форматирование исходных текстов.
Примерами таких настроек являются:
\begin{itemize}
  \item помещать ли фигурную скобку на той же строке, что и
    предыдущее выражение;
  \item форматирование списков --- всегда на одной строчке, переводить строчку после
    каждого элемента или печатать на одной строке,
    пока строчка меньше заданной рекомендуемой ширины.
%   \item и т.д.
\end{itemize}
На рис. ~\ref{fig:ideaFormatter} и~\ref{fig:eclipseFormatter} приведены диалоги задания
параметров форматирования в IntelliJ IDEA и Eclipse соответственно.

\begin{figure}[p]
	\centering
	\includegraphics[width=\textwidth]{ideaFormatter}
	\caption{Окно настройки форматтера IntelliJ IDEA}
	\label{fig:ideaFormatter}
\end{figure}

\begin{figure}[p]
	\centering
	\includegraphics[width=\textwidth]{eclipseFormatter}
	\caption{Окно настройки форматтера Eclipse}
	\label{fig:eclipseFormatter}
\end{figure}

У встроенных форматтеров есть следующие особенности.
Во-первых, это невозможность выразить
нестандартный СК, так как настройки форматтеров, несмотря на их большое
количество,
дают ограниченную вариативность для текстовых представлений форматируемых
программ. В целом, количество принципиально разных
стилей форматирования, которые можно задать с помощью данных настроек,
невелико.
Во-вторых, в случае, если надо
придерживаться СК уже существующего проекта, то по коду этого проекта необходимо
вручную задать все настройки форматтера.
Причем наборы настроек в разных IDE не совпадают, что увеличивает сложность
поддержки единого СК, если разработчики используют отличные от друг друга IDE.
Для решения данной проблемы существуют плагины, позволяющие использовать
внешние фоматтеры\footnote{\cd{http://plugins.jetbrains.com/plugin/6546}}.
Кроме того, есть средства, позволяющие экспортировать настройки форматтера в
XML-файл для их использования в другой
IDE\footnote{\cd{http://blog.jetbrains.com/idea/2014/01/intellij-idea-13-importing-code-formatter-settings-from-eclipse/}}.

Важным достоинством описанных форматтеров является малое время работы.
Это достигается в том числе и за счет того, что часто форматируется не весь
файл, а только его часть (см. \cite{eclipse}),
и из-за детерминированности представлений узлов синтаксического дерева,
которая следует из специфики настроек форматтера.
Однако форматирование целого большого проекта занимает существенное время.
Например, обработка кода IntelliJ IDEA Community Edition занимает более часа
у встроенного в IDEA форматтера.


\section{Реализация}
В процессе работы над проектом было выделено две основных задачи: реализация
взаимодействия с пользователем и расширение функциональности для переформатирования программного текста с фильтрацией шаблонов.
%диаграмму что ли нарисавать?
Предполагаемый процесс взаимодействия с пользователем: пользователь выделяет
участок программного кода, в котором будут производиться изменение, и вызывает 
действие (Action), которое извлекает шаблон из выделенного участка
кода. Затем пользователь меняет форматирование этого же участка, выделяет его
снова и вызывает действие, которое извлекает новый шаблон форматирования и 
сохраняет его.

После этого происходит перепечатывание элементов (PsiElement), имеющих старое
форматирование, в файле, в котором пользователь менял программный текст.

\subsection{Переформатирование}
\subsubsection{Извлечение элемента для форматирования}
% тут будет диаграмма:
% 1) Получаем текст
% 2) Корректируем отступы + пример
% 3) Извлечение элемента (для того чтобы знать тип) -- вся рутина
% 4) Склеить текст и элемент
В первую очередь необходимо получить элемент, форматирование которого пользователь хочет изменить.
% Где-то надо ссылку на диаграмму, наверное, тут
Весь этот процесс изображен на рис.~\ref{fig:getElemDiag}.

\begin{figure}[h]
	\centering
	\includegraphics[width=\textwidth]{images/getElemDiag.jpg}
	\caption{Диаграмма этапов извлечения элемента из текста}
	\label{fig:getElemDiag}
\end{figure}

На первом этапе с помощью функций, предоставляемых IntelliJ IDEA API, можно получить текст элемента. 
Однако этот текст не всегда корректный, так как может содержать в себе лишние отступы.
Например, на рис. \ref{fig:offsetExample} изображены методы doSomething() и doAnotherThing(). 
Они имеют отступ относительно оператора ветвления, равный четырем пробелам.
Кроме того, они имеют отступ относительно метода foo(), равный восьми пробелам. 
Чтобы шаблон был корректным, на втором этапе убираются лишние пробелы (в данном случае четыре).
Далее получаем из текста сам элемент. 
Это необходимо, чтобы узнать его тип. %ссылку на дерево PsiElement'ов.
Затем (на четвертом этапе) из корректного текста и полученного элемента создается новый элемент того же типа, но с корректными отступами.

Все это необходимо проделать, так как IntelliJ IDEA API не предоставляет функционала по изменению текста элемента явным образом. 

\begin{figure}[h]
	\centering
	\lstinputlisting[language=C++]{codes/offsetExample.txt}
	\caption{Общий отступ элементов}
	\label{fig:offsetExample}
\end{figure}

Аналогичные действия необходимо повторить для получения элемента из измененного
пользователем текста. 
Из этого элемента извлекается шаблон и сохраняется принтером для дальнейшего использования.

%\subsubsection{Извлечение шаблона}
%Шаблон, как и прежде, извлекается методами класса \lstinline[language=C++]{PsiElementComponent}.

%\subsubsection{Добавление шаблона}
%В связи с тем, что возникла необходимость добавления одного шаблона, класс %Printer обзавелся публичным методом, реализующим этот процесс. Остальные
%действия аналогичны тем при заполнении списка шаблонов из файла.

\subsubsection{Сравнение с шаблоном}
Как уже было отмечено в обзоре, при перепечатывании принтер обходит все элементы (PsiElement) дерева разбора и применяет к ним новые шаблоны. 
В рамках реализуемого подхода происходит перепечатывание лишь тех элементов, которые имели старое форматирование.
Это необходимо, если хочется поддерживать два и более различных варианта форматирования элементов некоторого типа. 
На рис. \ref{templates} изображены три блока оператора ветвления. 
Без фильтрации шаблонов все три блока будут изменены и получат одинаковое форматирование. %рис. сделать 
Однако первые два блока и так легко читаемы и понятны. 
Третий блок, будучи однострочным, уже не будет восприниматься так же хорошо.
Кроме того, существуют ограничения на ширину вывода, то есть в одной строке не можеть быть символов больше некоторой наперед заданной величины.

%\begin{figure}[t]
%    \centering
%    \lstinputlisting[language=C++]{codes/sameTmpltElems.txt}
%    \caption{Шаблоны}
%    \label{fig:sameTmpltElems}
%\end{figure}
%добавить вторую картинку как стало бы
\fvset{frame=lines,framesep=7pt}
\begin{figure}[ht]
\noindent\begin{minipage}{.5\textwidth}
    \lstinputlisting[language=C++]{codes/sameTmpltElems.txt}
\caption*{а) До переформатирования}    
\end{minipage}\hfill
\begin{minipage}{.5\textwidth}
    \lstinputlisting[language=C++]{codes/sameTmpltElems2.txt}
\caption*{б) После переформатирования}    
\end{minipage}
\caption{Переформатирование элементов одного типа без фильтрации шаблонов} 
\label{templates}
\end{figure}


Чтобы сделать настройку форматирования более гибкой и избавиться от подобного рода проблем, необходимо производить сравнение шаблонов текущего элемента дерева разбора и элемента со старым форматированием. 
Переформатировать элемент -- только в случае совпадения шаблонов.
Шаблоны совпадают, если совпадет их текстовое представление: шаблон записывается в виде строки, которая содержит необходимые элементы: ключевые слова, пробелы, переносы строк, скобки, а так же метки для поддеревьев, в которых указывается дополнительная информация, например, количество выражений в поддереве (одно или несколько).
В случае оператора ветвления (рис.~\ref{iftmpltcode}а) его поддеревьями являются: условие (condition), ветка, соответствующая выполнению условия (then block) и содержащая несколько подвыражений, ветка, соответствующая невыполнению условия (else block) и содержащая одно подвыражение. %ifTmplt ifTmpltCode
Соответствующий ему шаблон изображен на рисунке~\ref{iftmpltcode}б.
\fvset{frame=lines,framesep=7pt}
\begin{figure}[ht]
\noindent\begin{minipage}{.5\textwidth}
    \lstinputlisting[language=C++]{codes/ifTmpltCode.txt}
\caption*{а) Оператор ветвления}    
\end{minipage}\hfill
\begin{minipage}{.5\textwidth}
    \lstinputlisting[language=C++]{codes/ifTmplt.txt}
\caption*{б) Шаблон для оператора ветвления}    
\end{minipage}
\caption{Оператор ветвления и соответствующий ему шаблон}    
\label{iftmpltcode}
\end{figure}

%\begin{figure}[t]
%    \centering
%    \lstinputlisting[language=C++]{codes/sameTmpltElems.txt}
%    \caption{Шаблоны}
%    \label{fig:sameTmpltElems}
%\end{figure}
%добавить вторую картинку как стало бы

\subsection{Оценка производительности}
Описанный выше способ  давал очень низкую производительность. 
В таблице~\ref{table:slow} приведено время, затрачиваемое принтером на переформатирование файлов различной длины. 
Измерения производились на примере двух блоков: оператора ветвления и метода.

\begin{table}[h]
\begin{tabular}{cc|c|c|c|c|}
\cline{2-5}
\multicolumn{1}{ c| }{ } & \multicolumn{2}{p{6cm} |}{Формат. операторов ветвления} & \multicolumn{2}{p{5cm} |}{Форматирование методов} \\
\hline
\multicolumn{1}{ |c| }{Размер файла (строк)} & Кол-во & Время (сек.) & Кол-во & Время (сек.) \\ 
\hline
\multicolumn{1}{ |c| }{>7000}           & 520   & 35    & 660  & 58 \\
\multicolumn{1}{ |c| }{1000..2000}       & 145   & 2.7   & 90   & 4.4 \\
\multicolumn{1}{ |c| }{500..1000}       & 54    & 0.73  & 72   & 1.34 \\
\multicolumn{1}{ |c| }{0..500}          & 35    & 0.2   & 36   & 0.67 \\
\hline
\end{tabular}
\caption{Время переформатирования файлов I}
\label{table:slow}
\end{table}


Причина столь низкой производительности кроется в том, что для переформатирования элементов используется тот же способ, что и при форматировании по образцу:
принтер обходит все дерево разбора и сравнивает шаблоны.
%Однако нет необходимости проверять на совпадение шаблоны, полученные из каждого элемента дерева разбора, и шаблон, полученного из элемента со старым форматированием.
Однако нет необходимости сравнивать шаблон, полученный из элемента со старым форматированием, с шаблоном, полученным из каждого элемента дерева разбора.
Количество вариантов для переформатирования можно сузить лишь для элементов того же типа.

Для уменьшения затрат времени при обходе дерева элементов создается список кандидатов на форматирование (элементы того же типа, что и элемент со старым форматированием). 
После этого из элементов созданного списка извлекаются шаблоны.
Если полученный шаблон совпадает с шаблоном элемента со старым форматированием, то происходит перепечатывание элемента и замена его в дереве разбора.
Время, затрачиваемое в этом случае на переформатирование, приведено в таблице~\ref{table:fast}.

%\begin{table}[h]
%\begin{tabular}{ l | p{4cm} | p{4cm} }
%  Размер файла (строк) & Формат. операторов ветвления (сек.) & Формат. методов %(сек.) \\ 
%  \hline
%  >7000         & 6.6   & 5.8 \\
%  1000..2000    & 0.72  & 1   \\
%  500..1000     & 0.125 & 0.67 \\
%  0..500        & 0.145 & 0.325 \\
%\end{tabular}
%\caption{Время переформатирования файлов II}
%\label{table:fast}
%\end{table}


\begin{table}[h]
\begin{tabular}{cc|c|c|c|c|}
\cline{2-5}
\multicolumn{1}{ c| }{ } & \multicolumn{2}{p{6cm} |}{Формат. операторов ветвления} & \multicolumn{2}{p{5cm} |}{Форматирование  методов} \\
\hline
\multicolumn{1}{ |c| }{Размер файла (строк)} & Кол-во & Время (сек.) & Кол-во & Время (сек.) \\ 
\hline
\multicolumn{1}{ |c| }{>7000}           & 520   & 6.6   & 660  & 5.8 \\
\multicolumn{1}{ |c| }{1000..2000}      & 145   & 0.72  & 90   & 1 \\
\multicolumn{1}{ |c| }{500..1000}       & 54    & 0.225 & 72   & 0.67 \\
\multicolumn{1}{ |c| }{0..500}          & 35    & 0.145 & 36   & 0.325 \\
\hline
\end{tabular}
\caption{Время переформатирования файлов II}
\label{table:fast}
\end{table}
\section{Заключение}

В данной работе описан подход к композициональному символьному исполнению без раскрутки. Была предложена концепция композициональной памяти с символьной адресацией. Был доказан некоторый набор свойств КСП, дающий основание для подхода в стиле систем переписывания, где символьные кучи могут сами выступать как символы. Это даёт возможность автоматически порождать уравнения на состояния, решения которых в точности отражают поведения функций, работающих с динамической памятью. Было показано как свести задачу решения уравнений на состояния к задаче проверки безопасности чистых функций второго порядка.

Данная работа нацелена на теоретические основания композиционального анализа динамической памяти. Мы оставляем апробацию этого подхода на будущее. Другим направлением будущих исследований может быть расширение нашего формализма на композициональный анализ параллельных программ.

\begin{thebibliography}{99}
  \bibitem{podkopaev:diploma1}
  А.В.Подкопаев. Полиномиальной сложности оптимальные принтер-комбинаторы с выбором.
  Дипломная работа, кафедра системного программирования, математико-механический факультет СПбГУ, 2014.

  \bibitem{stratego:xt}
  Stratego/XT. \url{http://strategoxt.org/Stratego/StrategoDocumentation}, 2015.

  \bibitem{intellij:plugdev}
  IntelliJ IDEA Plugin Development Guide. \url{https://confluence.jetbrains.com/display/IDEADEV/PluginDevelopment},
  2015.
\end{thebibliography}

%\usepackage{listings}  
%\usepackage{hyperref}
%\usepackage{verbments}
%\usepackage{minted}
%\usepackage{caption}
%\usepackage{fancybox}
%\newfontfamily{\cyrillicfonttt}{CMU Serif}

\title{Декларативное форматирование в режиме онлайн}

\titlerunning{Декларативное форматирование в режиме онлайн}

\author{Озерных Игорь Станиславович}

\authorrunning{И.С.Озерных}

\tocauthor{И.С.Озерных}
\institute{Санкт-Петербургский государственный университет\\
\email{igor.ozernykh@gmail.com}}

\maketitle             

\begin{abstract}
Аккуратное, соответствующие выбранному стилю оформление (форматирование)
программных текстов критично для их восприятия.
Существуют средства автоматического форматирования кода. Одним
из них является метод синтаксических шаблонов. Его преимуществом является декларативность, которая
достигается за счет задания стиля форматирования с помощью образца кода. Так, естественным ограничением
описанного метода является необходимость наличия достаточно полного образца на целевом языке.
Данная работа посвящена расширению границ применимости метода синтаксических шаблонов
за счет возможности задания шаблонов в режиме онлайн.
\end{abstract}

\section*{Введение}

Этап поддержки и сопровождения в жизненном цикле программного обеспечения может занимать до 90\% времени существования программного продукта. На этом этапе особенно важно понимание программного текста поддерживаемой системы, что зачастую бывает сложной задачей. Необходимым условием для лучшего понимания программного текста является его аккуратное оформление, отражающее структуру 
программы. Существуют различные стандарты оформления кода (стили кодирования)~--- наборы правил и соглашений, используемые при написании исходного кода на некотором языке программирования.
Рассмотрим их на примере стандарта кодирования, принятого в компании Google\footnote{\texttt{https://google-styleguide.googlecode.com/svn/trunk/cppguide.html}}, и стандарта, используемого в проекте GNU\footnote{\texttt{http://www.gnu.org/prep/standards/standards.html}} (см. рис.~\ref{codingstandards}) для языка C++.


\fvset{frame=lines,framesep=7pt}
\begin{figure}[ht]
\noindent\begin{minipage}{.5\textwidth}
    \lstinputlisting[language=Java]{Ozernykh/codes/exGoogleCC.txt}
\caption*{а) Стиль кодирования Google}    
\end{minipage}\hfill
\begin{minipage}{.5\textwidth}
    \lstinputlisting[language=Java]{Ozernykh/codes/exGNUCC.txt}
\caption*{б) Стиль кодирования GNU}    
\end{minipage}
\caption{Различные стили кодирования для языка С++}    
\label{codingstandards}
\end{figure}

Когда программист работает над проектом, он придерживается некоторого стиля кодирования. 
Если файлы проекта с одним стилем кодирования присоединяется к уже существующему проекту с другим стилем кодирования, то их форматирование нужно привести к тому же виду. 
Для решения этой задачи можно воспользоваться уже существующими средствами, например, форматтерами (программами, которые форматируют исходный текст) в IDE таких, как Eclipse, IntelliJ IDEA, Visual Studio и др. 
Однако в этом случае необходимо вручную задавать настройки форматирования.
Кроме того, количество принципиально разных стилей форматирования, которые можно задать с помощью данных настроек, невелико.

%Однако в этом случае необходимо иметь настройки форматирования, заданные программистом, но в целом, количество принципиально разных стилей форматировани, которые можно задать с помощью данных настроек, невелико. %причем эти настройки могут быть различными в разных IDE.
% * <Anton Podkopaev> 12:15:51 22 May 2015 UTC+0300:
% Нужно переработать это предложение.

Еще одним примером может послужить принтер-плагин для IntelliJ IDEA~\cite{podkopaev:diploma1}, который позволяет форматировать исходный код проекта по образцу. 
В качестве образца используется код из некоторого репозитория. 
Из него выделяются шаблоны форматирования для структур языка, которые применяются к структурам целевого кода. 
Под \textit{шаблоном} понимаются данные, сопоставление которых с элементом синтаксического дерева дает текстовое представление этого элемента (и его потомков). 
Например, на рис.~\ref{fig:ifTree} представлено дерево разбора для оператора ветвления. 


\begin{figure}[h]
	\centering
	\includegraphics[width=\textwidth]{Ozernykh/images/ifTree.jpg}
	\caption{Представление оператора ветвления в виде дерева разбора}
	\label{fig:ifTree}
\end{figure}

На рис.~\ref{fig:tmpltcodeintro}а изображен шаблон форматирования, который может быть применен к нему (несколько подвыражений, соответствующих выполнению условия, и одно подвыражение, соответствующее невыполненению условия).
На рис.~\ref{fig:tmpltcodeintro}б представлен результат применения этого шаблона к дереву разбора для оператора ветвления.

% \begin{figure}[h]
% 	\centering
% 	\lstinputlisting[language=c++]{codes/ifTmpltIntro.txt}
% 	\caption{Шаблон для оператора ветвления}
% 	\label{fig:ifTmpltIntro}
% \end{figure}

% На рис.~\ref{fig:ifCodeIntro} представлен результат применения этого шаблона к дереву разбора для оператора ветвления.

% \begin{figure}[h]
% 	\centering
% 	\lstinputlisting[language=c++]{codes/ifCodeIntro.txt}
% 	\caption{Представление дерева разбора при применении шаблона}
% 	\label{fig:ifCodeIntro}
% \end{figure}


\fvset{frame=lines,framesep=7pt}
\begin{figure}[ht]
\noindent\begin{minipage}{.5\textwidth}
    \lstinputlisting[language=Java]{Ozernykh/codes/ifTmpltIntro.txt}
\caption*{а) Шаблон для оператора\\
ветвления}    
\end{minipage}\hfill
\begin{minipage}{.5\textwidth}
    \lstinputlisting[language=Java]{Ozernykh/codes/ifCodeIntro.txt}
\caption*{б) Текст, полученный при применении шаблона к дереву разбора}    
\end{minipage}
\caption{Оператор ветвления и шаблон для него}    
\label{fig:tmpltcodeintro}
\end{figure}

Однако не всегда этот репозиторий существует. Кроме того, необходимо наличие заданного форматирования для всех структур языка.
% * <Anton Podkopaev> 12:19:02 22 May 2015 UTC+0300:
% 'Однако не всегда удобно иметь этот репозиторий под рукой.' --- поменять. Не про удобство надо говорить.

Шаблоны можно извлекать из уже существующего кода и применять их к другим элементам того же типа, но отформатированных иным способом. 
Например, пользователь меняет форматирование некоторой структуры языка (в частности, оператор ветвления), программа извлекает из полученного участка кода новый шаблон и применяет его к структурам того же типа, содержащим старое форматирование. 
Назовем такой способ задания шаблонов~--- форматирование в режиме
онлайн.
%картинки

Целью данной работы является расширение функциональности описанного 
принтер-плагина путем добавления возможности задания шаблонов в режиме онлайн. 

\section{Обзор предметной области}

Одним из подходов к заданию принтеров являются принтер-комбинаторы.
В этой главе приведен обзор классических принтер-комбинаторов, а также
принтер-комбинаторов с дополнительным оператором \emph{выбора}, которые
в рамках данной работы использованы для реализации принтеров, задаваемых
с помощью шаблонов. Для решения проблем эффективности
комбинаторов с выбором в данной работе был использован подход BURS.
Апробация подхода задания принтеров с помощью шаблонов
была произведена для языка Java, для которого есть средства форматирования,
встроенные в IDE.

\subsection{Принтер-комбинаторы}

Базовыми принтер-комбинаторными библиотеками являются библиотеки
Джона Хьюза\cite{hughes} и Филиппа Вадлера\cite{wadler}, которые
представляют собой функциональную переработку алгоритма
Оппена\cite{oppen}.
Ограничимся рассмотрением библиотеки Вадлера, так как
библиотека Хьюза обладает теми же свойствами, которые принципиальны в контексте
данной работы.

В библиотеке ключевым типом является документ.
Он представляет сущность, которая потом может
быть переведена в строковое представление алгоритмом принтера.
Основные конструкторы для составления документа таковы:
\begin{itemize}
  \item атомарная строка, которая печатается как есть;
  \item \emph{разделитель};
  \item последовательная композиция двух документов;
  \item набор связанных документов.
\end{itemize}

Определяющей особенностью данного подхода является то, что все разделители
в рамках одного набора могут быть совместно заменены алгоритмом принтера
на пробельный символ или на перевод строки. Выбор для каждого набора разделителей
основывается на том, что вывод должен поместиться в заданную ширину,
используя минимальное число строк.

Основная проблема такого подхода заключается в его слабой выразительной силе.
Документы, построенные по синтаксическому дереву печатаемой программы,
обрабатываются слишком единообразно, что иногда приводит к нежелательному результату.
Пусть, к примеру, нужно напечатать программу на языке
Python\footnote{\cd{http://python.org}}.
Между последовательными операторами в случае их печати на одной строчке необходимо
добавить дополнительный разделитель (``;''), иначе программа станет некорректной
(см. рис.~\ref{fig:seqEx}). Однако, описанные принтер-комбинаторы не предоставляют
возможности задать такое поведение.
Кроме того, с помощью таких принтер-комбинаторов невозможно выразить разные проектные СК,
так как они всегда печатают текст в одном стиле, который жестко ``зашит'' в их код.

\begin{figure}[h!]
	\centering
	\null\hfill
	\subfloat[Корректный код]{
		\centering
    \makebox[.4\textwidth] {
		  \lstinputlisting[language=Python]{codes/pythonCode.py}
    }
	}
	\null\hfill
	\subfloat[Некорректный код]{
		\centering
    \makebox[.4\textwidth] {
	  	\lstinputlisting[language=Python]{codes/pythonCodeBad.py}
    }
	}
	\hfill\null
	\caption{Пример работы принтер-комбинаторов для языка Python}
  \label{fig:seqEx}	
\end{figure}

Более подробный обзор библиотек Хьюза и Вадлера представлен в \cite{myCoursePaper}.

Большинство принтер-комбинаторных библиотек\cite{
swierstraChitil, swierstra04, peytonJones, kiselyov, chitil}
являются развитиями работ Хьюза и Вадлера. Так, в отличие от базовых, среди
них есть реализации с линейной сложностью обработки документа от его размера и
\emph{online} алгоритмы, которые не требуют просмотра всего документа
для начала печати его текстового представления. Но все они обладают тем же
интерфейсом, а значит в них также неразрешимы задачи, в которых требуется
задавать варианты текстовых представлений, которые отличаются не только
пробелами и переводами строк.

\subsection{Принтер-комбинаторы с выбором}

Существенно от описанных выше отличаются библиотеки, предоставляющие в своем
интерфейсе комбинатор \emph{выбора}, который позволяет задавать для одного поддерева
принципиально разные варианты раскладок.
Так, в работах~\cite{jongeEveryOccasion, jongeReengine} используется
оператор \lstinline[language = Haskell]{ALT},
но алгоритм принтера устроен так, что среди двух альтернатив выбирается первая, если она
помещается в заданную ширину, и вторая --- иначе,
что не дает оптимальный результат на выходе.

Оптимальные принтер-комбинаторы с выбором были впервые представлены в работе~\cite{swierstra}.
Их реализация является частью Utrecht Tools
Library\footnote{\cd{http://www.cs.uu.nl/wiki/HUT/WebHome}}
(практическая реализация несколько изменена по
отношению к той, что описана в статье, но отличие несущественно).
В данном подходе текст строится из блоков прямоугольной формы с возможно неполной последней
строчкой (см. рис. \ref{fig:basicFormat}). В реализации на Haskell блоки представляются
структурой \lstinline[language = Haskell]{Format}:
\begin{lstlisting}
    data Format = Elem { height        :: Int
                       , lastLineWidth :: Int
                       , width         :: Int
                       , txtstr        :: Int -> String -> String
                       }
\end{lstlisting}

Первые три поля структуры определяют геометрические размеры блока, а последнее --- функция,
которая используется для преобразования блока в текст. Здесь используется функция, а не
просто строчка, чтобы можно было преобразовывать вложенные блоки за линейное время. Первый
аргумент \lstinline[language = Haskell]{txtstr} задает сдвиг блока.

\begin{figure}
  \centering
  \subfloat[Блок текста]{
    \raisebox{5mm}{
      \centering
      \vspace{0pt}
      \tikz[scale = 2.0]{
        \draw (0,0) -- (1,0) -- (1,0.2) -- (2,0.2) -- (2,1) -- (0, 1) -- cycle;
      }
    }
    \label{fig:basicFormat}
  }
  ~
  \subfloat[Горизонтальная композиция]{
    \centering
    \tikz[scale = 2.0]{
       \draw (0,0) -- (1,0) -- (1,0.2) -- (2,0.2) -- (2,1) -- (0, 1) -- cycle;
       \draw (1.1,-0.9) -- (2.1,-0.9) -- (2.1,-0.7) -- (3.1,-0.7) -- (3.1,0.1) -- (1.1,0.1) -- cycle;
     }
     \label{fig:beside}
  }
  %\hfill
  ~
  \subfloat[Вертикальная композиция]{
    \makebox[.28\textwidth] {
    \centering
    \tikz[scale = 2.0]{
       \draw (0,0) -- (1,0) -- (1,0.2) -- (2,0.2) -- (2,1) -- (0, 1) -- cycle;
       \draw (0,-1.1) -- (1,-1.1) -- (1,-0.9) -- (2,-0.9) -- (2,-0.1) -- (0,-0.1) -- cycle;
    }
    \label{fig:above}
    }
  }
  \caption{Примитивы блока текста}
  \label{fig:basicConcat}
\end{figure}

Для работы с блоками текста используется следующие четыре примитива:
\begin{lstlisting}
    s2fmt     :: String -> Format
    indentFmt :: Int -> Format -> Format
    aboveFmt  :: Format -> Format -> Format
    besideFmt :: Format -> Format -> Format
\end{lstlisting}

Функция \lstinline[language = Haskell]{s2fmt} создает блок текста,
состоящий из одной строчки; \lstinline[language = Haskell]{indentFmt} по блоку создает новый,
сдвинутый на заданное число позиций. Действие примитивов композиции
\lstinline[language = Haskell]{besideFmt} и \lstinline[language = Haskell]{aboveFmt}
показано на рис.~\ref{fig:beside} и \ref{fig:above} соответственно.

Также, как и в библиотеках без комбинатора выбора, в работе~\cite{swierstra} используется понятие
\emph{документа}. Здесь документ можно рассматривать как множество возможных раскладок
(набор \lstinline[language = Haskell]{Format}-элементов). Документы описываются типом
\lstinline[language = Haskell]{Doc}, экземпляры которого на этапе построения представляют собой
деревья применений приведенных ниже комбинаторов, а на этапе обработки алгоритмом
принтера им в соответствие ставится итоговое множество вариантов текстовых представлений.

Документ конструируется с помощью следующих комбинаторов, которые симметричны
примитивам построения блоков текста:
\begin{lstlisting}
    text   :: String -> Doc
    indent :: Int -> Doc -> Doc
    beside :: Doc -> Doc -> Doc
    above  :: Doc -> Doc -> Doc
\end{lstlisting}

В дополнение появляется пятый комбинатор для документов:
\begin{lstlisting}
    choice :: Doc -> Doc -> Doc
\end{lstlisting}

Этот комбинатор и является тем самым комбинатором выбора. Он представляет
\emph{объединение} множеств раскладок документов, которые были переданы как
аргументы комбинатора. Заметим, что только этот комбинатор может произвести
документ с несколькими раскладками из одновариантных аргументов.

Оригинальная реализация существенно опирается на ленивые вычисления. В \cite{swierstra}
множество вариантов, соответствующее экземпляру \lstinline[language = Haskell]{Doc},
представляется ленивым списком всех возможных раскладок, удовлетворяющих ограничению
на максимальную ширину. Этот список отсортирован в порядке ``ухудшения''
раскладок, то есть в голове списка лежит ``лучшая'', в терминах оптимальности,
раскладка из возможных при заданной ширине. В случае
\lstinline{beside}- и
\lstinline{above}-композиций
документов полная (без учета ленивости) сложность вычисления списка нового документа
составляет $O(n \times m)$, где $n$ и $m$ --- размеры списков, соответствующих 
соединяемым документам. Размер нового списка также порядка $n \times m$.
Он не обязательно точно равен $n \times m$, так как некоторые из полученных
представлений могут не подпадать под ограничение на ширину, соответственно их можно
отбросить на этапе построения нового списка.

Выбор лучшего представления для самого верхнеуровнего документа происходит
просто --- нужно из соответствующего списка взять первый элемент.
Это создает впечатление, что общее число операций,
благодаря ленивым вычислениям, существенно уменьшается.
Но это не так из-за реализации обработки документа,
построенного с помощью комбинатора \lstinline[language = Haskell]{beside}, которая
вынуждает полное вычисление дочерних списков. Более того, из-за свойств отношения
порядка, построенного по критерию оптимальности, в рамках данной модели списков
принципиально нельзя построить ленивую обработку комбинатора
\lstinline[language = Haskell]{beside}.
Так, выбор оптимальной раскладки документа в
\cite{swierstra} имеет в худшем случае экспоненциальную сложность от числа
комбинаторов, использованных при его построении.
В \cite{swiComb} приведены оптимизации для данной модели, но они
не улучшают асимптотику решения для деревьев в общем случае.

\subsection{BURS}

Bottom-Up Rewrite System (BURS)\cite{burs} --- это метод динамического
программирования на деревьях, изначально появившийся в контексте задачи выбора
инструкций для генерации машинного кода. Основой BURS является
регулярная грамматика с древовидными правилами\cite{tata}, для которых задана
стоимость применения подстановки,
то есть грамматика со следующим набором правил: 

$$
\begin{array}{rcll}
  N &:& \alpha& [c]\\
  N &:& \alpha\; (K_1,\dots,K_n)& [c]
\end{array}
$$

Здесь $N, K_i$ это нетерминалы, $\alpha$ --- терминал,
$c$ --- функция стоимости, заданная для каждого правила.
Как и для обычной линейной грамматики, вводится стартовый
нетерминал S. Считается, что терминальное дерево выводится в
данной грамматике, если его можно получить с помощью правил
подстановки из одноузлового дерева $S$.
Каждая подстановка заменяет нетерминал $N$, находящийся в листе дерева, на дерево 
$\alpha\;(K_1,\dots,K_n)$, если в грамматике есть правило
$N:\alpha\;(K_1,\dots,K_n)$. 
Для каждой подстановки вычисляется стоимость ее применения с помощью
функции стоимости $c$.
Аргументами функции могут служить терминальная метка $\alpha$ и стоимости
вывода поддеревьев.
Задачей, которую решает BURS, является поиск вывода наименьшей стоимости
для заданного дерева по заданной грамматике.
Такой вывод может быть найден двухпроходным алгоритмом.

Первый проход (\emph{пометка}) обрабатывает дерево снизу вверх и вычисляет
для каждого узла набор троек $(K,\;R,\;c)$, где $K$ --- нетерминал, из которого может быть
выведено поддерево с корнем в обрабатываемом узле,
$R$ --- первое правило, которое используется для вывода минимальной стоимости из $K$,
$c$ --- стоимость такого вывода.
Процесс пометки происходит следующим образом:

\begin{itemize}
\item для листовой вершины, помеченной терминалом $\alpha$, в множество троек
этого вершины добавляется $(K,\;R,\;c\:(\alpha))$ для каждого правила $R=K:\;\alpha\;[c]$;
\item для промежуточной вершины, помеченной терминалом $\alpha$,
с непосредственными поддеревьями $v_1,\dots,v_n$
в множество добавляется тройка $(K,\;R,\;c\:(\alpha,c_1,\dots,c_n))$ для каждого правила
$R=K:\;\alpha\;(K_1,\dots,K_n)\;[c]$, где $(K_i,\;R_i,\;c_i)$ входит в множество троек для
$v_i$; если есть несколько правил вывода из нетерминала $K$, то выбирается правило,
минимизирующее стоимость вывода.
\end{itemize}

Второй проход (\emph{свертка}) просматривает дерево сверху вниз, используя сделанные пометки.
Первое правило из минимального вывода определяется тройкой $(S,\;R,\;c)$ для корневого узла
(если такой тройки нет, то вывод из $S$ невозможен).
Это правило однозначно определяет нетерминалы $K_i$ для каждого непосредственного поддерева
и процесс повторяется.

Для пометки дерева потенциально необходимо каждое правило грамматики применить к каждому узлу.
При фиксированной грамматике алгоритмическая сложность первого прохода --- $O\:(|R|)$,
где $|R|$ --- количество правил
(размер множества троек для каждого узла ограничен числом нетерминалов, которое не больше,
чем количество правил). Свертка также имеет линейную сложность.


\newpage
\subsection{Средства форматирования кода в IDE}

Интегрированные среды разработки программного обеспечения
(IDE) предоставляют средства
для форматирования программного кода. Рассмотрим их на примере двух Java IDE ---
IntelliJ IDEA\footnote{\cd{http://jetbrains.com/idea/}} и
Eclipse\footnote{\cd{http://eclipse.org}}.
Для задания требуемого СК используется широкий набор настроек \emph{форматтера},
подпрограммы IDE, отвечающей за форматирование исходных текстов.
Примерами таких настроек являются:
\begin{itemize}
  \item помещать ли фигурную скобку на той же строке, что и
    предыдущее выражение;
  \item форматирование списков --- всегда на одной строчке, переводить строчку после
    каждого элемента или печатать на одной строке,
    пока строчка меньше заданной рекомендуемой ширины.
%   \item и т.д.
\end{itemize}
На рис. ~\ref{fig:ideaFormatter} и~\ref{fig:eclipseFormatter} приведены диалоги задания
параметров форматирования в IntelliJ IDEA и Eclipse соответственно.

\begin{figure}[p]
	\centering
	\includegraphics[width=\textwidth]{ideaFormatter}
	\caption{Окно настройки форматтера IntelliJ IDEA}
	\label{fig:ideaFormatter}
\end{figure}

\begin{figure}[p]
	\centering
	\includegraphics[width=\textwidth]{eclipseFormatter}
	\caption{Окно настройки форматтера Eclipse}
	\label{fig:eclipseFormatter}
\end{figure}

У встроенных форматтеров есть следующие особенности.
Во-первых, это невозможность выразить
нестандартный СК, так как настройки форматтеров, несмотря на их большое
количество,
дают ограниченную вариативность для текстовых представлений форматируемых
программ. В целом, количество принципиально разных
стилей форматирования, которые можно задать с помощью данных настроек,
невелико.
Во-вторых, в случае, если надо
придерживаться СК уже существующего проекта, то по коду этого проекта необходимо
вручную задать все настройки форматтера.
Причем наборы настроек в разных IDE не совпадают, что увеличивает сложность
поддержки единого СК, если разработчики используют отличные от друг друга IDE.
Для решения данной проблемы существуют плагины, позволяющие использовать
внешние фоматтеры\footnote{\cd{http://plugins.jetbrains.com/plugin/6546}}.
Кроме того, есть средства, позволяющие экспортировать настройки форматтера в
XML-файл для их использования в другой
IDE\footnote{\cd{http://blog.jetbrains.com/idea/2014/01/intellij-idea-13-importing-code-formatter-settings-from-eclipse/}}.

Важным достоинством описанных форматтеров является малое время работы.
Это достигается в том числе и за счет того, что часто форматируется не весь
файл, а только его часть (см. \cite{eclipse}),
и из-за детерминированности представлений узлов синтаксического дерева,
которая следует из специфики настроек форматтера.
Однако форматирование целого большого проекта занимает существенное время.
Например, обработка кода IntelliJ IDEA Community Edition занимает более часа
у встроенного в IDEA форматтера.


\section{Реализация}
В процессе работы над проектом было выделено две основных задачи: реализация
взаимодействия с пользователем и расширение функциональности для переформатирования программного текста с фильтрацией шаблонов.
%диаграмму что ли нарисавать?
Предполагаемый процесс взаимодействия с пользователем: пользователь выделяет
участок программного кода, в котором будут производиться изменение, и вызывает 
действие (Action), которое извлекает шаблон из выделенного участка
кода. Затем пользователь меняет форматирование этого же участка, выделяет его
снова и вызывает действие, которое извлекает новый шаблон форматирования и 
сохраняет его.

После этого происходит перепечатывание элементов (PsiElement), имеющих старое
форматирование, в файле, в котором пользователь менял программный текст.

\subsection{Переформатирование}
\subsubsection{Извлечение элемента для форматирования}
% тут будет диаграмма:
% 1) Получаем текст
% 2) Корректируем отступы + пример
% 3) Извлечение элемента (для того чтобы знать тип) -- вся рутина
% 4) Склеить текст и элемент
В первую очередь необходимо получить элемент, форматирование которого пользователь хочет изменить.
% Где-то надо ссылку на диаграмму, наверное, тут
Весь этот процесс изображен на рис.~\ref{fig:getElemDiag}.

\begin{figure}[h]
	\centering
	\includegraphics[width=\textwidth]{images/getElemDiag.jpg}
	\caption{Диаграмма этапов извлечения элемента из текста}
	\label{fig:getElemDiag}
\end{figure}

На первом этапе с помощью функций, предоставляемых IntelliJ IDEA API, можно получить текст элемента. 
Однако этот текст не всегда корректный, так как может содержать в себе лишние отступы.
Например, на рис. \ref{fig:offsetExample} изображены методы doSomething() и doAnotherThing(). 
Они имеют отступ относительно оператора ветвления, равный четырем пробелам.
Кроме того, они имеют отступ относительно метода foo(), равный восьми пробелам. 
Чтобы шаблон был корректным, на втором этапе убираются лишние пробелы (в данном случае четыре).
Далее получаем из текста сам элемент. 
Это необходимо, чтобы узнать его тип. %ссылку на дерево PsiElement'ов.
Затем (на четвертом этапе) из корректного текста и полученного элемента создается новый элемент того же типа, но с корректными отступами.

Все это необходимо проделать, так как IntelliJ IDEA API не предоставляет функционала по изменению текста элемента явным образом. 

\begin{figure}[h]
	\centering
	\lstinputlisting[language=C++]{codes/offsetExample.txt}
	\caption{Общий отступ элементов}
	\label{fig:offsetExample}
\end{figure}

Аналогичные действия необходимо повторить для получения элемента из измененного
пользователем текста. 
Из этого элемента извлекается шаблон и сохраняется принтером для дальнейшего использования.

%\subsubsection{Извлечение шаблона}
%Шаблон, как и прежде, извлекается методами класса \lstinline[language=C++]{PsiElementComponent}.

%\subsubsection{Добавление шаблона}
%В связи с тем, что возникла необходимость добавления одного шаблона, класс %Printer обзавелся публичным методом, реализующим этот процесс. Остальные
%действия аналогичны тем при заполнении списка шаблонов из файла.

\subsubsection{Сравнение с шаблоном}
Как уже было отмечено в обзоре, при перепечатывании принтер обходит все элементы (PsiElement) дерева разбора и применяет к ним новые шаблоны. 
В рамках реализуемого подхода происходит перепечатывание лишь тех элементов, которые имели старое форматирование.
Это необходимо, если хочется поддерживать два и более различных варианта форматирования элементов некоторого типа. 
На рис. \ref{templates} изображены три блока оператора ветвления. 
Без фильтрации шаблонов все три блока будут изменены и получат одинаковое форматирование. %рис. сделать 
Однако первые два блока и так легко читаемы и понятны. 
Третий блок, будучи однострочным, уже не будет восприниматься так же хорошо.
Кроме того, существуют ограничения на ширину вывода, то есть в одной строке не можеть быть символов больше некоторой наперед заданной величины.

%\begin{figure}[t]
%    \centering
%    \lstinputlisting[language=C++]{codes/sameTmpltElems.txt}
%    \caption{Шаблоны}
%    \label{fig:sameTmpltElems}
%\end{figure}
%добавить вторую картинку как стало бы
\fvset{frame=lines,framesep=7pt}
\begin{figure}[ht]
\noindent\begin{minipage}{.5\textwidth}
    \lstinputlisting[language=C++]{codes/sameTmpltElems.txt}
\caption*{а) До переформатирования}    
\end{minipage}\hfill
\begin{minipage}{.5\textwidth}
    \lstinputlisting[language=C++]{codes/sameTmpltElems2.txt}
\caption*{б) После переформатирования}    
\end{minipage}
\caption{Переформатирование элементов одного типа без фильтрации шаблонов} 
\label{templates}
\end{figure}


Чтобы сделать настройку форматирования более гибкой и избавиться от подобного рода проблем, необходимо производить сравнение шаблонов текущего элемента дерева разбора и элемента со старым форматированием. 
Переформатировать элемент -- только в случае совпадения шаблонов.
Шаблоны совпадают, если совпадет их текстовое представление: шаблон записывается в виде строки, которая содержит необходимые элементы: ключевые слова, пробелы, переносы строк, скобки, а так же метки для поддеревьев, в которых указывается дополнительная информация, например, количество выражений в поддереве (одно или несколько).
В случае оператора ветвления (рис.~\ref{iftmpltcode}а) его поддеревьями являются: условие (condition), ветка, соответствующая выполнению условия (then block) и содержащая несколько подвыражений, ветка, соответствующая невыполнению условия (else block) и содержащая одно подвыражение. %ifTmplt ifTmpltCode
Соответствующий ему шаблон изображен на рисунке~\ref{iftmpltcode}б.
\fvset{frame=lines,framesep=7pt}
\begin{figure}[ht]
\noindent\begin{minipage}{.5\textwidth}
    \lstinputlisting[language=C++]{codes/ifTmpltCode.txt}
\caption*{а) Оператор ветвления}    
\end{minipage}\hfill
\begin{minipage}{.5\textwidth}
    \lstinputlisting[language=C++]{codes/ifTmplt.txt}
\caption*{б) Шаблон для оператора ветвления}    
\end{minipage}
\caption{Оператор ветвления и соответствующий ему шаблон}    
\label{iftmpltcode}
\end{figure}

%\begin{figure}[t]
%    \centering
%    \lstinputlisting[language=C++]{codes/sameTmpltElems.txt}
%    \caption{Шаблоны}
%    \label{fig:sameTmpltElems}
%\end{figure}
%добавить вторую картинку как стало бы

\subsection{Оценка производительности}
Описанный выше способ  давал очень низкую производительность. 
В таблице~\ref{table:slow} приведено время, затрачиваемое принтером на переформатирование файлов различной длины. 
Измерения производились на примере двух блоков: оператора ветвления и метода.

\begin{table}[h]
\begin{tabular}{cc|c|c|c|c|}
\cline{2-5}
\multicolumn{1}{ c| }{ } & \multicolumn{2}{p{6cm} |}{Формат. операторов ветвления} & \multicolumn{2}{p{5cm} |}{Форматирование методов} \\
\hline
\multicolumn{1}{ |c| }{Размер файла (строк)} & Кол-во & Время (сек.) & Кол-во & Время (сек.) \\ 
\hline
\multicolumn{1}{ |c| }{>7000}           & 520   & 35    & 660  & 58 \\
\multicolumn{1}{ |c| }{1000..2000}       & 145   & 2.7   & 90   & 4.4 \\
\multicolumn{1}{ |c| }{500..1000}       & 54    & 0.73  & 72   & 1.34 \\
\multicolumn{1}{ |c| }{0..500}          & 35    & 0.2   & 36   & 0.67 \\
\hline
\end{tabular}
\caption{Время переформатирования файлов I}
\label{table:slow}
\end{table}


Причина столь низкой производительности кроется в том, что для переформатирования элементов используется тот же способ, что и при форматировании по образцу:
принтер обходит все дерево разбора и сравнивает шаблоны.
%Однако нет необходимости проверять на совпадение шаблоны, полученные из каждого элемента дерева разбора, и шаблон, полученного из элемента со старым форматированием.
Однако нет необходимости сравнивать шаблон, полученный из элемента со старым форматированием, с шаблоном, полученным из каждого элемента дерева разбора.
Количество вариантов для переформатирования можно сузить лишь для элементов того же типа.

Для уменьшения затрат времени при обходе дерева элементов создается список кандидатов на форматирование (элементы того же типа, что и элемент со старым форматированием). 
После этого из элементов созданного списка извлекаются шаблоны.
Если полученный шаблон совпадает с шаблоном элемента со старым форматированием, то происходит перепечатывание элемента и замена его в дереве разбора.
Время, затрачиваемое в этом случае на переформатирование, приведено в таблице~\ref{table:fast}.

%\begin{table}[h]
%\begin{tabular}{ l | p{4cm} | p{4cm} }
%  Размер файла (строк) & Формат. операторов ветвления (сек.) & Формат. методов %(сек.) \\ 
%  \hline
%  >7000         & 6.6   & 5.8 \\
%  1000..2000    & 0.72  & 1   \\
%  500..1000     & 0.125 & 0.67 \\
%  0..500        & 0.145 & 0.325 \\
%\end{tabular}
%\caption{Время переформатирования файлов II}
%\label{table:fast}
%\end{table}


\begin{table}[h]
\begin{tabular}{cc|c|c|c|c|}
\cline{2-5}
\multicolumn{1}{ c| }{ } & \multicolumn{2}{p{6cm} |}{Формат. операторов ветвления} & \multicolumn{2}{p{5cm} |}{Форматирование  методов} \\
\hline
\multicolumn{1}{ |c| }{Размер файла (строк)} & Кол-во & Время (сек.) & Кол-во & Время (сек.) \\ 
\hline
\multicolumn{1}{ |c| }{>7000}           & 520   & 6.6   & 660  & 5.8 \\
\multicolumn{1}{ |c| }{1000..2000}      & 145   & 0.72  & 90   & 1 \\
\multicolumn{1}{ |c| }{500..1000}       & 54    & 0.225 & 72   & 0.67 \\
\multicolumn{1}{ |c| }{0..500}          & 35    & 0.145 & 36   & 0.325 \\
\hline
\end{tabular}
\caption{Время переформатирования файлов II}
\label{table:fast}
\end{table}
\section{Заключение}

В данной работе описан подход к композициональному символьному исполнению без раскрутки. Была предложена концепция композициональной памяти с символьной адресацией. Был доказан некоторый набор свойств КСП, дающий основание для подхода в стиле систем переписывания, где символьные кучи могут сами выступать как символы. Это даёт возможность автоматически порождать уравнения на состояния, решения которых в точности отражают поведения функций, работающих с динамической памятью. Было показано как свести задачу решения уравнений на состояния к задаче проверки безопасности чистых функций второго порядка.

Данная работа нацелена на теоретические основания композиционального анализа динамической памяти. Мы оставляем апробацию этого подхода на будущее. Другим направлением будущих исследований может быть расширение нашего формализма на композициональный анализ параллельных программ.

\begin{thebibliography}{99}
  \bibitem{podkopaev:diploma1}
  А.В.Подкопаев. Полиномиальной сложности оптимальные принтер-комбинаторы с выбором.
  Дипломная работа, кафедра системного программирования, математико-механический факультет СПбГУ, 2014.

  \bibitem{stratego:xt}
  Stratego/XT. \url{http://strategoxt.org/Stratego/StrategoDocumentation}, 2015.

  \bibitem{intellij:plugdev}
  IntelliJ IDEA Plugin Development Guide. \url{https://confluence.jetbrains.com/display/IDEADEV/PluginDevelopment},
  2015.
\end{thebibliography}

% % %\usepackage{listings}  
%\usepackage{hyperref}
%\usepackage{verbments}
%\usepackage{minted}
%\usepackage{caption}
%\usepackage{fancybox}
%\newfontfamily{\cyrillicfonttt}{CMU Serif}

\title{Декларативное форматирование в режиме онлайн}

\titlerunning{Декларативное форматирование в режиме онлайн}

\author{Озерных Игорь Станиславович}

\authorrunning{И.С.Озерных}

\tocauthor{И.С.Озерных}
\institute{Санкт-Петербургский государственный университет\\
\email{igor.ozernykh@gmail.com}}

\maketitle             

\begin{abstract}
Аккуратное, соответствующие выбранному стилю оформление (форматирование)
программных текстов критично для их восприятия.
Существуют средства автоматического форматирования кода. Одним
из них является метод синтаксических шаблонов. Его преимуществом является декларативность, которая
достигается за счет задания стиля форматирования с помощью образца кода. Так, естественным ограничением
описанного метода является необходимость наличия достаточно полного образца на целевом языке.
Данная работа посвящена расширению границ применимости метода синтаксических шаблонов
за счет возможности задания шаблонов в режиме онлайн.
\end{abstract}

\section*{Введение}

Этап поддержки и сопровождения в жизненном цикле программного обеспечения может занимать до 90\% времени существования программного продукта. На этом этапе особенно важно понимание программного текста поддерживаемой системы, что зачастую бывает сложной задачей. Необходимым условием для лучшего понимания программного текста является его аккуратное оформление, отражающее структуру 
программы. Существуют различные стандарты оформления кода (стили кодирования)~--- наборы правил и соглашений, используемые при написании исходного кода на некотором языке программирования.
Рассмотрим их на примере стандарта кодирования, принятого в компании Google\footnote{\texttt{https://google-styleguide.googlecode.com/svn/trunk/cppguide.html}}, и стандарта, используемого в проекте GNU\footnote{\texttt{http://www.gnu.org/prep/standards/standards.html}} (см. рис.~\ref{codingstandards}) для языка C++.


\fvset{frame=lines,framesep=7pt}
\begin{figure}[ht]
\noindent\begin{minipage}{.5\textwidth}
    \lstinputlisting[language=Java]{Ozernykh/codes/exGoogleCC.txt}
\caption*{а) Стиль кодирования Google}    
\end{minipage}\hfill
\begin{minipage}{.5\textwidth}
    \lstinputlisting[language=Java]{Ozernykh/codes/exGNUCC.txt}
\caption*{б) Стиль кодирования GNU}    
\end{minipage}
\caption{Различные стили кодирования для языка С++}    
\label{codingstandards}
\end{figure}

Когда программист работает над проектом, он придерживается некоторого стиля кодирования. 
Если файлы проекта с одним стилем кодирования присоединяется к уже существующему проекту с другим стилем кодирования, то их форматирование нужно привести к тому же виду. 
Для решения этой задачи можно воспользоваться уже существующими средствами, например, форматтерами (программами, которые форматируют исходный текст) в IDE таких, как Eclipse, IntelliJ IDEA, Visual Studio и др. 
Однако в этом случае необходимо вручную задавать настройки форматирования.
Кроме того, количество принципиально разных стилей форматирования, которые можно задать с помощью данных настроек, невелико.

%Однако в этом случае необходимо иметь настройки форматирования, заданные программистом, но в целом, количество принципиально разных стилей форматировани, которые можно задать с помощью данных настроек, невелико. %причем эти настройки могут быть различными в разных IDE.
% * <Anton Podkopaev> 12:15:51 22 May 2015 UTC+0300:
% Нужно переработать это предложение.

Еще одним примером может послужить принтер-плагин для IntelliJ IDEA~\cite{podkopaev:diploma1}, который позволяет форматировать исходный код проекта по образцу. 
В качестве образца используется код из некоторого репозитория. 
Из него выделяются шаблоны форматирования для структур языка, которые применяются к структурам целевого кода. 
Под \textit{шаблоном} понимаются данные, сопоставление которых с элементом синтаксического дерева дает текстовое представление этого элемента (и его потомков). 
Например, на рис.~\ref{fig:ifTree} представлено дерево разбора для оператора ветвления. 


\begin{figure}[h]
	\centering
	\includegraphics[width=\textwidth]{Ozernykh/images/ifTree.jpg}
	\caption{Представление оператора ветвления в виде дерева разбора}
	\label{fig:ifTree}
\end{figure}

На рис.~\ref{fig:tmpltcodeintro}а изображен шаблон форматирования, который может быть применен к нему (несколько подвыражений, соответствующих выполнению условия, и одно подвыражение, соответствующее невыполненению условия).
На рис.~\ref{fig:tmpltcodeintro}б представлен результат применения этого шаблона к дереву разбора для оператора ветвления.

% \begin{figure}[h]
% 	\centering
% 	\lstinputlisting[language=c++]{codes/ifTmpltIntro.txt}
% 	\caption{Шаблон для оператора ветвления}
% 	\label{fig:ifTmpltIntro}
% \end{figure}

% На рис.~\ref{fig:ifCodeIntro} представлен результат применения этого шаблона к дереву разбора для оператора ветвления.

% \begin{figure}[h]
% 	\centering
% 	\lstinputlisting[language=c++]{codes/ifCodeIntro.txt}
% 	\caption{Представление дерева разбора при применении шаблона}
% 	\label{fig:ifCodeIntro}
% \end{figure}


\fvset{frame=lines,framesep=7pt}
\begin{figure}[ht]
\noindent\begin{minipage}{.5\textwidth}
    \lstinputlisting[language=Java]{Ozernykh/codes/ifTmpltIntro.txt}
\caption*{а) Шаблон для оператора\\
ветвления}    
\end{minipage}\hfill
\begin{minipage}{.5\textwidth}
    \lstinputlisting[language=Java]{Ozernykh/codes/ifCodeIntro.txt}
\caption*{б) Текст, полученный при применении шаблона к дереву разбора}    
\end{minipage}
\caption{Оператор ветвления и шаблон для него}    
\label{fig:tmpltcodeintro}
\end{figure}

Однако не всегда этот репозиторий существует. Кроме того, необходимо наличие заданного форматирования для всех структур языка.
% * <Anton Podkopaev> 12:19:02 22 May 2015 UTC+0300:
% 'Однако не всегда удобно иметь этот репозиторий под рукой.' --- поменять. Не про удобство надо говорить.

Шаблоны можно извлекать из уже существующего кода и применять их к другим элементам того же типа, но отформатированных иным способом. 
Например, пользователь меняет форматирование некоторой структуры языка (в частности, оператор ветвления), программа извлекает из полученного участка кода новый шаблон и применяет его к структурам того же типа, содержащим старое форматирование. 
Назовем такой способ задания шаблонов~--- форматирование в режиме
онлайн.
%картинки

Целью данной работы является расширение функциональности описанного 
принтер-плагина путем добавления возможности задания шаблонов в режиме онлайн. 

\section{Обзор предметной области}

Одним из подходов к заданию принтеров являются принтер-комбинаторы.
В этой главе приведен обзор классических принтер-комбинаторов, а также
принтер-комбинаторов с дополнительным оператором \emph{выбора}, которые
в рамках данной работы использованы для реализации принтеров, задаваемых
с помощью шаблонов. Для решения проблем эффективности
комбинаторов с выбором в данной работе был использован подход BURS.
Апробация подхода задания принтеров с помощью шаблонов
была произведена для языка Java, для которого есть средства форматирования,
встроенные в IDE.

\subsection{Принтер-комбинаторы}

Базовыми принтер-комбинаторными библиотеками являются библиотеки
Джона Хьюза\cite{hughes} и Филиппа Вадлера\cite{wadler}, которые
представляют собой функциональную переработку алгоритма
Оппена\cite{oppen}.
Ограничимся рассмотрением библиотеки Вадлера, так как
библиотека Хьюза обладает теми же свойствами, которые принципиальны в контексте
данной работы.

В библиотеке ключевым типом является документ.
Он представляет сущность, которая потом может
быть переведена в строковое представление алгоритмом принтера.
Основные конструкторы для составления документа таковы:
\begin{itemize}
  \item атомарная строка, которая печатается как есть;
  \item \emph{разделитель};
  \item последовательная композиция двух документов;
  \item набор связанных документов.
\end{itemize}

Определяющей особенностью данного подхода является то, что все разделители
в рамках одного набора могут быть совместно заменены алгоритмом принтера
на пробельный символ или на перевод строки. Выбор для каждого набора разделителей
основывается на том, что вывод должен поместиться в заданную ширину,
используя минимальное число строк.

Основная проблема такого подхода заключается в его слабой выразительной силе.
Документы, построенные по синтаксическому дереву печатаемой программы,
обрабатываются слишком единообразно, что иногда приводит к нежелательному результату.
Пусть, к примеру, нужно напечатать программу на языке
Python\footnote{\cd{http://python.org}}.
Между последовательными операторами в случае их печати на одной строчке необходимо
добавить дополнительный разделитель (``;''), иначе программа станет некорректной
(см. рис.~\ref{fig:seqEx}). Однако, описанные принтер-комбинаторы не предоставляют
возможности задать такое поведение.
Кроме того, с помощью таких принтер-комбинаторов невозможно выразить разные проектные СК,
так как они всегда печатают текст в одном стиле, который жестко ``зашит'' в их код.

\begin{figure}[h!]
	\centering
	\null\hfill
	\subfloat[Корректный код]{
		\centering
    \makebox[.4\textwidth] {
		  \lstinputlisting[language=Python]{codes/pythonCode.py}
    }
	}
	\null\hfill
	\subfloat[Некорректный код]{
		\centering
    \makebox[.4\textwidth] {
	  	\lstinputlisting[language=Python]{codes/pythonCodeBad.py}
    }
	}
	\hfill\null
	\caption{Пример работы принтер-комбинаторов для языка Python}
  \label{fig:seqEx}	
\end{figure}

Более подробный обзор библиотек Хьюза и Вадлера представлен в \cite{myCoursePaper}.

Большинство принтер-комбинаторных библиотек\cite{
swierstraChitil, swierstra04, peytonJones, kiselyov, chitil}
являются развитиями работ Хьюза и Вадлера. Так, в отличие от базовых, среди
них есть реализации с линейной сложностью обработки документа от его размера и
\emph{online} алгоритмы, которые не требуют просмотра всего документа
для начала печати его текстового представления. Но все они обладают тем же
интерфейсом, а значит в них также неразрешимы задачи, в которых требуется
задавать варианты текстовых представлений, которые отличаются не только
пробелами и переводами строк.

\subsection{Принтер-комбинаторы с выбором}

Существенно от описанных выше отличаются библиотеки, предоставляющие в своем
интерфейсе комбинатор \emph{выбора}, который позволяет задавать для одного поддерева
принципиально разные варианты раскладок.
Так, в работах~\cite{jongeEveryOccasion, jongeReengine} используется
оператор \lstinline[language = Haskell]{ALT},
но алгоритм принтера устроен так, что среди двух альтернатив выбирается первая, если она
помещается в заданную ширину, и вторая --- иначе,
что не дает оптимальный результат на выходе.

Оптимальные принтер-комбинаторы с выбором были впервые представлены в работе~\cite{swierstra}.
Их реализация является частью Utrecht Tools
Library\footnote{\cd{http://www.cs.uu.nl/wiki/HUT/WebHome}}
(практическая реализация несколько изменена по
отношению к той, что описана в статье, но отличие несущественно).
В данном подходе текст строится из блоков прямоугольной формы с возможно неполной последней
строчкой (см. рис. \ref{fig:basicFormat}). В реализации на Haskell блоки представляются
структурой \lstinline[language = Haskell]{Format}:
\begin{lstlisting}
    data Format = Elem { height        :: Int
                       , lastLineWidth :: Int
                       , width         :: Int
                       , txtstr        :: Int -> String -> String
                       }
\end{lstlisting}

Первые три поля структуры определяют геометрические размеры блока, а последнее --- функция,
которая используется для преобразования блока в текст. Здесь используется функция, а не
просто строчка, чтобы можно было преобразовывать вложенные блоки за линейное время. Первый
аргумент \lstinline[language = Haskell]{txtstr} задает сдвиг блока.

\begin{figure}
  \centering
  \subfloat[Блок текста]{
    \raisebox{5mm}{
      \centering
      \vspace{0pt}
      \tikz[scale = 2.0]{
        \draw (0,0) -- (1,0) -- (1,0.2) -- (2,0.2) -- (2,1) -- (0, 1) -- cycle;
      }
    }
    \label{fig:basicFormat}
  }
  ~
  \subfloat[Горизонтальная композиция]{
    \centering
    \tikz[scale = 2.0]{
       \draw (0,0) -- (1,0) -- (1,0.2) -- (2,0.2) -- (2,1) -- (0, 1) -- cycle;
       \draw (1.1,-0.9) -- (2.1,-0.9) -- (2.1,-0.7) -- (3.1,-0.7) -- (3.1,0.1) -- (1.1,0.1) -- cycle;
     }
     \label{fig:beside}
  }
  %\hfill
  ~
  \subfloat[Вертикальная композиция]{
    \makebox[.28\textwidth] {
    \centering
    \tikz[scale = 2.0]{
       \draw (0,0) -- (1,0) -- (1,0.2) -- (2,0.2) -- (2,1) -- (0, 1) -- cycle;
       \draw (0,-1.1) -- (1,-1.1) -- (1,-0.9) -- (2,-0.9) -- (2,-0.1) -- (0,-0.1) -- cycle;
    }
    \label{fig:above}
    }
  }
  \caption{Примитивы блока текста}
  \label{fig:basicConcat}
\end{figure}

Для работы с блоками текста используется следующие четыре примитива:
\begin{lstlisting}
    s2fmt     :: String -> Format
    indentFmt :: Int -> Format -> Format
    aboveFmt  :: Format -> Format -> Format
    besideFmt :: Format -> Format -> Format
\end{lstlisting}

Функция \lstinline[language = Haskell]{s2fmt} создает блок текста,
состоящий из одной строчки; \lstinline[language = Haskell]{indentFmt} по блоку создает новый,
сдвинутый на заданное число позиций. Действие примитивов композиции
\lstinline[language = Haskell]{besideFmt} и \lstinline[language = Haskell]{aboveFmt}
показано на рис.~\ref{fig:beside} и \ref{fig:above} соответственно.

Также, как и в библиотеках без комбинатора выбора, в работе~\cite{swierstra} используется понятие
\emph{документа}. Здесь документ можно рассматривать как множество возможных раскладок
(набор \lstinline[language = Haskell]{Format}-элементов). Документы описываются типом
\lstinline[language = Haskell]{Doc}, экземпляры которого на этапе построения представляют собой
деревья применений приведенных ниже комбинаторов, а на этапе обработки алгоритмом
принтера им в соответствие ставится итоговое множество вариантов текстовых представлений.

Документ конструируется с помощью следующих комбинаторов, которые симметричны
примитивам построения блоков текста:
\begin{lstlisting}
    text   :: String -> Doc
    indent :: Int -> Doc -> Doc
    beside :: Doc -> Doc -> Doc
    above  :: Doc -> Doc -> Doc
\end{lstlisting}

В дополнение появляется пятый комбинатор для документов:
\begin{lstlisting}
    choice :: Doc -> Doc -> Doc
\end{lstlisting}

Этот комбинатор и является тем самым комбинатором выбора. Он представляет
\emph{объединение} множеств раскладок документов, которые были переданы как
аргументы комбинатора. Заметим, что только этот комбинатор может произвести
документ с несколькими раскладками из одновариантных аргументов.

Оригинальная реализация существенно опирается на ленивые вычисления. В \cite{swierstra}
множество вариантов, соответствующее экземпляру \lstinline[language = Haskell]{Doc},
представляется ленивым списком всех возможных раскладок, удовлетворяющих ограничению
на максимальную ширину. Этот список отсортирован в порядке ``ухудшения''
раскладок, то есть в голове списка лежит ``лучшая'', в терминах оптимальности,
раскладка из возможных при заданной ширине. В случае
\lstinline{beside}- и
\lstinline{above}-композиций
документов полная (без учета ленивости) сложность вычисления списка нового документа
составляет $O(n \times m)$, где $n$ и $m$ --- размеры списков, соответствующих 
соединяемым документам. Размер нового списка также порядка $n \times m$.
Он не обязательно точно равен $n \times m$, так как некоторые из полученных
представлений могут не подпадать под ограничение на ширину, соответственно их можно
отбросить на этапе построения нового списка.

Выбор лучшего представления для самого верхнеуровнего документа происходит
просто --- нужно из соответствующего списка взять первый элемент.
Это создает впечатление, что общее число операций,
благодаря ленивым вычислениям, существенно уменьшается.
Но это не так из-за реализации обработки документа,
построенного с помощью комбинатора \lstinline[language = Haskell]{beside}, которая
вынуждает полное вычисление дочерних списков. Более того, из-за свойств отношения
порядка, построенного по критерию оптимальности, в рамках данной модели списков
принципиально нельзя построить ленивую обработку комбинатора
\lstinline[language = Haskell]{beside}.
Так, выбор оптимальной раскладки документа в
\cite{swierstra} имеет в худшем случае экспоненциальную сложность от числа
комбинаторов, использованных при его построении.
В \cite{swiComb} приведены оптимизации для данной модели, но они
не улучшают асимптотику решения для деревьев в общем случае.

\subsection{BURS}

Bottom-Up Rewrite System (BURS)\cite{burs} --- это метод динамического
программирования на деревьях, изначально появившийся в контексте задачи выбора
инструкций для генерации машинного кода. Основой BURS является
регулярная грамматика с древовидными правилами\cite{tata}, для которых задана
стоимость применения подстановки,
то есть грамматика со следующим набором правил: 

$$
\begin{array}{rcll}
  N &:& \alpha& [c]\\
  N &:& \alpha\; (K_1,\dots,K_n)& [c]
\end{array}
$$

Здесь $N, K_i$ это нетерминалы, $\alpha$ --- терминал,
$c$ --- функция стоимости, заданная для каждого правила.
Как и для обычной линейной грамматики, вводится стартовый
нетерминал S. Считается, что терминальное дерево выводится в
данной грамматике, если его можно получить с помощью правил
подстановки из одноузлового дерева $S$.
Каждая подстановка заменяет нетерминал $N$, находящийся в листе дерева, на дерево 
$\alpha\;(K_1,\dots,K_n)$, если в грамматике есть правило
$N:\alpha\;(K_1,\dots,K_n)$. 
Для каждой подстановки вычисляется стоимость ее применения с помощью
функции стоимости $c$.
Аргументами функции могут служить терминальная метка $\alpha$ и стоимости
вывода поддеревьев.
Задачей, которую решает BURS, является поиск вывода наименьшей стоимости
для заданного дерева по заданной грамматике.
Такой вывод может быть найден двухпроходным алгоритмом.

Первый проход (\emph{пометка}) обрабатывает дерево снизу вверх и вычисляет
для каждого узла набор троек $(K,\;R,\;c)$, где $K$ --- нетерминал, из которого может быть
выведено поддерево с корнем в обрабатываемом узле,
$R$ --- первое правило, которое используется для вывода минимальной стоимости из $K$,
$c$ --- стоимость такого вывода.
Процесс пометки происходит следующим образом:

\begin{itemize}
\item для листовой вершины, помеченной терминалом $\alpha$, в множество троек
этого вершины добавляется $(K,\;R,\;c\:(\alpha))$ для каждого правила $R=K:\;\alpha\;[c]$;
\item для промежуточной вершины, помеченной терминалом $\alpha$,
с непосредственными поддеревьями $v_1,\dots,v_n$
в множество добавляется тройка $(K,\;R,\;c\:(\alpha,c_1,\dots,c_n))$ для каждого правила
$R=K:\;\alpha\;(K_1,\dots,K_n)\;[c]$, где $(K_i,\;R_i,\;c_i)$ входит в множество троек для
$v_i$; если есть несколько правил вывода из нетерминала $K$, то выбирается правило,
минимизирующее стоимость вывода.
\end{itemize}

Второй проход (\emph{свертка}) просматривает дерево сверху вниз, используя сделанные пометки.
Первое правило из минимального вывода определяется тройкой $(S,\;R,\;c)$ для корневого узла
(если такой тройки нет, то вывод из $S$ невозможен).
Это правило однозначно определяет нетерминалы $K_i$ для каждого непосредственного поддерева
и процесс повторяется.

Для пометки дерева потенциально необходимо каждое правило грамматики применить к каждому узлу.
При фиксированной грамматике алгоритмическая сложность первого прохода --- $O\:(|R|)$,
где $|R|$ --- количество правил
(размер множества троек для каждого узла ограничен числом нетерминалов, которое не больше,
чем количество правил). Свертка также имеет линейную сложность.


\newpage
\subsection{Средства форматирования кода в IDE}

Интегрированные среды разработки программного обеспечения
(IDE) предоставляют средства
для форматирования программного кода. Рассмотрим их на примере двух Java IDE ---
IntelliJ IDEA\footnote{\cd{http://jetbrains.com/idea/}} и
Eclipse\footnote{\cd{http://eclipse.org}}.
Для задания требуемого СК используется широкий набор настроек \emph{форматтера},
подпрограммы IDE, отвечающей за форматирование исходных текстов.
Примерами таких настроек являются:
\begin{itemize}
  \item помещать ли фигурную скобку на той же строке, что и
    предыдущее выражение;
  \item форматирование списков --- всегда на одной строчке, переводить строчку после
    каждого элемента или печатать на одной строке,
    пока строчка меньше заданной рекомендуемой ширины.
%   \item и т.д.
\end{itemize}
На рис. ~\ref{fig:ideaFormatter} и~\ref{fig:eclipseFormatter} приведены диалоги задания
параметров форматирования в IntelliJ IDEA и Eclipse соответственно.

\begin{figure}[p]
	\centering
	\includegraphics[width=\textwidth]{ideaFormatter}
	\caption{Окно настройки форматтера IntelliJ IDEA}
	\label{fig:ideaFormatter}
\end{figure}

\begin{figure}[p]
	\centering
	\includegraphics[width=\textwidth]{eclipseFormatter}
	\caption{Окно настройки форматтера Eclipse}
	\label{fig:eclipseFormatter}
\end{figure}

У встроенных форматтеров есть следующие особенности.
Во-первых, это невозможность выразить
нестандартный СК, так как настройки форматтеров, несмотря на их большое
количество,
дают ограниченную вариативность для текстовых представлений форматируемых
программ. В целом, количество принципиально разных
стилей форматирования, которые можно задать с помощью данных настроек,
невелико.
Во-вторых, в случае, если надо
придерживаться СК уже существующего проекта, то по коду этого проекта необходимо
вручную задать все настройки форматтера.
Причем наборы настроек в разных IDE не совпадают, что увеличивает сложность
поддержки единого СК, если разработчики используют отличные от друг друга IDE.
Для решения данной проблемы существуют плагины, позволяющие использовать
внешние фоматтеры\footnote{\cd{http://plugins.jetbrains.com/plugin/6546}}.
Кроме того, есть средства, позволяющие экспортировать настройки форматтера в
XML-файл для их использования в другой
IDE\footnote{\cd{http://blog.jetbrains.com/idea/2014/01/intellij-idea-13-importing-code-formatter-settings-from-eclipse/}}.

Важным достоинством описанных форматтеров является малое время работы.
Это достигается в том числе и за счет того, что часто форматируется не весь
файл, а только его часть (см. \cite{eclipse}),
и из-за детерминированности представлений узлов синтаксического дерева,
которая следует из специфики настроек форматтера.
Однако форматирование целого большого проекта занимает существенное время.
Например, обработка кода IntelliJ IDEA Community Edition занимает более часа
у встроенного в IDEA форматтера.


\section{Реализация}
В процессе работы над проектом было выделено две основных задачи: реализация
взаимодействия с пользователем и расширение функциональности для переформатирования программного текста с фильтрацией шаблонов.
%диаграмму что ли нарисавать?
Предполагаемый процесс взаимодействия с пользователем: пользователь выделяет
участок программного кода, в котором будут производиться изменение, и вызывает 
действие (Action), которое извлекает шаблон из выделенного участка
кода. Затем пользователь меняет форматирование этого же участка, выделяет его
снова и вызывает действие, которое извлекает новый шаблон форматирования и 
сохраняет его.

После этого происходит перепечатывание элементов (PsiElement), имеющих старое
форматирование, в файле, в котором пользователь менял программный текст.

\subsection{Переформатирование}
\subsubsection{Извлечение элемента для форматирования}
% тут будет диаграмма:
% 1) Получаем текст
% 2) Корректируем отступы + пример
% 3) Извлечение элемента (для того чтобы знать тип) -- вся рутина
% 4) Склеить текст и элемент
В первую очередь необходимо получить элемент, форматирование которого пользователь хочет изменить.
% Где-то надо ссылку на диаграмму, наверное, тут
Весь этот процесс изображен на рис.~\ref{fig:getElemDiag}.

\begin{figure}[h]
	\centering
	\includegraphics[width=\textwidth]{images/getElemDiag.jpg}
	\caption{Диаграмма этапов извлечения элемента из текста}
	\label{fig:getElemDiag}
\end{figure}

На первом этапе с помощью функций, предоставляемых IntelliJ IDEA API, можно получить текст элемента. 
Однако этот текст не всегда корректный, так как может содержать в себе лишние отступы.
Например, на рис. \ref{fig:offsetExample} изображены методы doSomething() и doAnotherThing(). 
Они имеют отступ относительно оператора ветвления, равный четырем пробелам.
Кроме того, они имеют отступ относительно метода foo(), равный восьми пробелам. 
Чтобы шаблон был корректным, на втором этапе убираются лишние пробелы (в данном случае четыре).
Далее получаем из текста сам элемент. 
Это необходимо, чтобы узнать его тип. %ссылку на дерево PsiElement'ов.
Затем (на четвертом этапе) из корректного текста и полученного элемента создается новый элемент того же типа, но с корректными отступами.

Все это необходимо проделать, так как IntelliJ IDEA API не предоставляет функционала по изменению текста элемента явным образом. 

\begin{figure}[h]
	\centering
	\lstinputlisting[language=C++]{codes/offsetExample.txt}
	\caption{Общий отступ элементов}
	\label{fig:offsetExample}
\end{figure}

Аналогичные действия необходимо повторить для получения элемента из измененного
пользователем текста. 
Из этого элемента извлекается шаблон и сохраняется принтером для дальнейшего использования.

%\subsubsection{Извлечение шаблона}
%Шаблон, как и прежде, извлекается методами класса \lstinline[language=C++]{PsiElementComponent}.

%\subsubsection{Добавление шаблона}
%В связи с тем, что возникла необходимость добавления одного шаблона, класс %Printer обзавелся публичным методом, реализующим этот процесс. Остальные
%действия аналогичны тем при заполнении списка шаблонов из файла.

\subsubsection{Сравнение с шаблоном}
Как уже было отмечено в обзоре, при перепечатывании принтер обходит все элементы (PsiElement) дерева разбора и применяет к ним новые шаблоны. 
В рамках реализуемого подхода происходит перепечатывание лишь тех элементов, которые имели старое форматирование.
Это необходимо, если хочется поддерживать два и более различных варианта форматирования элементов некоторого типа. 
На рис. \ref{templates} изображены три блока оператора ветвления. 
Без фильтрации шаблонов все три блока будут изменены и получат одинаковое форматирование. %рис. сделать 
Однако первые два блока и так легко читаемы и понятны. 
Третий блок, будучи однострочным, уже не будет восприниматься так же хорошо.
Кроме того, существуют ограничения на ширину вывода, то есть в одной строке не можеть быть символов больше некоторой наперед заданной величины.

%\begin{figure}[t]
%    \centering
%    \lstinputlisting[language=C++]{codes/sameTmpltElems.txt}
%    \caption{Шаблоны}
%    \label{fig:sameTmpltElems}
%\end{figure}
%добавить вторую картинку как стало бы
\fvset{frame=lines,framesep=7pt}
\begin{figure}[ht]
\noindent\begin{minipage}{.5\textwidth}
    \lstinputlisting[language=C++]{codes/sameTmpltElems.txt}
\caption*{а) До переформатирования}    
\end{minipage}\hfill
\begin{minipage}{.5\textwidth}
    \lstinputlisting[language=C++]{codes/sameTmpltElems2.txt}
\caption*{б) После переформатирования}    
\end{minipage}
\caption{Переформатирование элементов одного типа без фильтрации шаблонов} 
\label{templates}
\end{figure}


Чтобы сделать настройку форматирования более гибкой и избавиться от подобного рода проблем, необходимо производить сравнение шаблонов текущего элемента дерева разбора и элемента со старым форматированием. 
Переформатировать элемент -- только в случае совпадения шаблонов.
Шаблоны совпадают, если совпадет их текстовое представление: шаблон записывается в виде строки, которая содержит необходимые элементы: ключевые слова, пробелы, переносы строк, скобки, а так же метки для поддеревьев, в которых указывается дополнительная информация, например, количество выражений в поддереве (одно или несколько).
В случае оператора ветвления (рис.~\ref{iftmpltcode}а) его поддеревьями являются: условие (condition), ветка, соответствующая выполнению условия (then block) и содержащая несколько подвыражений, ветка, соответствующая невыполнению условия (else block) и содержащая одно подвыражение. %ifTmplt ifTmpltCode
Соответствующий ему шаблон изображен на рисунке~\ref{iftmpltcode}б.
\fvset{frame=lines,framesep=7pt}
\begin{figure}[ht]
\noindent\begin{minipage}{.5\textwidth}
    \lstinputlisting[language=C++]{codes/ifTmpltCode.txt}
\caption*{а) Оператор ветвления}    
\end{minipage}\hfill
\begin{minipage}{.5\textwidth}
    \lstinputlisting[language=C++]{codes/ifTmplt.txt}
\caption*{б) Шаблон для оператора ветвления}    
\end{minipage}
\caption{Оператор ветвления и соответствующий ему шаблон}    
\label{iftmpltcode}
\end{figure}

%\begin{figure}[t]
%    \centering
%    \lstinputlisting[language=C++]{codes/sameTmpltElems.txt}
%    \caption{Шаблоны}
%    \label{fig:sameTmpltElems}
%\end{figure}
%добавить вторую картинку как стало бы

\subsection{Оценка производительности}
Описанный выше способ  давал очень низкую производительность. 
В таблице~\ref{table:slow} приведено время, затрачиваемое принтером на переформатирование файлов различной длины. 
Измерения производились на примере двух блоков: оператора ветвления и метода.

\begin{table}[h]
\begin{tabular}{cc|c|c|c|c|}
\cline{2-5}
\multicolumn{1}{ c| }{ } & \multicolumn{2}{p{6cm} |}{Формат. операторов ветвления} & \multicolumn{2}{p{5cm} |}{Форматирование методов} \\
\hline
\multicolumn{1}{ |c| }{Размер файла (строк)} & Кол-во & Время (сек.) & Кол-во & Время (сек.) \\ 
\hline
\multicolumn{1}{ |c| }{>7000}           & 520   & 35    & 660  & 58 \\
\multicolumn{1}{ |c| }{1000..2000}       & 145   & 2.7   & 90   & 4.4 \\
\multicolumn{1}{ |c| }{500..1000}       & 54    & 0.73  & 72   & 1.34 \\
\multicolumn{1}{ |c| }{0..500}          & 35    & 0.2   & 36   & 0.67 \\
\hline
\end{tabular}
\caption{Время переформатирования файлов I}
\label{table:slow}
\end{table}


Причина столь низкой производительности кроется в том, что для переформатирования элементов используется тот же способ, что и при форматировании по образцу:
принтер обходит все дерево разбора и сравнивает шаблоны.
%Однако нет необходимости проверять на совпадение шаблоны, полученные из каждого элемента дерева разбора, и шаблон, полученного из элемента со старым форматированием.
Однако нет необходимости сравнивать шаблон, полученный из элемента со старым форматированием, с шаблоном, полученным из каждого элемента дерева разбора.
Количество вариантов для переформатирования можно сузить лишь для элементов того же типа.

Для уменьшения затрат времени при обходе дерева элементов создается список кандидатов на форматирование (элементы того же типа, что и элемент со старым форматированием). 
После этого из элементов созданного списка извлекаются шаблоны.
Если полученный шаблон совпадает с шаблоном элемента со старым форматированием, то происходит перепечатывание элемента и замена его в дереве разбора.
Время, затрачиваемое в этом случае на переформатирование, приведено в таблице~\ref{table:fast}.

%\begin{table}[h]
%\begin{tabular}{ l | p{4cm} | p{4cm} }
%  Размер файла (строк) & Формат. операторов ветвления (сек.) & Формат. методов %(сек.) \\ 
%  \hline
%  >7000         & 6.6   & 5.8 \\
%  1000..2000    & 0.72  & 1   \\
%  500..1000     & 0.125 & 0.67 \\
%  0..500        & 0.145 & 0.325 \\
%\end{tabular}
%\caption{Время переформатирования файлов II}
%\label{table:fast}
%\end{table}


\begin{table}[h]
\begin{tabular}{cc|c|c|c|c|}
\cline{2-5}
\multicolumn{1}{ c| }{ } & \multicolumn{2}{p{6cm} |}{Формат. операторов ветвления} & \multicolumn{2}{p{5cm} |}{Форматирование  методов} \\
\hline
\multicolumn{1}{ |c| }{Размер файла (строк)} & Кол-во & Время (сек.) & Кол-во & Время (сек.) \\ 
\hline
\multicolumn{1}{ |c| }{>7000}           & 520   & 6.6   & 660  & 5.8 \\
\multicolumn{1}{ |c| }{1000..2000}      & 145   & 0.72  & 90   & 1 \\
\multicolumn{1}{ |c| }{500..1000}       & 54    & 0.225 & 72   & 0.67 \\
\multicolumn{1}{ |c| }{0..500}          & 35    & 0.145 & 36   & 0.325 \\
\hline
\end{tabular}
\caption{Время переформатирования файлов II}
\label{table:fast}
\end{table}
\section{Заключение}

В данной работе описан подход к композициональному символьному исполнению без раскрутки. Была предложена концепция композициональной памяти с символьной адресацией. Был доказан некоторый набор свойств КСП, дающий основание для подхода в стиле систем переписывания, где символьные кучи могут сами выступать как символы. Это даёт возможность автоматически порождать уравнения на состояния, решения которых в точности отражают поведения функций, работающих с динамической памятью. Было показано как свести задачу решения уравнений на состояния к задаче проверки безопасности чистых функций второго порядка.

Данная работа нацелена на теоретические основания композиционального анализа динамической памяти. Мы оставляем апробацию этого подхода на будущее. Другим направлением будущих исследований может быть расширение нашего формализма на композициональный анализ параллельных программ.

\begin{thebibliography}{99}
  \bibitem{podkopaev:diploma1}
  А.В.Подкопаев. Полиномиальной сложности оптимальные принтер-комбинаторы с выбором.
  Дипломная работа, кафедра системного программирования, математико-механический факультет СПбГУ, 2014.

  \bibitem{stratego:xt}
  Stratego/XT. \url{http://strategoxt.org/Stratego/StrategoDocumentation}, 2015.

  \bibitem{intellij:plugdev}
  IntelliJ IDEA Plugin Development Guide. \url{https://confluence.jetbrains.com/display/IDEADEV/PluginDevelopment},
  2015.
\end{thebibliography}

%\usepackage{listings}  
%\usepackage{hyperref}
%\usepackage{verbments}
%\usepackage{minted}
%\usepackage{caption}
%\usepackage{fancybox}
%\newfontfamily{\cyrillicfonttt}{CMU Serif}

\title{Декларативное форматирование в режиме онлайн}

\titlerunning{Декларативное форматирование в режиме онлайн}

\author{Озерных Игорь Станиславович}

\authorrunning{И.С.Озерных}

\tocauthor{И.С.Озерных}
\institute{Санкт-Петербургский государственный университет\\
\email{igor.ozernykh@gmail.com}}

\maketitle             

\begin{abstract}
Аккуратное, соответствующие выбранному стилю оформление (форматирование)
программных текстов критично для их восприятия.
Существуют средства автоматического форматирования кода. Одним
из них является метод синтаксических шаблонов. Его преимуществом является декларативность, которая
достигается за счет задания стиля форматирования с помощью образца кода. Так, естественным ограничением
описанного метода является необходимость наличия достаточно полного образца на целевом языке.
Данная работа посвящена расширению границ применимости метода синтаксических шаблонов
за счет возможности задания шаблонов в режиме онлайн.
\end{abstract}

\section*{Введение}

Этап поддержки и сопровождения в жизненном цикле программного обеспечения может занимать до 90\% времени существования программного продукта. На этом этапе особенно важно понимание программного текста поддерживаемой системы, что зачастую бывает сложной задачей. Необходимым условием для лучшего понимания программного текста является его аккуратное оформление, отражающее структуру 
программы. Существуют различные стандарты оформления кода (стили кодирования)~--- наборы правил и соглашений, используемые при написании исходного кода на некотором языке программирования.
Рассмотрим их на примере стандарта кодирования, принятого в компании Google\footnote{\texttt{https://google-styleguide.googlecode.com/svn/trunk/cppguide.html}}, и стандарта, используемого в проекте GNU\footnote{\texttt{http://www.gnu.org/prep/standards/standards.html}} (см. рис.~\ref{codingstandards}) для языка C++.


\fvset{frame=lines,framesep=7pt}
\begin{figure}[ht]
\noindent\begin{minipage}{.5\textwidth}
    \lstinputlisting[language=Java]{Ozernykh/codes/exGoogleCC.txt}
\caption*{а) Стиль кодирования Google}    
\end{minipage}\hfill
\begin{minipage}{.5\textwidth}
    \lstinputlisting[language=Java]{Ozernykh/codes/exGNUCC.txt}
\caption*{б) Стиль кодирования GNU}    
\end{minipage}
\caption{Различные стили кодирования для языка С++}    
\label{codingstandards}
\end{figure}

Когда программист работает над проектом, он придерживается некоторого стиля кодирования. 
Если файлы проекта с одним стилем кодирования присоединяется к уже существующему проекту с другим стилем кодирования, то их форматирование нужно привести к тому же виду. 
Для решения этой задачи можно воспользоваться уже существующими средствами, например, форматтерами (программами, которые форматируют исходный текст) в IDE таких, как Eclipse, IntelliJ IDEA, Visual Studio и др. 
Однако в этом случае необходимо вручную задавать настройки форматирования.
Кроме того, количество принципиально разных стилей форматирования, которые можно задать с помощью данных настроек, невелико.

%Однако в этом случае необходимо иметь настройки форматирования, заданные программистом, но в целом, количество принципиально разных стилей форматировани, которые можно задать с помощью данных настроек, невелико. %причем эти настройки могут быть различными в разных IDE.
% * <Anton Podkopaev> 12:15:51 22 May 2015 UTC+0300:
% Нужно переработать это предложение.

Еще одним примером может послужить принтер-плагин для IntelliJ IDEA~\cite{podkopaev:diploma1}, который позволяет форматировать исходный код проекта по образцу. 
В качестве образца используется код из некоторого репозитория. 
Из него выделяются шаблоны форматирования для структур языка, которые применяются к структурам целевого кода. 
Под \textit{шаблоном} понимаются данные, сопоставление которых с элементом синтаксического дерева дает текстовое представление этого элемента (и его потомков). 
Например, на рис.~\ref{fig:ifTree} представлено дерево разбора для оператора ветвления. 


\begin{figure}[h]
	\centering
	\includegraphics[width=\textwidth]{Ozernykh/images/ifTree.jpg}
	\caption{Представление оператора ветвления в виде дерева разбора}
	\label{fig:ifTree}
\end{figure}

На рис.~\ref{fig:tmpltcodeintro}а изображен шаблон форматирования, который может быть применен к нему (несколько подвыражений, соответствующих выполнению условия, и одно подвыражение, соответствующее невыполненению условия).
На рис.~\ref{fig:tmpltcodeintro}б представлен результат применения этого шаблона к дереву разбора для оператора ветвления.

% \begin{figure}[h]
% 	\centering
% 	\lstinputlisting[language=c++]{codes/ifTmpltIntro.txt}
% 	\caption{Шаблон для оператора ветвления}
% 	\label{fig:ifTmpltIntro}
% \end{figure}

% На рис.~\ref{fig:ifCodeIntro} представлен результат применения этого шаблона к дереву разбора для оператора ветвления.

% \begin{figure}[h]
% 	\centering
% 	\lstinputlisting[language=c++]{codes/ifCodeIntro.txt}
% 	\caption{Представление дерева разбора при применении шаблона}
% 	\label{fig:ifCodeIntro}
% \end{figure}


\fvset{frame=lines,framesep=7pt}
\begin{figure}[ht]
\noindent\begin{minipage}{.5\textwidth}
    \lstinputlisting[language=Java]{Ozernykh/codes/ifTmpltIntro.txt}
\caption*{а) Шаблон для оператора\\
ветвления}    
\end{minipage}\hfill
\begin{minipage}{.5\textwidth}
    \lstinputlisting[language=Java]{Ozernykh/codes/ifCodeIntro.txt}
\caption*{б) Текст, полученный при применении шаблона к дереву разбора}    
\end{minipage}
\caption{Оператор ветвления и шаблон для него}    
\label{fig:tmpltcodeintro}
\end{figure}

Однако не всегда этот репозиторий существует. Кроме того, необходимо наличие заданного форматирования для всех структур языка.
% * <Anton Podkopaev> 12:19:02 22 May 2015 UTC+0300:
% 'Однако не всегда удобно иметь этот репозиторий под рукой.' --- поменять. Не про удобство надо говорить.

Шаблоны можно извлекать из уже существующего кода и применять их к другим элементам того же типа, но отформатированных иным способом. 
Например, пользователь меняет форматирование некоторой структуры языка (в частности, оператор ветвления), программа извлекает из полученного участка кода новый шаблон и применяет его к структурам того же типа, содержащим старое форматирование. 
Назовем такой способ задания шаблонов~--- форматирование в режиме
онлайн.
%картинки

Целью данной работы является расширение функциональности описанного 
принтер-плагина путем добавления возможности задания шаблонов в режиме онлайн. 

\section{Обзор предметной области}

Одним из подходов к заданию принтеров являются принтер-комбинаторы.
В этой главе приведен обзор классических принтер-комбинаторов, а также
принтер-комбинаторов с дополнительным оператором \emph{выбора}, которые
в рамках данной работы использованы для реализации принтеров, задаваемых
с помощью шаблонов. Для решения проблем эффективности
комбинаторов с выбором в данной работе был использован подход BURS.
Апробация подхода задания принтеров с помощью шаблонов
была произведена для языка Java, для которого есть средства форматирования,
встроенные в IDE.

\subsection{Принтер-комбинаторы}

Базовыми принтер-комбинаторными библиотеками являются библиотеки
Джона Хьюза\cite{hughes} и Филиппа Вадлера\cite{wadler}, которые
представляют собой функциональную переработку алгоритма
Оппена\cite{oppen}.
Ограничимся рассмотрением библиотеки Вадлера, так как
библиотека Хьюза обладает теми же свойствами, которые принципиальны в контексте
данной работы.

В библиотеке ключевым типом является документ.
Он представляет сущность, которая потом может
быть переведена в строковое представление алгоритмом принтера.
Основные конструкторы для составления документа таковы:
\begin{itemize}
  \item атомарная строка, которая печатается как есть;
  \item \emph{разделитель};
  \item последовательная композиция двух документов;
  \item набор связанных документов.
\end{itemize}

Определяющей особенностью данного подхода является то, что все разделители
в рамках одного набора могут быть совместно заменены алгоритмом принтера
на пробельный символ или на перевод строки. Выбор для каждого набора разделителей
основывается на том, что вывод должен поместиться в заданную ширину,
используя минимальное число строк.

Основная проблема такого подхода заключается в его слабой выразительной силе.
Документы, построенные по синтаксическому дереву печатаемой программы,
обрабатываются слишком единообразно, что иногда приводит к нежелательному результату.
Пусть, к примеру, нужно напечатать программу на языке
Python\footnote{\cd{http://python.org}}.
Между последовательными операторами в случае их печати на одной строчке необходимо
добавить дополнительный разделитель (``;''), иначе программа станет некорректной
(см. рис.~\ref{fig:seqEx}). Однако, описанные принтер-комбинаторы не предоставляют
возможности задать такое поведение.
Кроме того, с помощью таких принтер-комбинаторов невозможно выразить разные проектные СК,
так как они всегда печатают текст в одном стиле, который жестко ``зашит'' в их код.

\begin{figure}[h!]
	\centering
	\null\hfill
	\subfloat[Корректный код]{
		\centering
    \makebox[.4\textwidth] {
		  \lstinputlisting[language=Python]{codes/pythonCode.py}
    }
	}
	\null\hfill
	\subfloat[Некорректный код]{
		\centering
    \makebox[.4\textwidth] {
	  	\lstinputlisting[language=Python]{codes/pythonCodeBad.py}
    }
	}
	\hfill\null
	\caption{Пример работы принтер-комбинаторов для языка Python}
  \label{fig:seqEx}	
\end{figure}

Более подробный обзор библиотек Хьюза и Вадлера представлен в \cite{myCoursePaper}.

Большинство принтер-комбинаторных библиотек\cite{
swierstraChitil, swierstra04, peytonJones, kiselyov, chitil}
являются развитиями работ Хьюза и Вадлера. Так, в отличие от базовых, среди
них есть реализации с линейной сложностью обработки документа от его размера и
\emph{online} алгоритмы, которые не требуют просмотра всего документа
для начала печати его текстового представления. Но все они обладают тем же
интерфейсом, а значит в них также неразрешимы задачи, в которых требуется
задавать варианты текстовых представлений, которые отличаются не только
пробелами и переводами строк.

\subsection{Принтер-комбинаторы с выбором}

Существенно от описанных выше отличаются библиотеки, предоставляющие в своем
интерфейсе комбинатор \emph{выбора}, который позволяет задавать для одного поддерева
принципиально разные варианты раскладок.
Так, в работах~\cite{jongeEveryOccasion, jongeReengine} используется
оператор \lstinline[language = Haskell]{ALT},
но алгоритм принтера устроен так, что среди двух альтернатив выбирается первая, если она
помещается в заданную ширину, и вторая --- иначе,
что не дает оптимальный результат на выходе.

Оптимальные принтер-комбинаторы с выбором были впервые представлены в работе~\cite{swierstra}.
Их реализация является частью Utrecht Tools
Library\footnote{\cd{http://www.cs.uu.nl/wiki/HUT/WebHome}}
(практическая реализация несколько изменена по
отношению к той, что описана в статье, но отличие несущественно).
В данном подходе текст строится из блоков прямоугольной формы с возможно неполной последней
строчкой (см. рис. \ref{fig:basicFormat}). В реализации на Haskell блоки представляются
структурой \lstinline[language = Haskell]{Format}:
\begin{lstlisting}
    data Format = Elem { height        :: Int
                       , lastLineWidth :: Int
                       , width         :: Int
                       , txtstr        :: Int -> String -> String
                       }
\end{lstlisting}

Первые три поля структуры определяют геометрические размеры блока, а последнее --- функция,
которая используется для преобразования блока в текст. Здесь используется функция, а не
просто строчка, чтобы можно было преобразовывать вложенные блоки за линейное время. Первый
аргумент \lstinline[language = Haskell]{txtstr} задает сдвиг блока.

\begin{figure}
  \centering
  \subfloat[Блок текста]{
    \raisebox{5mm}{
      \centering
      \vspace{0pt}
      \tikz[scale = 2.0]{
        \draw (0,0) -- (1,0) -- (1,0.2) -- (2,0.2) -- (2,1) -- (0, 1) -- cycle;
      }
    }
    \label{fig:basicFormat}
  }
  ~
  \subfloat[Горизонтальная композиция]{
    \centering
    \tikz[scale = 2.0]{
       \draw (0,0) -- (1,0) -- (1,0.2) -- (2,0.2) -- (2,1) -- (0, 1) -- cycle;
       \draw (1.1,-0.9) -- (2.1,-0.9) -- (2.1,-0.7) -- (3.1,-0.7) -- (3.1,0.1) -- (1.1,0.1) -- cycle;
     }
     \label{fig:beside}
  }
  %\hfill
  ~
  \subfloat[Вертикальная композиция]{
    \makebox[.28\textwidth] {
    \centering
    \tikz[scale = 2.0]{
       \draw (0,0) -- (1,0) -- (1,0.2) -- (2,0.2) -- (2,1) -- (0, 1) -- cycle;
       \draw (0,-1.1) -- (1,-1.1) -- (1,-0.9) -- (2,-0.9) -- (2,-0.1) -- (0,-0.1) -- cycle;
    }
    \label{fig:above}
    }
  }
  \caption{Примитивы блока текста}
  \label{fig:basicConcat}
\end{figure}

Для работы с блоками текста используется следующие четыре примитива:
\begin{lstlisting}
    s2fmt     :: String -> Format
    indentFmt :: Int -> Format -> Format
    aboveFmt  :: Format -> Format -> Format
    besideFmt :: Format -> Format -> Format
\end{lstlisting}

Функция \lstinline[language = Haskell]{s2fmt} создает блок текста,
состоящий из одной строчки; \lstinline[language = Haskell]{indentFmt} по блоку создает новый,
сдвинутый на заданное число позиций. Действие примитивов композиции
\lstinline[language = Haskell]{besideFmt} и \lstinline[language = Haskell]{aboveFmt}
показано на рис.~\ref{fig:beside} и \ref{fig:above} соответственно.

Также, как и в библиотеках без комбинатора выбора, в работе~\cite{swierstra} используется понятие
\emph{документа}. Здесь документ можно рассматривать как множество возможных раскладок
(набор \lstinline[language = Haskell]{Format}-элементов). Документы описываются типом
\lstinline[language = Haskell]{Doc}, экземпляры которого на этапе построения представляют собой
деревья применений приведенных ниже комбинаторов, а на этапе обработки алгоритмом
принтера им в соответствие ставится итоговое множество вариантов текстовых представлений.

Документ конструируется с помощью следующих комбинаторов, которые симметричны
примитивам построения блоков текста:
\begin{lstlisting}
    text   :: String -> Doc
    indent :: Int -> Doc -> Doc
    beside :: Doc -> Doc -> Doc
    above  :: Doc -> Doc -> Doc
\end{lstlisting}

В дополнение появляется пятый комбинатор для документов:
\begin{lstlisting}
    choice :: Doc -> Doc -> Doc
\end{lstlisting}

Этот комбинатор и является тем самым комбинатором выбора. Он представляет
\emph{объединение} множеств раскладок документов, которые были переданы как
аргументы комбинатора. Заметим, что только этот комбинатор может произвести
документ с несколькими раскладками из одновариантных аргументов.

Оригинальная реализация существенно опирается на ленивые вычисления. В \cite{swierstra}
множество вариантов, соответствующее экземпляру \lstinline[language = Haskell]{Doc},
представляется ленивым списком всех возможных раскладок, удовлетворяющих ограничению
на максимальную ширину. Этот список отсортирован в порядке ``ухудшения''
раскладок, то есть в голове списка лежит ``лучшая'', в терминах оптимальности,
раскладка из возможных при заданной ширине. В случае
\lstinline{beside}- и
\lstinline{above}-композиций
документов полная (без учета ленивости) сложность вычисления списка нового документа
составляет $O(n \times m)$, где $n$ и $m$ --- размеры списков, соответствующих 
соединяемым документам. Размер нового списка также порядка $n \times m$.
Он не обязательно точно равен $n \times m$, так как некоторые из полученных
представлений могут не подпадать под ограничение на ширину, соответственно их можно
отбросить на этапе построения нового списка.

Выбор лучшего представления для самого верхнеуровнего документа происходит
просто --- нужно из соответствующего списка взять первый элемент.
Это создает впечатление, что общее число операций,
благодаря ленивым вычислениям, существенно уменьшается.
Но это не так из-за реализации обработки документа,
построенного с помощью комбинатора \lstinline[language = Haskell]{beside}, которая
вынуждает полное вычисление дочерних списков. Более того, из-за свойств отношения
порядка, построенного по критерию оптимальности, в рамках данной модели списков
принципиально нельзя построить ленивую обработку комбинатора
\lstinline[language = Haskell]{beside}.
Так, выбор оптимальной раскладки документа в
\cite{swierstra} имеет в худшем случае экспоненциальную сложность от числа
комбинаторов, использованных при его построении.
В \cite{swiComb} приведены оптимизации для данной модели, но они
не улучшают асимптотику решения для деревьев в общем случае.

\subsection{BURS}

Bottom-Up Rewrite System (BURS)\cite{burs} --- это метод динамического
программирования на деревьях, изначально появившийся в контексте задачи выбора
инструкций для генерации машинного кода. Основой BURS является
регулярная грамматика с древовидными правилами\cite{tata}, для которых задана
стоимость применения подстановки,
то есть грамматика со следующим набором правил: 

$$
\begin{array}{rcll}
  N &:& \alpha& [c]\\
  N &:& \alpha\; (K_1,\dots,K_n)& [c]
\end{array}
$$

Здесь $N, K_i$ это нетерминалы, $\alpha$ --- терминал,
$c$ --- функция стоимости, заданная для каждого правила.
Как и для обычной линейной грамматики, вводится стартовый
нетерминал S. Считается, что терминальное дерево выводится в
данной грамматике, если его можно получить с помощью правил
подстановки из одноузлового дерева $S$.
Каждая подстановка заменяет нетерминал $N$, находящийся в листе дерева, на дерево 
$\alpha\;(K_1,\dots,K_n)$, если в грамматике есть правило
$N:\alpha\;(K_1,\dots,K_n)$. 
Для каждой подстановки вычисляется стоимость ее применения с помощью
функции стоимости $c$.
Аргументами функции могут служить терминальная метка $\alpha$ и стоимости
вывода поддеревьев.
Задачей, которую решает BURS, является поиск вывода наименьшей стоимости
для заданного дерева по заданной грамматике.
Такой вывод может быть найден двухпроходным алгоритмом.

Первый проход (\emph{пометка}) обрабатывает дерево снизу вверх и вычисляет
для каждого узла набор троек $(K,\;R,\;c)$, где $K$ --- нетерминал, из которого может быть
выведено поддерево с корнем в обрабатываемом узле,
$R$ --- первое правило, которое используется для вывода минимальной стоимости из $K$,
$c$ --- стоимость такого вывода.
Процесс пометки происходит следующим образом:

\begin{itemize}
\item для листовой вершины, помеченной терминалом $\alpha$, в множество троек
этого вершины добавляется $(K,\;R,\;c\:(\alpha))$ для каждого правила $R=K:\;\alpha\;[c]$;
\item для промежуточной вершины, помеченной терминалом $\alpha$,
с непосредственными поддеревьями $v_1,\dots,v_n$
в множество добавляется тройка $(K,\;R,\;c\:(\alpha,c_1,\dots,c_n))$ для каждого правила
$R=K:\;\alpha\;(K_1,\dots,K_n)\;[c]$, где $(K_i,\;R_i,\;c_i)$ входит в множество троек для
$v_i$; если есть несколько правил вывода из нетерминала $K$, то выбирается правило,
минимизирующее стоимость вывода.
\end{itemize}

Второй проход (\emph{свертка}) просматривает дерево сверху вниз, используя сделанные пометки.
Первое правило из минимального вывода определяется тройкой $(S,\;R,\;c)$ для корневого узла
(если такой тройки нет, то вывод из $S$ невозможен).
Это правило однозначно определяет нетерминалы $K_i$ для каждого непосредственного поддерева
и процесс повторяется.

Для пометки дерева потенциально необходимо каждое правило грамматики применить к каждому узлу.
При фиксированной грамматике алгоритмическая сложность первого прохода --- $O\:(|R|)$,
где $|R|$ --- количество правил
(размер множества троек для каждого узла ограничен числом нетерминалов, которое не больше,
чем количество правил). Свертка также имеет линейную сложность.


\newpage
\subsection{Средства форматирования кода в IDE}

Интегрированные среды разработки программного обеспечения
(IDE) предоставляют средства
для форматирования программного кода. Рассмотрим их на примере двух Java IDE ---
IntelliJ IDEA\footnote{\cd{http://jetbrains.com/idea/}} и
Eclipse\footnote{\cd{http://eclipse.org}}.
Для задания требуемого СК используется широкий набор настроек \emph{форматтера},
подпрограммы IDE, отвечающей за форматирование исходных текстов.
Примерами таких настроек являются:
\begin{itemize}
  \item помещать ли фигурную скобку на той же строке, что и
    предыдущее выражение;
  \item форматирование списков --- всегда на одной строчке, переводить строчку после
    каждого элемента или печатать на одной строке,
    пока строчка меньше заданной рекомендуемой ширины.
%   \item и т.д.
\end{itemize}
На рис. ~\ref{fig:ideaFormatter} и~\ref{fig:eclipseFormatter} приведены диалоги задания
параметров форматирования в IntelliJ IDEA и Eclipse соответственно.

\begin{figure}[p]
	\centering
	\includegraphics[width=\textwidth]{ideaFormatter}
	\caption{Окно настройки форматтера IntelliJ IDEA}
	\label{fig:ideaFormatter}
\end{figure}

\begin{figure}[p]
	\centering
	\includegraphics[width=\textwidth]{eclipseFormatter}
	\caption{Окно настройки форматтера Eclipse}
	\label{fig:eclipseFormatter}
\end{figure}

У встроенных форматтеров есть следующие особенности.
Во-первых, это невозможность выразить
нестандартный СК, так как настройки форматтеров, несмотря на их большое
количество,
дают ограниченную вариативность для текстовых представлений форматируемых
программ. В целом, количество принципиально разных
стилей форматирования, которые можно задать с помощью данных настроек,
невелико.
Во-вторых, в случае, если надо
придерживаться СК уже существующего проекта, то по коду этого проекта необходимо
вручную задать все настройки форматтера.
Причем наборы настроек в разных IDE не совпадают, что увеличивает сложность
поддержки единого СК, если разработчики используют отличные от друг друга IDE.
Для решения данной проблемы существуют плагины, позволяющие использовать
внешние фоматтеры\footnote{\cd{http://plugins.jetbrains.com/plugin/6546}}.
Кроме того, есть средства, позволяющие экспортировать настройки форматтера в
XML-файл для их использования в другой
IDE\footnote{\cd{http://blog.jetbrains.com/idea/2014/01/intellij-idea-13-importing-code-formatter-settings-from-eclipse/}}.

Важным достоинством описанных форматтеров является малое время работы.
Это достигается в том числе и за счет того, что часто форматируется не весь
файл, а только его часть (см. \cite{eclipse}),
и из-за детерминированности представлений узлов синтаксического дерева,
которая следует из специфики настроек форматтера.
Однако форматирование целого большого проекта занимает существенное время.
Например, обработка кода IntelliJ IDEA Community Edition занимает более часа
у встроенного в IDEA форматтера.


\section{Реализация}
В процессе работы над проектом было выделено две основных задачи: реализация
взаимодействия с пользователем и расширение функциональности для переформатирования программного текста с фильтрацией шаблонов.
%диаграмму что ли нарисавать?
Предполагаемый процесс взаимодействия с пользователем: пользователь выделяет
участок программного кода, в котором будут производиться изменение, и вызывает 
действие (Action), которое извлекает шаблон из выделенного участка
кода. Затем пользователь меняет форматирование этого же участка, выделяет его
снова и вызывает действие, которое извлекает новый шаблон форматирования и 
сохраняет его.

После этого происходит перепечатывание элементов (PsiElement), имеющих старое
форматирование, в файле, в котором пользователь менял программный текст.

\subsection{Переформатирование}
\subsubsection{Извлечение элемента для форматирования}
% тут будет диаграмма:
% 1) Получаем текст
% 2) Корректируем отступы + пример
% 3) Извлечение элемента (для того чтобы знать тип) -- вся рутина
% 4) Склеить текст и элемент
В первую очередь необходимо получить элемент, форматирование которого пользователь хочет изменить.
% Где-то надо ссылку на диаграмму, наверное, тут
Весь этот процесс изображен на рис.~\ref{fig:getElemDiag}.

\begin{figure}[h]
	\centering
	\includegraphics[width=\textwidth]{images/getElemDiag.jpg}
	\caption{Диаграмма этапов извлечения элемента из текста}
	\label{fig:getElemDiag}
\end{figure}

На первом этапе с помощью функций, предоставляемых IntelliJ IDEA API, можно получить текст элемента. 
Однако этот текст не всегда корректный, так как может содержать в себе лишние отступы.
Например, на рис. \ref{fig:offsetExample} изображены методы doSomething() и doAnotherThing(). 
Они имеют отступ относительно оператора ветвления, равный четырем пробелам.
Кроме того, они имеют отступ относительно метода foo(), равный восьми пробелам. 
Чтобы шаблон был корректным, на втором этапе убираются лишние пробелы (в данном случае четыре).
Далее получаем из текста сам элемент. 
Это необходимо, чтобы узнать его тип. %ссылку на дерево PsiElement'ов.
Затем (на четвертом этапе) из корректного текста и полученного элемента создается новый элемент того же типа, но с корректными отступами.

Все это необходимо проделать, так как IntelliJ IDEA API не предоставляет функционала по изменению текста элемента явным образом. 

\begin{figure}[h]
	\centering
	\lstinputlisting[language=C++]{codes/offsetExample.txt}
	\caption{Общий отступ элементов}
	\label{fig:offsetExample}
\end{figure}

Аналогичные действия необходимо повторить для получения элемента из измененного
пользователем текста. 
Из этого элемента извлекается шаблон и сохраняется принтером для дальнейшего использования.

%\subsubsection{Извлечение шаблона}
%Шаблон, как и прежде, извлекается методами класса \lstinline[language=C++]{PsiElementComponent}.

%\subsubsection{Добавление шаблона}
%В связи с тем, что возникла необходимость добавления одного шаблона, класс %Printer обзавелся публичным методом, реализующим этот процесс. Остальные
%действия аналогичны тем при заполнении списка шаблонов из файла.

\subsubsection{Сравнение с шаблоном}
Как уже было отмечено в обзоре, при перепечатывании принтер обходит все элементы (PsiElement) дерева разбора и применяет к ним новые шаблоны. 
В рамках реализуемого подхода происходит перепечатывание лишь тех элементов, которые имели старое форматирование.
Это необходимо, если хочется поддерживать два и более различных варианта форматирования элементов некоторого типа. 
На рис. \ref{templates} изображены три блока оператора ветвления. 
Без фильтрации шаблонов все три блока будут изменены и получат одинаковое форматирование. %рис. сделать 
Однако первые два блока и так легко читаемы и понятны. 
Третий блок, будучи однострочным, уже не будет восприниматься так же хорошо.
Кроме того, существуют ограничения на ширину вывода, то есть в одной строке не можеть быть символов больше некоторой наперед заданной величины.

%\begin{figure}[t]
%    \centering
%    \lstinputlisting[language=C++]{codes/sameTmpltElems.txt}
%    \caption{Шаблоны}
%    \label{fig:sameTmpltElems}
%\end{figure}
%добавить вторую картинку как стало бы
\fvset{frame=lines,framesep=7pt}
\begin{figure}[ht]
\noindent\begin{minipage}{.5\textwidth}
    \lstinputlisting[language=C++]{codes/sameTmpltElems.txt}
\caption*{а) До переформатирования}    
\end{minipage}\hfill
\begin{minipage}{.5\textwidth}
    \lstinputlisting[language=C++]{codes/sameTmpltElems2.txt}
\caption*{б) После переформатирования}    
\end{minipage}
\caption{Переформатирование элементов одного типа без фильтрации шаблонов} 
\label{templates}
\end{figure}


Чтобы сделать настройку форматирования более гибкой и избавиться от подобного рода проблем, необходимо производить сравнение шаблонов текущего элемента дерева разбора и элемента со старым форматированием. 
Переформатировать элемент -- только в случае совпадения шаблонов.
Шаблоны совпадают, если совпадет их текстовое представление: шаблон записывается в виде строки, которая содержит необходимые элементы: ключевые слова, пробелы, переносы строк, скобки, а так же метки для поддеревьев, в которых указывается дополнительная информация, например, количество выражений в поддереве (одно или несколько).
В случае оператора ветвления (рис.~\ref{iftmpltcode}а) его поддеревьями являются: условие (condition), ветка, соответствующая выполнению условия (then block) и содержащая несколько подвыражений, ветка, соответствующая невыполнению условия (else block) и содержащая одно подвыражение. %ifTmplt ifTmpltCode
Соответствующий ему шаблон изображен на рисунке~\ref{iftmpltcode}б.
\fvset{frame=lines,framesep=7pt}
\begin{figure}[ht]
\noindent\begin{minipage}{.5\textwidth}
    \lstinputlisting[language=C++]{codes/ifTmpltCode.txt}
\caption*{а) Оператор ветвления}    
\end{minipage}\hfill
\begin{minipage}{.5\textwidth}
    \lstinputlisting[language=C++]{codes/ifTmplt.txt}
\caption*{б) Шаблон для оператора ветвления}    
\end{minipage}
\caption{Оператор ветвления и соответствующий ему шаблон}    
\label{iftmpltcode}
\end{figure}

%\begin{figure}[t]
%    \centering
%    \lstinputlisting[language=C++]{codes/sameTmpltElems.txt}
%    \caption{Шаблоны}
%    \label{fig:sameTmpltElems}
%\end{figure}
%добавить вторую картинку как стало бы

\subsection{Оценка производительности}
Описанный выше способ  давал очень низкую производительность. 
В таблице~\ref{table:slow} приведено время, затрачиваемое принтером на переформатирование файлов различной длины. 
Измерения производились на примере двух блоков: оператора ветвления и метода.

\begin{table}[h]
\begin{tabular}{cc|c|c|c|c|}
\cline{2-5}
\multicolumn{1}{ c| }{ } & \multicolumn{2}{p{6cm} |}{Формат. операторов ветвления} & \multicolumn{2}{p{5cm} |}{Форматирование методов} \\
\hline
\multicolumn{1}{ |c| }{Размер файла (строк)} & Кол-во & Время (сек.) & Кол-во & Время (сек.) \\ 
\hline
\multicolumn{1}{ |c| }{>7000}           & 520   & 35    & 660  & 58 \\
\multicolumn{1}{ |c| }{1000..2000}       & 145   & 2.7   & 90   & 4.4 \\
\multicolumn{1}{ |c| }{500..1000}       & 54    & 0.73  & 72   & 1.34 \\
\multicolumn{1}{ |c| }{0..500}          & 35    & 0.2   & 36   & 0.67 \\
\hline
\end{tabular}
\caption{Время переформатирования файлов I}
\label{table:slow}
\end{table}


Причина столь низкой производительности кроется в том, что для переформатирования элементов используется тот же способ, что и при форматировании по образцу:
принтер обходит все дерево разбора и сравнивает шаблоны.
%Однако нет необходимости проверять на совпадение шаблоны, полученные из каждого элемента дерева разбора, и шаблон, полученного из элемента со старым форматированием.
Однако нет необходимости сравнивать шаблон, полученный из элемента со старым форматированием, с шаблоном, полученным из каждого элемента дерева разбора.
Количество вариантов для переформатирования можно сузить лишь для элементов того же типа.

Для уменьшения затрат времени при обходе дерева элементов создается список кандидатов на форматирование (элементы того же типа, что и элемент со старым форматированием). 
После этого из элементов созданного списка извлекаются шаблоны.
Если полученный шаблон совпадает с шаблоном элемента со старым форматированием, то происходит перепечатывание элемента и замена его в дереве разбора.
Время, затрачиваемое в этом случае на переформатирование, приведено в таблице~\ref{table:fast}.

%\begin{table}[h]
%\begin{tabular}{ l | p{4cm} | p{4cm} }
%  Размер файла (строк) & Формат. операторов ветвления (сек.) & Формат. методов %(сек.) \\ 
%  \hline
%  >7000         & 6.6   & 5.8 \\
%  1000..2000    & 0.72  & 1   \\
%  500..1000     & 0.125 & 0.67 \\
%  0..500        & 0.145 & 0.325 \\
%\end{tabular}
%\caption{Время переформатирования файлов II}
%\label{table:fast}
%\end{table}


\begin{table}[h]
\begin{tabular}{cc|c|c|c|c|}
\cline{2-5}
\multicolumn{1}{ c| }{ } & \multicolumn{2}{p{6cm} |}{Формат. операторов ветвления} & \multicolumn{2}{p{5cm} |}{Форматирование  методов} \\
\hline
\multicolumn{1}{ |c| }{Размер файла (строк)} & Кол-во & Время (сек.) & Кол-во & Время (сек.) \\ 
\hline
\multicolumn{1}{ |c| }{>7000}           & 520   & 6.6   & 660  & 5.8 \\
\multicolumn{1}{ |c| }{1000..2000}      & 145   & 0.72  & 90   & 1 \\
\multicolumn{1}{ |c| }{500..1000}       & 54    & 0.225 & 72   & 0.67 \\
\multicolumn{1}{ |c| }{0..500}          & 35    & 0.145 & 36   & 0.325 \\
\hline
\end{tabular}
\caption{Время переформатирования файлов II}
\label{table:fast}
\end{table}
\section{Заключение}

В данной работе описан подход к композициональному символьному исполнению без раскрутки. Была предложена концепция композициональной памяти с символьной адресацией. Был доказан некоторый набор свойств КСП, дающий основание для подхода в стиле систем переписывания, где символьные кучи могут сами выступать как символы. Это даёт возможность автоматически порождать уравнения на состояния, решения которых в точности отражают поведения функций, работающих с динамической памятью. Было показано как свести задачу решения уравнений на состояния к задаче проверки безопасности чистых функций второго порядка.

Данная работа нацелена на теоретические основания композиционального анализа динамической памяти. Мы оставляем апробацию этого подхода на будущее. Другим направлением будущих исследований может быть расширение нашего формализма на композициональный анализ параллельных программ.

\begin{thebibliography}{99}
  \bibitem{podkopaev:diploma1}
  А.В.Подкопаев. Полиномиальной сложности оптимальные принтер-комбинаторы с выбором.
  Дипломная работа, кафедра системного программирования, математико-механический факультет СПбГУ, 2014.

  \bibitem{stratego:xt}
  Stratego/XT. \url{http://strategoxt.org/Stratego/StrategoDocumentation}, 2015.

  \bibitem{intellij:plugdev}
  IntelliJ IDEA Plugin Development Guide. \url{https://confluence.jetbrains.com/display/IDEADEV/PluginDevelopment},
  2015.
\end{thebibliography}

%\usepackage{listings}  
%\usepackage{hyperref}
%\usepackage{verbments}
%\usepackage{minted}
%\usepackage{caption}
%\usepackage{fancybox}
%\newfontfamily{\cyrillicfonttt}{CMU Serif}

\title{Декларативное форматирование в режиме онлайн}

\titlerunning{Декларативное форматирование в режиме онлайн}

\author{Озерных Игорь Станиславович}

\authorrunning{И.С.Озерных}

\tocauthor{И.С.Озерных}
\institute{Санкт-Петербургский государственный университет\\
\email{igor.ozernykh@gmail.com}}

\maketitle             

\begin{abstract}
Аккуратное, соответствующие выбранному стилю оформление (форматирование)
программных текстов критично для их восприятия.
Существуют средства автоматического форматирования кода. Одним
из них является метод синтаксических шаблонов. Его преимуществом является декларативность, которая
достигается за счет задания стиля форматирования с помощью образца кода. Так, естественным ограничением
описанного метода является необходимость наличия достаточно полного образца на целевом языке.
Данная работа посвящена расширению границ применимости метода синтаксических шаблонов
за счет возможности задания шаблонов в режиме онлайн.
\end{abstract}

\section*{Введение}

Этап поддержки и сопровождения в жизненном цикле программного обеспечения может занимать до 90\% времени существования программного продукта. На этом этапе особенно важно понимание программного текста поддерживаемой системы, что зачастую бывает сложной задачей. Необходимым условием для лучшего понимания программного текста является его аккуратное оформление, отражающее структуру 
программы. Существуют различные стандарты оформления кода (стили кодирования)~--- наборы правил и соглашений, используемые при написании исходного кода на некотором языке программирования.
Рассмотрим их на примере стандарта кодирования, принятого в компании Google\footnote{\texttt{https://google-styleguide.googlecode.com/svn/trunk/cppguide.html}}, и стандарта, используемого в проекте GNU\footnote{\texttt{http://www.gnu.org/prep/standards/standards.html}} (см. рис.~\ref{codingstandards}) для языка C++.


\fvset{frame=lines,framesep=7pt}
\begin{figure}[ht]
\noindent\begin{minipage}{.5\textwidth}
    \lstinputlisting[language=Java]{Ozernykh/codes/exGoogleCC.txt}
\caption*{а) Стиль кодирования Google}    
\end{minipage}\hfill
\begin{minipage}{.5\textwidth}
    \lstinputlisting[language=Java]{Ozernykh/codes/exGNUCC.txt}
\caption*{б) Стиль кодирования GNU}    
\end{minipage}
\caption{Различные стили кодирования для языка С++}    
\label{codingstandards}
\end{figure}

Когда программист работает над проектом, он придерживается некоторого стиля кодирования. 
Если файлы проекта с одним стилем кодирования присоединяется к уже существующему проекту с другим стилем кодирования, то их форматирование нужно привести к тому же виду. 
Для решения этой задачи можно воспользоваться уже существующими средствами, например, форматтерами (программами, которые форматируют исходный текст) в IDE таких, как Eclipse, IntelliJ IDEA, Visual Studio и др. 
Однако в этом случае необходимо вручную задавать настройки форматирования.
Кроме того, количество принципиально разных стилей форматирования, которые можно задать с помощью данных настроек, невелико.

%Однако в этом случае необходимо иметь настройки форматирования, заданные программистом, но в целом, количество принципиально разных стилей форматировани, которые можно задать с помощью данных настроек, невелико. %причем эти настройки могут быть различными в разных IDE.
% * <Anton Podkopaev> 12:15:51 22 May 2015 UTC+0300:
% Нужно переработать это предложение.

Еще одним примером может послужить принтер-плагин для IntelliJ IDEA~\cite{podkopaev:diploma1}, который позволяет форматировать исходный код проекта по образцу. 
В качестве образца используется код из некоторого репозитория. 
Из него выделяются шаблоны форматирования для структур языка, которые применяются к структурам целевого кода. 
Под \textit{шаблоном} понимаются данные, сопоставление которых с элементом синтаксического дерева дает текстовое представление этого элемента (и его потомков). 
Например, на рис.~\ref{fig:ifTree} представлено дерево разбора для оператора ветвления. 


\begin{figure}[h]
	\centering
	\includegraphics[width=\textwidth]{Ozernykh/images/ifTree.jpg}
	\caption{Представление оператора ветвления в виде дерева разбора}
	\label{fig:ifTree}
\end{figure}

На рис.~\ref{fig:tmpltcodeintro}а изображен шаблон форматирования, который может быть применен к нему (несколько подвыражений, соответствующих выполнению условия, и одно подвыражение, соответствующее невыполненению условия).
На рис.~\ref{fig:tmpltcodeintro}б представлен результат применения этого шаблона к дереву разбора для оператора ветвления.

% \begin{figure}[h]
% 	\centering
% 	\lstinputlisting[language=c++]{codes/ifTmpltIntro.txt}
% 	\caption{Шаблон для оператора ветвления}
% 	\label{fig:ifTmpltIntro}
% \end{figure}

% На рис.~\ref{fig:ifCodeIntro} представлен результат применения этого шаблона к дереву разбора для оператора ветвления.

% \begin{figure}[h]
% 	\centering
% 	\lstinputlisting[language=c++]{codes/ifCodeIntro.txt}
% 	\caption{Представление дерева разбора при применении шаблона}
% 	\label{fig:ifCodeIntro}
% \end{figure}


\fvset{frame=lines,framesep=7pt}
\begin{figure}[ht]
\noindent\begin{minipage}{.5\textwidth}
    \lstinputlisting[language=Java]{Ozernykh/codes/ifTmpltIntro.txt}
\caption*{а) Шаблон для оператора\\
ветвления}    
\end{minipage}\hfill
\begin{minipage}{.5\textwidth}
    \lstinputlisting[language=Java]{Ozernykh/codes/ifCodeIntro.txt}
\caption*{б) Текст, полученный при применении шаблона к дереву разбора}    
\end{minipage}
\caption{Оператор ветвления и шаблон для него}    
\label{fig:tmpltcodeintro}
\end{figure}

Однако не всегда этот репозиторий существует. Кроме того, необходимо наличие заданного форматирования для всех структур языка.
% * <Anton Podkopaev> 12:19:02 22 May 2015 UTC+0300:
% 'Однако не всегда удобно иметь этот репозиторий под рукой.' --- поменять. Не про удобство надо говорить.

Шаблоны можно извлекать из уже существующего кода и применять их к другим элементам того же типа, но отформатированных иным способом. 
Например, пользователь меняет форматирование некоторой структуры языка (в частности, оператор ветвления), программа извлекает из полученного участка кода новый шаблон и применяет его к структурам того же типа, содержащим старое форматирование. 
Назовем такой способ задания шаблонов~--- форматирование в режиме
онлайн.
%картинки

Целью данной работы является расширение функциональности описанного 
принтер-плагина путем добавления возможности задания шаблонов в режиме онлайн. 

\section{Обзор предметной области}

Одним из подходов к заданию принтеров являются принтер-комбинаторы.
В этой главе приведен обзор классических принтер-комбинаторов, а также
принтер-комбинаторов с дополнительным оператором \emph{выбора}, которые
в рамках данной работы использованы для реализации принтеров, задаваемых
с помощью шаблонов. Для решения проблем эффективности
комбинаторов с выбором в данной работе был использован подход BURS.
Апробация подхода задания принтеров с помощью шаблонов
была произведена для языка Java, для которого есть средства форматирования,
встроенные в IDE.

\subsection{Принтер-комбинаторы}

Базовыми принтер-комбинаторными библиотеками являются библиотеки
Джона Хьюза\cite{hughes} и Филиппа Вадлера\cite{wadler}, которые
представляют собой функциональную переработку алгоритма
Оппена\cite{oppen}.
Ограничимся рассмотрением библиотеки Вадлера, так как
библиотека Хьюза обладает теми же свойствами, которые принципиальны в контексте
данной работы.

В библиотеке ключевым типом является документ.
Он представляет сущность, которая потом может
быть переведена в строковое представление алгоритмом принтера.
Основные конструкторы для составления документа таковы:
\begin{itemize}
  \item атомарная строка, которая печатается как есть;
  \item \emph{разделитель};
  \item последовательная композиция двух документов;
  \item набор связанных документов.
\end{itemize}

Определяющей особенностью данного подхода является то, что все разделители
в рамках одного набора могут быть совместно заменены алгоритмом принтера
на пробельный символ или на перевод строки. Выбор для каждого набора разделителей
основывается на том, что вывод должен поместиться в заданную ширину,
используя минимальное число строк.

Основная проблема такого подхода заключается в его слабой выразительной силе.
Документы, построенные по синтаксическому дереву печатаемой программы,
обрабатываются слишком единообразно, что иногда приводит к нежелательному результату.
Пусть, к примеру, нужно напечатать программу на языке
Python\footnote{\cd{http://python.org}}.
Между последовательными операторами в случае их печати на одной строчке необходимо
добавить дополнительный разделитель (``;''), иначе программа станет некорректной
(см. рис.~\ref{fig:seqEx}). Однако, описанные принтер-комбинаторы не предоставляют
возможности задать такое поведение.
Кроме того, с помощью таких принтер-комбинаторов невозможно выразить разные проектные СК,
так как они всегда печатают текст в одном стиле, который жестко ``зашит'' в их код.

\begin{figure}[h!]
	\centering
	\null\hfill
	\subfloat[Корректный код]{
		\centering
    \makebox[.4\textwidth] {
		  \lstinputlisting[language=Python]{codes/pythonCode.py}
    }
	}
	\null\hfill
	\subfloat[Некорректный код]{
		\centering
    \makebox[.4\textwidth] {
	  	\lstinputlisting[language=Python]{codes/pythonCodeBad.py}
    }
	}
	\hfill\null
	\caption{Пример работы принтер-комбинаторов для языка Python}
  \label{fig:seqEx}	
\end{figure}

Более подробный обзор библиотек Хьюза и Вадлера представлен в \cite{myCoursePaper}.

Большинство принтер-комбинаторных библиотек\cite{
swierstraChitil, swierstra04, peytonJones, kiselyov, chitil}
являются развитиями работ Хьюза и Вадлера. Так, в отличие от базовых, среди
них есть реализации с линейной сложностью обработки документа от его размера и
\emph{online} алгоритмы, которые не требуют просмотра всего документа
для начала печати его текстового представления. Но все они обладают тем же
интерфейсом, а значит в них также неразрешимы задачи, в которых требуется
задавать варианты текстовых представлений, которые отличаются не только
пробелами и переводами строк.

\subsection{Принтер-комбинаторы с выбором}

Существенно от описанных выше отличаются библиотеки, предоставляющие в своем
интерфейсе комбинатор \emph{выбора}, который позволяет задавать для одного поддерева
принципиально разные варианты раскладок.
Так, в работах~\cite{jongeEveryOccasion, jongeReengine} используется
оператор \lstinline[language = Haskell]{ALT},
но алгоритм принтера устроен так, что среди двух альтернатив выбирается первая, если она
помещается в заданную ширину, и вторая --- иначе,
что не дает оптимальный результат на выходе.

Оптимальные принтер-комбинаторы с выбором были впервые представлены в работе~\cite{swierstra}.
Их реализация является частью Utrecht Tools
Library\footnote{\cd{http://www.cs.uu.nl/wiki/HUT/WebHome}}
(практическая реализация несколько изменена по
отношению к той, что описана в статье, но отличие несущественно).
В данном подходе текст строится из блоков прямоугольной формы с возможно неполной последней
строчкой (см. рис. \ref{fig:basicFormat}). В реализации на Haskell блоки представляются
структурой \lstinline[language = Haskell]{Format}:
\begin{lstlisting}
    data Format = Elem { height        :: Int
                       , lastLineWidth :: Int
                       , width         :: Int
                       , txtstr        :: Int -> String -> String
                       }
\end{lstlisting}

Первые три поля структуры определяют геометрические размеры блока, а последнее --- функция,
которая используется для преобразования блока в текст. Здесь используется функция, а не
просто строчка, чтобы можно было преобразовывать вложенные блоки за линейное время. Первый
аргумент \lstinline[language = Haskell]{txtstr} задает сдвиг блока.

\begin{figure}
  \centering
  \subfloat[Блок текста]{
    \raisebox{5mm}{
      \centering
      \vspace{0pt}
      \tikz[scale = 2.0]{
        \draw (0,0) -- (1,0) -- (1,0.2) -- (2,0.2) -- (2,1) -- (0, 1) -- cycle;
      }
    }
    \label{fig:basicFormat}
  }
  ~
  \subfloat[Горизонтальная композиция]{
    \centering
    \tikz[scale = 2.0]{
       \draw (0,0) -- (1,0) -- (1,0.2) -- (2,0.2) -- (2,1) -- (0, 1) -- cycle;
       \draw (1.1,-0.9) -- (2.1,-0.9) -- (2.1,-0.7) -- (3.1,-0.7) -- (3.1,0.1) -- (1.1,0.1) -- cycle;
     }
     \label{fig:beside}
  }
  %\hfill
  ~
  \subfloat[Вертикальная композиция]{
    \makebox[.28\textwidth] {
    \centering
    \tikz[scale = 2.0]{
       \draw (0,0) -- (1,0) -- (1,0.2) -- (2,0.2) -- (2,1) -- (0, 1) -- cycle;
       \draw (0,-1.1) -- (1,-1.1) -- (1,-0.9) -- (2,-0.9) -- (2,-0.1) -- (0,-0.1) -- cycle;
    }
    \label{fig:above}
    }
  }
  \caption{Примитивы блока текста}
  \label{fig:basicConcat}
\end{figure}

Для работы с блоками текста используется следующие четыре примитива:
\begin{lstlisting}
    s2fmt     :: String -> Format
    indentFmt :: Int -> Format -> Format
    aboveFmt  :: Format -> Format -> Format
    besideFmt :: Format -> Format -> Format
\end{lstlisting}

Функция \lstinline[language = Haskell]{s2fmt} создает блок текста,
состоящий из одной строчки; \lstinline[language = Haskell]{indentFmt} по блоку создает новый,
сдвинутый на заданное число позиций. Действие примитивов композиции
\lstinline[language = Haskell]{besideFmt} и \lstinline[language = Haskell]{aboveFmt}
показано на рис.~\ref{fig:beside} и \ref{fig:above} соответственно.

Также, как и в библиотеках без комбинатора выбора, в работе~\cite{swierstra} используется понятие
\emph{документа}. Здесь документ можно рассматривать как множество возможных раскладок
(набор \lstinline[language = Haskell]{Format}-элементов). Документы описываются типом
\lstinline[language = Haskell]{Doc}, экземпляры которого на этапе построения представляют собой
деревья применений приведенных ниже комбинаторов, а на этапе обработки алгоритмом
принтера им в соответствие ставится итоговое множество вариантов текстовых представлений.

Документ конструируется с помощью следующих комбинаторов, которые симметричны
примитивам построения блоков текста:
\begin{lstlisting}
    text   :: String -> Doc
    indent :: Int -> Doc -> Doc
    beside :: Doc -> Doc -> Doc
    above  :: Doc -> Doc -> Doc
\end{lstlisting}

В дополнение появляется пятый комбинатор для документов:
\begin{lstlisting}
    choice :: Doc -> Doc -> Doc
\end{lstlisting}

Этот комбинатор и является тем самым комбинатором выбора. Он представляет
\emph{объединение} множеств раскладок документов, которые были переданы как
аргументы комбинатора. Заметим, что только этот комбинатор может произвести
документ с несколькими раскладками из одновариантных аргументов.

Оригинальная реализация существенно опирается на ленивые вычисления. В \cite{swierstra}
множество вариантов, соответствующее экземпляру \lstinline[language = Haskell]{Doc},
представляется ленивым списком всех возможных раскладок, удовлетворяющих ограничению
на максимальную ширину. Этот список отсортирован в порядке ``ухудшения''
раскладок, то есть в голове списка лежит ``лучшая'', в терминах оптимальности,
раскладка из возможных при заданной ширине. В случае
\lstinline{beside}- и
\lstinline{above}-композиций
документов полная (без учета ленивости) сложность вычисления списка нового документа
составляет $O(n \times m)$, где $n$ и $m$ --- размеры списков, соответствующих 
соединяемым документам. Размер нового списка также порядка $n \times m$.
Он не обязательно точно равен $n \times m$, так как некоторые из полученных
представлений могут не подпадать под ограничение на ширину, соответственно их можно
отбросить на этапе построения нового списка.

Выбор лучшего представления для самого верхнеуровнего документа происходит
просто --- нужно из соответствующего списка взять первый элемент.
Это создает впечатление, что общее число операций,
благодаря ленивым вычислениям, существенно уменьшается.
Но это не так из-за реализации обработки документа,
построенного с помощью комбинатора \lstinline[language = Haskell]{beside}, которая
вынуждает полное вычисление дочерних списков. Более того, из-за свойств отношения
порядка, построенного по критерию оптимальности, в рамках данной модели списков
принципиально нельзя построить ленивую обработку комбинатора
\lstinline[language = Haskell]{beside}.
Так, выбор оптимальной раскладки документа в
\cite{swierstra} имеет в худшем случае экспоненциальную сложность от числа
комбинаторов, использованных при его построении.
В \cite{swiComb} приведены оптимизации для данной модели, но они
не улучшают асимптотику решения для деревьев в общем случае.

\subsection{BURS}

Bottom-Up Rewrite System (BURS)\cite{burs} --- это метод динамического
программирования на деревьях, изначально появившийся в контексте задачи выбора
инструкций для генерации машинного кода. Основой BURS является
регулярная грамматика с древовидными правилами\cite{tata}, для которых задана
стоимость применения подстановки,
то есть грамматика со следующим набором правил: 

$$
\begin{array}{rcll}
  N &:& \alpha& [c]\\
  N &:& \alpha\; (K_1,\dots,K_n)& [c]
\end{array}
$$

Здесь $N, K_i$ это нетерминалы, $\alpha$ --- терминал,
$c$ --- функция стоимости, заданная для каждого правила.
Как и для обычной линейной грамматики, вводится стартовый
нетерминал S. Считается, что терминальное дерево выводится в
данной грамматике, если его можно получить с помощью правил
подстановки из одноузлового дерева $S$.
Каждая подстановка заменяет нетерминал $N$, находящийся в листе дерева, на дерево 
$\alpha\;(K_1,\dots,K_n)$, если в грамматике есть правило
$N:\alpha\;(K_1,\dots,K_n)$. 
Для каждой подстановки вычисляется стоимость ее применения с помощью
функции стоимости $c$.
Аргументами функции могут служить терминальная метка $\alpha$ и стоимости
вывода поддеревьев.
Задачей, которую решает BURS, является поиск вывода наименьшей стоимости
для заданного дерева по заданной грамматике.
Такой вывод может быть найден двухпроходным алгоритмом.

Первый проход (\emph{пометка}) обрабатывает дерево снизу вверх и вычисляет
для каждого узла набор троек $(K,\;R,\;c)$, где $K$ --- нетерминал, из которого может быть
выведено поддерево с корнем в обрабатываемом узле,
$R$ --- первое правило, которое используется для вывода минимальной стоимости из $K$,
$c$ --- стоимость такого вывода.
Процесс пометки происходит следующим образом:

\begin{itemize}
\item для листовой вершины, помеченной терминалом $\alpha$, в множество троек
этого вершины добавляется $(K,\;R,\;c\:(\alpha))$ для каждого правила $R=K:\;\alpha\;[c]$;
\item для промежуточной вершины, помеченной терминалом $\alpha$,
с непосредственными поддеревьями $v_1,\dots,v_n$
в множество добавляется тройка $(K,\;R,\;c\:(\alpha,c_1,\dots,c_n))$ для каждого правила
$R=K:\;\alpha\;(K_1,\dots,K_n)\;[c]$, где $(K_i,\;R_i,\;c_i)$ входит в множество троек для
$v_i$; если есть несколько правил вывода из нетерминала $K$, то выбирается правило,
минимизирующее стоимость вывода.
\end{itemize}

Второй проход (\emph{свертка}) просматривает дерево сверху вниз, используя сделанные пометки.
Первое правило из минимального вывода определяется тройкой $(S,\;R,\;c)$ для корневого узла
(если такой тройки нет, то вывод из $S$ невозможен).
Это правило однозначно определяет нетерминалы $K_i$ для каждого непосредственного поддерева
и процесс повторяется.

Для пометки дерева потенциально необходимо каждое правило грамматики применить к каждому узлу.
При фиксированной грамматике алгоритмическая сложность первого прохода --- $O\:(|R|)$,
где $|R|$ --- количество правил
(размер множества троек для каждого узла ограничен числом нетерминалов, которое не больше,
чем количество правил). Свертка также имеет линейную сложность.


\newpage
\subsection{Средства форматирования кода в IDE}

Интегрированные среды разработки программного обеспечения
(IDE) предоставляют средства
для форматирования программного кода. Рассмотрим их на примере двух Java IDE ---
IntelliJ IDEA\footnote{\cd{http://jetbrains.com/idea/}} и
Eclipse\footnote{\cd{http://eclipse.org}}.
Для задания требуемого СК используется широкий набор настроек \emph{форматтера},
подпрограммы IDE, отвечающей за форматирование исходных текстов.
Примерами таких настроек являются:
\begin{itemize}
  \item помещать ли фигурную скобку на той же строке, что и
    предыдущее выражение;
  \item форматирование списков --- всегда на одной строчке, переводить строчку после
    каждого элемента или печатать на одной строке,
    пока строчка меньше заданной рекомендуемой ширины.
%   \item и т.д.
\end{itemize}
На рис. ~\ref{fig:ideaFormatter} и~\ref{fig:eclipseFormatter} приведены диалоги задания
параметров форматирования в IntelliJ IDEA и Eclipse соответственно.

\begin{figure}[p]
	\centering
	\includegraphics[width=\textwidth]{ideaFormatter}
	\caption{Окно настройки форматтера IntelliJ IDEA}
	\label{fig:ideaFormatter}
\end{figure}

\begin{figure}[p]
	\centering
	\includegraphics[width=\textwidth]{eclipseFormatter}
	\caption{Окно настройки форматтера Eclipse}
	\label{fig:eclipseFormatter}
\end{figure}

У встроенных форматтеров есть следующие особенности.
Во-первых, это невозможность выразить
нестандартный СК, так как настройки форматтеров, несмотря на их большое
количество,
дают ограниченную вариативность для текстовых представлений форматируемых
программ. В целом, количество принципиально разных
стилей форматирования, которые можно задать с помощью данных настроек,
невелико.
Во-вторых, в случае, если надо
придерживаться СК уже существующего проекта, то по коду этого проекта необходимо
вручную задать все настройки форматтера.
Причем наборы настроек в разных IDE не совпадают, что увеличивает сложность
поддержки единого СК, если разработчики используют отличные от друг друга IDE.
Для решения данной проблемы существуют плагины, позволяющие использовать
внешние фоматтеры\footnote{\cd{http://plugins.jetbrains.com/plugin/6546}}.
Кроме того, есть средства, позволяющие экспортировать настройки форматтера в
XML-файл для их использования в другой
IDE\footnote{\cd{http://blog.jetbrains.com/idea/2014/01/intellij-idea-13-importing-code-formatter-settings-from-eclipse/}}.

Важным достоинством описанных форматтеров является малое время работы.
Это достигается в том числе и за счет того, что часто форматируется не весь
файл, а только его часть (см. \cite{eclipse}),
и из-за детерминированности представлений узлов синтаксического дерева,
которая следует из специфики настроек форматтера.
Однако форматирование целого большого проекта занимает существенное время.
Например, обработка кода IntelliJ IDEA Community Edition занимает более часа
у встроенного в IDEA форматтера.


\section{Реализация}
В процессе работы над проектом было выделено две основных задачи: реализация
взаимодействия с пользователем и расширение функциональности для переформатирования программного текста с фильтрацией шаблонов.
%диаграмму что ли нарисавать?
Предполагаемый процесс взаимодействия с пользователем: пользователь выделяет
участок программного кода, в котором будут производиться изменение, и вызывает 
действие (Action), которое извлекает шаблон из выделенного участка
кода. Затем пользователь меняет форматирование этого же участка, выделяет его
снова и вызывает действие, которое извлекает новый шаблон форматирования и 
сохраняет его.

После этого происходит перепечатывание элементов (PsiElement), имеющих старое
форматирование, в файле, в котором пользователь менял программный текст.

\subsection{Переформатирование}
\subsubsection{Извлечение элемента для форматирования}
% тут будет диаграмма:
% 1) Получаем текст
% 2) Корректируем отступы + пример
% 3) Извлечение элемента (для того чтобы знать тип) -- вся рутина
% 4) Склеить текст и элемент
В первую очередь необходимо получить элемент, форматирование которого пользователь хочет изменить.
% Где-то надо ссылку на диаграмму, наверное, тут
Весь этот процесс изображен на рис.~\ref{fig:getElemDiag}.

\begin{figure}[h]
	\centering
	\includegraphics[width=\textwidth]{images/getElemDiag.jpg}
	\caption{Диаграмма этапов извлечения элемента из текста}
	\label{fig:getElemDiag}
\end{figure}

На первом этапе с помощью функций, предоставляемых IntelliJ IDEA API, можно получить текст элемента. 
Однако этот текст не всегда корректный, так как может содержать в себе лишние отступы.
Например, на рис. \ref{fig:offsetExample} изображены методы doSomething() и doAnotherThing(). 
Они имеют отступ относительно оператора ветвления, равный четырем пробелам.
Кроме того, они имеют отступ относительно метода foo(), равный восьми пробелам. 
Чтобы шаблон был корректным, на втором этапе убираются лишние пробелы (в данном случае четыре).
Далее получаем из текста сам элемент. 
Это необходимо, чтобы узнать его тип. %ссылку на дерево PsiElement'ов.
Затем (на четвертом этапе) из корректного текста и полученного элемента создается новый элемент того же типа, но с корректными отступами.

Все это необходимо проделать, так как IntelliJ IDEA API не предоставляет функционала по изменению текста элемента явным образом. 

\begin{figure}[h]
	\centering
	\lstinputlisting[language=C++]{codes/offsetExample.txt}
	\caption{Общий отступ элементов}
	\label{fig:offsetExample}
\end{figure}

Аналогичные действия необходимо повторить для получения элемента из измененного
пользователем текста. 
Из этого элемента извлекается шаблон и сохраняется принтером для дальнейшего использования.

%\subsubsection{Извлечение шаблона}
%Шаблон, как и прежде, извлекается методами класса \lstinline[language=C++]{PsiElementComponent}.

%\subsubsection{Добавление шаблона}
%В связи с тем, что возникла необходимость добавления одного шаблона, класс %Printer обзавелся публичным методом, реализующим этот процесс. Остальные
%действия аналогичны тем при заполнении списка шаблонов из файла.

\subsubsection{Сравнение с шаблоном}
Как уже было отмечено в обзоре, при перепечатывании принтер обходит все элементы (PsiElement) дерева разбора и применяет к ним новые шаблоны. 
В рамках реализуемого подхода происходит перепечатывание лишь тех элементов, которые имели старое форматирование.
Это необходимо, если хочется поддерживать два и более различных варианта форматирования элементов некоторого типа. 
На рис. \ref{templates} изображены три блока оператора ветвления. 
Без фильтрации шаблонов все три блока будут изменены и получат одинаковое форматирование. %рис. сделать 
Однако первые два блока и так легко читаемы и понятны. 
Третий блок, будучи однострочным, уже не будет восприниматься так же хорошо.
Кроме того, существуют ограничения на ширину вывода, то есть в одной строке не можеть быть символов больше некоторой наперед заданной величины.

%\begin{figure}[t]
%    \centering
%    \lstinputlisting[language=C++]{codes/sameTmpltElems.txt}
%    \caption{Шаблоны}
%    \label{fig:sameTmpltElems}
%\end{figure}
%добавить вторую картинку как стало бы
\fvset{frame=lines,framesep=7pt}
\begin{figure}[ht]
\noindent\begin{minipage}{.5\textwidth}
    \lstinputlisting[language=C++]{codes/sameTmpltElems.txt}
\caption*{а) До переформатирования}    
\end{minipage}\hfill
\begin{minipage}{.5\textwidth}
    \lstinputlisting[language=C++]{codes/sameTmpltElems2.txt}
\caption*{б) После переформатирования}    
\end{minipage}
\caption{Переформатирование элементов одного типа без фильтрации шаблонов} 
\label{templates}
\end{figure}


Чтобы сделать настройку форматирования более гибкой и избавиться от подобного рода проблем, необходимо производить сравнение шаблонов текущего элемента дерева разбора и элемента со старым форматированием. 
Переформатировать элемент -- только в случае совпадения шаблонов.
Шаблоны совпадают, если совпадет их текстовое представление: шаблон записывается в виде строки, которая содержит необходимые элементы: ключевые слова, пробелы, переносы строк, скобки, а так же метки для поддеревьев, в которых указывается дополнительная информация, например, количество выражений в поддереве (одно или несколько).
В случае оператора ветвления (рис.~\ref{iftmpltcode}а) его поддеревьями являются: условие (condition), ветка, соответствующая выполнению условия (then block) и содержащая несколько подвыражений, ветка, соответствующая невыполнению условия (else block) и содержащая одно подвыражение. %ifTmplt ifTmpltCode
Соответствующий ему шаблон изображен на рисунке~\ref{iftmpltcode}б.
\fvset{frame=lines,framesep=7pt}
\begin{figure}[ht]
\noindent\begin{minipage}{.5\textwidth}
    \lstinputlisting[language=C++]{codes/ifTmpltCode.txt}
\caption*{а) Оператор ветвления}    
\end{minipage}\hfill
\begin{minipage}{.5\textwidth}
    \lstinputlisting[language=C++]{codes/ifTmplt.txt}
\caption*{б) Шаблон для оператора ветвления}    
\end{minipage}
\caption{Оператор ветвления и соответствующий ему шаблон}    
\label{iftmpltcode}
\end{figure}

%\begin{figure}[t]
%    \centering
%    \lstinputlisting[language=C++]{codes/sameTmpltElems.txt}
%    \caption{Шаблоны}
%    \label{fig:sameTmpltElems}
%\end{figure}
%добавить вторую картинку как стало бы

\subsection{Оценка производительности}
Описанный выше способ  давал очень низкую производительность. 
В таблице~\ref{table:slow} приведено время, затрачиваемое принтером на переформатирование файлов различной длины. 
Измерения производились на примере двух блоков: оператора ветвления и метода.

\begin{table}[h]
\begin{tabular}{cc|c|c|c|c|}
\cline{2-5}
\multicolumn{1}{ c| }{ } & \multicolumn{2}{p{6cm} |}{Формат. операторов ветвления} & \multicolumn{2}{p{5cm} |}{Форматирование методов} \\
\hline
\multicolumn{1}{ |c| }{Размер файла (строк)} & Кол-во & Время (сек.) & Кол-во & Время (сек.) \\ 
\hline
\multicolumn{1}{ |c| }{>7000}           & 520   & 35    & 660  & 58 \\
\multicolumn{1}{ |c| }{1000..2000}       & 145   & 2.7   & 90   & 4.4 \\
\multicolumn{1}{ |c| }{500..1000}       & 54    & 0.73  & 72   & 1.34 \\
\multicolumn{1}{ |c| }{0..500}          & 35    & 0.2   & 36   & 0.67 \\
\hline
\end{tabular}
\caption{Время переформатирования файлов I}
\label{table:slow}
\end{table}


Причина столь низкой производительности кроется в том, что для переформатирования элементов используется тот же способ, что и при форматировании по образцу:
принтер обходит все дерево разбора и сравнивает шаблоны.
%Однако нет необходимости проверять на совпадение шаблоны, полученные из каждого элемента дерева разбора, и шаблон, полученного из элемента со старым форматированием.
Однако нет необходимости сравнивать шаблон, полученный из элемента со старым форматированием, с шаблоном, полученным из каждого элемента дерева разбора.
Количество вариантов для переформатирования можно сузить лишь для элементов того же типа.

Для уменьшения затрат времени при обходе дерева элементов создается список кандидатов на форматирование (элементы того же типа, что и элемент со старым форматированием). 
После этого из элементов созданного списка извлекаются шаблоны.
Если полученный шаблон совпадает с шаблоном элемента со старым форматированием, то происходит перепечатывание элемента и замена его в дереве разбора.
Время, затрачиваемое в этом случае на переформатирование, приведено в таблице~\ref{table:fast}.

%\begin{table}[h]
%\begin{tabular}{ l | p{4cm} | p{4cm} }
%  Размер файла (строк) & Формат. операторов ветвления (сек.) & Формат. методов %(сек.) \\ 
%  \hline
%  >7000         & 6.6   & 5.8 \\
%  1000..2000    & 0.72  & 1   \\
%  500..1000     & 0.125 & 0.67 \\
%  0..500        & 0.145 & 0.325 \\
%\end{tabular}
%\caption{Время переформатирования файлов II}
%\label{table:fast}
%\end{table}


\begin{table}[h]
\begin{tabular}{cc|c|c|c|c|}
\cline{2-5}
\multicolumn{1}{ c| }{ } & \multicolumn{2}{p{6cm} |}{Формат. операторов ветвления} & \multicolumn{2}{p{5cm} |}{Форматирование  методов} \\
\hline
\multicolumn{1}{ |c| }{Размер файла (строк)} & Кол-во & Время (сек.) & Кол-во & Время (сек.) \\ 
\hline
\multicolumn{1}{ |c| }{>7000}           & 520   & 6.6   & 660  & 5.8 \\
\multicolumn{1}{ |c| }{1000..2000}      & 145   & 0.72  & 90   & 1 \\
\multicolumn{1}{ |c| }{500..1000}       & 54    & 0.225 & 72   & 0.67 \\
\multicolumn{1}{ |c| }{0..500}          & 35    & 0.145 & 36   & 0.325 \\
\hline
\end{tabular}
\caption{Время переформатирования файлов II}
\label{table:fast}
\end{table}
\section{Заключение}

В данной работе описан подход к композициональному символьному исполнению без раскрутки. Была предложена концепция композициональной памяти с символьной адресацией. Был доказан некоторый набор свойств КСП, дающий основание для подхода в стиле систем переписывания, где символьные кучи могут сами выступать как символы. Это даёт возможность автоматически порождать уравнения на состояния, решения которых в точности отражают поведения функций, работающих с динамической памятью. Было показано как свести задачу решения уравнений на состояния к задаче проверки безопасности чистых функций второго порядка.

Данная работа нацелена на теоретические основания композиционального анализа динамической памяти. Мы оставляем апробацию этого подхода на будущее. Другим направлением будущих исследований может быть расширение нашего формализма на композициональный анализ параллельных программ.

\begin{thebibliography}{99}
  \bibitem{podkopaev:diploma1}
  А.В.Подкопаев. Полиномиальной сложности оптимальные принтер-комбинаторы с выбором.
  Дипломная работа, кафедра системного программирования, математико-механический факультет СПбГУ, 2014.

  \bibitem{stratego:xt}
  Stratego/XT. \url{http://strategoxt.org/Stratego/StrategoDocumentation}, 2015.

  \bibitem{intellij:plugdev}
  IntelliJ IDEA Plugin Development Guide. \url{https://confluence.jetbrains.com/display/IDEADEV/PluginDevelopment},
  2015.
\end{thebibliography}

%\usepackage{listings}  
%\usepackage{hyperref}
%\usepackage{verbments}
%\usepackage{minted}
%\usepackage{caption}
%\usepackage{fancybox}
%\newfontfamily{\cyrillicfonttt}{CMU Serif}

\title{Декларативное форматирование в режиме онлайн}

\titlerunning{Декларативное форматирование в режиме онлайн}

\author{Озерных Игорь Станиславович}

\authorrunning{И.С.Озерных}

\tocauthor{И.С.Озерных}
\institute{Санкт-Петербургский государственный университет\\
\email{igor.ozernykh@gmail.com}}

\maketitle             

\begin{abstract}
Аккуратное, соответствующие выбранному стилю оформление (форматирование)
программных текстов критично для их восприятия.
Существуют средства автоматического форматирования кода. Одним
из них является метод синтаксических шаблонов. Его преимуществом является декларативность, которая
достигается за счет задания стиля форматирования с помощью образца кода. Так, естественным ограничением
описанного метода является необходимость наличия достаточно полного образца на целевом языке.
Данная работа посвящена расширению границ применимости метода синтаксических шаблонов
за счет возможности задания шаблонов в режиме онлайн.
\end{abstract}

\section*{Введение}

Этап поддержки и сопровождения в жизненном цикле программного обеспечения может занимать до 90\% времени существования программного продукта. На этом этапе особенно важно понимание программного текста поддерживаемой системы, что зачастую бывает сложной задачей. Необходимым условием для лучшего понимания программного текста является его аккуратное оформление, отражающее структуру 
программы. Существуют различные стандарты оформления кода (стили кодирования)~--- наборы правил и соглашений, используемые при написании исходного кода на некотором языке программирования.
Рассмотрим их на примере стандарта кодирования, принятого в компании Google\footnote{\texttt{https://google-styleguide.googlecode.com/svn/trunk/cppguide.html}}, и стандарта, используемого в проекте GNU\footnote{\texttt{http://www.gnu.org/prep/standards/standards.html}} (см. рис.~\ref{codingstandards}) для языка C++.


\fvset{frame=lines,framesep=7pt}
\begin{figure}[ht]
\noindent\begin{minipage}{.5\textwidth}
    \lstinputlisting[language=Java]{Ozernykh/codes/exGoogleCC.txt}
\caption*{а) Стиль кодирования Google}    
\end{minipage}\hfill
\begin{minipage}{.5\textwidth}
    \lstinputlisting[language=Java]{Ozernykh/codes/exGNUCC.txt}
\caption*{б) Стиль кодирования GNU}    
\end{minipage}
\caption{Различные стили кодирования для языка С++}    
\label{codingstandards}
\end{figure}

Когда программист работает над проектом, он придерживается некоторого стиля кодирования. 
Если файлы проекта с одним стилем кодирования присоединяется к уже существующему проекту с другим стилем кодирования, то их форматирование нужно привести к тому же виду. 
Для решения этой задачи можно воспользоваться уже существующими средствами, например, форматтерами (программами, которые форматируют исходный текст) в IDE таких, как Eclipse, IntelliJ IDEA, Visual Studio и др. 
Однако в этом случае необходимо вручную задавать настройки форматирования.
Кроме того, количество принципиально разных стилей форматирования, которые можно задать с помощью данных настроек, невелико.

%Однако в этом случае необходимо иметь настройки форматирования, заданные программистом, но в целом, количество принципиально разных стилей форматировани, которые можно задать с помощью данных настроек, невелико. %причем эти настройки могут быть различными в разных IDE.
% * <Anton Podkopaev> 12:15:51 22 May 2015 UTC+0300:
% Нужно переработать это предложение.

Еще одним примером может послужить принтер-плагин для IntelliJ IDEA~\cite{podkopaev:diploma1}, который позволяет форматировать исходный код проекта по образцу. 
В качестве образца используется код из некоторого репозитория. 
Из него выделяются шаблоны форматирования для структур языка, которые применяются к структурам целевого кода. 
Под \textit{шаблоном} понимаются данные, сопоставление которых с элементом синтаксического дерева дает текстовое представление этого элемента (и его потомков). 
Например, на рис.~\ref{fig:ifTree} представлено дерево разбора для оператора ветвления. 


\begin{figure}[h]
	\centering
	\includegraphics[width=\textwidth]{Ozernykh/images/ifTree.jpg}
	\caption{Представление оператора ветвления в виде дерева разбора}
	\label{fig:ifTree}
\end{figure}

На рис.~\ref{fig:tmpltcodeintro}а изображен шаблон форматирования, который может быть применен к нему (несколько подвыражений, соответствующих выполнению условия, и одно подвыражение, соответствующее невыполненению условия).
На рис.~\ref{fig:tmpltcodeintro}б представлен результат применения этого шаблона к дереву разбора для оператора ветвления.

% \begin{figure}[h]
% 	\centering
% 	\lstinputlisting[language=c++]{codes/ifTmpltIntro.txt}
% 	\caption{Шаблон для оператора ветвления}
% 	\label{fig:ifTmpltIntro}
% \end{figure}

% На рис.~\ref{fig:ifCodeIntro} представлен результат применения этого шаблона к дереву разбора для оператора ветвления.

% \begin{figure}[h]
% 	\centering
% 	\lstinputlisting[language=c++]{codes/ifCodeIntro.txt}
% 	\caption{Представление дерева разбора при применении шаблона}
% 	\label{fig:ifCodeIntro}
% \end{figure}


\fvset{frame=lines,framesep=7pt}
\begin{figure}[ht]
\noindent\begin{minipage}{.5\textwidth}
    \lstinputlisting[language=Java]{Ozernykh/codes/ifTmpltIntro.txt}
\caption*{а) Шаблон для оператора\\
ветвления}    
\end{minipage}\hfill
\begin{minipage}{.5\textwidth}
    \lstinputlisting[language=Java]{Ozernykh/codes/ifCodeIntro.txt}
\caption*{б) Текст, полученный при применении шаблона к дереву разбора}    
\end{minipage}
\caption{Оператор ветвления и шаблон для него}    
\label{fig:tmpltcodeintro}
\end{figure}

Однако не всегда этот репозиторий существует. Кроме того, необходимо наличие заданного форматирования для всех структур языка.
% * <Anton Podkopaev> 12:19:02 22 May 2015 UTC+0300:
% 'Однако не всегда удобно иметь этот репозиторий под рукой.' --- поменять. Не про удобство надо говорить.

Шаблоны можно извлекать из уже существующего кода и применять их к другим элементам того же типа, но отформатированных иным способом. 
Например, пользователь меняет форматирование некоторой структуры языка (в частности, оператор ветвления), программа извлекает из полученного участка кода новый шаблон и применяет его к структурам того же типа, содержащим старое форматирование. 
Назовем такой способ задания шаблонов~--- форматирование в режиме
онлайн.
%картинки

Целью данной работы является расширение функциональности описанного 
принтер-плагина путем добавления возможности задания шаблонов в режиме онлайн. 

\section{Обзор предметной области}

Одним из подходов к заданию принтеров являются принтер-комбинаторы.
В этой главе приведен обзор классических принтер-комбинаторов, а также
принтер-комбинаторов с дополнительным оператором \emph{выбора}, которые
в рамках данной работы использованы для реализации принтеров, задаваемых
с помощью шаблонов. Для решения проблем эффективности
комбинаторов с выбором в данной работе был использован подход BURS.
Апробация подхода задания принтеров с помощью шаблонов
была произведена для языка Java, для которого есть средства форматирования,
встроенные в IDE.

\subsection{Принтер-комбинаторы}

Базовыми принтер-комбинаторными библиотеками являются библиотеки
Джона Хьюза\cite{hughes} и Филиппа Вадлера\cite{wadler}, которые
представляют собой функциональную переработку алгоритма
Оппена\cite{oppen}.
Ограничимся рассмотрением библиотеки Вадлера, так как
библиотека Хьюза обладает теми же свойствами, которые принципиальны в контексте
данной работы.

В библиотеке ключевым типом является документ.
Он представляет сущность, которая потом может
быть переведена в строковое представление алгоритмом принтера.
Основные конструкторы для составления документа таковы:
\begin{itemize}
  \item атомарная строка, которая печатается как есть;
  \item \emph{разделитель};
  \item последовательная композиция двух документов;
  \item набор связанных документов.
\end{itemize}

Определяющей особенностью данного подхода является то, что все разделители
в рамках одного набора могут быть совместно заменены алгоритмом принтера
на пробельный символ или на перевод строки. Выбор для каждого набора разделителей
основывается на том, что вывод должен поместиться в заданную ширину,
используя минимальное число строк.

Основная проблема такого подхода заключается в его слабой выразительной силе.
Документы, построенные по синтаксическому дереву печатаемой программы,
обрабатываются слишком единообразно, что иногда приводит к нежелательному результату.
Пусть, к примеру, нужно напечатать программу на языке
Python\footnote{\cd{http://python.org}}.
Между последовательными операторами в случае их печати на одной строчке необходимо
добавить дополнительный разделитель (``;''), иначе программа станет некорректной
(см. рис.~\ref{fig:seqEx}). Однако, описанные принтер-комбинаторы не предоставляют
возможности задать такое поведение.
Кроме того, с помощью таких принтер-комбинаторов невозможно выразить разные проектные СК,
так как они всегда печатают текст в одном стиле, который жестко ``зашит'' в их код.

\begin{figure}[h!]
	\centering
	\null\hfill
	\subfloat[Корректный код]{
		\centering
    \makebox[.4\textwidth] {
		  \lstinputlisting[language=Python]{codes/pythonCode.py}
    }
	}
	\null\hfill
	\subfloat[Некорректный код]{
		\centering
    \makebox[.4\textwidth] {
	  	\lstinputlisting[language=Python]{codes/pythonCodeBad.py}
    }
	}
	\hfill\null
	\caption{Пример работы принтер-комбинаторов для языка Python}
  \label{fig:seqEx}	
\end{figure}

Более подробный обзор библиотек Хьюза и Вадлера представлен в \cite{myCoursePaper}.

Большинство принтер-комбинаторных библиотек\cite{
swierstraChitil, swierstra04, peytonJones, kiselyov, chitil}
являются развитиями работ Хьюза и Вадлера. Так, в отличие от базовых, среди
них есть реализации с линейной сложностью обработки документа от его размера и
\emph{online} алгоритмы, которые не требуют просмотра всего документа
для начала печати его текстового представления. Но все они обладают тем же
интерфейсом, а значит в них также неразрешимы задачи, в которых требуется
задавать варианты текстовых представлений, которые отличаются не только
пробелами и переводами строк.

\subsection{Принтер-комбинаторы с выбором}

Существенно от описанных выше отличаются библиотеки, предоставляющие в своем
интерфейсе комбинатор \emph{выбора}, который позволяет задавать для одного поддерева
принципиально разные варианты раскладок.
Так, в работах~\cite{jongeEveryOccasion, jongeReengine} используется
оператор \lstinline[language = Haskell]{ALT},
но алгоритм принтера устроен так, что среди двух альтернатив выбирается первая, если она
помещается в заданную ширину, и вторая --- иначе,
что не дает оптимальный результат на выходе.

Оптимальные принтер-комбинаторы с выбором были впервые представлены в работе~\cite{swierstra}.
Их реализация является частью Utrecht Tools
Library\footnote{\cd{http://www.cs.uu.nl/wiki/HUT/WebHome}}
(практическая реализация несколько изменена по
отношению к той, что описана в статье, но отличие несущественно).
В данном подходе текст строится из блоков прямоугольной формы с возможно неполной последней
строчкой (см. рис. \ref{fig:basicFormat}). В реализации на Haskell блоки представляются
структурой \lstinline[language = Haskell]{Format}:
\begin{lstlisting}
    data Format = Elem { height        :: Int
                       , lastLineWidth :: Int
                       , width         :: Int
                       , txtstr        :: Int -> String -> String
                       }
\end{lstlisting}

Первые три поля структуры определяют геометрические размеры блока, а последнее --- функция,
которая используется для преобразования блока в текст. Здесь используется функция, а не
просто строчка, чтобы можно было преобразовывать вложенные блоки за линейное время. Первый
аргумент \lstinline[language = Haskell]{txtstr} задает сдвиг блока.

\begin{figure}
  \centering
  \subfloat[Блок текста]{
    \raisebox{5mm}{
      \centering
      \vspace{0pt}
      \tikz[scale = 2.0]{
        \draw (0,0) -- (1,0) -- (1,0.2) -- (2,0.2) -- (2,1) -- (0, 1) -- cycle;
      }
    }
    \label{fig:basicFormat}
  }
  ~
  \subfloat[Горизонтальная композиция]{
    \centering
    \tikz[scale = 2.0]{
       \draw (0,0) -- (1,0) -- (1,0.2) -- (2,0.2) -- (2,1) -- (0, 1) -- cycle;
       \draw (1.1,-0.9) -- (2.1,-0.9) -- (2.1,-0.7) -- (3.1,-0.7) -- (3.1,0.1) -- (1.1,0.1) -- cycle;
     }
     \label{fig:beside}
  }
  %\hfill
  ~
  \subfloat[Вертикальная композиция]{
    \makebox[.28\textwidth] {
    \centering
    \tikz[scale = 2.0]{
       \draw (0,0) -- (1,0) -- (1,0.2) -- (2,0.2) -- (2,1) -- (0, 1) -- cycle;
       \draw (0,-1.1) -- (1,-1.1) -- (1,-0.9) -- (2,-0.9) -- (2,-0.1) -- (0,-0.1) -- cycle;
    }
    \label{fig:above}
    }
  }
  \caption{Примитивы блока текста}
  \label{fig:basicConcat}
\end{figure}

Для работы с блоками текста используется следующие четыре примитива:
\begin{lstlisting}
    s2fmt     :: String -> Format
    indentFmt :: Int -> Format -> Format
    aboveFmt  :: Format -> Format -> Format
    besideFmt :: Format -> Format -> Format
\end{lstlisting}

Функция \lstinline[language = Haskell]{s2fmt} создает блок текста,
состоящий из одной строчки; \lstinline[language = Haskell]{indentFmt} по блоку создает новый,
сдвинутый на заданное число позиций. Действие примитивов композиции
\lstinline[language = Haskell]{besideFmt} и \lstinline[language = Haskell]{aboveFmt}
показано на рис.~\ref{fig:beside} и \ref{fig:above} соответственно.

Также, как и в библиотеках без комбинатора выбора, в работе~\cite{swierstra} используется понятие
\emph{документа}. Здесь документ можно рассматривать как множество возможных раскладок
(набор \lstinline[language = Haskell]{Format}-элементов). Документы описываются типом
\lstinline[language = Haskell]{Doc}, экземпляры которого на этапе построения представляют собой
деревья применений приведенных ниже комбинаторов, а на этапе обработки алгоритмом
принтера им в соответствие ставится итоговое множество вариантов текстовых представлений.

Документ конструируется с помощью следующих комбинаторов, которые симметричны
примитивам построения блоков текста:
\begin{lstlisting}
    text   :: String -> Doc
    indent :: Int -> Doc -> Doc
    beside :: Doc -> Doc -> Doc
    above  :: Doc -> Doc -> Doc
\end{lstlisting}

В дополнение появляется пятый комбинатор для документов:
\begin{lstlisting}
    choice :: Doc -> Doc -> Doc
\end{lstlisting}

Этот комбинатор и является тем самым комбинатором выбора. Он представляет
\emph{объединение} множеств раскладок документов, которые были переданы как
аргументы комбинатора. Заметим, что только этот комбинатор может произвести
документ с несколькими раскладками из одновариантных аргументов.

Оригинальная реализация существенно опирается на ленивые вычисления. В \cite{swierstra}
множество вариантов, соответствующее экземпляру \lstinline[language = Haskell]{Doc},
представляется ленивым списком всех возможных раскладок, удовлетворяющих ограничению
на максимальную ширину. Этот список отсортирован в порядке ``ухудшения''
раскладок, то есть в голове списка лежит ``лучшая'', в терминах оптимальности,
раскладка из возможных при заданной ширине. В случае
\lstinline{beside}- и
\lstinline{above}-композиций
документов полная (без учета ленивости) сложность вычисления списка нового документа
составляет $O(n \times m)$, где $n$ и $m$ --- размеры списков, соответствующих 
соединяемым документам. Размер нового списка также порядка $n \times m$.
Он не обязательно точно равен $n \times m$, так как некоторые из полученных
представлений могут не подпадать под ограничение на ширину, соответственно их можно
отбросить на этапе построения нового списка.

Выбор лучшего представления для самого верхнеуровнего документа происходит
просто --- нужно из соответствующего списка взять первый элемент.
Это создает впечатление, что общее число операций,
благодаря ленивым вычислениям, существенно уменьшается.
Но это не так из-за реализации обработки документа,
построенного с помощью комбинатора \lstinline[language = Haskell]{beside}, которая
вынуждает полное вычисление дочерних списков. Более того, из-за свойств отношения
порядка, построенного по критерию оптимальности, в рамках данной модели списков
принципиально нельзя построить ленивую обработку комбинатора
\lstinline[language = Haskell]{beside}.
Так, выбор оптимальной раскладки документа в
\cite{swierstra} имеет в худшем случае экспоненциальную сложность от числа
комбинаторов, использованных при его построении.
В \cite{swiComb} приведены оптимизации для данной модели, но они
не улучшают асимптотику решения для деревьев в общем случае.

\subsection{BURS}

Bottom-Up Rewrite System (BURS)\cite{burs} --- это метод динамического
программирования на деревьях, изначально появившийся в контексте задачи выбора
инструкций для генерации машинного кода. Основой BURS является
регулярная грамматика с древовидными правилами\cite{tata}, для которых задана
стоимость применения подстановки,
то есть грамматика со следующим набором правил: 

$$
\begin{array}{rcll}
  N &:& \alpha& [c]\\
  N &:& \alpha\; (K_1,\dots,K_n)& [c]
\end{array}
$$

Здесь $N, K_i$ это нетерминалы, $\alpha$ --- терминал,
$c$ --- функция стоимости, заданная для каждого правила.
Как и для обычной линейной грамматики, вводится стартовый
нетерминал S. Считается, что терминальное дерево выводится в
данной грамматике, если его можно получить с помощью правил
подстановки из одноузлового дерева $S$.
Каждая подстановка заменяет нетерминал $N$, находящийся в листе дерева, на дерево 
$\alpha\;(K_1,\dots,K_n)$, если в грамматике есть правило
$N:\alpha\;(K_1,\dots,K_n)$. 
Для каждой подстановки вычисляется стоимость ее применения с помощью
функции стоимости $c$.
Аргументами функции могут служить терминальная метка $\alpha$ и стоимости
вывода поддеревьев.
Задачей, которую решает BURS, является поиск вывода наименьшей стоимости
для заданного дерева по заданной грамматике.
Такой вывод может быть найден двухпроходным алгоритмом.

Первый проход (\emph{пометка}) обрабатывает дерево снизу вверх и вычисляет
для каждого узла набор троек $(K,\;R,\;c)$, где $K$ --- нетерминал, из которого может быть
выведено поддерево с корнем в обрабатываемом узле,
$R$ --- первое правило, которое используется для вывода минимальной стоимости из $K$,
$c$ --- стоимость такого вывода.
Процесс пометки происходит следующим образом:

\begin{itemize}
\item для листовой вершины, помеченной терминалом $\alpha$, в множество троек
этого вершины добавляется $(K,\;R,\;c\:(\alpha))$ для каждого правила $R=K:\;\alpha\;[c]$;
\item для промежуточной вершины, помеченной терминалом $\alpha$,
с непосредственными поддеревьями $v_1,\dots,v_n$
в множество добавляется тройка $(K,\;R,\;c\:(\alpha,c_1,\dots,c_n))$ для каждого правила
$R=K:\;\alpha\;(K_1,\dots,K_n)\;[c]$, где $(K_i,\;R_i,\;c_i)$ входит в множество троек для
$v_i$; если есть несколько правил вывода из нетерминала $K$, то выбирается правило,
минимизирующее стоимость вывода.
\end{itemize}

Второй проход (\emph{свертка}) просматривает дерево сверху вниз, используя сделанные пометки.
Первое правило из минимального вывода определяется тройкой $(S,\;R,\;c)$ для корневого узла
(если такой тройки нет, то вывод из $S$ невозможен).
Это правило однозначно определяет нетерминалы $K_i$ для каждого непосредственного поддерева
и процесс повторяется.

Для пометки дерева потенциально необходимо каждое правило грамматики применить к каждому узлу.
При фиксированной грамматике алгоритмическая сложность первого прохода --- $O\:(|R|)$,
где $|R|$ --- количество правил
(размер множества троек для каждого узла ограничен числом нетерминалов, которое не больше,
чем количество правил). Свертка также имеет линейную сложность.


\newpage
\subsection{Средства форматирования кода в IDE}

Интегрированные среды разработки программного обеспечения
(IDE) предоставляют средства
для форматирования программного кода. Рассмотрим их на примере двух Java IDE ---
IntelliJ IDEA\footnote{\cd{http://jetbrains.com/idea/}} и
Eclipse\footnote{\cd{http://eclipse.org}}.
Для задания требуемого СК используется широкий набор настроек \emph{форматтера},
подпрограммы IDE, отвечающей за форматирование исходных текстов.
Примерами таких настроек являются:
\begin{itemize}
  \item помещать ли фигурную скобку на той же строке, что и
    предыдущее выражение;
  \item форматирование списков --- всегда на одной строчке, переводить строчку после
    каждого элемента или печатать на одной строке,
    пока строчка меньше заданной рекомендуемой ширины.
%   \item и т.д.
\end{itemize}
На рис. ~\ref{fig:ideaFormatter} и~\ref{fig:eclipseFormatter} приведены диалоги задания
параметров форматирования в IntelliJ IDEA и Eclipse соответственно.

\begin{figure}[p]
	\centering
	\includegraphics[width=\textwidth]{ideaFormatter}
	\caption{Окно настройки форматтера IntelliJ IDEA}
	\label{fig:ideaFormatter}
\end{figure}

\begin{figure}[p]
	\centering
	\includegraphics[width=\textwidth]{eclipseFormatter}
	\caption{Окно настройки форматтера Eclipse}
	\label{fig:eclipseFormatter}
\end{figure}

У встроенных форматтеров есть следующие особенности.
Во-первых, это невозможность выразить
нестандартный СК, так как настройки форматтеров, несмотря на их большое
количество,
дают ограниченную вариативность для текстовых представлений форматируемых
программ. В целом, количество принципиально разных
стилей форматирования, которые можно задать с помощью данных настроек,
невелико.
Во-вторых, в случае, если надо
придерживаться СК уже существующего проекта, то по коду этого проекта необходимо
вручную задать все настройки форматтера.
Причем наборы настроек в разных IDE не совпадают, что увеличивает сложность
поддержки единого СК, если разработчики используют отличные от друг друга IDE.
Для решения данной проблемы существуют плагины, позволяющие использовать
внешние фоматтеры\footnote{\cd{http://plugins.jetbrains.com/plugin/6546}}.
Кроме того, есть средства, позволяющие экспортировать настройки форматтера в
XML-файл для их использования в другой
IDE\footnote{\cd{http://blog.jetbrains.com/idea/2014/01/intellij-idea-13-importing-code-formatter-settings-from-eclipse/}}.

Важным достоинством описанных форматтеров является малое время работы.
Это достигается в том числе и за счет того, что часто форматируется не весь
файл, а только его часть (см. \cite{eclipse}),
и из-за детерминированности представлений узлов синтаксического дерева,
которая следует из специфики настроек форматтера.
Однако форматирование целого большого проекта занимает существенное время.
Например, обработка кода IntelliJ IDEA Community Edition занимает более часа
у встроенного в IDEA форматтера.


\section{Реализация}
В процессе работы над проектом было выделено две основных задачи: реализация
взаимодействия с пользователем и расширение функциональности для переформатирования программного текста с фильтрацией шаблонов.
%диаграмму что ли нарисавать?
Предполагаемый процесс взаимодействия с пользователем: пользователь выделяет
участок программного кода, в котором будут производиться изменение, и вызывает 
действие (Action), которое извлекает шаблон из выделенного участка
кода. Затем пользователь меняет форматирование этого же участка, выделяет его
снова и вызывает действие, которое извлекает новый шаблон форматирования и 
сохраняет его.

После этого происходит перепечатывание элементов (PsiElement), имеющих старое
форматирование, в файле, в котором пользователь менял программный текст.

\subsection{Переформатирование}
\subsubsection{Извлечение элемента для форматирования}
% тут будет диаграмма:
% 1) Получаем текст
% 2) Корректируем отступы + пример
% 3) Извлечение элемента (для того чтобы знать тип) -- вся рутина
% 4) Склеить текст и элемент
В первую очередь необходимо получить элемент, форматирование которого пользователь хочет изменить.
% Где-то надо ссылку на диаграмму, наверное, тут
Весь этот процесс изображен на рис.~\ref{fig:getElemDiag}.

\begin{figure}[h]
	\centering
	\includegraphics[width=\textwidth]{images/getElemDiag.jpg}
	\caption{Диаграмма этапов извлечения элемента из текста}
	\label{fig:getElemDiag}
\end{figure}

На первом этапе с помощью функций, предоставляемых IntelliJ IDEA API, можно получить текст элемента. 
Однако этот текст не всегда корректный, так как может содержать в себе лишние отступы.
Например, на рис. \ref{fig:offsetExample} изображены методы doSomething() и doAnotherThing(). 
Они имеют отступ относительно оператора ветвления, равный четырем пробелам.
Кроме того, они имеют отступ относительно метода foo(), равный восьми пробелам. 
Чтобы шаблон был корректным, на втором этапе убираются лишние пробелы (в данном случае четыре).
Далее получаем из текста сам элемент. 
Это необходимо, чтобы узнать его тип. %ссылку на дерево PsiElement'ов.
Затем (на четвертом этапе) из корректного текста и полученного элемента создается новый элемент того же типа, но с корректными отступами.

Все это необходимо проделать, так как IntelliJ IDEA API не предоставляет функционала по изменению текста элемента явным образом. 

\begin{figure}[h]
	\centering
	\lstinputlisting[language=C++]{codes/offsetExample.txt}
	\caption{Общий отступ элементов}
	\label{fig:offsetExample}
\end{figure}

Аналогичные действия необходимо повторить для получения элемента из измененного
пользователем текста. 
Из этого элемента извлекается шаблон и сохраняется принтером для дальнейшего использования.

%\subsubsection{Извлечение шаблона}
%Шаблон, как и прежде, извлекается методами класса \lstinline[language=C++]{PsiElementComponent}.

%\subsubsection{Добавление шаблона}
%В связи с тем, что возникла необходимость добавления одного шаблона, класс %Printer обзавелся публичным методом, реализующим этот процесс. Остальные
%действия аналогичны тем при заполнении списка шаблонов из файла.

\subsubsection{Сравнение с шаблоном}
Как уже было отмечено в обзоре, при перепечатывании принтер обходит все элементы (PsiElement) дерева разбора и применяет к ним новые шаблоны. 
В рамках реализуемого подхода происходит перепечатывание лишь тех элементов, которые имели старое форматирование.
Это необходимо, если хочется поддерживать два и более различных варианта форматирования элементов некоторого типа. 
На рис. \ref{templates} изображены три блока оператора ветвления. 
Без фильтрации шаблонов все три блока будут изменены и получат одинаковое форматирование. %рис. сделать 
Однако первые два блока и так легко читаемы и понятны. 
Третий блок, будучи однострочным, уже не будет восприниматься так же хорошо.
Кроме того, существуют ограничения на ширину вывода, то есть в одной строке не можеть быть символов больше некоторой наперед заданной величины.

%\begin{figure}[t]
%    \centering
%    \lstinputlisting[language=C++]{codes/sameTmpltElems.txt}
%    \caption{Шаблоны}
%    \label{fig:sameTmpltElems}
%\end{figure}
%добавить вторую картинку как стало бы
\fvset{frame=lines,framesep=7pt}
\begin{figure}[ht]
\noindent\begin{minipage}{.5\textwidth}
    \lstinputlisting[language=C++]{codes/sameTmpltElems.txt}
\caption*{а) До переформатирования}    
\end{minipage}\hfill
\begin{minipage}{.5\textwidth}
    \lstinputlisting[language=C++]{codes/sameTmpltElems2.txt}
\caption*{б) После переформатирования}    
\end{minipage}
\caption{Переформатирование элементов одного типа без фильтрации шаблонов} 
\label{templates}
\end{figure}


Чтобы сделать настройку форматирования более гибкой и избавиться от подобного рода проблем, необходимо производить сравнение шаблонов текущего элемента дерева разбора и элемента со старым форматированием. 
Переформатировать элемент -- только в случае совпадения шаблонов.
Шаблоны совпадают, если совпадет их текстовое представление: шаблон записывается в виде строки, которая содержит необходимые элементы: ключевые слова, пробелы, переносы строк, скобки, а так же метки для поддеревьев, в которых указывается дополнительная информация, например, количество выражений в поддереве (одно или несколько).
В случае оператора ветвления (рис.~\ref{iftmpltcode}а) его поддеревьями являются: условие (condition), ветка, соответствующая выполнению условия (then block) и содержащая несколько подвыражений, ветка, соответствующая невыполнению условия (else block) и содержащая одно подвыражение. %ifTmplt ifTmpltCode
Соответствующий ему шаблон изображен на рисунке~\ref{iftmpltcode}б.
\fvset{frame=lines,framesep=7pt}
\begin{figure}[ht]
\noindent\begin{minipage}{.5\textwidth}
    \lstinputlisting[language=C++]{codes/ifTmpltCode.txt}
\caption*{а) Оператор ветвления}    
\end{minipage}\hfill
\begin{minipage}{.5\textwidth}
    \lstinputlisting[language=C++]{codes/ifTmplt.txt}
\caption*{б) Шаблон для оператора ветвления}    
\end{minipage}
\caption{Оператор ветвления и соответствующий ему шаблон}    
\label{iftmpltcode}
\end{figure}

%\begin{figure}[t]
%    \centering
%    \lstinputlisting[language=C++]{codes/sameTmpltElems.txt}
%    \caption{Шаблоны}
%    \label{fig:sameTmpltElems}
%\end{figure}
%добавить вторую картинку как стало бы

\subsection{Оценка производительности}
Описанный выше способ  давал очень низкую производительность. 
В таблице~\ref{table:slow} приведено время, затрачиваемое принтером на переформатирование файлов различной длины. 
Измерения производились на примере двух блоков: оператора ветвления и метода.

\begin{table}[h]
\begin{tabular}{cc|c|c|c|c|}
\cline{2-5}
\multicolumn{1}{ c| }{ } & \multicolumn{2}{p{6cm} |}{Формат. операторов ветвления} & \multicolumn{2}{p{5cm} |}{Форматирование методов} \\
\hline
\multicolumn{1}{ |c| }{Размер файла (строк)} & Кол-во & Время (сек.) & Кол-во & Время (сек.) \\ 
\hline
\multicolumn{1}{ |c| }{>7000}           & 520   & 35    & 660  & 58 \\
\multicolumn{1}{ |c| }{1000..2000}       & 145   & 2.7   & 90   & 4.4 \\
\multicolumn{1}{ |c| }{500..1000}       & 54    & 0.73  & 72   & 1.34 \\
\multicolumn{1}{ |c| }{0..500}          & 35    & 0.2   & 36   & 0.67 \\
\hline
\end{tabular}
\caption{Время переформатирования файлов I}
\label{table:slow}
\end{table}


Причина столь низкой производительности кроется в том, что для переформатирования элементов используется тот же способ, что и при форматировании по образцу:
принтер обходит все дерево разбора и сравнивает шаблоны.
%Однако нет необходимости проверять на совпадение шаблоны, полученные из каждого элемента дерева разбора, и шаблон, полученного из элемента со старым форматированием.
Однако нет необходимости сравнивать шаблон, полученный из элемента со старым форматированием, с шаблоном, полученным из каждого элемента дерева разбора.
Количество вариантов для переформатирования можно сузить лишь для элементов того же типа.

Для уменьшения затрат времени при обходе дерева элементов создается список кандидатов на форматирование (элементы того же типа, что и элемент со старым форматированием). 
После этого из элементов созданного списка извлекаются шаблоны.
Если полученный шаблон совпадает с шаблоном элемента со старым форматированием, то происходит перепечатывание элемента и замена его в дереве разбора.
Время, затрачиваемое в этом случае на переформатирование, приведено в таблице~\ref{table:fast}.

%\begin{table}[h]
%\begin{tabular}{ l | p{4cm} | p{4cm} }
%  Размер файла (строк) & Формат. операторов ветвления (сек.) & Формат. методов %(сек.) \\ 
%  \hline
%  >7000         & 6.6   & 5.8 \\
%  1000..2000    & 0.72  & 1   \\
%  500..1000     & 0.125 & 0.67 \\
%  0..500        & 0.145 & 0.325 \\
%\end{tabular}
%\caption{Время переформатирования файлов II}
%\label{table:fast}
%\end{table}


\begin{table}[h]
\begin{tabular}{cc|c|c|c|c|}
\cline{2-5}
\multicolumn{1}{ c| }{ } & \multicolumn{2}{p{6cm} |}{Формат. операторов ветвления} & \multicolumn{2}{p{5cm} |}{Форматирование  методов} \\
\hline
\multicolumn{1}{ |c| }{Размер файла (строк)} & Кол-во & Время (сек.) & Кол-во & Время (сек.) \\ 
\hline
\multicolumn{1}{ |c| }{>7000}           & 520   & 6.6   & 660  & 5.8 \\
\multicolumn{1}{ |c| }{1000..2000}      & 145   & 0.72  & 90   & 1 \\
\multicolumn{1}{ |c| }{500..1000}       & 54    & 0.225 & 72   & 0.67 \\
\multicolumn{1}{ |c| }{0..500}          & 35    & 0.145 & 36   & 0.325 \\
\hline
\end{tabular}
\caption{Время переформатирования файлов II}
\label{table:fast}
\end{table}
\section{Заключение}

В данной работе описан подход к композициональному символьному исполнению без раскрутки. Была предложена концепция композициональной памяти с символьной адресацией. Был доказан некоторый набор свойств КСП, дающий основание для подхода в стиле систем переписывания, где символьные кучи могут сами выступать как символы. Это даёт возможность автоматически порождать уравнения на состояния, решения которых в точности отражают поведения функций, работающих с динамической памятью. Было показано как свести задачу решения уравнений на состояния к задаче проверки безопасности чистых функций второго порядка.

Данная работа нацелена на теоретические основания композиционального анализа динамической памяти. Мы оставляем апробацию этого подхода на будущее. Другим направлением будущих исследований может быть расширение нашего формализма на композициональный анализ параллельных программ.

\begin{thebibliography}{99}
  \bibitem{podkopaev:diploma1}
  А.В.Подкопаев. Полиномиальной сложности оптимальные принтер-комбинаторы с выбором.
  Дипломная работа, кафедра системного программирования, математико-механический факультет СПбГУ, 2014.

  \bibitem{stratego:xt}
  Stratego/XT. \url{http://strategoxt.org/Stratego/StrategoDocumentation}, 2015.

  \bibitem{intellij:plugdev}
  IntelliJ IDEA Plugin Development Guide. \url{https://confluence.jetbrains.com/display/IDEADEV/PluginDevelopment},
  2015.
\end{thebibliography}


\end{document}
