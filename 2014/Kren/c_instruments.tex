\section{Используемые особенности языка С++}

Реализация сборки мусора в виде внешней библиотеки существенно опирается 
на некоторые особенности как языка С++, так и его компиляторов и среды поддержки времени 
исполнения. К таковым особенностям относятся:

\begin{enumerate}
\item функция \lstinline{typeid} и возможность идентификации типа во время исполнения (runtime type identification, RTTI);
\item инициализация по размещению (placement new);
\item шаблоны, в том числе шаблоны с переменным числом аргументов (templates, variadic templates);
\item конструкторы, деструкторы и порядок их вызова.
\end{enumerate}

Далее рассмотрим вышеперечисленные особенности подробнее.

\subsection{Функция \lstinline{typeid} и идентификация типа\\
во время исполнения} 

Идентификация типа во время исполнения позволяет получить метаинформацию о типе объекта во время работы программы. 
В сборщике мусора эта метаинформация используется для установления связи между объектом и метаинформацией сборщика
мусора, соответствующей этому типу. Для того, чтобы это стало возможно, поддержка RTTI должна быть включена 
соответствующими опциями компилятора.

На пользовательском уровне возможности RTTI представлены функцией \lstinline{typeid}\footnote{\url{http://www.c-cpp.ru/books/identifikaciya-tipa-vo-vremya-ispolneniya-rtti}},
которая получает указатель на объект и возвращает ссылку на объект типа \lstinline{typeinfo}. Данный тип
содержит информацию о типе объекта, в частности, имя этого типа, которое и используется в сборщике мусора.

\subsection{Инициализация по размещению, шаблоны,\\
шаблоны с произвольным числом аргументов} 

Инициализация по размещению и шаблоны с произвольным числом аргументов в реализации сборщика мусора используются 
совместно для того, чтобы выразить примитив выделения памяти. Основная задача, которая при этом возникает 
кроме собственно выделения~--- определение типа объекта, занимающего данную область памяти, и её аннотирование
соответствующей метаинформацией. 

Отследить момент выделения памяти можно было бы путем перегрузки глобального оператора \lstinline{new}; однако
в этом случае было бы трудно узнать, каков тип объекта, для которого выделяется память. Использование же
инициализации по размещению, шаблонов и шаблонов с произвольным числом аргументов позволяет инкапсулировать
всю нужную функциональность в одной функции, которая отведет память, вызовет конструктор, передав
ему нужное количество аргументов, и кроме того выполнит все остальные необходимые действия.

 \subsection{Конструкторы, деструкторы и порядок их вызова} 

Важным фактом является то, что в С++ момент создания/удаления объекта можно отследить с помощью 
конструктора/деструктора\footnote{\url{http://www.developerfusion.com/article/133063/constructors-in-c11/}}. 
В сборщике мусора это используется для того, чтобы, во-первых, построить метаинформацию для данного типа (каждый
``умный указатель'' при создании регистрирует себя в некоторой структуре данных) и, во-вторых, чтобы регистрировать
корневые указатели во внешнем пуле корней.
