\newpage

\subsection{Библиотека Вадлера}

В \cite{wadler} Филипп Вадлер описал свою комбинаторную бибилиотеку для форматированного вывода на языке Haskell. Она является модификацией библиотеки Хьюза, описанной в предыдущем разделе. Код библиотеки сократился с $\approx$ 110 строк до $\approx$ 80 строк, и, по исследованию Вадлера, на 30\% увеличилась скорость вычисления раскладки документа.

Рассмотрим основые комбинаторы этой библиотеки:
% \inputminted{haskell}{Podkopaev/codes/wadlerBasicOperations.hs}
\lstinputlisting[language=Haskell]{Podkopaev/codes/wadlerBasicOperations.hs}

Вадлер решил отказаться от двух разных способов соединения документов, оставив лишь горизонтальную склейку. Но для того, чтобы можно было выражать не только однострочные документы, в библиотеке Вадлера появилась функция “\lstinline[language=Haskell]{line}”. Она создает документ, который может быть переведен в символ новой строки или в пробел.
Функция “\lstinline[language=Haskell]{group}” имеет то же назначение, что и оператор “\lstinline[language=Haskell]{sep}” в библиотеке Хьюза, но работает не со списком документов, а с одним документом, и по сути предоставляет альтернативы для алгоритма перевода документа в “\lstinline[language=Haskell]{String}”: в документе, на который подействовал “\lstinline[language=Haskell]{group}”, либо все вхождения “\lstinline[language=Haskell]{line}” заменяются на пробел, либо остаются переводами строки (если они не являются частью вложенных “\lstinline[language=Haskell]{group}”-документов).

В таком виде библиотека потеряла в выразительности, что признается в статье Вадлера. Но кроме потери выразительности, есть еще один серьезный недостаток, возникающий из-за оператора “\lstinline[language=Haskell]{group}”. То, что любой документ им может быть преобразован в однострочный, делает библиотеку неприменимой в некоторых ситуациях. 

Рассмотрим следующий пример. Пусть нам надо написать принтер для языка Python\footnote{http://python.org}. Для конструкции последовательных операторов принтер изображен на рисунке~\ref{fig:pythonPrinter}.
По-другому его написать нельзя --- мы хотим, чтобы последовательные операторы печатались на новых строках. Но если такая конструкция попадет внутрь “\lstinline[language=Haskell]{group}”-документа, то последовательные строчки могут склеиться пробелом, что сделает код некорректным, так как в Python несколько операторов на одной строке должны разделяться символом “;”.

\begin{figure}[h!]
	% \inputminted{haskell}{Podkopaev/codes/pythonPrinter.hs}
	\lstinputlisting[language=Haskell]{Podkopaev/codes/pythonPrinter.hs}
	\caption{Принтер для последовательных операторов в Python}
	\label{fig:pythonPrinter}
\end{figure}

Так корректный код (см. рисунок~\ref{fig:pythonCode}) может превратиться в некорректный (см. рисунок~\ref{fig:pythonCodeBad}).
\begin{figure}[h!]
	\centering
	\null\hfill
	\subfloat[]{
		\centering
		\lstinputlisting[language=Python]{Podkopaev/codes/pythonCode.py}
		\label{fig:pythonCode}	
	}
	\null\hfill
	\subfloat[]{
		\centering
		\lstinputlisting[language=Python]{Podkopaev/codes/pythonCodeBad.py}
		\label{fig:pythonCodeBad}
	}
	\hfill
	\null
	\caption{Пример работы принтера для языка Python}
\end{figure}