\section*{Введение}
Графы и графовые базы данных имеют широкое применение в таких областях, как биоинформатика, логистика, социальные сети и многие другие. Запросы к таким базам формулируются как задача поиска путей в графе, удовлетворяющих некоторым ограничениям. Во многих случаях такие ограничения формулируются в виде некоторой грамматики, наиболее выразительным её представлением является контекстно-свободная (КС) грамматика. В таком случае задача сводится к поиску путей в графе, которые бы соответствовали строкам в контекстно-свободном языке. Такую задачу назовем синтаксическим анализом графа. 

Одним из примеров применения синтаксического анализа графа является поиск подпоследовательности геномов в задаче биоинформатики. Из окружающей среды берется образец, по нему необходимо выявить подпоследовательности генов для классификации организмов в данном образце. Для этого по образцу строится метагеномная сборка, являющаяся комбинацией генов. Сборка представляет собой граф с последовательностями символов на ребрах. В таком графе необходимо найти подстроки, позволяющие провести классификацию. Такую задачу можно решить при помощи синтаксического анализа графов.

Существуют различные подходы к синтаксическому анализу графов (например,~\cite{GrigRagCFPQuerying},~\cite{Hellings},~\cite{Sevon}), одно из которых основано на алгоритма Generalised LL (GLL)~\cite{GrigRagCFPQuerying}. Алгоритм GLL позволяет без модификаций грамматики анализировать все КС-языки. Результатом работы данного алгоритма является компактное представление леса разбора Shared Packed Parse Forest (SPPF)~\cite{SPPF}. Данное представление разбора позволяет производить дополнительный анализ и получать таким образом дополнительную информацию, а так же производить семантические действия уже после разбора. Данное решение построено по принципу генераторов синтаксических анализаторов, когда по грамматике генерируется код анализаторов, что не всегда удобно, при написании запросов к графовым базам данных.

Существуют различные решения для поиска путей в графовых базах данных. Это встроенные инструменты и языки для запросов. К примеру, для базы данных Neo4j~\cite{Neo4j} существуют такие языки запросов как Cypher~\cite{Cypher} и openCypher~\cite{openCypher}. Однако они не поддерживают синтаксис запросов в стилеконтекстно-свободной грамматики и не возвращают пути в виде деревьев разбора. При работе с графовыми базами данных было бы удобно строить запрос к ним на языке, на котором написано целевое приложение. Это возможно реализовать техникой парсер-комбинаторов. Парсер-комбинаторы позволяют создавать синтаксические анализаторы динамически непосредственно в коде программы на некотором языке. Все существующие библиотеки парсер-комбинаторов анализируют только линейный вход — строки. Нашей задачей стала разработка библиотеки для синтаксического анализа графов. Существует библиотека Meerkat~\cite{Meerkat} на языке Scala~\cite{Scala}, реализующая синтаксический анализ строк методом парсер-комбинаторов, используя идеи, схожие с используемыми в алгоритме GLL~\cite{GLL}. Она обладает рядом преимуществ:

\begin{itemize} 
\item результатом работы библиотеки является лес разбора SPPF;
\item библиотеку возможно запустить на JVM;
\item разбор происходит в худшем случае за $O(n^3)$, где n – длина входной последовательности.
\end{itemize}

Было принято решение использовать данную библиотеку для решения задачи.

Таким образом в рамках данной работы была поставлена задача модифицировать существующую библиотеку для синтаксического анализа графов с помощью техники парсер-комбинаторов.
