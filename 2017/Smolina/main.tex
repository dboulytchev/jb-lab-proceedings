\title{Синтаксический анализ графов с помощью парсер-комбинаторов}
\titlerunning{Синтаксический анализ графов с помощью парсер-комбинаторов}
\author{Смолина Софья Константиновна}
\authorrunning{С.К. Смолина}
\tocauthor{С.К. Смолина}
\institute{Санкт-Петербургский государственный электротехнический университет «ЛЭТИ» им. В.И.Ульянова (Ленина)\\
\email{sofysmol@gmail.com}}
\maketitle

\begin{abstract}
Запросы к графовым базам данных могут быть сформулированы как
задача поиска путей в графе, удовлетворяющих некоторым ограничениям.
Часто такие ограничения задаются грамматиками. В таком случае задача
сводится к поиску путей в графе, которые бы соответствовали строкам в
некотором языке. Такую задачу назовем синтаксическим анализом графов.
Существует несколько подходов решения данной проблемы, но все они
подразумевают использование парсер-генераторов, что не всегда удобно. В
нашей работе к задаче анализа графов была адаптирована библиотека парсер-
комбинаторов Meerkat. По итогам работы был получена библиотека для синтаксического
анализатора графов.
\end{abstract}

\section*{Введение}
Графы и графовые базы данных имеют широкое применение в таких областях, как биоинформатика, логистика, социальные сети и многие другие. Запросы к таким базам формулируются как задача поиска путей в графе, удовлетворяющих некоторым ограничениям. Во многих случаях такие ограничения формулируются в виде некоторой грамматики, наиболее выразительным её представлением является контекстно-свободная (КС) грамматика. В таком случае задача сводится к поиску путей в графе, которые бы соответствовали строкам в контекстно-свободном языке. Такую задачу назовем синтаксическим анализом графа. 

Одним из примеров применения синтаксического анализа графа является поиск подпоследовательности геномов в задаче биоинформатики. Из окружающей среды берется образец, по нему необходимо выявить подпоследовательности генов для классификации организмов в данном образце. Для этого по образцу строится метагеномная сборка, являющаяся комбинацией генов. Сборка представляет собой граф с последовательностями символов на ребрах. В таком графе необходимо найти подстроки, позволяющие провести классификацию. Такую задачу можно решить при помощи синтаксического анализа графов.

Существуют различные подходы к синтаксическому анализу графов (например,~\cite{GrigRagCFPQuerying},~\cite{Hellings},~\cite{Sevon}), одно из которых основано на алгоритма Generalised LL (GLL)~\cite{GrigRagCFPQuerying}. Алгоритм GLL позволяет без модификаций грамматики анализировать все КС-языки. Результатом работы данного алгоритма является компактное представление леса разбора Shared Packed Parse Forest (SPPF)~\cite{SPPF}. Данное представление разбора позволяет производить дополнительный анализ и получать таким образом дополнительную информацию, а так же производить семантические действия уже после разбора. Данное решение построено по принципу генераторов синтаксических анализаторов, когда по грамматике генерируется код анализаторов, что не всегда удобно, при написании запросов к графовым базам данных.

Существуют различные решения для поиска путей в графовых базах данных. Это встроенные инструменты и языки для запросов. К примеру, для базы данных Neo4j~\cite{Neo4j} существуют такие языки запросов как Cypher~\cite{Cypher} и openCypher~\cite{openCypher}. Однако они не поддерживают синтаксис запросов в стилеконтекстно-свободной грамматики и не возвращают пути в виде деревьев разбора. При работе с графовыми базами данных было бы удобно строить запрос к ним на языке, на котором написано целевое приложение. Это возможно реализовать техникой парсер-комбинаторов. Парсер-комбинаторы позволяют создавать синтаксические анализаторы динамически непосредственно в коде программы на некотором языке. Все существующие библиотеки парсер-комбинаторов анализируют только линейный вход — строки. Нашей задачей стала разработка библиотеки для синтаксического анализа графов. Существует библиотека Meerkat~\cite{Meerkat} на языке Scala~\cite{Scala}, реализующая синтаксический анализ строк методом парсер-комбинаторов, используя идеи, схожие с используемыми в алгоритме GLL~\cite{GLL}. Она обладает рядом преимуществ:

\begin{itemize} 
\item результатом работы библиотеки является лес разбора SPPF;
\item библиотеку возможно запустить на JVM;
\item разбор происходит в худшем случае за $O(n^3)$, где n – длина входной последовательности.
\end{itemize}

Было принято решение использовать данную библиотеку для решения задачи.

Таким образом в рамках данной работы была поставлена задача модифицировать существующую библиотеку для синтаксического анализа графов с помощью техники парсер-комбинаторов.

\section{Постановка задачи}
Целью данной работы является создание решения для синтаксического анализа регулярных множеств, применимого для работы со входными данными большого размера. Для достижения поставленной цели были поставлены следующие задачи:

\begin{itemize}  
\item Разработать алгоритм синтаксического анализа динамически  формируемого кода на основе алгоритма GLL. 
\item Доказать корректность предложенного алгоритма.
\item Применить к задаче поиска на входных данных большого размера.
\item Реализовать предложенный алгоритм в рамках проекта YaccConstructor.
\item Произвести эксперименты и сравнение.
\end{itemize}

\section{Обзор}
\subsection{Синтаксический анализ графов}
При работе с графами, например в графовых базах данных, возникает необходимость выполнения запросов поиска путей, удовлетворяющих заданным ограничениям. Ограничения задаются, как правило, регулярной грамматикой, однако контекстно-свободные грамматики представляют собой более выразительный язык запросов. Контекстно-свободная грамматика (КС-грамматика) $G$ --- четверка $(T, N, P, S)$, где $N$ --- множество нетерминалов, $T$ --- множество терминалов ($T \cap N = \varnothing$), $P = \{ A \rightarrow \alpha \mid A \in N, \alpha \in (N \cup T)^*\}$ --- множество правил грамматики и $S \in N$ --- стартовый символ. Грамматика $G = (\{+, -, a\}, \{ E, N \}, P, E)$, множество правил которой $P$ представлены на листинге~\ref{grmG1}, задает язык арифметических выражений со сложением и умножением над переменными $a$.

\begin{listing}
\caption{Правила грамматики $G$}
\label{grmG1}
\centering
$\begin{array}{ll}
E \rightarrow & N \ + \ N \mid N \ - \ N
\\
N \rightarrow & a
\end{array}$
 \end{listing}

Итак, задача выполнения запросов является задачей поиска в ориентированном графе всех путей, представляющих собой строки языка, заданные КС-грамматикой. Одно из возможных решений такой задачи --- это модификация алгоритмов классического синтаксического анализа строк. Работа~\cite{Sevon} модифицирует алгоритм Эрли. Данный подход позволяет записывать запросы к графовым структурам данных с указанием направления поиска: прямое направление, когда поиск производится в направлении ребер графа, или обратном --- по обратным ребрам. Данный алгоритм строит только некоторое приближение к результату: обработка циклов входного графа осуществляется только до некоторой глубины, специфицируемой пользователем, в результате чего некоторые пути могут быть утеряны. Результатом работы алгоритма является подграф, содержащий пути разбора. 

Для последующего анализа путей из результата удобно иметь информацию об их синтаксической структуре, например в форме абстрактных синтаксических деревьев. Так как множество путей может быть бесконечным, то и синтаксических деревьев может быть бесконечно много, поэтому возникает вопрос об их представлении. В алгоритмах обобщенного синтаксического анализа, в случае существования нескольких деревьев разбора для одной строки (в виду неоднозначностей грамматики), используется компактное представление леса разбора SPPF (Shared Packed Parse Forest). В структуре SPPF переиспользуются общие фрагменты разных выводов, за счет его размер полиномиален от размера входной строки. Существуют модификации обобщенных алгоритмов, применимые для выполнения запросов в контекстно-свободных ограничениях к графам: алгоритмы на основе RNGLR~\cite{RNGLR} и GLL~\cite{GrigRagCFPQuerying}. Первый позволяет построить лес разбора SPPF, второй --- бинаризованный лес разбора Binarized Shared Packed Parse Forest~\cite{SPPF} с лучшими пространственными характеристиками. 

Binarized Shared Packed Parse Forest имеет размер $O(n^3)$, где n --- длина входной строки. В отличие от обычного дерева разбора, внутренние узлы которого всегда соответствуют нетерминалам грамматик, в BSPPF используются дополнительные типы узлов: упакованный узел (packed node) и промежуточный узел (intermediate node). Упакованный узел создаётся для представления неоднозначностей вывода (его дети соответствуют разным продукциям). Промежуточный узел используется для бинаризации, когда правило длины больше 2 представляется как цепочка применений правил длины 2. Именно за счет бинаризации достигается кубический размер представления леса разбора. В SPPF могут быть циклы. На рис.~\ref{fig:sppfV} представлено, как выглядит SPPF, состоящее из деревьев на рис.~\ref{fig:sppfA} и рис.~\ref{fig:sppfB}. А на рис.~\ref{fig:sppfG} представлена его бинаризованная форма. Пример взят из статьи~\cite{IzmCombinator}.

 \begin{figure}[t]
 \centering
    \subfloat[Дерево разбора 1]{
        \label{fig:sppfA}
        \includegraphics[width=0.35\textwidth]{Smolina/pics/SppfA.png}
    }
    \subfloat[Дерево разбора 2]{
        \label{fig:sppfB}
        \includegraphics[width=0.35\textwidth]{Smolina/pics/SppfB.png}        
    }

    ~\\~
    \subfloat[SPPF]{
        \label{fig:sppfV}
        \includegraphics[width=0.35\textwidth]{Smolina/pics/SppfV.png}        
    }
    \subfloat[Бинаризинное SPPF]{
        \label{fig:sppfG}
        \includegraphics[width=0.45\textwidth]{Smolina/pics/SppfG.png}        
    }
 \caption{Граф SPPF для двух вариантов вывода}
\end{figure}

Решение на RNGLR и GLL построено на основе генерации синтаксических анализаторов. Такое решение не является удобным при работе с графами и графовыми базами данных, так как при добавлении малейших изменений необходимо генерировать руками новый синтаксический анализатор, а также требуется использование дополнительного предметно-ориентированного языка для задания запроса. 

В сфере промышленных графовых баз данных существуют свои языки запросов. К примеру для графовой базы данных Neo4j~\cite{Neo4j} существуют языки Cypher~\cite{Cypher} и openCypher~\cite{openCypher}, а для OrientDB~\cite{OrientDB} используется язык SQL~\cite{Sql}. Ни один из них не поддерживает формат запроса в виде контекстно-свободной грамматики. Более того, результат запроса всегда --- простые строки, что усложняет их дальнейший анализ.

Таким образом, нашей целью стала разработка решения, в котором грамматику можно специфицировать в коде целевого приложения. Один из возможных подходов к решению данной задачи --- использование техники парсер-комбинаторов~\cite{HOFunParsing}.

\subsection{Техника парсер-комбинаторов}
Комбинатор --- это функция высшего порядка, которая из набора функций строит новую функцию. Возможность принимать функции как аргументы, комбинировать их и возвращать как результат является важной особенностью функциональных языков программирования. Парсер-комбинатор --- это функция высшего порядка, которая на вход получает множество синтаксических анализаторов и возвращает новый синтаксический анализатор. 

Для синтаксического анализа необходимо научиться анализировать элементарные сущности (терминалы, нетерминалы), осуществлять последовательное применение анализаторов и поддержать возможность осуществлять выбор анализатора для разбора суффикса сроки. Эти требования задают минимальный набор комбинаторов. Техника парсер-комбинаторов позволяет из элементарных анализаторов конструировать более сложные. Интеграция с языком программирования приложения, в котором применяется синтаксический анализатор, добавляет гибкости и расширяемости в сравнении с генераторами синтаксических анализаторов. Приведём пример реализации простейшего парсер-комбинатора на Scala.

В листинге~\ref{parser1} представлен синтаксический анализатор, который принимает на вход строковую последовательность, затем разбирает строку, начинающуюся с определенного терминала, и возвращает результат разбора и необработанный суффикс строки.

\begin{listing}
\caption{Синтаксический анализатор терминала}
\label{parser1}
\centering
\includegraphics[width=0.7\textwidth]{Smolina/pics/parser1.png}
\end{listing}

Для того чтобы получить синтаксический анализатор подстроки, можно воспользоваться парсер-комбинатором,
который составлял последовательность из анализаторов символов --- парсер-комбинатор последовательности seq. На листинге~\ref{parserSeq} приведен пример реализации.

\begin{listing}
\caption{Парсер-комбинатор последовательности}
\label{parserSeq}
\centering
\includegraphics[width=0.7\textwidth]{Smolina/pics/parserSeq.png}
\end{listing}

Теперь, чтобы получить синтаксический анализатор, начинающийся с подстроки ``ABC'', мы можем воспользоваться элементарными синтаксическими анализаторами символов и парсер-комбинатором последовательности (см. листиниг~\ref{parserABC}).

\begin{listing}
\caption{Парсер-комбинатор строки “ABC”}
\label{parserABC}
\centering
\includegraphics[width=0.9\textwidth]{Smolina/pics/parserABC.png}
\end{listing}

Простые парсер-комбинаторы, основанные на рекурсивном спуске, представляют собой интуитивно ясную модель и поэтому удобны для
отладки. Однако они имеют экспоненциальную сложность относительно размеров грамматики~\cite{Popov}. Это связано с тем, что в наивной реализации рекурсивного спуска при откате не сохраняются результаты и разбор префикса строки одним и тем же синтаксическим анализатором может происходить многократно. Мемоизация~\cite{Memoization} позволяет решить эту проблему за счет переиспользования результатов применения синтаксических анализаторов к подстрокам. Таким образом, однажды выполненное вычисление никогда не повторяется --- результат просто берется из таблицы.

Другой проблемой, ассоциируемой с парсер-комбинаторами, является трудность обработки леворекурсивных определений анализаторов. Например, синтаксический анализ наивным рекурсивным спуском в соответствии с грамматикой, представленной в листинге~\ref{grmG2}, никогда не завершится.

 \begin{listing}
\caption{Леворекурсивная грамматика}
\label{grmG2}
\centering
$\begin{array}{rl}
E \rightarrow E \ + \ a \ | \ a
\end{array}$
 \end{listing}

Данная проблема имеет несколько решений, одно из которых основано на ограничении числа вызовов нетерминала некоторой константой, связанной с длиной входной последовательности ~\cite{ParserComb}. Количество применений каждого распознавателя к каждой позиции в строке ограничивается длиной неразобранного суффикса строки. Данный подход не применим, если длина последовательности неизвестна, например, при считывании символов с сетевого сокета. Во-вторых, такой подход обладает сложностью $O(n^4)$, вместо ожидаемой $O(n^3)$, где n --- длина последовательности. Подобные проблемы решаются техникой Continuation Parsing Style (CPS)~\cite{MemoizationInTopDown}. В отличие от первого метода, цепочка леворекурсивных вызовов завершается, когда происходит второй вызов синтаксического анализатора в данной позиции входа. Затем результаты для леворекурсивных синтаксических анализаторов эффективно вычисляются в цикле: пока создается новый результат, завершенные пути синтаксического анализа, записанные как продолжения, перезапускаются в новой входной позиции. Как результат, обработка рекурсивных правил более эффективна и не требует знания длины последовательности. Более подробно об этом речь пойдет в следующей главе.



\section{Разработка библиотеки для синтаксического анализа графов}
\subsection{Библиотека Meerkat}
Meerkat — это библиотека парсер-комбинаторов, разработанная на языке программирования Scala Али Афрузе и Анастасией Измайловой~\cite{IzmCombinator}. Библиотека предназначена для синтаксического анализа строк. Анализ, при использовании библиотеки, осуществляется за $O(n^3)$, где n – длина последовательности, а также осуществляет построение компактного представления леса разбора Binarized Shared Packed Parse Forest (BSPPF). В ней решены проблемы левой рекурсии, а также экспоненциальной сложности за счет использования техники мемоизации и Continuation-Passing Style, предложенной Марком Джонсоном~\cite{MemoizationInTopDown}.

\subsection{Распознаватель в стиле парсер-комбинаторов}
В терминах данной библиотеки базовый распознователь – это функция типа $Recognizer$, которая определяется как функция $Int => Result[Int]$ (принимает тип $Int$ и возвращает тип $Result[Int]$). Базовый распознаватель — это частичная функция, он принимает как аргумент позицию во входном потоке и возвращает значение $success$, соответствующее успешному разбору и содержащее следующую позицию, или значение $failure$, соответствующее ошибке при анализе.

Для составления любой КС-грамматики необходимо реализовать функциональность, который бы позволил реализовывать синтаксический анализатор терминала и пустой строки, а также функциональность, который бы позволил бы их объединять в последовательности и составлять правила. В библиотеке Meerkat для этих целей реализованы анализаторы $terminal$, и $epsilon$, а так же комбинаторы $seq$ и $rule$ (см. листининг~\ref{parserAll}).

\begin{listing}
\caption{Распознаватели в стиле парсер-комбинаторов}
\label{parserAll}
\centering
\includegraphics[width=0.9\textwidth]{Smolina/pics/combinators.png}
\end{listing}

Анализаторы представляют собой базовые распознаватели для терминала и пустой строки. Анализатор $terminal$ возвращает распознаватель, который принимает индекс текущей строки и возвращает $success$ в случае, когда суффикс строки с текущего символа равен значению терминала. Входной поток в данном случае предполагается глобальной переменной, но в реализации передаётся как параметр функции. 

Комбинатор $seq$ представляет собой последовательную композицию распознавателей. Он получает на вход текущую позицию и список распознавателей, затем применяет последовательно каждый из распознавателей к входному потоку, начиная с текущей позиции, и передает предыдущие результаты дальнейшим распознавателям. Если какой-либо из распознавателей вернёт $failure$, то дальнейший разбор производиться не будет и комбинатор $seq$ вернёт $failure$. Это одна из реализаций монадического
парсер-комбинатора.

 Комбинатор $rule$ используется, чтобы определить нетерминал с именем $nt$ и списком распознавателей $alts$, которые представляют из себя альтернативы правила грамматики. Данный комбинатор принимает как параметр текущую позицию во входном потоке и применяет каждый анализатор к этому символу, пока хотя бы один не вернёт $success$. 

Для того чтобы в строго типизированных языках можно было реализовать рекурсивные распознаватели, введен дополнительный комбинатор fix (см. листинг~\ref{fix}). Это комбинатор неподвижной точки — функция, которая вычисляет состояние, при котором заданное отображение возвращает в неё же.

\begin{listing}
\caption{Комбинатор fix}
\label{fix}
\centering
\includegraphics[width=0.9\textwidth]{Smolina/pics/fix.png}
\end{listing} 

Комбинирование элементарных анализаторов и представленных выше парсер-комбинаторов позволяет представить КС-грамматику и избежать зацикливания при использовании ~\cite{GLL}.

\subsection{Полный перебор с использованием Continuation-Passing Style}
У базовых распознавателей есть свои недостатки. Одна из проблем связана с тем, что работа распознавателя тесно связана с расположением правил в грамматике, первый распознаватель имеет более высокий приоритет. Например, когда распознавателю для грамматики $G_3$ (см. листиниг~\ref{grmG3}) будет представлена на вход строка “ab”, он вернёт $failure$. Это произойдет по причине того, что когда будет применен первый распознаватель из двух альтернативных, он распознает часть строки успешно, но дальнейшее распознавание завершиться ошибкой, и второй распознаватель так никогда и не будет применен к началу строки. Ещё одной проблемой базовых распознавателей является то, что результатом успешного распознавания всегда является единственное решение, когда как их может быть и несколько. Необходимо порождать все возможные выводы данной строки. Одним из подходов для решения этой задачи является техника Continuation-Passing
Style (CPS).

\begin{listing}
\caption{Грамматика $G_3$}
\label{grmG3}
\centering
$\begin{array}{rl}
A \rightarrow a \ | \ a \ b
\end{array}$
 \end{listing}

 Идея программирования в стиле Continuation-Passing состоит в том, что передача управления происходит через механизм продолжений.
Продолжение в данном контексте представляет собой состояние программы в конкретный момент времени, которое возможно сохранить и использовать для перехода в данное состояние.

Для того чтобы преобразовать базовые распознаватели в CPS авторы библиотеки Meerkat изменили тип $Result[T]$ и сопровождающие его функции $success$ и $failure$. Любой результат теперь должен быть представим как композиция двух функций, используя метод $flatMap$, или как комбинация двух возможных альтернатив – результатов, используя метод $orElse$. Реализация идеи в библиотеке Meerkat представлена на  листинге~\ref{result}.

\begin{listing}
\caption{Result[T] для CPS распознавателей}
\label{result}
\centering
\includegraphics[width=0.8\textwidth]{Smolina/pics/result.png}
\end{listing}

CPS распознаватель принимает на вход позицию и возвращает функцию типа $Result[T]$. Данная функция принимает на вход продолжение типа $K[Int]$ и возвращает $Unit$. Продолжение в данном случае является функцией, которая представляет собой остаток от процесса распознавания. Вместо возвращения значения, распознаватель возвращает либо $success$, вызывая продолжение со следующей позицией, либо $failure$, не вызывая продолжения.

Таким образом, используя базовые распознаватели в терминах Continuation-Passing Style, авторами статьи~\cite{GLL} был получен полный перебор всех решений.

\subsection{Мемоизация и поддержка левой рекурсии}
Ранее были отмечены две проблемы у наивных реализаций парсер-комбинаторов – это проблема левой рекурсии и экспоненциального роста. Рассмотрим подробнее, как решены эти проблемы в библиотеке Meerkat. Проблема экспоненциального роста возникает при поиске всех деревьев разбора и связана с тем, что анализ входной последовательности распознавателем происходит без сохранения промежуточных результатов анализа.

Проблема левой рекурсии возникает с определенным типом грамматики (пример приведён на листинге~\ref{grmG2}). В представленном случае нетерминал бесконечно будет вызывает сам себя, так и не считав ни одного символа из входного потока. Такая ситуация может возникнуть либо явно,
либо когда префикс перед рекурсивным определением обращается в пустую строку.

В библиотеке Meerkat данные проблемы решаются мемоизацией и техникой CPS. Мемоизацией называют механизм, который позволяет сохранять вычисленные результаты и в дальнейшем переиспользовать их. При вычислении распознавателя, в первую очередь алгоритм смотрит на таблицу мемоизации и проверяет, не было ли вычислено значение данного распознавателя в данной позиции ранее. В случае, когда вычисление происходит в первый раз, результат работы распознавателя записывается в таблицу мемоизации. Иначе же распознаватель не вычисляется, а результат его работы берется из таблицы. На листинге~\ref{memo} представлен такой механизм для техники CPS, реализованный в библиотеке Meerkat.

 Функция $memo$ превращает произвольный распознаватель CPS в мемоизированный CPS-распознаватель. Мемоизированный CPS-распознаватель при каждом вызове в позиции i обращается к таблице $table$. Таблица $table$ содержит функцию для работы с продолжениями для каждого символа. В случае, когда для позиции i применяется немемоизированый распознаватель в первый раз, его модификация сохраняется и представляет собой переменную res. Результат распознавателя не вычисляется в тот же момент, а только после обновления таблицы table. Если распознаватель был вызван в данной позиции раньше, его результат возвращается из table и не перевычисляется. Таким образом решается проблема экспоненциального роста.

\begin{listing}
\caption{Мемоизация для CPS}
\label{memo}
\centering
\includegraphics[width=0.8\textwidth]{Smolina/pics/memo.png}
\end{listing}

CPS-распознаватель имеет доступ к двум списками $Rs$ и $Ks$. Список $Rs$ содержит все позиции входа, созданные немемоизированным распознавателем, когда он достигает успеха в позиции $i$. Список $Ks$ содержит все продолжения, которые передаются в функцию $memo\_result$, когда они вызываются в позиции $i$. Если мемоизированный результат вызывается в первый раз, текущее продолжение добавляется к $Ks$, а исходный немемоизованный результат вызывается с новым продолжением $k\_i$. Продолжение $k\_i$ создается только после первого вызова мемоизированного распознавателя в $i$. Каждый раз, когда немемоизированный распознаватель завершается с успехом в позиции $i$, $k\_i$ проверяет, была ли эта входная позиция получена как результат раньше, если нет, сначала записывает ее в $Rs$, а затем запускает все записанные продолжения. С другой стороны, если вызванный мемоизированный результат был вызван ранее, текущее
продолжение $k$ добавляется к $Ks$ и вызывается для каждой входной позиции,записанной в $Rs$.

Теперь, когда мемоизированный леворекурсивный CPS-распознаватель вызывается во входной позиции $i$, его завершение гарантируется, поскольку соответствующий немемоизированный распознаватель никогда не будет вызван в $i$ более одного раза. В то же время часть пути выполнения, которая привела к леворекурсивному вызову и может создавать новые позиции ввода для левого рекурсивного распознавателя в $i$, эффективно записывается как продолжение. Каждое продолжение фиксирует следующий шаг в альтернативе после того, как текущий вызов возвращается. Продолжения будут выполняться для любой входной позиции, создаваемой леворекурсивным распознавателем в точке $i$, до тех пор, пока создаются новые позиции ввода~\cite{IzmCombinator}. Таким образом решается проблема левой рекурсии.

\subsection{ Библиотека для синтаксического анализа графов}
Основной задачей работы является разработка решения для синтаксического анализа графа, которое бы позволило не только писать запросы непосредственно в коде программы, но и получать результат этих запросов в компактной форме SPPF.

Данная работа требует решения промежуточных шагов:
\begin{enumerate}
\item входным типом данных библиотеки Meerkat являются строки, необходимо изменить его на граф;
\item строки можно рассматривать как линейную последовательность ребер и вершин, необходимо преобразовать библиотеку до того, чтобы стало возможным анализировать последовательность вершин с несколькими исходящими ребрами (деревья) с различными метками, а так же граф с циклами;
\item необходимо преобразовать библиотеку до того, чтобы стало возможным анализ графов с одинаковыми ребрами из одной вершины;
\item необходимо добавить функциональность для решения частных задач синтаксического анализа;
\item необходимо добавить функциональность для работы с графовойбазой данных Neo4j.
\end{enumerate}

\subsubsection{Изменение входного типа данных на граф}


Входной последовательностью в библиотеке Meerkat являются строки. Для работы с ними разработчиками библиотеки был разработан класс $Input$, который в конструкторе получает строку. Для добавления нового типа входных данных класс $Input$ был преобразован в интерфейс $Input$, который имеет следующие методы:
\begin{itemize}
\item $startWith$: принимает на вход строку $prefix$ и позицию в последовательности $n$. Проверяет, является ли prefix префиксом суффикса строки, начинающийся с позиции $n$;
\item $matchRegex$: принимает на вход регулярное выражение и позицию в последовательности. Проверяет, содержится ли в суффиксе строки с позиции $n$, строка удовлетворяющая данному регулярному выражению.
\end{itemize}

Класс $InputString$ реализует интерфейс для строки, класс $InputGraph$ – для графа. Для графа был разработан интерфейс $IGraph$, задающий методы необходимые для реализации класса $InputGraph$. Для тестирования данный интерфейс был реализован для типа данных $scalax.collection.Graph$~\cite{Graph}. Этопростоя в использовании реализация графа, в которой кратко и наглядно можно представить узлы и связи между ними.

\subsubsection{Преобразование системы для анализа дерева и графа с циклами}


Дерево представляет собой частный случай связного графа, в котором нет циклов. Пример дерева представлен на рис.~\ref{Graph1}. Обязательным условием на данном этапе является то, что метки на ребрах из одной вершины должны быть различны. Необходимо преобразовать библиотеку до того, чтобы стало возможным синтаксический анализ деревьев.

\begin{figure}

 \centering
 \includegraphics[width=0.4\textwidth]{Smolina/pics/Graph1.png}
 \caption{Дерево $A_1$}
 \label{Graph1}
\end{figure}

Для решения данной задачи необходимо обратить свое внимание на синтаксический анализатор терминала, реализованный в библиотеке Meerkat на листинге~\ref{parser1}. Данный метод принимает на вход некоторую строку, и позицию во входной последовательности. Если начиная с этой позиции существует подстрока, которая равна терминалу, то результат считается успешным и возвращается значение $success$, который содержит следующий символ, с которого следует продолжать разбор. 

В строке мы анализируем префикс длины $k$, начинающийся с позиции $n$ входного потока, в случае успеха, следующей позицией для анализа становится позиция $k+n$. В случае же с деревом это не верно. В дереве мы просматриваем все исходящие ребра, анализируем их на соответствии со значением терминал. Если для синтаксического анализатора существует ребро с соответствующим терминалом, для него следующей позицией становится номер вершины, в которую направлено ребро. Из чего следует, тип данных метода $startWith$ необходимо изменить с Boolean на $Option[Int]$.

Тип $Option$ является контейнером, энкапсулирующим понятие опционального значения. В случае успешного завершения, возвращается результат $Some(i)$, где $i$ – номер следующей вершины, или же $None$.

В графах с циклами, в отличие от деревьев, в одну и ту же вершину может существовать несколько путей, которые при этом могут образовывать циклы. В наивной реализации парсер-комбинаторов каждый путь приходилось бы пересчитывать заново, а в случае с циклом, вычисление бы никогда не завершилось. Однако применение техники мемоизации совместно с CPS позволяет избежать данных проблем. С использованием мемоизации вычисление каждого анализатора в каждой вершине происходит всего один раз. Для каждой вершины существует свой набор продолжений, который комбинируется и переиспользуется. Благодаря этим свойствам дальнейших изменений для поддержания графов с циклами не потребовалось. Циклы в графе обрабатываются подобно леворекурсивной грамматике.

Таким образом была решена задача синтаксического анализа для
деревьев и графов с циклами. Проиллюстрируем полученные результаты на примере 1.

\textsc{Пример 1.} 
Входной последовательностью является граф представленный на рис.~\ref{Graph2}. Данный граф цикличен. Из вершины 0 исходит
ребро с меткой «а» в вершину 1, а из вершины 1 исходит ребро с меткой «b». Стартовой вершиной считаем вершину 0. Проведем результат
синтаксического анализа в соответствии с грамматикой $G_4$ на рис.~\ref{grmG4}.

\begin{figure}
 \centering
 \includegraphics[width=0.2\textwidth]{Smolina/pics/Graph2.png}
 \caption{Цикличный граф $A_2$}
 \label{Graph2}
\end{figure}

\begin{listing}
\caption{Грамматика $G_4$}
\label{grmG4}
\centering
$\begin{array}{rl}
E \rightarrow a \ b \ E \ | \ a \ b
\end{array}$
 \end{listing}

\begin{figure}
 \centering
 \includegraphics[width=0.4\textwidth]{Smolina/pics/Tree1.png}
 \caption{Результат работы синтаксического анализатора $E$ на
графе $A_2$}
 \label{Tree1}
\end{figure}

Результатом синтаксического анализа преобразованной библиотекой Meerkat является синтаксический лес разбора, представленый на рис~\ref{Tree1}. Из корневой вершины исходит две ветви, два возможных решения. Первое решение строка “ab”. Второе решение состоит из двух ветвей. Правая ветвь также конструирует строку “ab”, a левая ветвь указывает на корень дерева, это означает, что подстрока тоже должна принадлежать языку. Таким образом результатом анализа является бесконечное множество деревьев разбора.

\subsubsection{Преобразование системы для анализа графов с одинаковыми ребрами из одной вершины}


Ранее мы требовали, чтобы исходящие ребра имели уникальные метки. В этом случае из каждой вершины может существовать единственный путь, начинающийся с данного терминала. Если отказаться от данного ограничения, то таких путей может существовать несколько или не существовать вовсе. Итого, необходимо изменить тип и реализацию метода $startWith$. Теперь результатом его работы будет тип $Set[Int]$. Когда множество оказывается не пусто, будем считать, что разбор прошел успешно ($success$)
иначе не успешно ($failure$).

Метод $terminal$ также требует изменений. Все успешные пути, полученные методом $startWith$, комбинируются при помощи $orElse$ класса
$Result$. В результате чего получаем единственное продолжение, которое возвращается для дальнейшей обработки. Пример 2 демонстрирует работу новой реализации.

\textsc{Пример 2.} 
Входная последовательность – граф $A_3$ на рис.~\ref{Graph3}. Из стартовой вершины 0 исходят 3 ребра с одинаковыми метками. Проведем синтаксический анализ в соответствии с грамматикой $G_5$ на листинге~\ref{grmG5}.

\begin{figure}
 \centering
 \includegraphics[width=0.6\textwidth]{Smolina/pics/Graph3.png}
 \caption{Граф $A_3$}
 \label{Graph3}
\end{figure}

\begin{listing}
\caption{Грамматика $G_5$}
\label{grmG5}
\centering
$\begin{array}{rl}
E \rightarrow a \ c \ d \ E \ | \ a \ d
\end{array}$
 \end{listing}

\begin{figure}
 \centering
 \includegraphics[width=\textwidth]{Smolina/pics/Trees2.png}
 \caption{– Результат работы синтаксического анализатора $E$ на графе $A_3$}
 \label{Trees2}
\end{figure}

Результат работы представлен на рис.~\ref{Trees2}. В графе $A_3$ из вершины 0 существуют 3 пути, соответствующие грамматике $G_5$ (0-5-6-7, 0-1-2-4 и 0-3-2-4). В результате получен граф SPPF, где несколько вершин выделены как начальные. Вершины, соответствующие общим подпутям при этом переиспользуются.

SPPF представляет собой компактное представление множества деревьев разбора одной строки. В случае синтаксического анализа графов необходимо построить множество деревьев разбора множества строк, что возможно сделать, если модифицировать SPPF таким образом, чтобы у него было несколько «корневых» узлов. За счет такого определения можно обеспечить переиспользование деревьев разбора для общих подпутей.

В библиотеке Meerkat во время анализа строится такая структура данных, как $SPPFLookup$. Она представляет собой множество уникальных узлов, создаваемых во время анализа, которые ссылаются друг на друга. При успешном результате один или несколько узлов, как в нашем случае, берутся как корни дерева.

\subsubsection{Функциональность для решения частных задач синтаксического анализа}


Раннее мы говорили о поиске всех путей в графе, удовлетворяющих заданным ограничениям. Существуют и другие семантики запросов, при которых необходимо идентифицировать некое подмножество всех путей, удовлетворяющих ограничениям. Такими семантиками являются~\cite{Hellings}:

\begin{itemize}
\item поиск всех путей в графе;
\item поиск путей, начинающихся с какой-либо вершины из данного
множества;
\item поиск путей, завершающихся в какой-нибудь вершине из данного
множества;
\item поиск путей, начинающихся в какой-нибудь вершине множества
начальных вершин и заканчивающихся в вершине из множества
конечных вершин.
\end{itemize}

Для того чтобы находить решения удовлетворяющие каждой из данных семантик, не нужно вносить изменений в процесс разбора. Задача решается благодаря структуре данных $SPPFLookup$, которая была описана выше. После конструирования $SPPFLookup$, необходимо лишь произвести фильтрацию стартовых нетерминальных символов, то есть выбрать корневые вершины, которые удовлетворяют требованиям. В итоге все вычисления производятся один раз и уже из полученных результатов выбираются необходимые.

\subsubsection{Интеграция библиотеки с графовой базой данных Neo4j}


После модификации библиотеки Meerkat для анализа графов, было решено применить результат работы на существующей графовой базе
данных. Нами была выбрана популярная графовая база данных Neo4j~\cite{Neo4j}. Она обладает наглядным и удобным в использовании интерфейсом, имеет REST API для доступа из любого языка программирования и большое количество пользователей по всему миру.

 Для интегрирования был реализован класс $Neo4jGraph$, реализующий методы интерфейса $IGraph$. Каждая вершина в графовой базе данных имеет свой уникальный идентификатор. Мы используем его как индекс во входном потоке. Класс использует REST API для работы с базой данных,предоставленное компанией Neo4j.

 \textsc{Пример 2.} 
Для тестирования была взята графовая база данных Wine. Фрагмент представлен на рис.~\ref{GraphWine}. Каждая вершина представляет собой вид вина, регион или тело вина. Между ними представлены два вида связи: 
\begin{itemize}
\item $locatedIn$ – указывает на расположение объекта, из которого
исходит ребро;
\item $hasBody$ – указывает на цвет вина
\end{itemize}

\begin{figure}
 \centering
 \includegraphics[width=0.9\textwidth]{Smolina/pics/GraphWine.png}
 \caption{Фрагмент графовой базы данных Wine}
 \label{GraphWine}
\end{figure}

Для того чтобы узнать в каких регионах производят Riesling, необходимо выполнить запрос представленный на листинге~\ref{grmG6}. 

\begin{listing}
\caption{Грамматика $G_6$}
\label{grmG6}
\centering
$\begin{array}{rl}
S \rightarrow locatedId \ S | \ locatedIn
\end{array}$
 \end{listing}

В процессе работы было получено 3 пути, начинающиеся с вершины 9 и заканчивающиеся в вершинах 2, 3, 4. Получив соответствующие этим вершинам названия, мы узнали расположение региона, где производится вино Riesling: Central Texas, Texas, US.
 
 \textsc{Пример 3.}

 Другой пример, демонстрирующий необходимость в запросах, представленных в КС-грамматиках, это поиск потомков одного
поколения.

База данных моделирует семейное древо (см. рис.~\ref{GraphFamily}). Требуется найти всех потомков одного поколения, например, все братьев и сестер конкретного узла. Нахождение всех этих потомков может быть произведено запросом в виде грамматики $G_7$ на листинге~\ref{grmG7}. Данная грамматика представляет собой язык правильных скобочных последовательностей, где одна скобка – ребенок, другой родитель. Технически данные в базе данных представлены только переходы от предков к потомкам, поэтому отношение «ребенок» необходимо симулировать как обратное ребро. Для обработки данной ситуации был добавлен комбинатор «not», который говорит о том, что анализ должен продолжаться в обратном направлении, от узла, в которое входит ребро, в узел, из которого исходит ребро.

\begin{figure}
 \centering
 \includegraphics[width=0.8\textwidth]{Smolina/pics/GraphFamily.png}
 \caption{Фрагмент графовой базы данных «Семейное древо»}
 \label{GraphFamily}
\end{figure}

\begin{listing}
\caption{Грамматика $G_7$}
\label{grmG7}
\centering
$\begin{array}{rl}
S \rightarrow child \ S \ parent| \ epsilon
\end{array}$
 \end{listing}

 Найдем всех потомков одного поколения для Кати из базы данных «Семейное древо». Для этого запишем запрос в виде грамматики
представленной на листинге~\ref{grmG8}.

\begin{listing}
\caption{Грамматика $G_8$}
\label{grmG8}
\centering
$\begin{array}{rl}
S \rightarrow - \ parent \ S \ parent| \ epsilon
\end{array}$
 \end{listing}

 На выходе работы синтаксического анализатора было получено три дерева: дерево из вершины 17 в вершину 17, из 17 в 18 и из 17 в 19. Это
говорит о том, что в одном поколении находятся Катя, Кристина и Игорь. Так же из полученных деревьев можно узнать, кто является ближайшим общим родственником.

\section*{Заключение}
Поиск путей в графах, удовлетворяющих контекстно-свободным ограничениям может использоваться для решения различных задач, в том числе выполнения запросов к графовым базам данных. Существующие решения в основном основаны на генераторах синтаксических анализаторов, поэтому в данной работе была поставлена задача разработка нового инструмента для синтаксического анализа графов, основанном на технике парсер-комбинаторах, который бы:

\begin{itemize}
\item поддерживавал контекстно-свободные запросы;
\item конструировал в качестве результата работы лес разбора множества строк, удовлетворяющих ограничениям;
\item позволял взаимодействовать с одной из графовых баз данных.
\end{itemize}

Для модификации библиотеки Meerkat для синтаксического анализа графов были сделаны следующие шаги:
\begin{itemize}
\item преобразованы типы входных данных;
\item преобразован анализатор терминальных символов $terminal$;
\item добавлена функциональность для фильтрации возвращаемых деревьев;
\item добавлены средства для взаимодействия с графовой базой данных Neo4j;
\item добавлена дополнительная функциональность для указания направления поиска.
\end{itemize}

В результате работы была разработана библиотека синтаксического анализа графов, основанная на технике парсер-комбинаторов. Использование парсер-комбинаторов позволяет писать более модульный, человекочитаемый и поддерживаемый код синтаксических анализаторов. Более того, такие анализаторы не требуют исполнения дополнительного предметно-ориентированного языка для запросов или выполнения отдельного шага генерации анализатора по спецификации.

На данном этапе рекомендуется использование данной библиотеки для небольших объемов данных. В дальнейшем планируется улучшение производительности, поддержка некоторых форматов контекстно-зависимых грамматик, а также вычисления семантических действий.

\begin{thebibliography}{99}

\bibitem{GrigRagCFPQuerying}
  Grigorev S., Ragozina A. Context-Free Path Querying with Structural Representation of Result //arXiv preprint arXiv:1612.08872. – 2016.

\bibitem{Hellings}
  Hellings J. Conjunctive context-free path queries. – 2014. 

\bibitem{Sevon}
  Sevon P., Eronen L. Subgraph queries by context-free grammars

\bibitem{SPPF}
  Scott E., Johnstone A. GLL parse-tree generation //Science of
Computer Programming. – 2013. – Т. 78. – №. 10. – С. 1828-1844. 

\bibitem{Neo4j}
Webber J. A programmatic introduction to neo4j //Proceedings of the
3rd annual conference on Systems, programming, and applications: software
for humanity. – ACM, 2012. – С. 217-218.

\bibitem{Cypher}
Team N. Cypher Query Language. – 2013. 

\bibitem{openCypher}
Frost R. A., Hafiz R., Callaghan P. Parser combinators for
ambiguous left-recursive grammars //International Symposium on Practical
Aspects of Declarative Languages. – Springer Berlin Heidelberg, 2008. – С.
167-181.

\bibitem{Meerkat}
Izmaylova A., Afroozeh A., Storm T. Practical, general parser
combinators //Proceedings of the 2016 ACM SIGPLAN Workshop on
Partial Evaluation and Program Manipulation. – ACM, 2016. – С. 1-12

\bibitem{Scala}
Odersky M., Spoon L., Venners B. Programming in scala. – Artima
Inc, 2008. 

\bibitem{GLL}
Afroozeh A., Izmaylova A. Faster, Practical GLL Parsing //CC. –
2015. – Т. 15. – С. 89-108. 

\bibitem{RNGLR}
Verbitskaia E., Grigorev S., Avdyukhin D. Relaxed Parsing of
Regular Approximations of String-Embedded Languages //International
Andrei Ershov Memorial Conference on Perspectives of System Informatics.
– Springer International Publishing, 2015. – С. 291-302. 

\bibitem{IzmCombinator}
Izmaylova A., Afroozeh A., Storm T. Practical, general parser
combinators //Proceedings of the 2016 ACM SIGPLAN Workshop on
Partial Evaluation and Program Manipulation. – ACM, 2016. – С. 1-12.

\bibitem{OrientDB}
Tesoriero C. Getting Started with OrientDB. – Packt Publishing Ltd,
2013. 

\bibitem{Sql}
Date C. J., Darwen H. A Guide to the SQL Standard. – New York :
Addison-Wesley, 1987. – Т. 3. 

\bibitem{HOFunParsing}
G. Hutton. Higher-order Functions for Parsing. Journal of Functional
Programming, 2(3):323–343, July 1992.

\bibitem{Popov}
Попов Д. Оптимизирующие парсер-комбинаторы // Практика
функционального программирования. 2010, вып. №05. С. 162-186

\bibitem{Memoization}
Norvig P. Techniques for automatic memoization with applications to
context-free parsing //Computational Linguistics. – 1991. – Т. 17. – №. 1. –
С. 91-98

\bibitem{ParserComb}
Frost R. A., Hafiz R., Callaghan P. Parser combinators for
ambiguous left-recursive grammars //International Symposium on Practical
Aspects of Declarative Languages. – Springer Berlin Heidelberg, 2008. – С.
167-181.

\bibitem{MemoizationInTopDown}
Johnson M. Memoization in top-down parsing //Computational
Linguistics. – 1995. – Т. 21. – №. 3. – С. 405-417.

\bibitem{Graph}
Scala-graph library documentation [Электронный ресурс]. — URL:
www.scala-graph.org (дата обращения: 03.05.2016).

\bibitem{Hellings}
Hellings J. Path results for context-free grammar queries on graphs
//arXiv preprint arXiv:1502.02242. – 2015.

\end{thebibliography}
