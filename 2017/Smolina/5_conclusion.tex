\section*{Заключение}
Поиск путей в графах, удовлетворяющих контекстно-свободным ограничениям может использоваться для решения различных задач, в том числе выполнения запросов к графовым базам данных. Существующие решения в основном основаны на генераторах синтаксических анализаторов, поэтому в данной работе была поставлена задача разработка нового инструмента для синтаксического анализа графов, основанном на технике парсер-комбинаторах, который бы:

\begin{itemize}
\item поддерживавал контекстно-свободные запросы;
\item конструировал в качестве результата работы лес разбора множества строк, удовлетворяющих ограничениям;
\item позволял взаимодействовать с одной из графовых баз данных.
\end{itemize}

Для модификации библиотеки Meerkat для синтаксического анализа графов были сделаны следующие шаги:
\begin{itemize}
\item преобразованы типы входных данных;
\item преобразован анализатор терминальных символов $terminal$;
\item добавлена функциональность для фильтрации возвращаемых деревьев;
\item добавлены средства для взаимодействия с графовой базой данных Neo4j;
\item добавлена дополнительная функциональность для указания направления поиска.
\end{itemize}

В результате работы была разработана библиотека синтаксического анализа графов, основанная на технике парсер-комбинаторов. Использование парсер-комбинаторов позволяет писать более модульный, человекочитаемый и поддерживаемый код синтаксических анализаторов. Более того, такие анализаторы не требуют исполнения дополнительного предметно-ориентированного языка для запросов или выполнения отдельного шага генерации анализатора по спецификации.

На данном этапе рекомендуется использование данной библиотеки для небольших объемов данных. В дальнейшем планируется улучшение производительности, поддержка некоторых форматов контекстно-зависимых грамматик, а также вычисления семантических действий.