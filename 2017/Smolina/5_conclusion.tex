\section*{Заключение}
Синтаксический анализ графов является важной составляющей при работе с графами. Это один из подходов, который может использоваться для
работы с графовыми базами данных.

В ходе данной работе представлен краткий обзор существующих решений. Были выделены их плюсы и минусы, из чего была поставлена задача о создании нового инструмента для синтаксического анализа графов, который бы позволял:

\begin{itemize}
\item записывать запросы в коде программы в виде контекстно-свободной грамматики;
\item конструирует в качестве результата работы лес разбора
множества строк в графе;
\item взаимодействовать с одной из графовых баз данных.
\end{itemize}

Для выполнения данных задач были изучена техника парсер-комбинаторов, мемоизации и CPS. За основу решения была взята библиотека
синтаксического анализа Meerkat, краткий обзор работы которой был представлен во второй главе.

Для преобразования библиотеки Meerkat в библиотеку синтаксического анализа графов были сделаны следующие шаги:
\begin{itemize}
\item введены новые типы входных данных;
\item преобразован анализатор терминальных символов $terminal$;
\item добавлена функциональность для фильтрации возвращаемых деревьев;
\item добавлены средства для взаимодействия с графовой базой данных Neo4j;
\item добавлена дополнительная функциональность для указания направления поиска.
\end{itemize}

Таким образом поставленную задачу можно считать решенной. Нами была разработана библиотека синтаксического анализа графов, основанная
на технике парсер-комбинаторов. Использование парсер-комбинаторов позволяет писать более модульный, человекочитаемый и более просто поддерживаемый код синтаксических анализаторов. Более того, такие анализаторы становятся возможным создавать непосредственно в коде
целевой программы, что упрощает формирование запросов 

На данном этапе рекомендуется использование данной библиотеки для небольших объемов данных. В дальнейшем планируется модифицировать
данную библиотеку для больших объемов данных, расширить формат запроса до контекстно-зависимой грамматики, а также добавить такой функциональности, как вычисление семантических действий.